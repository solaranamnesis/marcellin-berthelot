\documentclass[a4paper, 11pt, oneside, polutonikogreek, french]{article}
\usepackage[sfdefault]{biolinum}
\usepackage[T1]{fontenc}
% Load encoding definitions (after font package)

\usepackage{textalpha}
\usepackage{bbding}

% Babel package:
\usepackage[french]{babel}
\usepackage{listings}
\lstset{basicstyle=\ttfamily}

% With XeTeX$\$LuaTeX, load fontspec after babel to use Unicode
% fonts for Latin script and LGR for Greek:
\ifdefined\luatexversion \usepackage{fontspec}\fi
\ifdefined\XeTeXrevision \usepackage{fontspec}\fi

% "Lipsiakos" italic font `cbleipzig`:
\newcommand*{\lishape}{\fontencoding{LGR}\fontfamily{cmr}%
		       \fontshape{li}\selectfont}
\DeclareTextFontCommand{\textli}{\lishape}

\usepackage{booktabs}
\usepackage{graphicx}
\setlength{\emergencystretch}{15pt}
\graphicspath{ {./ } }
\usepackage[figurename=]{caption}
\usepackage{float}
\usepackage{fancyhdr}
\usepackage{microtype}
\usepackage{wasysym}
\usepackage{svg}

\newcommand*\svgA{\includesvg[height=0.7em]{svgs/01.svg}}
\newcommand*\svgB{\includesvg[height=0.7em]{svgs/02.svg}}
\newcommand*\svgC{\raisebox{0.5ex}{\includesvg[width=1em]{svgs/03.svg}}}

\begin{document}
\begin{titlepage} % Suppresses headers and footers on the title page
	\centering % Centre everything on the title page
	%\scshape % Use small caps for all text on the title page

	%------------------------------------------------
	%	Title
	%------------------------------------------------
	
	\rule{\textwidth}{1.6pt}\vspace*{-\baselineskip}\vspace*{2pt} % Thick horizontal rule
	\rule{\textwidth}{0.4pt} % Thin horizontal rule
	
	\vspace{1\baselineskip} % Whitespace above the title
	
	{\scshape\Huge Collection des Anciens Alchimistes Grecs}
	
	\vspace{1\baselineskip} % Whitespace above the title

	\rule{\textwidth}{0.4pt}\vspace*{-\baselineskip}\vspace{3.2pt} % Thin horizontal rule
	\rule{\textwidth}{1.6pt} % Thick horizontal rule
	
	\vspace{1\baselineskip} % Whitespace after the title block
	
	%------------------------------------------------
	%	Subtitle
	%------------------------------------------------
	
	{\scshape \normalsize Publiée sous les Auspices du Ministère de l'Instruction Publique}
	
	{\scshape Par \Large Marcellin Berthelot} % Subtitle or further description
	
	\vspace*{1\baselineskip} % Whitespace under the subtitle
	
        {\scshape\scriptsize Sénateur, Membre de l'Institut, Professeur au Collège de France \\Avec la Collaboration de \\\large M. Ch.-Em. Ruelle,\\\scriptsize Bibliothécaire a la Bibliothèque Sainte-Geneviève} % Subtitle or further description
  
        \vspace{4\baselineskip}
  
	{\scshape \normalsize Seconde Livraison.} % Subtitle or further description
  

	%------------------------------------------------
	%	Editor(s)
	%------------------------------------------------
        \vspace*{\fill}

	\vspace{1\baselineskip}

	{\small\scshape Paris 1888}
	
	{\small\scshape{Georges Steinheil, Éditeur \\2, Rue Casimir-Delavigne, 2}}
	
	\vspace{0.5\baselineskip} % Whitespace after the title block

    \scshape Internet Archive Online Edition  % Publication year
	
	{\scshape\small Utilisation non commerciale --- Partage dans les mêmes conditions 4.0 International} % Publisher
\end{titlepage}
\setlength{\parskip}{1mm plus1mm minus1mm}
\clearpage
\tableofcontents
\clearpage
\section{Texte Grec.}
\subsection{Troisième Partie. --- Zosime.}
\subsubsection{3. --- 1. ΖΩΣΙΜΟΥ ΤΟΥ ΘΕΙΟΥ ΠΕΡΙ ΑΡΕΤΗΣ.}
\begin{center}
< ΠΡΛΞΙΣ Α >
\end{center}
\paragraph{}
\emph{Transcrit sur} M, f. 92 v. --- \emph{Collationné} (le § 1, seul existant) \emph{sur} M, f. 115 r. (= M\textsuperscript{2}) ;--- \emph{sur} A, f. 85 r. ;--- \emph{sur} K, f. 1 r. ;--- \emph{sur} Lc, page 265. --- \emph{Nous noterons ici, une fois pour toutes, que les leçons de} M \emph{différentes de celles de} A K \emph{ont été reproduites, dans le manuscrit} K, \emph{soit en marge, soit sur la ligne par une main élégante} (Kmg.), \emph{contemporaine du ms. Il en est de même des leçons de} M. \emph{omises dans le texte de} A K.

\bigskip

1. Θέσις ὑδάτων, καὶ κίνησις, καὶ αὔξησις, καὶ ἀποσωμάτωσις,\footnote{Titre dans AK : Ζωσίμου ἀρετῆς περὶ συνθέσεως ὑδάτων αʹ. --- Dans M\textsuperscript{2} (Ζωσίμου ἀρετῆς omis) : περὶ συνθέσεως ὑδάτων.} καὶ ἐπισωμάτωσις, καὶ ἀποσπασμὸς πνεύματος ἀπὸ σώματος, καὶ σύνδεσμος πνεύματος μετὰ σώματος, οὐ ξένων ἢ ἐπεισάκτων φύσεων,\footnote{μετὰ] ἐπὶ AKLc. --- ού ξένων --- φύσεων] Réd. de Lc : οὐ ξένον ἢ ἐπείσακτον πρᾶγμά ἐστι τῶν φύσεων. --- φύσεως M\textsuperscript{2}.} ἀλλ ᾽ αὐτὴ καὶ μόνη εὶς ἐαυτὴν (f. 93 r.) ἡ μονοειδὴς φύσις\footnote{μόνη] μόνον M\textsuperscript{2}.} κέκτηται τά τε στερεόστρακα τῶν μετάλλων καὶ τὰ ὑγρόδρυα τῶν\footnote{τὰ στερέα ὄστρακα M\textsuperscript{2} AK Lc.} βοτανῶν · καὶ ἐν τούτῳ τῷ μονοειδῇ καὶ πολυχρώμῳ σχήματι\footnote{τῷ] τῶ et au-dessus : καὶ K. --- καὶ πολυχρ]. τῶ πολυχρ. M\textsuperscript{2} A. --- σχήματι σώζεται] σχηματίζεται M\textsuperscript{2} A ; σχηματισώ ζεται K ; πολυχρ. πράγματι σχηματίζεται Lc. f. mel.} σώζεται ἡ τῶν πάντων πολύλεκτος καὶ παμποίκιλος ζήτησις\footnote{ἡ τῶν πάντων --- τὴν λῆξιν] Réd. de M\textsuperscript{2} A : ἡ τοῦ παντὸς πολυτύλικτος (πολυτύληκτος M\textsuperscript{2}) παμποικιλία καὶ ζήτησις · ὄθεν ... ὑποβάλλει τὴν λῆξιν. Réd. de K : ἡ τοῦ παντός πολυτύλικτος (en marge : πολυλεκτὸς) παμποικιλία καὶ ζητ. · ὄθεν κ. τ. λ. (comme dans M).} · ὅθεν καὶ σεληνιαζομένης τῆς φύσεως τῷ μέτρῳ τῷ χρονικῷ ὑποβάλλεται, καὶ τὴν λῆξιν καὶ τὴν αὔξησιν δι ᾽ ἦς ὑποφεύγει ἡ φύσις.\footnote{δι ᾽ ἧς ὑποφεύγει ἡ φύσις] Leçon de M\textsuperscript{2} ALc ; δὶς ἱπεύει ἡ φ. M.}

2. Καὶ ταῦτα λαλῶν ἀπεκοιμήθην, καὶ ὅρω ἱερουργόν τινα ἑστῶτα ἔμπροσθέν μου ἐπάνω βωμοῦ φιαλοειδοῦς. Ἔνθα δεκαπέντε\footnote{μου] τοῦ AKLc. --- φιαλ.] τοῦ φιαλ. Lc. --- δεκαπέντε] αἱ AK ; τὰς Lc.} κλίμακας πρὸς ἀνάβασιν εἶχεν ὁ αὐτὸς βωμός. Ἔνθα ὁ ίερεὺς ἵστατο, καὶ φωνῆς ἄνωθεν ἤκουσα λεγούσης μοι · « Πεπλήρωκα τοῦ\footnote{ἄνωθεν ἤκ. λεγ. μοι] ἤκ. λεγ. μοι ἄνωθεν Lc. --- πεπληρώκαται A ; πεπληρώκατε K.} κατιέναι με ταύτας τὰς δεκαπέντε σκοτοφεγγεῖς κλίμακας, καὶ ἀνιέναι με τὰς φωτολαμπεῖς κλίμακας. Καὶ ἔστιν καὶ ὁ ἱερουργῶν καινουργῶν με, ἀποβαλλόμενος τὴν τοῦ σώματος παχύτητα, καὶ\footnote{καὶ καινουργῶν Lc. --- ἀποβαλλόμενος] ὃς ἀποβάλλει Lc. --- Après παχύτητα] ἀπ ᾽ ἐμοῦ add. Lc, f. mel.} ἐξ ἀνάγκης ἱερατευόμενος πνεῦμα τελοῦμαι. Καὶ ἀκούσας τῆς\footnote{καὶ ἐξ ἀνάγκης --- ἀκούσας] Réd. de Lc : ἐγὼ δὲ ἐξ ἀν. ίερατεύομαι καὶ πνευματοτελειοῦμαι · ἐγὼ δὲ ἀκούσας. --- τελοῦμαι] τελειούμενοι AK (biffé dans K) et Kmg. : τελοῦμαι.} φωνῆς αὐτοῦ ἐν τῷ φιαλοβωμῷ ἑστῶτος, ἠρώτων βουλόμενος\footnote{αὐτοῦ τοῦ Lc. --- ἑστῶτι MK. --- ἠρωτοῦν M ; ἡρώτονμαι μαθεῖν A ; ἠρώτουν με μαθεῖν K (με sous-pointillé). --- Ajouté βουλόμενος d'après Lc.} μαθεῖν παρ ᾽ αὐτοῦ τίς ὑπάρχει. Ὁ δὲ ἰσχνοφώνως ἀπεκρίνατό μοι\footnote{ὁ δὲ --- ἀπεκρίν.] οὗτος ὁ ἰσχνόφωνος · αὐτὸς δὲ ἀπεκρίν Lc. --- ἰσχνοφώνος A ; ίσχνόφωνος K.} λέγων · « Ἐγώ εἰμι ὁ Ἴ ω ν ὁ ἰερεὐς τῶν ἀδύτων, καὶ βίαν\footnote{ὁ Ἴων] οἶων A ; ὁ ὢν Lc.} ἀφόρητον ὑπομένω. Ἦλθεν γάρ τις περὶ τὸν ὄρθρον δρομαίως, καὶ\footnote{Ajouté καὶ d'après Lc.} ἐχειρώσατό με μαχαίρη διελών με, καὶ διασπάσας κατὰ σύστασιν ἀρμονίας. Καὶ ἀποδερματώσας τὴν κεφαλήν μου τῷ ξίφει τῷ\footnote{Avant τὴν κεφαλήν] πᾶσαν add. A.} ὑπ ᾽ αὐτοῦ κρατουμένῳ, τὰ ὀστέα ταῖς σαρξὶ συνέπλεξεν, καὶ τῷ πυρὶ τῷ διαχείρως κατέκαιεν, ἕως ἂν ἔμαθον μετασωματούμενος πνεῦμα γενέσθαι. Καὶ αὕτη μου ἐστὶν ἡ ἀφόρητος βία. » Καὶ ὡς ἔτι ταῦτά μοι διελέγετο, καὶ ἐξεβιαζόμην αὐτὸν εὶς τὸ λέγειν, ὥσπερ\footnote{διελέγετο] ἔλεγε Lc. --- ὥσπερ] F. l. ὄπως.} αἷμα γεγόνασιν οἱ ὀφθαλμοὶ αὐτοῦ. Καὶ ἤμεσεν πάσας τὰς σάρκας\footnote{γεγόνασιν οἱ ὀφθ. αὐτοῦ ὥσπερ αἷμα Lc.} αὐτοῦ. Καὶ (f. 93 v.) εἶδον αὐτὸν ὡς τοὐναντίον ἀνθρωπάριον\footnote{αύτοῦ] F. l. αὐτοῦ. --- τοὐναντίον om. Lc, f. mel.} κολοβόν · καὶ τοῖς ὀδοῦσιν ἐαυτοῦ ἐαυτὸν μασσώμενον, καὶ συμπίπτοντα.\footnote{ἑαυτοῦ] αὐτοῦ Lc ; om. A. --- μασσῶντα Lc.}

3. Καὶ φοβηθεὶς διυπνίσθην καὶ ἐνεθυμήθην ; « Μὴ οὕτως ἄρα\footnote{διυπνίσθην] δὲ ὑπνίσθην MAK.} ἐστὶν ἡ τῶν ὑδάτων θέσις ; » Ἔδοξα πείθειν ἑαυτὸν νενοηκέναι καλῶς. Καὶ πάλιν ἀπεκοιμήθην. Καὶ εἶδον τὸν αὐτὸν φιαλοβωμὸν, καὶ ἐπάνω ὕδωρ καχλάζον, καὶ πολὐν λαὸν εἰς αὐτὸν ἄπειρον ὄντα. Καὶ οὐκ ἧν\footnote{κοχλάζον MAK ici et plus loin (p. suiv., l. 8).} τις ἵνα ἐρωτήσω αὐτὸν ἔξω τοῦ βωμοῦ. Καὶ ἀνέρχομαι ἐπὶ τὸ ἰδέσθαι\footnote{Lc place ἔξω τοῦ βωμοῦ aussitôt après τις. --- ἀνέρχομαι ἐπὶ τὸ ίδέσθαι] ἀνερχόμενος ἐπιτηδεύεσθαι AK ; ἀνερχόμενος δὲ πρὸς τὸ ἐπιτηδεύεσθαι Lc.} τὴν θέαν εὶς τὸν βωμόν. Καὶ ὁρῶ πεπολιωμένον ξηρουργὸν ἀνθρωπάριον\footnote{εὶς τὸν βωμὸν] τοῦ βωμοῦ K. --- Καὶ ἰδοὐ ὁρῶ Lc. --- ξηρουργὸν] υ au-dessus de η dans M.} λέγοντά μοι. « Τί σκοπεῖς. » Ἀπεκρινάμην αὐτῷ ὅτι θαυμάζω\footnote{λέγοντά μοι] καὶ λέγει μοι AKLc. --- Καὶ ἀπέκριν. ALc.} τοῦ ὕδατος τὸν βρασμὸν καὶ τῶν ἀνθρώπων συγκαιομένων καὶ ζώντων.\footnote{καὶ τῶν ἀνθρ. συγκ. καὶ ζ.] καὶ τοὐς ἀνθρώπους καὶ ζῶντας (τοὐς ζ. Lc) συγκαιομένους AKLc.} Καὶ ἀπεκρίνατο μοι λέγων. « Αὕτη ἡ θέα ἣν ὁρᾷς εἴσοδός ἐστι καὶ ἔξοδος καὶ μεταβολή. Ἐπηρώτησα οὖν αὐτὸν πάλιν. « Ποία μεταβολή\footnote{μεταβολῆ M. --- καὶ ἐπηρ. αύτὸν πάλιν Lc. --- ποῖα μέταβολῆ M.} ; » Καὶ ἀπεκρίνατο λέγων. « Τόπος ἀσκήσεως τῆς λεγομένης\footnote{μοὶ λέγων AKLc. --- τόπος] F. l. τρόπος. --- τόπος ἀσκ. οὗτος τῆς λεγ. ταρ. ἐστίν Lc.} ταριχείας. Οἱ γὰρ θέλοντες ἄνθρωποι ἀρετῆς τυχεῖν ὧδε εἰσέρχονται, καὶ γίνονται πνεύματα, φυγόντες τὸ σῶμα. » Ἔλεγον οὖν αὐτῷ. « Καὶ σὺ πνεῦμα εἶ ; » Καὶ ἀπεκρίνατο λέγων. « Καὶ πνεῦμα καὶ\footnote{μοι λέγων AKLc.} φύλαξ πνευμάτων. » Καὶ ἐν τῷ ὁμιλεῖν ἡμᾶς ταῦτα, καὶ προστιθεμένου τοῦ βρασμοῦ καὶ τοῦ λαοῦ ὁλολύζοντος, εἶδον ἄνθρωπον χαλκοῦν δέλτον μολυβδίνην κατέχοντα ἐν τῆ χειρὶ αὐτοῦ. Καὶ ἐξεῖπεν\footnote{αὐτοῦ] F. l. αὑτοῦ. --- Réd. de Lc : καὶ ἐξεῖπέ μοι τῇ φωνῇ · ὄρα ταύτῃ τῇ δέλτῳ ἐν ταῖς κ. π. ἐπιτρ. καθεσθῆναι, κελεύω δὲ ἕκαστον.} τῆ φωνῆ βλέπων τὴν δέλτον. « Τοῖς ἐν ταῖς κολάσεσι πᾶσιν ἐπιτρέπω\footnote{ἐπιτρέπων AK.} καθευθῆναι καὶ ἕκαστον ἐν τῇ χειρὶ αὐτοῦ λαβεῖν δέλτον μολυβδίνην, καὶ\footnote{καθευθῆναι] AK. F. l. καθαρθῆναι, avoir été purifié (αρ diffère peu de ευ dans les mss. du Xe siècle).} χειρὶ γράφειν, καὶ τὰς ὄψεις < ἔχειν > ἄνω καὶ τὰ στόματα ὑμῶν ἀνεωγμένα,\footnote{Réd. de Lc : ... γράφειν ἕως ἄν αὐξ. ἡ σταφ. αὐτῶν καὶ τὰ στόμ. αὐτῶν ἀνεωγ. καὶ τὰς ὄψ. ἄνω ἔχειν. Ajouté ἔχειν d'après Lc.} ἕως ἂν αὐξήσῃ ἡ σταφυλὴ ὑμῶν. Καὶ τῷ λόγῳ τὸ ἔργον ἠκολούθει, καὶ λέγει μοι ὁ οἰκοδεσπότης. « Ἐθεώρησας · ἐξέτεινας τὸν αὐχένα σου ἄνω, καὶ εἶδες τὸ πραχθέν ; » Καὶ εἶπον ὄτι εἶδον, καὶ λέγει μοι ὅτι « Τοῦτον ὃν εἶδες χαλκάνθρωπον (f. 94 r.),\footnote{Lc place les mots καὶ τὰς ἰδίας σάρκας ἐξιοῦντα (\emph{sic}) après χαλκάνθρωπον, ce qui vaut mieux.} οὗτός ἐστιν ὁ ἱερουργῶν καὶ ἱερουργούμενος, καὶ τὰς ἰδίας σάρκας ἐξεμοῦντα.\footnote{ἐξεμοῦντα] ἐξιοῦντα AK.} Καὶ αὐτῶ ἐδόθη ἡ ἐξουσία τοῦ ὕδατος τούτου καὶ τῶν τιμωρουμένων.\footnote{τούτου --- τιμωρουμένων] τούτου καὶ ἔστιν ὁ τιμωρούμενος Lc, f. mel.}

4. Καὶ ταῦτα ἐμφαντασθεὶς διυπνίσθην πάλιν. Καὶ εἶπον πρὸς\footnote{Réd. de Lc : Καὶ ταῦτα ἐφαντάσθην καὶ πάλιν διυπνίσθην.} ἑαυτόν · « Τίς ἡ αἰτία τῆς ὀπτασίας ταύτης ; Μὴ ἆρα τοῦτό ἐστιν\footnote{ἑαυτὸν] ἐμαῦτὸν Lc. --- Après ταύτης] τί τοῦτο εἶναι AK (sous-pointillé dans Κ) ; τί τοῦτό ἐστι ; Lc.} τὸ ὕδωρ τὸ λευκόν τε καὶ ξανθὸν τὸ καχλάζον, τὸ θείον ; » Καὶ ηὗρον ὅτι μᾶλλον καλῶς ἐνόησα. Καὶ εἶπον ὅτι καλὸν τὸ λέγειν, καὶ καλὸν τὸ ἀκούειν, καὶ καλὸν τὸ διδόναι, καὶ καλὸν τὸ λαμβάνειν, καὶ καλὸν τὸ πενητεύειν, καὶ καλὸν τὸ πλουτεῖν. Καὶ πῶς ἡ φύσις μανθάνει διδόναι καὶ λαμβάνειν ; Δίδωσιν ὁ χαλκάνθρωπος, καὶ λαμβάνει ὁ ὑγρόλιθος · δίδωσι τὸ μέταλλον, καὶ λαμβάνει ἡ βοτάνη · δίδουσιν οἱ ἄστερες, καὶ λαμβάνει τὰ ἄνθη · δίδωσιν ὁ οὐρανὸς,\footnote{λαμβάνουσιν AK.} καὶ λαμβάνει ἡ γῆ δίδουσιν αἱ βρονταὶ τοῦ ἐκτροχίζοντος πυρός.\footnote{ἐκτροχίζοντος] ἐκ τοῦ τροχίζοντος AKLc (ἐκ sous-pointillé dans K). F. l. ἐκτροχάζοντος.} Καὶ συμπλέκονται τὰ πάντα, καὶ ἀποπλέκονται τὰ πάντα, καὶ μίσγονται\footnote{καὶ μίσγονται --- φιαλοβωμῷ] Réd. de Lc : καὶ συντίθενται τὰ π., καὶ μιγνύονται τ. π. καὶ ἀποκίρνανται τ. π., καὶ κυβερνᾶται τ. π. καὶ ἀποβρέχονται τὰ πάντα, κ. τ. λ.} τὰ πάντα, καὶ συντίθενται τὰ πάντα, καὶ κίρναται τὰ πάντα,\footnote{κιρνᾶται M, κυβέρνατε A ; κυβερνᾶται KLc.} καὶ ἀποκίρναται τὰ πάντα, καὶ βρέξει τὰ πάντα, καὶ ἀποβρέξει τὰ πάντα,\footnote{βρέχει A, f. mel. --- ἀποβρέχονται Lc.} καὶ ἀνθεῖ τὰ πάντα, καὶ ἐξανθεῖ τὰ πάντα ἐν τῷ φιαλοβωμῷ. Ἕκαστον\footnote{ἕκαστον --- Καὶ τὰ πάντα] Réd. de AΚ : ἔκκοπον ἄριστον μεθόδῳ καὶ συγκόματι καὶ συνκεράσματι τετραστίχω, ἡ (εἰ A) τῶν ὅλ. συμπλ. ἐστιν καὶ φυσήματι καὶ τὰς τάξεις τηροῦσα τῆς μεθ. αὔξ. καὶ ὀλιγοῦσα, καὶ πάντα. --- Réd. de Lc : ἀρίστῳ μεθόδῳ καὶ συγκόμματι καὶ οὐγγιασμῷ, καὶ συγκεράσματι τετραστοίχῳ · ἡ δὲ τῶν ὅλ. πραγματεία συμπλ. ἐστι καὶ ἀποπλ. καὶ ὁ π. σ. οὐκ ἀ. μεθ. γίν · ἡ μέθ. φυσ. ἐ. καὶ φυσ. κ. ἐκφυσ. κ. τὰς τ. τηρ. τῆς μεθ., αὐξάνουσα καὶ ἐλαττοῦοα. κ. τὰ πάντα ...} γὰρ μεθόδῳ καὶ σηκώματι καὶ οὐγγιασμῶ τετραστοίχῳ, ἡ τῶν ὅλων συμπλοκὴ, καὶ ἀποπλοκὴ, καὶ, ὁ πᾶς σύνδεσμος ἄνευ μεθόδου\footnote{καὶ ἀποπλοκὴ restitué en marge de M et de K.} οὐ γίνεται. Ἡ μέθοδος φυσική ἐστιν, καὶ φυσῶσα καὶ ἐκφυσῶσα, καὶ τὰς τάξεις τηροῦσα τῆς μεθόδου, αὔξουσα καὶ λήγουσα.\footnote{τηροῦσα] στηροῦσα M.} Καὶ τὰ πάντα ὡς ἐν συντόμῳ σύμφωνα τῇ διαιρέσει καὶ τῇ ἑνώσει,\footnote{ἐν συντόμῳ] συντόμως Lc. --- Après ἑνώσει] Réd. de Lc : ποιοῦσα τῇ μεθόδῳ μηδενὸς ὑποληφθέντος · ἡ γὰρ μέθοδος ἐκστρέφει τὴν φύσιν, καὶ ἡ φύσις στρεφ. κ. τ. λ.} τῆς μεθόδου μηδὲν ὑπολειφθείσης, ἐκστρέφει τὴν φύσιν. Ἡ γὰρ φύσις στρεφομένη εἰς ἑαυτὴν στρέφεται · καὶ αὕτη ἐστὶν ἡ τοῦ παντὸς κόσμου τῆς ἀρετῆς φύσις καὶ σύνδεσμος.\footnote{A mg. : σῆ.}

5. Καὶ ἵνα μὴ διὰ πολλῶν σοι γράφω, φίλτατε, κτῖσαι ναὸν\footnote{Ἵνα δὲ μή σοι δ. π. γρ. ὦ φιλτ. κτίσον ν. μ. ψιμμυθ. κ. τ. λ. Lc. --- γράφω] λέγω ἢ γρ. AK.} μονόλιθον ψιμυθοειδῆ, ἀ- (f. 94 v.) λαβαστροειδῆ, προκοννήσιον,\footnote{Προικοννήσιον Lc.} μήτε ἀρχὴν ἔχοντα, μήτε τέλος ἐν τῇ οἰκοδομῇ · πηγὴν δὲ ἔσωθεν ἔχουσαν ὕδατος καθαρωτάτου, καὶ φῶς ἐξαστράπτον ἡλιακόν. Περιέργασαι\footnote{καὶ περιεργάζου ποῦ ἐστιν ἡ εἴσ. Lc.} δὲ πόθεν ἡ εἴσοδος τοῦ ναοῦ, καὶ λάβε ἐπὶ χεῖράς σου ξίφος, καὶ οὕτως\footnote{λαβὼν AKLc.} ζήτει τὴν εἴσοδον. Στενόστομος γάρ ἐστιν ὁ τόπος ὅθεν ἐστὶν ἡ ἄνοιξις\footnote{στενόστομος γὰρ] στενὸς γάρ μοι AK ; στενὸς γάρ Lc. --- ὅθεν] ἔνθα AK ; ὅπου Lc.} τῆς εἰσόδου · καὶ δράκων παράκειται τῇ εἰσόδῳ, φυλάττων τὸν ναόν.\footnote{καὶ δράκων] δράκων δέ τις Lc. --- Lc mg. : Ligne verticale, en guise de guillemets jusqu'à la fin du paragraphe.} Καὶ τοῦτον χειρωσάμενος, πρῶτον θῦσον · καὶ ἀποδερματώσας αὐτὸν,\footnote{ἀποδερμάτωσον AKLc. --- αὐτὸν om. AKLc.} καὶ λαβὼν τὰς σάρκας αὐτοῦ μετὰ τῶν ὀστέων, διέλῃς μέλη [μέλη],\footnote{καὶ λαβὼν τὰς σάρκας αὐτοῦ --- καὶ ἀνάβηθι] Réd. de Lc : καὶ λαβὼν τ. σ. αὐτοῦ, δίελε εἰς τὰ μέλη αὐτοῦ καὶ σύνθες πάντα τὰ μέλη τοῖς μέλεσι μ. τ. ὀστέων · καὶ ποίησον σεαυτῷ βάσιν πρὸς τὸ στ. καὶ ἀνάβηθι. --- μέλη μέλη] F. l. μέλη μεληδὸν.} καὶ συνθεὶς μέλος [μέλος] μετὰ τῶν ὀστέων πρὸς τὸ στόμιον τοῦ ναοῦ\footnote{μέλος μέλος] F. l. μέλος μέλει.} ποίησον ἑαυτῷ βάσιν, καὶ ἀνάβηθι, καὶ εἴσελθε, καὶ εὑρήσεις ἐκεῖ τὸ ζητούμενον χρῆμα. Τὸν γὰρ ἱερέα τὸν χαλκάνθρωπον ὃν ὁρᾷς ἐν τῇ\footnote{τὸν γὰρ ἱερέα τὸν χαλκ.] ὁ γὰρ ἱερεὺς ὁ ὢν χαλκάνθρωπος Lc. --- Rapprocher de ce passage le morceau 3, 35.} πηγῇ καθήμενον καὶ τὸ χρῆμα συνάγοντα · ἐκεῖνον δὲ οὐχ ὡς χαλκάνθρωπον\footnote{τὸ χρῆμα] F. l. τὸ χρῶμα (\emph{M. B.}). --- οὐχ ὁρᾶς AK qui omettent ὡς. --- Réd. de Lc : Οὐχ ὁρᾷς δὲ αὐτὸν εἶναι χαλκ.} · μετέβη γὰρ τοῦ χρώματος τῆς φύσεως, καί γέγονεν ἀργυράνθρωπος,\footnote{μετέθη τὰ τοῦ χρώματος A ; μετέχθη γὰρ (ajouté) τὰ τ. χρ. K ; μεταβάλλεται ἐκ τοῦ χρ. Lc.} ὅν μετ ᾽ ὀλίγον ἐὰν θελήσῃς ἕξεις χρυσάνθρωπον.\footnote{ἕξεις] εὑρήσεις AKLc.}

6. Τοῦτο τὸ προοίμιόν ἐστιν εἴσοδος τοῦ ἀνοίγεσθαί σοι τὰ παρακάτω\footnote{τοῦτο --- γινομένων] Réd. de Lc : Καὶ τοῦτο ἔστω σοι τὸ πρ. Ἀνοίγονται δέ σοι μετέπειτα τὰ ἄνθη τῶν λόγ. καὶ αἱ ζητ. τῆς ἀρετῆς κ. τ. σ. κ. τῆς φύσεως, κ. τῆς φρ. καὶ τὰ δ. τοῦ νοῦ καὶ αἱ μεθ. αἱ δρ. καὶ αἱ ἀπ. τῶν κεκρ. ῥ. φανερῶν γενομένων.} ἄνθη λόγων, καὶ ζητήσεις ἀρετῶν, καὶ σοφίας, καὶ φρονήσεως, καὶ νοῦ δόγματα, καὶ μέθοδοι δραστικαὶ, καὶ ἀποκαλύψεις κεκρυμμένων ῥήσεων εἰς φανερὸν γινομένων · καὶ τὸ πᾶν ὁ τῆς ἀρετῆς μεθοδεύει ὁ χρόνος.\footnote{καὶ τὰ πάντα τῆς ἀρ. AK. --- Réd. de Lc : τὰ δὲ πάντα τ. ἀρ. μεθοδεύσει σοι χρόνος · καὶ ἡ φύσις ἡ νικ. τὰς φ., ἀποτ. τελεία φύσις.}

7. Καὶ τί ἐστιν « νικῶσα φύσις τὰς φύσεις, » καὶ « ἀποτελεῖται καὶ γίνεται ἰλιγγιῶσα, » καὶ « ἐκθλιβομένη πρὸς τὴν ζήτησιν, κοινὸν πρόσωπον τοῦ παντὸς τῆς ἐργασίας ὁρωμένης, ἀναλαμβάνει\footnote{κοινὸν πρόσωπον ... ] κοινοῦ προσώπου τ. π. τ. ἐ. ὁρᾶται Lc.} καὶ τὴν οἰκείαν ὕλην τοῦ εἴδους κατεσθίει ; » Καὶ « εἶθ ᾽ οὕτως πεσοῦσα\footnote{καὶ τὴν οἰκ.] καὶ om. AK. --- Réd. de Lc : καὶ ἀναλαμβ. τὴν οἰκ. ὕλην καὶ τὸν ἰὸν κατεστίει · εἶθ οῦτως ... --- Τοῦ ἰοῦ δὲ κατεσθίον AK.} τοῦ προτέρου σχήματος θνήσκειν οἴεται ; » Καὶ « ὅταν βαρβαρίζουσα\footnote{θνήσκειν οἴεται] θνήσκει Lc. --- ὅτε βαρβαρίζειν AK : ἣ καὶ ὅτε βαρβαρίζει Lc.} μιμεῖται οἷον ἰουδαϊκὴν ἔχοντος, τότε διεκδικήσασα ἑαυτὴν ἡ τάλαινα\footnote{ἐκδικήσαντα AK. --- Réd. de Lc : μιμεῖται τὸν τὴν ἰουδ. γλῶσσαν λαλοῦντα, ποτὲ δὲ ἐκδικήσαντα.} κουφοτέρα ἑαυτῆς γίνεται, μίξιν ἔχουσα τῶν ἰδίων (f. 95 r.) μελῶν ; » Καὶ « τὸ ὑγρὸν ἅμα πυρὶ καὶ τελεσφορεῖται ; »

8. Ἐν τούτοις τοῖς νοήμασι τοῦ νοῦ σαφῶς ἐκστρέψας τὴν\footnote{Ἐν τούτοις οὗν Lc.} φύσιν ἐπίστηθι, καὶ τὴν πολύϋλον ὡς μονόϋλον λογίζου, μηδενὶ σαφῶς καταλέγων τὴν τοιαύτην ἀρετὴν, ἀλλ ᾽ αὐτὸς ἑαυτῷ ἀρκέσθητι, μή πως καὶ λέγων ἑαυτὸν ἀνέλῃς. Ἡ γὰρ σιωπὴ διδάσκει τὴν ἀρετήν. Καλὸν ἰδεῖν τῶν τεσσάρων μετάλλων τὰς μεταβολὰς,\footnote{κάλλιστον δὲ ἔστιν ἰδεῖν Lc.} μολύβδου, χαλκοῦ, ἀσήμου, ἀργύρου, κασσιτέρου εἰς τὸ γενέσθαι\footnote{μολύβδου ... ] ἤγουν τοῦ μολ., τοῦ χ., τοῦ κασ., τοῦ ἀργ., ἵνα γένωνται τέλειος χρυσός Lc ; même leçon dans AK jusqu'à ἀργ., moins le mot ἤγουν, --- Les mots ἀσήμου et ἀργύρου sont la traduction du signe lunaire ; l'un des deux est de trop. Lc écrit ἀργύρου en toutes lettres.} τέλειον χρυσόν. Λαβὼν ἅλας νότισον τὸ θεῖον τὸ ἀγλαΐζον τὸ κηρομελές\footnote{νότισον] πότησον AK ; πότισον Lc. --- Le π et le ν diffère peu dans la cursive du 4\textsuperscript{e} au 7\textsuperscript{e} siècle.} · δῆσον ὁποτέρων τὴν ἰσχὺν, καὶ χάλκανθον μεσίτευε,\footnote{δῆσον ὅτι τὴν ἰ. ἔχων καὶ χαλκ. AK ; καὶ δῆσον ὅτι τὴν ἰ. ἔχει, καὶ μεσ. χαλκ. Lc. F. l. νόησον.} καὶ ποίησον ὄξος ἐξ αὐτῶν πρωτοζύμιον ἀργοὺς καὶ χαλκάνθου · κατὰ\footnote{αὐτῶν] αὐτοῦ Lc. --- πρωτοζώμιον AKLc. --- ἀργοῦς Lc. --- καὶ χαλκάνθου · ] Réd. de Lc : τὸν δἐ χάλκανθον ποίει κ. β., καὶ ἐν τούτοις. --- χάλκανθον AK, --- καταβαθμὸν M ; καταβαθμῶν AK.} βαθμὸν δὲ καὶ ἐν τούτοις τὸν λευκοειδῆ δαμάσεις χαλκὸν ἀνάγκῃ,\footnote{ἀνάγκῃ] ἀνάγαγε AK ; καὶ ἀνάγαγε αὐτὸν καὶ εὑρ. Lc.} καὶ εὑρήσεις μετὰ πέμπτην μέθοδον ὑπὸ τὰς γʹ αἰθάλας, ἑξῆς γίνεται\footnote{μέθοδον, ὑπὸ δὲ τὰς τρεῖς αἰθ. Lc.} ὁ λεγόμενος χρυσός. Ἰδοὺ καὶ τὴν ὕλην ἀπέχεις δαμάζων τὸ μονόειδον\footnote{Ἰδοὺ καὶ] εἰ δὲ καὶ AKLc. --- δάμαζε Lc. --- τὸ μον. ὡς πολ.] τὸ μον. τὸ ἐκ πολλῶν εἰδῶν AK ; τὸ μον. ὡς πολ., ἤγουν τὸ ἐκ πολλῶν εἰδῶν κατασκευαζόμενον Lc, qui poursuit avec la πρᾶξις βʹ.} ὡς πολύειδον.

\bigskip
\centerline{\EightStarTaper}
\centerline{\EightStarTaper\EightStarTaper}
\bigskip

\subsubsection[3. --- 2. ΖΩΣΙΜΟΣ ΛΕΓΕΙ ΠΕΡΙ ΤΗΣ ΑΣΒΕΣΤΟΥ.]{3. --- 2. ΖΩΣΙΜΟΣ ΛΕΓΕΙ ΠΕΡΙ ΤΗΣ ΑΣΒΕΣΤΟΥ.\footnote{Titre dans A : Ὁ Ζώσ. ἔφη περὶ τῆς ἀσβ.}}
\paragraph{}
\emph{Transcrit} (§ 1 et 2) \emph{sur} M, f. 95 r., et (§ 3) \emph{sur} A, f. 8 v. --- \emph{Collationné la copie de} M \emph{sur} A, f. 8 r.

\bigskip

1. Δῆλα ὑμῖν ποιοῦμαι · γινώσκεται γὰρ ὅτι ὁ λίθος ὁ ἀλαβαστρίτης\footnote{γινώσκεται] F. l. γινώσκετε.} ἐγκέφαλος κέκληται διὰ τὸ κάτοχον αὐτὸν εἶναι πάσης βαφῆς φευκτῆς. Λαβὼν οὖν τὸν ἀλαβάστρινον λίθον, ὄπτα νυχθήμερον, καὶ ἔχε ἄσβεστον, καὶ λάβε ὄξος δριμύτατον καὶ κατάσβεσον · καὶ θαυμάσεις · θείαν γὰρ ποίησιν τὴν ἐπιφάνειαν λευκοτάτην ποιεῖ. Καὶ ἔα καταστῆναι, καὶ ἐπίβαλλε αὐτῷ ὄξους δριμυτάτου οὐκ ἐμφίμῳ ἀλλ ᾽ ἀπώμῳ,\footnote{αὐτῷ] αὐστιῶ (ρ au-dessus de ι) A.} ἵνα τὴν ἐπιτρέχουσαν αἰθάλην καθ ᾽ ἑκάστην ἐπαίρῃς · ἔτι λαβὼν ὄξος δριμὺ δι ᾽ ἑπτὰ ἡμερῶν τὴν αἰθάλην ἐπαίρῃς, οὕτως ποίει\footnote{οὕτως] τοῦτο A, f. mel.} ἄχρις ἂν ἡ αἰθάλη μὴ ἀναπέμπηται. Καὶ ἔασον ἡμέρας τεσσαράκοντα\footnote{αἱ αἰθάλαι μὴ ἀναπέμπονται (\emph{sic}) A.} ἐν ἡλίῳ καὶ δρόσῳ τῇ ἐμπροθέσμῳ, γλύκανον ὕδατι ὑετίῳ.\footnote{γλύκασον M.} Καὶ ξηράνας ἐν ἡλίῳ ἔχε τὸ μυστή- (f. 95 v.) ριον ἀμετάδοτον, ὃ οὐδεὶς τῶν προφητῶν ἐτόλμησεν μυσταγωγῆσαι τῷ λόγῳ, ἀλλὰ μόνον τοῖς νοήμοσιν αὐτῶν ἐμυσταγώγουν. Τοῦτο γὰρ τὸ κεφάλαιον\footnote{νοήμοσιν] νεύμασιν mss. Corr. conj. Même variante et même correction, ci-après ligne 13. --- αὐτῶν] αὐστιῶ (ρ au-dessus de ι) A. F. l. αὐτοὶ. --- Tοῦτο --- οὐ λίθον] Réd. de A : Τοῦτον δὲ ἐκάλ. λίθον οὐ λίθον.} ἐκάλεσαν ἐν ταῖς λοξαῖς γραφαῖς λίθον τὸν οὐ λίθον, τὸν ἄγνωστον καὶ πᾶσι γνωστὸν, τὸν ἄτιμον καὶ πολύτιμον, τὸν ἀδώρητον καὶ θεοδώρητον. Κἀγὼ δὲ αὐτὸν ἐγκωμιάσω τὸν ἀδώρητον καὶ θεοδώρητον, τὸν μόνον ἐν ταῖς ἡμῶν ἐργασίαις κρείττω τοῦ ὑλαίου.\footnote{τὸν μόνον --- μυστήριον] Réd. de A : τὸν μόνον ἐν ταῖς ἡμετέραις ἐργ. κρύπτον, τοῦτο γάρ ἐστι τὸ μιθρ. μυστ. --- Après ces mots, A se sépare de M jusqu'à la fin de notre § 2 et continue ainsi : Στέφανος δέ φησιν ·Λάβε ἐκ τῶν τεσσάρων στοιχείων ἀρσενικοῦ κ. τ. λ. jusqu'à μὴ ἀποκαλύψαι καὶ δημοσιεῦσαι (voir ci-après 4, 20).} Τοῦτο γάρ ἐστι τὸ φάρμακον τὸ τὴν δύναμιν ἔχον, τὸ μιθριακὸν μυστήριον.

2. Τὸ γὰρ πνεῦμα τοῦ πυρὸς ἑνοῦται τῷ λίθῳ, καὶ γίνεται πνεῦμα μονογενές. Τὰς δὲ ἐργασίας τοῦ λίθου ἑρμηνεύσω ὑμῖν. Κώμαρι συμμεμιγμένῳ μαργάρους ἀποτελεῖ · ἐπεί τοι γε αὐτὸν χρυσόλιθον ἐκάλεσαν · πάντα δὲ πνεῦμα σεύει τῇ δυνάμει τοῦ ξηρίου. Κἀγὼ κώμαριν\footnote{σεύει] σέβη M. Corr. conj.} μέλλω ἑρμηνεύειν ὑμῖν, ὃ οὐδεὶς ἐτόλμησεν μυσταγωγῆσαι · ἀλλὰ καὶ αὐτοὶ τοῖς νοήμοσι παρέδωκαν. Ἀπέχεται τὴν θηλυκὴν δύναμιν προτιμωτέραν αὐτήν. Αὕτη γὰρ καὶ μόνη ἡ λεύκωσις σεβασμία γέγονε παντὸς προφήτου. Ἑρμηνεύσω ὑμῖν καὶ τοῦ μαργάρου τὴν δύναμιν.\footnote{M mg : ὡς ἡμάρτηκε. (Main du 15\textsuperscript{e} siècle, peut-être celle de Bessarion.)} Ἐργασίαν ἔχει τῷ ἐλαίῳ ἑψόμενον ὅ ἐστιν θηλυκὴ δύναμις. Λαβὼν μαργάρου τὸ ἀσιτικῶ ἕψῃ ἐλαίῳ οὐκ ὑποφίμῳ ἀλλ ᾽ ἀπώμῳ ἐπὶ ὥρας τρεῖς,\footnote{ασιτικῶ (sans esprit) M. F. l. τοῦ ἀσιατικοῦ.} μέσοις φωσίν · καὶ λαβὼν ῥάκος ἐρίου, ἔκθλιβε ἐν τῷ μαργάρῳ, ἵνα ἀποβάλῃ τὸ ἔλαιον, καὶ ἔχε εἰς τὰς χρείας τῶν καταβαφῶν · ἡ γὰρ τελείωσις τοῦ ὑλαίου διὰ τοῦ μαργάρου ἐστίν.

3. Ἄρσις δὲ ἑρμηνεύεται ὁ κουφισμός · ἀνθ ᾽ ὧν αἴρεται καὶ κουφίζεται\footnote{ἀνθῶν ms. Corr. conj.} ἡ τοῦ ὕδατος ἐπίχυσις, ἐκ τῆς τοῦ σώματος συμπλοκῆς ἀνεμποδίστως τὸ μολύβδου πήσηται ὑπόμονος τούτῳ ποιῆσαι.\footnote{ἐμποδίστως τὸ signe du soufre et πέσηται biffés dans le ms. ; ὑπόμονος τούτο ποιῆσαι seulement à sa marge, après plusieurs mots biffés.} Ἀρκεστῶμεν τῇ θυείᾳ καὶ τῷ δοίδυκι ἐπὶ τῶν δύο βαφῶν · ἐπὶ δὲ τοῦ χαλκοῦ, ἐπεὶ περὶ τούτου Ζώσιμος καὶ ὑπὸ πλήθους ὑδάτων σηπόμενον διὰ τῆς τοῦ ἀέρος ὑγρότητος τε καὶ θερμότητος αὐξανόμενον ἄνθη φορεῖ κατὰ πολὺ γλυκύτητα, καὶ τῇ ποιότητι τῆς φύσεως καρποφορεῖ\footnote{φορᾶ A. --- F. l. κατὰ πολλὴν γλυκύτητα.} :

\bigskip
\centerline{\EightStarTaper}
\centerline{\EightStarTaper\EightStarTaper}
\bigskip

\subsubsection{3. --- 3. ΑΓΑΘΟΔΑΙΜΟΝΟΣ.}
\paragraph{}
\emph{Transcrit sur} M, f. 95 v., \emph{ainsi que l'article suiνant}.

\bigskip

Μετὰ τὴν τοῦ χαλκοῦ ἐξίωσιν καὶ μέλανσιν καὶ ἐς ὕστερον\footnote{Cette phrase est dans Stephanus, praxis 2, p. 204, éd. Ideler.} λεύκωσιν, τότε ἔσται βεβαία ξάνθωσις.

\bigskip
\centerline{\EightStarTaper}
\centerline{\EightStarTaper\EightStarTaper}
\bigskip

\subsubsection{3. --- 4. ΕΡΜΟΥ.}
\paragraph{}
Ἐὰν μὴ τὰ σώματα ἀσωματώσῃς καὶ τὰ ἀσώματα σωματώσῃς,\footnote{Cp. Olympiodore, § 40 ; ci-dessus, p. 93, l. 14.} οὐδὲν τὸ προσδοκώμενον ἔσται.

\bigskip
\centerline{\EightStarTaper}
\centerline{\EightStarTaper\EightStarTaper}
\bigskip

\subsubsection[3. --- 5. ΖΩΣΙΜΟΥ ΠΡΑΞΙΣ Β.]{3. --- 5. ΖΩΣΙΜΟΥ ΠΡΑΞΙΣ Β.\footnote{Titre dans Lc : Τοῦ αὐτοῦ Ζωσίμου πρᾶξις δευτέρα.}}
\paragraph{}
\emph{Transcrit sur} A, f. 87 v. --- \emph{Collationné sur} K, f. 2 v. --- \emph{sur} Lc, p. 289.

\bigskip

1. Μόλις ποτὲ εἰς ἐπιθυμίαν ἐλθὼν τοῦ ἀναβῆναι τὰς ἑπτὰ κλίμακας καὶ θεάσασθαι τὰς ἑπτὰ κολάσεις, καὶ δὴ ὡς ἔχει ἐν μιᾷ τῶν ἡμερῶν, ἤνυσα τὴν ὁδὸν τοῦ ἀναβῆναι. Διελθὼν δὲ πολλάκις ἀνῆλθον\footnote{διελθὼν δὲ π. ἀν.] καὶ διελθὼν π. ἀνοδίᾳ. ἀνῆλθον Lc.} ἔπειτα εἰς τὴν ὁδόν. Καὶ δὴ ἐν τῷ ἐπανέρχεσθαί με ἀπέτυχον πάσης ὁδοῦ,\footnote{καὶ δὴ ἐν τῷ ἐπ.] καὶ δι ἐν A ; κ. διεν K ; ἐν δὲ τῷ ἐπ. Lc. Corr. conj.} καὶ ἐν ἀθυμίᾳ πολλῇ γενόμενον, μὴ ἰδόντος μου πόθεν ἀπελθεῖν,\footnote{γενόμενον] γέγονα Lc. --- μὴ ἰδόντος μου --- ἠμφιεσμένον (p. suiv., l. 2)] Réd. de Lc : μὴ εἰδὼς ποῦ ἀπελθεῖν δυνηθῶ, ἐν τούτοις δὲ ὢν, καὶ σφόδρα ἀθυμῶν ἐτράπην εἰς ὕπνον, καὶ ὅρω κατ ᾽ ὄναρ τι ἀνθρ. ξυρ. ἠμφ.} ἐτράπην εἰς ὕπνον. Καὶ θεωρῶ κατὰ τὸν ὕπνον μου ξυρουργόν τινα ἀνθρωπάριον ἠμφιεσμένον στολὴν ἐρυθρὰν, καὶ βασιλικὴν ἐσθῆτα, κα ἱστάμενον ἔξω τῶν κολάσεων, καὶ λέγει μοι · « Τί ποιεῖς, ἄνθρωπε ; » Ἐγὼ δὲ πρὸς αὐτὸν ἔφην · « Ἵσταμαι ὧδε ὅτι πάσης ὁδοῦ ἀστοχήσας\footnote{πρὸς αὐτὸν] αὐτῷ Lc.} ὑπάρχω πλανώμενος. » Ὁ δὲ λέγει μοι · « Ἀκολούθει μοι. » Ἐγὼ δὲ\footnote{ὁ δὲ] ἐκεῖνο δὲ Lc. --- Ἐγὼ δὲ ἐξελθὼν] ὁ δὲ ἐξῆλθον AK. Réd. de Lc : ἐγὼ δὲ ἐξῆλθον καὶ ἠκολούθουν αὐτῷ.} ἐξελθὼν ἠκολούθουν αὐτῷ · πλησίον δὲ γενομένων τῶν κολάσεων,\footnote{γενόμενος Lc, f. mel. --- θεωρῶ --- ἀνθρωπάριον] Réd. de Lc : Ὅρω τὸ ὁδηγοῦν με ἐκενο τὸ ξ. ἀνθρ.} θεωρῶ τὸν ὁδηγοῦντα με, ἐκεῖνον ξυρουργὸν ἀνθρωπάριον · καὶ ἰδοὺ ἐνεβλήθη ἐν τῇ κολάσει, καὶ ὅλον αὐτοῦ τὸ σῶμα ἐδαπανήθη ὑπὸ τοῦ πυρός.\footnote{ἐν τῇ κολάσει] εἰς τὴν κόλασιν Lc. --- ὑπὸ τοῦ πυρὸς ἐδαπ. Lc. --- ἐδαπανίσθην A ; ἐδαπανήθην K.}

2. Ἰδὼν ἐγὼ ἐξέστην καὶ ἐτρόμαξα ἀπὸ τοῦ φόβου, καὶ διυπνίσθην,\footnote{Ἰδὼν] Τοῦτο ἰδὼν Lc, f. mel. --- ἐτρόμαξα] F. l. ἐτρόμησα.} καὶ λέγω ἐν ἑαυτῷ · « Ἆρα τί ἐστι τὸ ὁρώμενον ; » καὶ πάλιν διεσάφησα\footnote{ἐμαυτῷ Lc.} τὸν λόγον, καὶ διακρίνων ὅτι ὁ ξυρουργὸς ἐκεῖνος ἄνθρωπος\footnote{διακρίνων] εὗρον Lc. --- ξυρουργοῦντος AK. --- Lc mg. : barre verticale se rapportant aux lignes 12 et 13. --- ὁ ξυρ. --- ἄνθρωπος] τὸ ξυρουργὸν ἐκεῖνο ἀνθρωπάριον Lc.} ὁ χαλκάνθρωπός ἐστιν, [ἔχων] ἐσθῆτα ἐρυθράν ἐνδεδυμένος, καὶ εἶπον\footnote{Après ἐστιν] ὁ ἐσθ. ἐρ. ἐνδ. Lc. --- καὶ εἶπον ἐν ἐμαυτῷ Lc.} · « Καλῶς ἐπενόησα, οὗτος ἐστιν ὁ χαλκάνθρωπος · δεῖ δὲ πρῶτον ἐμβάλεῖν\footnote{δεῖ δὲ] ἀλλὰ δεῖ Lc.} αὐτὸν εἰς τὰς κολάσεις. » Πάλιν ἐπεθύμησεν ἡ ψυχή μου τοῦ\footnote{καὶ πάλιν Lc.} ἀναβῆναι καὶ τὴν τρίτην κλίμακα. Καὶ πάλιν μόνος τὴν ὁδὸν ἐπορευόμην, καὶ ὡς ἐγενόμην τῶν κολάσεων πλησίον, πάλιν ἐπλα- (f. 88 r.)\footnote{πλησίον τῶν κολ. Lc.} νήθην, μὴ εἰδὼς τὴν ὁδὸν, ἱστάμενος, ἀπονενοημένος.\footnote{Après ὁδὸν] καὶ πάλιν ἐστάθην ἀπονενοημένος. Lc.}

3. Καὶ πάλιν τῷ ὁμοίῳ τρόπῳ θεωρῶ πεπολιωμένον γηραιὸν λευκὸν πάνυ, ὥστε ἐκ τῆς πολλῆς λευκότητος αὐτοῦ οἱ ὀφθαλμοὶ\footnote{ὀφθαλμοὶ en signe AK.} ἀπεμαυρώθησαν. Τὸ δὲ ὄνομα αὐτοῦ ἐκαλεῖτο Ἀγαθοδαίμων. Καὶ στραφεὶς ὁ πεπολιωμένος ἐκεῖνος θεωρεῖ με ἐπὶ πλείστην ὥραν. Ἐγὼ δὲ τοῦτον ἐπεμελούμην · « Δεῖξόν μοι εὐθεῖαν ὁδόν. » Ὁ δὲ πρὸς\footnote{ἐπεμελούμην] ἐπιμ. AK ; παρεκάλουν Lc, mel. --- δεῖξαι Lc.} μὲ οὐκ ἀνεστράφη, ἀλλ ᾽ ἤνυσεν τὴν ὁδὸν αὐτοῦ σπουδαίως · καὶ\footnote{ἤνυσεν] ἤνεισεν A ; ἥνεισεν K --- καὶ] ἐγὼ δὲ Lc. --- αὑτοῦ] αὐτοῦ mss. ici et dans tout le morceau.} διερχόμενος δὲ ἔνθεν κἀκεῖθεν ἤνυον σπουδαίως τὸν βωμόν. Ὡς οὖν\footnote{ἤνυον] ἠνείουν AK.} ἤνυσα ἄνω ἐπὶ τοῦ βωμοῦ, θεωρῶ τὸν πεπολιωμένον γηραιὸν,\footnote{ἤνυσα] ἤν. καὶ ὐπῆρχον Lc ; ἤνεισα A ;--- ἐκεῖνον γηραιὸν Lc. --- καὶ] F. l. ὡς (ou ὃς ? ).} καὶ ἐνεβλήθη ἐν τῇ κολάσει. Ὦ οὐρανίων φύσεων δημιουργοὶ,\footnote{ὦ φύσεις οὐρ. Lc. --- εὐθὺς γὰρ Lc.} εὐθὺς ὅλος ὑπὸ τῆς φλογὸς πυρίφλεκτος γέγονεν · ὃν καὶ τὸ διήγημα,\footnote{ὃν] οὗ Lc, f. mel.} ἀδελφοὶ, φρικτόν · ἐκ γὰρ τῆς πολλῆς βίας τῆς κολάσεως οἱ ὀφθαλμοὶ αὐτοῦ πλήρεις αἵματος γεγόνασιν. Ἐπηρώτησα δὲ λέγων\footnote{αἱμάτων mss. Corr. conj. --- ἐπηρώτησα δὲ λ. αὐτὸν] εἶτα ἐπηρ. αύτὸν, λέγων Lc, f. mel.} αὐτὸν · « Τί ἐνταῦθα κατάκεισαι ; » Ὁ δὲ μόλις ἀνοίξας τὸ στόμα αὑτοῦ ἔφη μοι · « Ἐγώ εἰμι ὁ μολυβδάνθρωπος καὶ βίαν ὑπομένω ἀφόρητον. » Καὶ οὕτως ἐκ τοῦ πολλοῦ φόβου διυπνίσθην, καὶ ἐν ἐμοὶ τὴν αἰτίαν\footnote{φόβου] ὕπνου sous-pointillé. puis φόβου Lc. --- ἐν ἐμοὶ] ἐν ἐμαυτῷ Lc.} ἠρεύνων τοῦ πράγματος. Καὶ πάλιν διέκρινα καθ ᾽ ἑαυτὸν καὶ εἶπον\footnote{κατ ᾽ ἐμαυτὸν Lc.} · « Καλῶς ἐπενόησα ὅτι οὕτως δὴ ἐκβαλεῖν τὸν μόλυβδον, καὶ ἀληθῶς\footnote{δὴ] δεῖ Lc, f mel.} τὸ ὅραμά ἐστιν περὶ τῆς συνθέσεως τῶν ὑγρῶν.

\subsubsection[3. --- 5 bis. ΠΟΙΗΜΑ ΤΟΥ ΑΥΤΟΥ ΖΩΣΙΜΟΥ. ΠΡΑΞΙΣ Γ.]{3. --- 5 bis. ΠΟΙΗΜΑ ΤΟΥ ΑΥΤΟΥ ΖΩΣΙΜΟΥ. ΠΡΑΞΙΣ Γ.\footnote{Titre dans Lc : τοῦ αὐτοῦ Ζωσ. πρᾶξις τρίτη.}}
\paragraph{}
Καὶ πάλιν κατενόησα τὸν θεῖον καὶ ἱερὸν φιαλοβωμὸν,\footnote{Καὶ πάλιν κατεν.] πάλιν δὲ κατανοήσας Lc. F. l. κατήνυσα, je gagnai.} καὶ εἶδόν τινα ἱεροπρεπῆ λευκοποδήρην ἐνδε- (f. 88 v.) δυμένον ἱερουργοῦντα τὰ φοβερὰ ἐκεῖνα μυστήρια, καὶ εἶπον · « Ἆρα τίς ἐστιν οὗτος\footnote{τὰ φοβερὰ] τὰ ἱερὰ K et mg. : φοβερὰ ; τὰ ἱερὰ καὶ φοβ. Lc.} ; » καὶ ἀποκριθεὶς εἶπέ μοι · « Οὗτός ἐστιν ὁ ἱερεὺς τῶν ἀδύτων. Οὗτος βούλεται αἱματῶσαι τὰ σώματα, καὶ ὀμματῶσαι τὰ ὄμματα, καὶ τὰ νενεκρωμένα ἀναστῆσαι. Καὶ οὕτω πάλιν πεσὰν ἐκοιμήθην ἄλλον\footnote{ἄλλο Lc.} ὀλίγον, καὶ αὐτὸ δὴ ἐν τῷ ἐπανέρχεσθαί με ἐπὶ τὴν τετάρτην κλιμακα,\footnote{καὶ αὐτὸ δὴ] καὶ ἐν τῷ ἐπ. Lc. F. l. καὶ οὕτω vel καὶ αὐτὸς.} εἶδον κατ ᾽ ἀνατολὰς ἐρχόμενον, κατέχοντα ἐν τῇ χειρὶ αὑτοῦ\footnote{κατ ᾽ ἀνατολὰς] ἐξ ἀνατολῶν Lc. --- ἐρχόμενον ἄνθρωπον Lc.} μάχαιραν. Καὶ ἄλλος ὀπίσω αὐτοῦ φέρων περιηκονισμένον τινὰ λευκοφόρον\footnote{ἄλλος ὀπίσω] ἄλλον ὄπισθεν Lc, f. mel. --- φέρων] φέροντα Lc.} καὶ ὡραῖον τὴν ὄψιν, οὗ τὸ ὄνομα [αὐτοῦ] ἐκαλεῖτο μεσουράνισμα\footnote{αὐτοῦ om. Lc, mel.} ἡλίου, καὶ ὡς πλησίον ἦλθον τῶν κολάσεων, λέγων ὅτι\footnote{ἡλίου] signe commun au soleil et au cinabre AK ; κινναβάρεως (en toutes lettres) Lc. --- λέγων ὅτι ... ] λέγει μοι ὁ τὴν μαχ. κρ. Lc. mel.} μάχαιραν κρατῶν, « Περιέτεμε αὐτοῦ τὴν κεφαλὴν, καὶ τὰ κρέατα αὐτοῦ θήσων ἀνὰ μέρος, καὶ τὰς σάρκας αὐτοῦ ἀνὰ μέρος, ὅπως αἱ\footnote{θήσων] θὲς Lc. F. l. θύσων. --- ὅπως] ὅπου mss. Corr. conj.} σάρκες αὐτοῦ πρῶτον ἑψηθῶσιν ὀργανικῶς, καὶ τότε τῇ κολάσει παραπορευθῶσιν.\footnote{ἑψηθήτωσαν Lc. --- παραπορευθήτωσαν Lc.} » Καὶ οὕτως πάλιν ἔξυπνος γενόμενος εἶπον · « Καλῶς ἐπενόησα καὶ ὅτι περὶ ταῦτά ἐστιν τὰ ὑγρὰ τῆς μεταλλικῆς. » Καὶ πάλιν\footnote{ὅτι] ὁ AK (om. Lc). Corr. conj. --- τῆς μετ. τέχνης Lc.} ὁ βαστάζων τὴν μάχαιραν ἐφη · « Πεπληρώκατε τὴν κάτω ἑπτὰ κλίμακας.\footnote{τὴν κατὼ ἑ. κλ.] F. l. τὴν < τέχνην > κατὰ ἑ. κλ. --- ἑπτὰ κλίμακα A ; ἑπτὰ κλίματα K ; ἑπτακλήματα Lc. Corr. conj.} Ὁ δὲ ἕτερος ἔφη ἅμα τῷ ἐκβαλεῖν τοὺς κρουνοὺς δι ᾽ ὑγρῶν\footnote{κρούνους] χρόνους A.} πάντων · « Ἡ τέχνη πεπλήρωται. »

\subsubsection[3. --- 6. ΖΩΣΙΜΟΥ ΤΟΥ ΘΕΙΟΥ ΠΕΡΙ ΑΡΕΤΗΣ ΚΑΙ ΕΡΜΗΝΕΙΑΣ.]{3. --- 6. ΖΩΣΙΜΟΥ ΤΟΥ ΘΕΙΟΥ ΠΕΡΙ ΑΡΕΤΗΣ ΚΑΙ ΕΡΜΗΝΕΙΑΣ.\footnote{Titre dans E Lc : Ἀνεπιγράφου φιλοσόφου εἰς τὸ περὶ ἀρ. καὶ ἑρμ. τοῦ θείου Ζωσς. τοῦ Πανοπολίτου (ἤ Θηβαίου add. F).}}
\paragraph{}
\emph{Transcrit sur} A, f. 168 v. --- \emph{Collationné sur} K, f. 47 v. ;--- \emph{sur une copie de Laur.}, f. 253 r. (\emph{seulement depuis la ligne} 3 \emph{du} § 4 \emph{jusqu'à la ligne} 3 \emph{du} § 17) ;--- \emph{sur} E, \emph{première feuille de garde} ;--- \emph{sur} Lc, (copie de E ? ) p. 301.

\bigskip

1. Προσπαθείας < ἕνεκα > καὶ μεθερμηνείας τοῦ ἐνυπνιάζεσθαι\footnote{Ajouté ἕνεκα d'après une conjecture confirmée par E Lc. --- Rédaction de E Lc : Ὁ θεῖος Ζώσιμός φησιν ὅτι, ἕνεκα προσπαθείας καὶ μεθερμηνείας, τοῦτον τὸν τρόπον ἐνυπνιάσθη. Ἐδόκουν γὰρ, φησὶ, καὶ ἰδοὺ βωμὸς φιαλοειδὴς ὑπῆρχε, καὶ πνεῦμα πύρινον ἵστατο ἐπάνω τοῦ βωμοῦ, καὶ διηκόνει τοῖς τοῦ πυρὸς βρασμοῖς καὶ καχλασμοῖς (κ. καχλ. om. E), καὶ καύσεσι τῶν ἀνθρώπων ἀνερχομένων. Καὶ ἠρώτησα τούτων τινα τίς ἂν εἴη οὗτος ὁ βρασμὸς καὶ ὁ καχλασμὸς, καὶ (page 303) πῶς κ. τ. λ.} αὐτόν φησιν. Καὶ ἰδοὺ βωμὸς φιαλοειδὴς καὶ πνεῦμα πύρινον ἑστὼς ἐπὶ τοῦ βωμοῦ, καὶ διηκόνουν τοὺς τοῦ πυρὸς βρασμοὺς καὶ καχλασμοὺς [καὶ] καυσώδεις τῶν ἀνθρώπων ἀνερχομένων, καὶ ἠρώτησα, φησὶν, καὶ εἶπον ἐπὶ τὸν ἑστῶτα λαόν. Θαυμάζομαι γὰρ τὸν τοῦ ὕδατος βρασμὸν καὶ καχλασμὸν, καὶ πῶς οἱ ἄνθρωποι καιόμενοι ζῶσι. Καὶ ἀποκριθεὶς λέγει μοι · « Οὗτος ὃν ὁρᾷς βρασμὸς τόπος ἐστὶν\footnote{ὁ βρ. οὗτός ἐστι τόπος τῆς ἀσκ. ELc. (Cp. 3, 1, 3. p. 109. l. 11).} ἀσκήσεως τῆς λεγομένης ταριχείας · οἱ γὰρ βουλόμενοι ἄνθρωποι ἀρετῆς\footnote{Οἱ γὰρ ἄνθρ. βουλ. ELc.} τυχεῖν ῶδε εἰσέρχονται καὶ ἀποβάλλονται [διὰ τὸ εἶναι] σώματα\footnote{ἀποβάλλουσι ELc. --- διὰ τὸ εἶναι om. ELc, mel. --- τὰ σώματα, καὶ γὶν. πν. ELc.} πνεύματα γίνονται. Καὶ γὰρ πάλιν ἄσκησις ἔνθεν ἑρμηνεύεται ἐκ τοῦ\footnote{Καὶ γὰρ --- γίνονται (l. 10)]. Réd. de ELc : Διὸ καὶ οὗτος ὁ τόπος ἀσκ. ἑρμην. ὅτι τὰ σώμ. ἀποβάλλουσι τὴν παχ. ἑαυτῶν καὶ γίν. πν.} ἀσκῆσαι · οἷον γὰρ ἀποβαλλόμενα τὴν παχύτητα τοῦ σώματος πνεύματα γίνονται. »

2. Καί τι τοιοῦτον Δημόκριτός φησιν · « Οἰκονόμει ἕως γένηται\footnote{καί τι] καί τοι (ι au-dessus de τοι) K. --- Corr. de 1\textsuperscript{re} main. --- Réd. de ELc : Διὰ τοῦτο φ. ὁ Δημ.} ἰὸς ξανθὸς ὡς στίγμα χρυσοῦν διὰ τοῦ ἰοῦ τὸ πνεῦμα συμβαῖνον.\footnote{ὁ ἰὸς ELc. --- ὡς στίγμα] F. l. ὡς τῆγμα (ici et l. 14). --- ὅτι διὰ τοῦ ἰοῦ τὸ πν. συμβαίνει ELc.} » Καὶ γὰρ ὁ ἰὸς διὰ τοῦ ἀσωμάτου κατὰ τὸν ὄφιν ἑρμηνεύεται πνεῦμα,\footnote{Lc, mg., p. 303 du ms. : renvoi à la fig. de la p. 221. (Ci-après, 3, 11. Cp. Introduction de M. Berthelot, p. 132, fig. 11, n° 1.) Réciproquement, p. 221 du ms. : renvoi à la p. 303.} καὶ διὰ τὸ τέλειον τοῦ χρώματος ξανθὸν ὠς στίγμα χρυσοῦν προσαγορεύεται.\footnote{χρώμ. προσαγ. ξ. ὡς στ. χρυσοῦ ELc.} Καὶ οὕτω διὰ φωνῆς πρὸς φωνὴν συνάπτοντες τὴν ἔννοιαν, ὑπερφαίνουσιν ταύτην, ὅθεν καὶ δι ᾽ ὁμοειδοῦς πάλιν ἥξεώς φησιν\footnote{ἥξεως] ἡξέσω K ; om. ELc. F. l. ἕξεως.} · « Ὁἰκονόμει δὲ ἕως οὗ ῥεῦσαι δυνηθῇ, ῥεύσεις δὲ διὰ ῥύτεως, ἀντὶ τοῦ\footnote{ῥεύσεις δὲ --- ἐτήσιος λίθος (p. suiv., l. 3) om. ELc.} εἰπεῖν διὰ ῥεύσεως · τρέπουσι γὰρ τὸ Σ στοιχεῖον εἰς Τ · χρησάμενος\footnote{διαρεύσεως AK « Il y a ici un jeu de mots opposant ῥυτός, ῥύτις, ῥύτεως, à ῥεῦσις, ῥεύσεως. Voir le morceau 3, 7, 5. » (\emph{M. B.})} (f. 169 r.) τῇ λέξει, φησὶν ῥεύσης, ῥεύσης δὲ διὰ ῥεύσεως, ὃ ἑρμηνεύεται διὰ ῥεύσεως, ὡς εἴπομεν. Τούτῳ δὲ ὃ λέγει · « Οἰκονόμει δὲ\footnote{τούτῳ] F. l. τοῦτο.} ἕως < οὗ > ῥεῦσαι δυνηθῇ. » Ὅμως οἷόν ἐστιν τὸ ὁμορρευστῆσαι προκείμενον.\footnote{ἕως < οὗ > ] ὡς AK. --- ὅμως] F. l. ὁμοίως.}

3. Καὶ νῦν δὲ πάλιν διὰ τοῦ λέγειν σιδηρίτην, ὃν καὶ σιδηρίτην καλοῦσιν οἱ κάτω ἐνσημαινόμενοι · διαγινώσκεται, ἀναφερόμενον ὡς ἔλεγεν · χαλκὸς μόλυβδος ἐτήσιος λίθος. Ὁ γὰρ πυρίτης διὰ περιουσίαν\footnote{Ὁ γὰρ. πυρ.] πυρίτης δὲ λέγεται ELc.} χρώματος, ἤτοι τὸ περισσὸν ἐκκαιόμενον, ἤτοι πυρούμενον,\footnote{ἤτοι --- τὴν ἐξυδραργύρωσιν] Réd. de ELc : ἤγουν διὰ τὸ περισσῶς, ἐκκαίεσθαι καὶ πυροῦσθαι τὸν χαλκὸν (« Nota bene hic » ajouté par E.) Ὁμοίως δὲ καὶ ὁ ὀργ. λέγεται διὰ τὴν ἐξυδραργύρωσιν. --- τὸ] F. l. τὸν.} τὸν χαλκὸν ὑπαινίττεται · καὶ ὁμοίως τὸ ἀργυρίτης τὴν ἐξυδραργύρωσιν · ἐξυδραργυρούμενος γὰρ ὁ χαλκὸς ἀργυρίτης γίνεται,\footnote{κατ ᾽ ἐναντίαν --- λέγων (l. 9) om. ELc.} κατ ᾽ ἐναντίαν τοῦ ἐτησίου, ἥτις ἐστὶν ὑδράργυρος, κατ ᾽ ἐτυμολογίαν τοῦ ὅλου, ἥτις ποιεῖ τὴν μέλλουσαν ἀναφαίνεσθαι χρύσοπτα προσυπακούειν, λέγων « σιδηρίτης » διὰ τὴν ἐκ μολύβδου σύγκρασιν.\footnote{λέγων] F. l. λέγει. --- σιδερίτης jusqu'à ποιοῦσιν] Réd. de ELc : σιδερίτης δὲ λέγεται διὰ τὴν τοῦ σιδρου καὶ μολ. μέλανσιν · τοιοῦτος γὰρ γίνεται.} Συγκρινόμεναι γὰρ αἱ οὐσίαι σιδηρίτην ποιοῦσιν.

4. Ὁμοίως τί τοῦ σιδήρου καρδίαν ; ὅτε δὲ μάλιστα μάζα κλασθῇ\footnote{ELc omettent tout notre § 4. --- ὅτε δὲ] F. l. ὅτε δὴ. --- κλαστῆ AΚ.} ὡς ἐκ τῆς ῥεύσεως ταύτης, ῥῆσιν ποιοῦντες πρὸς τὰς ἀναλογίας [ῥήσεις], εὑρίσκομεν σαφῆ τὴν θεωρίαν, ὡς κατὰ τὸ κρυπτὸν τοῦτο ὑπεμφαίνει.\footnote{Avec le mot εὑρίσκομεν commence la copie du ms. Laur. (fol. 253, r°), rapportée de Florence par M. André Berthelot, ms. dont nous donnons ici les principales variantes. --- σαφῆ] leçon de Laur. ; σαφὴν AK.} Καὶ ἐν ἄλλοις ὁ Δημόκριτος λέγει · « Οἰκονόμει δὲ ἅλμῃ, ἢ ὀξάλμῃ, ἢ οὔρῳ ἅλμης, ἢ ἐπ ᾽ ἄμφω · τὸν σύλλογον ἐπάγω,\footnote{οὔρῳ ἅλμης] F. l. οὔρου ἅλμῃ.} φάσκει, ἢ ὡς ἐπινοεῖς ἐν τῇ γραφῇ, ἢ ὡς ἐπινοεῖται ἡ γραφὴ δυνάμενα\footnote{φάσκειν Laur.} καὶ διασκευαζόμενα ἐξ ἑτέρων ὑγρῶν, ἐπείπερ οὐδὲν τούτων διαμένει, ἄλλ ᾽ ἀπόχυται πλύνον τὴν σύνθεσιν (f. 169 v.) κατ ᾽ αὐτοῦ.\footnote{πλύνον] πλύνοντας A ; πλυνομένουσα (\emph{sic}) Laur.} »

5. Ἕνεκεν ἐκείνων ὁ ἀρχαιότατος Ὀστάνης ὡς ἐν τοῖς ἑαυτοῦ\footnote{ἕνεκεν --- ἀετὸς χαλκοῦς (p. suiv., l. 2)] Réd. de ELc : Ὁ δἐ ἀρχ. Ὀστ. ἐν τοῖς αὐτοῦ συγγράμμασιν εἴρηκεν ὅτι ὑπῆρχεν ἐν Περσίᾳ τις μέγας φιλόσοφος καλύμενος Σοφὰρ, ὅστις ἔγραψεν ὅτι ἔστι τις ἀετὸς χ. --- Fin de la collation de E, manuscrit de tout point semblable à Lc. --- ὡς om. Laur.} καταπαραδείγμασιν · Ἕτερος περί τινος Σωφὰρ, κατὰ τὴν Περσίδα\footnote{καταπαραδείγμασιν] κατὰπαραδ. AΚ ; παραδείγμασιν Laur. F. l. κάτω παραδ.} προαναφανέντος ἱστορεῖ · λέγει οὗτος ὀ θεῖος Σωφάρ\footnote{λέγει] λέγων Laur., f. mel. --- λέγει οὖτος --- ἀποβάλλει (l. 6)] Passage reproduit dans le morceau 3, 29, 19, avec quelques variantes : Φησὶν ὁ θεῖος Σοφαρ · εἶδον ἀετὸν χαλκὸν (χάλκινον Lb, p. 339) κατερχόμενον ἐν π. κ. καὶ λουόμενον καθ ᾽ ἡμ. καὶ ἐντ. ἀναπεμπόμενον (ἀναβεβώμενος A\textsuperscript{2}. f. 9 r.) τ ᾽ ὑπερ φύσει (ὑπὲρ φύσιν Lb ; lire comme ici ἐπείπερ φησὶν) ὁ γὰρ ἀετὸς ἐτυμ. καθ ᾽ ἡμ. λ. θ. ὡς καὶ δὴ ἕως (δι ᾽ ἑαυτοῦ Lb) καὶ δι ᾽ ἕτ. κ. τ. λ. --- Les variantes de A\textsuperscript{2} (f. 9 r.) sont pour la plupart conformes au texte que nous adoptons.} · « Ἔστι μὲν οὖν ἐν κίονι ἀετὸς χαλκοῦς, κατερχόμενος ἐν πηγῇ καθαρᾷ\footnote{κατερχύμενος --- δι ᾽ ὅλων (l. 7)] Réd. de Lc : Ὃς κατέρχεται εἰς τὸν κιόνα. Puis : Δεῖ οὖν δι ᾽ ὅλων κ. τ. λ. --- ἐν κιόνι --- χαλκοῦς] Réd. de Laur. : ἐν κιονίῳ καὶ φησὶν ὅτι ἰδοὺ ἀετὸν χαλκοῦν.} καὶ λουόμενος καθ ᾽ ἡμέραν, ἐντεῦθεν ἀνανεούμενος, ἐπείπερ φησίν\footnote{ἐντεῦθεν] ἐνταύθου Laur. F. l. ἐνταῦθα --- ἀνανευόμενος AΚ ; ἀνανεβώμενος Laur. (comme A\textsuperscript{2} dans 3, 29, 19. Corr. conj.)} · ὁ ἀετὸς ἐτυμολογούμενος καθ ᾽ ἡμέραν λούεσθαι θέλει. » Ὡς οὖν καὶ δι ᾽ ἑτέρων τὸ αὐτὸ αἰνιττόμενος τὴν καθ ᾽ ἡμέραν ἀπόλουσιν καὶ ἀπόπλυσιν ἀποβάλλει · χρὴ γὰρ ἀκριβῶς ἐπὶ τὸν τῆς παρούσης\footnote{F. l. χρὴ γὰρ ἀκριβῶς εἰπεῖν ἐπὶ τῆς π. ἐ.} ἐργασίας ... · ἀμφιβαλλόμενος οὖν διὰ φιλοσοφίας, δι ᾽ ὅλων τῶν τριάκοσίων\footnote{ἐργασίας ἀφικόμενος οὖν Laur ; ἀφηλώμενος (μα pour βα au-dessus de φη) A ; ἀφιβαλλόμενος K. Corr. conj.} ἑξήκοντα πέντε ἡμερῶν λούειν τὸν χάλκεον ἀετὸν καὶ ἀνανεοῦν, ὡς δεῖ καὶ ἑξῆς δι ᾽ ὅλης αὐτοῦ τῆς πραγματείας.\footnote{ἀνανεοῦν] ἀνανεὸν A ; ἀνανεῶν Κ Laur. --- καὶ ἕξεις αὐτὸν δι ᾽ ἕλ. Lc, mel. --- οὗτος γὰρ ὁ Ὀστ. φησίν] καὶ πάλιν ὁ Ὁστ.  φ. Lc.} Οὗτος γὰρ ὁ Ὀστάνης φησίν · « Ἀπόθλιψον τὴν σταφυλὴν, ὑπογράφει,\footnote{ὑπογράφει om. Lc, f. mel.} ἤγουν ἡ τῆς ῥεύσεως πλύσις ἐστὶ τοῦ μυστηρίου τούτου · τὸν ἰὸν\footnote{ἤγουν --- ἄπελθε] Réd. de Lc : ἤγουν πλύνε τὸν ἰὸν πολλάκις διὰ τῆς ῥεύσεως, καὶ τοῦτό ἐστι τὸ μυστήριον · καὶ πάλιν ὁ αὐτὸς Ὁστ. φησίν · ἄπελθε ...} δεῖ νοεῖν. » Καὶ νῦν ἐμφανέστατα ἐπάγει λέγων · « Ἄπελθε πρὸς τὰ ῥεύματα τοῦ Νείλου καὶ εὑρήσεις ἐνταῦθα λίθον ἔχοντα πνεῦμα. Τοῦτον λαβὼν διχοτόμησον, καὶ βάλλων τὴν χεῖρά σου εἰς τὰ\footnote{βαλὼν Lc.} ἐντὸς αὐτοῦ, [καὶ] ἐξάγαγε τὴν καρδίαν αὐτοῦ · ἡ γὰρ ψυγὴ αὐτοῦ\footnote{καὶ om. Lc. mel. --- Après καρδίαν] αὐτοῦ om. Laur.} ἐν τῇ καρδίᾳ αὐτοῦ ἐστιν. » Διὰ τὸ λέγειν · « Πορεύου εἰς τὰ ῥεύματα τοῦ Νείλου καὶ εὑρήσεις ἐκεῖ λίθον ἔχοντα πνεῦμα, » σαφῶς\footnote{ἐκεῖ λίθον om. AK. --- δείκνυσι σαφῶς Lc.} δείκνυσι τὸν τοῖς ῥεύμασι πλυνόμενον κατὰ τὴν ταριχείαν τοῦ ἡμετέρου\footnote{τὸν] τῶν AK Laur. --- τοῦ om. Lc, qui lit : ἡμέτερον λίθον comme Laur. (f. mel.).} λίθου, ἀνθ ᾽ ὧν καὶ πᾶς ὁ χαλκὸς λίθος ἐστὶ κατὰ τὴν σὴν μετάλλων\footnote{χαλκὸς λίθος] F. l. χαλκόλιθος (\emph{M. B.}). --- σὴν om. Lc. f. mel.} γένησιν, καὶ πᾶς ὁ μολυβδόλιθος. Τοῦτον οὖν τὸν λίθον εὑρήσεις, φησὶν,\footnote{μόλυβδος λίθος Laur ; μολυβδόχαλκος Lc.} (f. 170 r.) ἔχοντα πνεῦμα, ὅς ἐστι < τρόπος > τῆς ἐξυδραργυρώσεως.\footnote{πνεύματα Laur. --- ὅς] ὡς A ; ὅ Laur., f. mel.}

 

6. Ἐπειδὴ καὶ ὁ Δημόκριτος ἐκεῖνος ὁ ἐμοὶ ἀγαθώτατος ἐδιακρίθη\footnote{ἐπειδὴ καὶ ὁ Δημ. --- γάλακτι ὀνείῳ ἢ αἰγείῳ (l. 15)]. Passage reproduit dans le texte 3, 29, 21, d'après le ms. A, f. 139 r. (texte que nous désignons par un astérisque) avec quelques variantes rapportées ici. Ἐπειδὴ ὁ Δημ. ἐκ. ὁ ἐ. ἀγαθῶς λέγει · Δέξαι λίθον τὸν οὐ λ ... τὸν ὀμώνυμον (comme les mss. de Zosime), τ. ἀ. λέγω (λέγει Lb, p. 339, A\textsuperscript{3}) ... ὡς γὰρ ἐκ πᾶν (f. l. ἐπὰν ? ) ... πάντα ἐ. ν. λέγω κ. τ. λ. Dans le texte 3. 29, les bonnes variantes de A\textsuperscript{2}, de A\textsuperscript{3} et de Lb sont généralement conformes au texte de Zosime. --- ἐπειδὴ καὶ ὁ Δημ. --- λίθον] Réd. de Laur. : ἐπεὶ δὲ καὶ ὁ Δημ. ἐκεῖνος δὲ μοι ἀγαθότητος καὶ φησὶν δέξε λίθον. --- Réd. de Lc : Καὶ ὁ Δημ. δέ φησι · Δέξαι λίθον. --- ἐδιακρίθην AK Laur.} καθ ᾽ ἑαυτοῦ φησιν · « Δέξαι λίθον τὸν οὐ λίθον, τὸν ἄτιμον\footnote{F. l. καθ ᾽ ἑαυτὸν. --- Δέξαι λίθον] Cp. Stephanus, éd. Ideler, p. 217, l. 20-23.} καὶ πολύτιμον, τὸν πολύμορφον καὶ ἄμορφον, τὸν ἄγνωστον καὶ πᾶσι\footnote{Après πολύτιμον] καὶ τὰ ἑξῆς Lc, qui om. τὸν πολύμορφον jusqu'à αἰγείῳ (l. 15). --- πᾶσιν γνωστὸν Laur.} γνωστὸν, τὸν πολυώνυμον, καὶ ἀνώνυμον, τὸν ἀφροσέληνον λέγω.\footnote{ἀνώνυμον] ὁμώνυμον mss. Corr. conj.} Οὗτος γὰρ ὁ λίθος [ὥστε γὰρ] οὐκ ἔστι λίθος, καὶ πολύτιμος ὢν,\footnote{ὥστε γὰρ] γὰρ om. Laur. --- πολυτίμιτος (pour πολυτίμητος) Laur.} οὐδενὸς πιπράσκεται, μίαν ἔχει φύσιν καὶ ἓν ὄνομα, καὶ ἐν πολλοῖς\footnote{ἐπιπράσκεται AK Laur. F. l. ἐμπιπράσκεται. --- ἔχων $\star$ dans Lb (p. 339).} ὀνόμασι κέκληται, οὐχ ἁπλῶς λέγω, ἀλλ ᾽ ὡς ἔχει φύσεως, ὥστε ἐάν\footnote{ἐὰν γάρ τις εἴπῃ $\star$ Lb. --- ὡς γὰρ AK Laur.} τις εἴποι πυρίφευκτον, καὶ αἰθάλην λευκὴν < ἢ > λευκὸν χαλκὸν,\footnote{καὶ om $\star$ (dans Lb). ἢ restitué ici d'après $\star$.} οὐ ψεύδεται. Πάντα ἐπὶ νεφέλην λέγει, ἐπειδὴ παρὰ πάντα τὰ ἄλλα\footnote{λέγει] λέγων Laur ; λέγω $\star$ dans Lb.} φεύγει τὸ πῦρ, καὶ ἡ αἰθάλη ἐστὶν τῆς κινναβάρεως, καὶ αὕτη μόνη λευκαίνει τὸν χαλκόν. Καῦσον οὖν αὐτὸν πραέως καὶ σβέσον ἐν\footnote{καῦσον --- τὸν χαλκόν (l. 17) om. Laur ; hab. K. --- πραέως om. $\star$ (Lb).} γάλακτι ὀνείῳ ἢ αἰγείῳ. Ἀποδίδου τοίνυν καὶ ἐπισυγγενάμενος ὅτι\footnote{ἀποδίδου --- ἡ αἰθάλη] Réd. de Lc : ἀποδίδωσι δὲ μετὰ ταῦτα ὁ φιλόσοφος ὅτι ἡ αἰθ.} παρὰ πάντα τὰ ἄλλα φεύγει τὸ πῦρ, καὶ ἡ αἰθάλη ἐστὶ τῆς κινναβάρεως, καὶ αὕτη μόνη λευκαίνει τὸν χαλκόν.

7. Καὶ πῶς οἱ φιλόσοφοι σαφῶς παραδίδουσιν τὴν ἔννοιαν, ὅτι τὸν\footnote{καὶ πῶς --- οὐκ οἶδεν (l. 3)] om. Lc. --- παραδίδωσι AK Laur.} ἐξυδραργυρωθέντα πυρίτην λίθον καλεῖ ; Οὗτος οὖν ὁ ἀγαθώτατος φιλόσοφος\footnote{ἀγαθότητος AK Laur. ici et partout. Corr. conj.} · « Τίς οὐκ οἶδεν ὅτι ἡ αἰθάλη τῆς κινναβάρεως, [ἤγουν] ὑδράργυρός\footnote{ἡ αἰθάλη --- κατὰ φύσιν (p. suiv., l. 7)]. Barre verticale en marge de Lc. --- κινναβ. ἐστὶν ἡ ὐδράργ. Lc. --- ἤγουν om. Laur, f. mel. --- ὁ ὑδράργ. αὐτός ἐστιν Laur.} ἐστιν, δι ᾽ ἧς καὶ συντίθεται ; Διὸ καὶ εἴ τις ἐλλείωσας αὐτὴν\footnote{ἐλειώσας AK Laur ; ἐλείωσεν Lc, f. mel.} τὴν κιννάβαριν νιτρελαίω, ἀναφυράσας καὶ περικλείσας ἐν ἄγγε- (f. 170 v.) σιν διπλοῖς, ὑποκαύσας φωσὶν ἀλήκτοις πᾶσαν αἰθάλην λήψεται,\footnote{καὶ ὑποκαύσας Lc. --- αἰθ. λήψεται] τὴν αἰθ. ἔλαβεν Lc.} ἐγκεκαυμένην εἰς τὰ σώματα. Οὐκοῦν ὁ λίθος ὢν δι ᾽ οὗ ἔχει\footnote{Après σώματα] Lc. continue ainsi : Ἀφροσέληνον δὲ λέγεται ὅτι ὁ λίθος γίνεται ἐκ τῆς Ἀφροδίτῃς ἥ ἐστιν ὑδράργυρος, καὶ ἐκ τῆς σελήνης ἥ ἐστιν ἄργυρος · ὥσπερ γὰρ τὸ φῶς τῆς σελήνης κ. τ. λ. (p. suiv. l. 4).} σύμπηξιν ἐν τῷ σώματι τῆς μαγνησίας, οὐκ ἔστι λίθος · διὸ ἔχει φύσεις τῆς ῥεύσεως. Ἆρα οὐκ ἀκούεις αὐτοῦ τοῦ Δημοκρίτου τί ἀνώτερον λέγει ; « Λαβὼν ὑδράργυρον, πῆξον τὸ σῶμα τῆς μαγνησίας [ἥτις] τῷ μεμιγμένῳ, κατὰ μίαν τοῦ σώματος οὐσίαν, ἐν τῷ μολυβδοχάλκῳ. » Ἆρα οὐχὶ τοῦτό ἐστι τὸ ἀφροσέληνον ; πάντες γὰρ ἴσασιν ὅτι κατ ᾽ ἀναφορὰν τὴν Ἀφροδίτην καὶ σελήνην ἐκ τῶν δύο ὀνομάτων\footnote{F. l. τῆς Ἀφροδίτης καὶ σελήνης.} σύνθετον ὄνομα ἡμῖν μεθερμηνευόμενον ἀφροσέληνον · πάντες γὰρ ἴσασιν ὅτι κατ ᾽ ἀναφορὰν τῆς Ἀφροδίτης ἀστρολόγον τὸν χαλκὸν\footnote{ἀστρολόγων Laur. --- F. l. τῇ Ἀφροδίτη < οἱ > ἀστρολόγοι τὸν χ. ἀνατίθενται.} ἀνατίθεται. Οἱ μὲν ταχύτερον τὴν ὑδράργυρον λέγουσιν, εἰ δὲ πνεύματικώτερον\footnote{F. l. παχύτερον Cp. la fin du § 1. (C. \emph{E. R.}) εἰ δὲ] F. l. οἱ δὲ (\emph{M. B.}). --- τὸν ὑδράργ. Laur. ici et partout.} τὴν ὑδράργυρον, ἐπείπερ ἐν σελήνῃ ἐνρωηκὰ ἀπορία\footnote{ἐνροϊκα K. --- ἀπορία AK Laur., ici et partout. F. l. ἐνροὴ καὶ ἀπύρροια. On connaît ἐνρέω et ῥοή (C. \emph{E. R.}) ἀπορία, c'est le déclin de la lune exprimé comme le mercure par le croissant retourné. (\emph{M. B.}). Cp. p. 125, note sur la ligne 10, réd. de Lc (C. \emph{E. R.}).} ἐστὶν τοῦ φωτὸς, καὶ αὕτη ἡ ῥεῦσίς ἐστιν τῆς οἰκείας φύσεως ἐνδικαίως\footnote{ἐνδικαίως] εἴδη καὶ ὡς Laur.} τῶν ἄλλων πάντων τῶν ἄστρων · ὁ Ζεὺς μόνος προσηγορεύεται πρῶτον ἤλεκτρον, κατ ᾽ ἀναφορὰν < ἣν > ἔχει ἐκ τριῶν τὸ ἐλάχιστον\footnote{πρῶτον μὲν Laur. --- κατ ᾽ ἀναφ. --- ἠλέκτρου om. Laur.} παντὸς ἠλέκτρου συντιθεμένου.

8. Οὐκοῦν διὰ τὴν ἁπλῆν τῆς προσηγορίας < ὁ > μὲν ἄργυρος\footnote{ἁπλὴν mss. F. l. ἁπλόην.} κατ ᾽ ἀναφορὰν τῆς σελήνης, ὡς ἐντεῦθεν ὁ ἀγαθώτατος φιλόσοφος, οἰκείοις τοῖς ὀνόμασι κεχρημένος, ἐν τοῖς τῶν δύο πρὸς ἀργυρίων ὡς ἔφρασεν,\footnote{κεχρημένοις AK Laur. Corr. conj. --- ἐν τοῖς] ἐκ τοῖς AK Laur. F. l. ἐκ τῆς τ. δ. προσαργ. < μίξεως > ? ;--- ὡς ἔφρασεν] κατὰ μίαν ἀναφορὰν ἔφρ. Laur.} τὸ ἀφροσέληνον ἐκάλεσεν. Καὶ ἐπείπερ τὸ (f. 171 r.) φῶς ἀντὶ τῆς σελήνης\footnote{Καὶ ἐπεί περ] ὥσπερ γὰρ Lc. --- ἀντὶ om. Lc. f. mel.} πνευματικῶς ὁρᾶται (κατὰ γὰρ τὸ σῶμα γίνεται καὶ ἀπογίνεται), οὕτω καὶ αὕτη κατὰ τὸ σῶμα τῆς μαγνησίας γίνεται καὶ ἀπογίνεται\footnote{αὕτη] αὐτοῖς Laur. --- τὸ σῶμα αὐτῆς Lc. --- οὕτω --- κατὰ τὸ σῶμα] Réd. de Lc : οὕτω καὶ τὸ ζητούμενον ἡμῶν πνεῦμα κατὰ τ. σ.} · καὶ πνεῦμά ἐστιν κατὰ φύσιν. Ἀνθ ᾽ ὧν καὶ πάλιν ὡς διαιρουμένης\footnote{Après κατὰ φύσιν] Réd. de Lc : Διὸ καὶ ὁ Ζώσιμος ἠρώτησε τὸν ἑστῶτα ἐν τῷ φιαλοβωμῷ, οὕτω λέγων · καὶ σὺ (l. 9).} ἐρωτῶμεν ἐν τῇ κατ ᾽ ἐνέργειαν περὶ ἀρετῆς πραγματείᾳ διὰ Ζώσιμον,\footnote{πράγματι A Laur. ; πράγμασι K. Corr. conj.} ὡς δι ᾽ αὐτοῦ ἐρωτῶντες · « Καὶ σὺ ἄρα πνεῦμα εἶ ; » Ὁ δὲ ἀποκρίνεται\footnote{F. l. ὡς δὴ αὐτοῦ ἐρωτῶντος. --- καὶ σὺ] καὶ λέγ. καὶ σὺ Laur. --- ἀπεκρίνατο Lc. --- καὶ φησὶ om. Lc.} καὶ φησί · « Καὶ πνεῦμά εἰμι, καὶ φύλαξ πνευμάτων, πνεῦμα οὖσα κατὰ\footnote{πνεῦμα οὖσα --- ἀναλαμβάνει] Réd. de Lc : τὸ πν. γὰρ τὸ ὂν κατὰ τὴν πν. τοῦ ἀργύρου οὐσίαν ἀναλ.} πνευματικὴν [τοῦ ἐρωτῶντος] ἐν τῇ σελήνῃ οὐσίαν, ἀναλαμβάνει τὸ σῶμα τῶν συγκραθέντων στερεῶν, καὶ ποιεῖ αὐτῷ πνεῦμα λογχευόμενον,\footnote{αὐτὸ Lc, f. mel. --- λοχευόμενον Lc, f. mel.} ὡς ἐν βάθει ἑαυτῆς, ὃ ἔχει ψυχὴν ἐκ τῆς καρδίας καὶ εἰς ὄρυγμα\footnote{ἐαυτοῦ, καὶ ἔχει Lc. --- εἰς om. Lc.} ἐν στομάχῳ, κατὰ τὸ ὑέλιον τοῦ κινοῦντος τὴν δύναμιν ἑλκυούσασα\footnote{κατὰ τὸ ὑέλιον] Il y a eu probablement dans un ms. oncial ΚΑΤΑΤΟΥΗΛΙΟΥ (κατὰ τοῦ ἡλίου). Réd. de Lc : κατὰ τὸν ἥλιον τὸν κινοῦντα τ. δ. ἕλκει πρὸς ἑαυτὸ ἀλλοιωτικὴν δύναμιν καὶ αὕτη εἰς αἷμα κ. τ. λ.} πρὸς ἑαυτὴν πρὸς ἀλειωτικὴν, ἐξαλλοιοῦσα τοῦτο εἰς αἷμα κατάγει τὸν χυμὸν,\footnote{F. l. προσαλλοιωτικὴν. --- ἐξαλλείουσα τούτω AK Laur. Corr. conj.} καὶ κατὰ τὴν θελκτικὴν καὶ ἀποκριτικὴν, τὰ ἄλλα φυσικῶς\footnote{καὶ κατὰ τὴν θελκτικὴν (θελητικὴν A ; θερητικὴν Laur.) jusqu'à κατεργαζομένη] Réd. de Lc : καὶ ἔστι θελκτικὴ καὶ ἀποκριτικὴ ἅπαντα φυσ. κατεργ.} κατεργαζομένη. Ἢ γὰρ οὐδὲ τοῦτο ἤκουσας, ὥς φησιν, τὴν πολυθρύλλητον\footnote{κατεργαζομένην AK Laur. --- ἢ γὰρ --- ἀνακράζοντες] Réd. de Lc : Διὸ φησὶν ὁ φιλόσοφος · περιμ. χ., περιμάχου ὑδρ ... (Cp. Stephanus, leçon 4, p. 217 éd. Ideler).} φωνὴν ἀνακράζοντες. « Περιμάχου χαλκὸν, μάχου ὑδράργυρον,\footnote{F. l. ἀνακράζοντος. --- F. l. πυρὶ μάχου.} καὶ ἀσωμάτωσον τελείως εἰς φθορὰν τὴν τέχνην, καὶ ὡς οὐδὲν ἐπὶ\footnote{τῇ τέχνῃ Lc, qui continue ainsi : καὶ γὰρ τὸ τῆς μαγνησίας σῶμα (ci-après, l. 4). --- οὐδὲν] F. l. οὐδενὶ.} τούτου κέχρηται, πλὴν τῆς ὑδραργύρου καὶ τῆς μαγνησίας, καὶ εἰσὶν ἄμφω διὰ τὴν σύμπηξιν. « Λαβὼν, φησὶ, τὴν ὑδράργυρον < καὶ > τὸ\footnote{A mg. σῆ.} τῆς μαγνησίας σῶμα, καὶ πνεῦμα ἔχει διὰ τὴν ἐξυδραργύρωσιν\footnote{καὶ εὑρίσκεται --- προγέγραπται om. Lc.} · » καὶ « εὑρίσκεται, φησὶν, πρὸς τοῦ Νείλου τὰ ῥεύματα, ἀνθ ᾽ ὧν καὶ διὰ ῥεύσεως ὁμορρευστῆσαι, ὡς προγέγραπται · » καὶ, ὥς φησιν, « Οὐδὲν ὑπολέλειπται,\footnote{ὡς προγεγρ.] ὡς om. Laur., f. mel. --- ὥς φησιν om. Laur. --- καὶ πάλιν φησὶν Lc.} οὐδὲν ὑστερεῖ (f. 171 v.), πλὴν τῆς νεφέλης · ἤτοι\footnote{Après νεφέλης] Réd. de Lc : καὶ τοῦ ὕδατος ἡ ἄρσις, ἤγουν πλὴν τοῦ διορατικοῦ καὶ διανοητικοῦ · διορῶμεν γὰρ τὸ σῶμα τῆς μαγνησίας, διανοοῦμεν δὲ τὴν δύναμιν αὐτῆς ὡς πρὸς τὰ προσφωνούμενα.} < διὰ > τοῦ διορατικοῦ καὶ τοῦ διανοητικοῦ δυνάμενος διορᾶν καὶ διανοεῖσθαι πρὸς τὰ προσφωνούμενα.

9. Τί γὰρ ὁ Ἑρμῆς καὶ αὖθις προστάττων διαλέγεται τὸ ἀπὸ\footnote{Le rédige ainsi le début de notre § 9 : Ὁ δὲ Ἑρμῆς φησι, τὸ ἀπὸ τῆς σελ. ἀπορροίας ἐκπίπτον, ἤγουν ὥσπερ τὸ τῆς σελήνης φῶς αὐξαίνει καὶ μειοῦται, οὕτω καὶ ὁ ἡμέτερος ἄργυρος μειοῦται μὲν διὰ τῆς ἀσωματώσεως, ἀντιστρόφως τῆς σελήνης. Ἡ δὲ ἀπόρροια καὶ ἡ εἴσροια διὰ μακρᾶς καὶ μετρίας ἐκπυρώσεως ὀφείλει (sur δεῖ, gratté) γίνεσθαι, ἵνα (page 319) φυλαχθῇ τὸ πνεῦμα κ. τ. λ. --- Τί γὰρ ὁ Ἑρμῆς Laur. --- τὸ ἀπὸ σελ. --- τὴν φύσιν] Cette phrase se retrouve dans Stephanus, p. 203.} τῆς σεληνιακῆς ἀπορίας ἐκπίπτον, ποῦ εὑρίσκεται, καὶ ποῦ οἰκονομεῖται,\footnote{ἀπορίας] ἀπορροίας Ideler.} καὶ πῶς ἄκαυστον ἔχει τὴν φύσιν, παρ ᾽ ἐμοὶ εὑρήσεις καὶ Ἀγαθοδαίμονος · διὰ γὰρ τοῦ λέγειν ἀπορίας πάλιν τῆς ῥεύσης\footnote{F. l. Ἀγαθοδαίμονι. --- F. l. ῥευστῆς.} ἀνάπτησον, καὶ καταδηλότερον γίνεται διὰ τὸ ἐπαγαγεῖν τὸ ἀπὸ τῆς\footnote{F. l. ἀνάπτυσον.} σεληνιακῆς ἀπορίας ἐκπίπτῃ κατὰ τὴν τῆς σελήνης οὐσίαν. Κατεχόμενον\footnote{F. l. ἐκπίπτει, ici et plus loin.} γὰρ τὸ σῶμα ἐκπίπτῃ διὰ τῆς ἀπορίας, καὶ γὰρ σεληνιάζεται\footnote{καὶ γὰρ --- τῆς μαγνησίας om. Lc.} ἡ φύσις τῆς μαγνησίας σεληνοειδὴς ὅλη γινομένη, καὶ κατὰ καιρὸν τῆς ἀπορίας ἐκφυσᾶται · ὡς ἰὸν ἐκπίπτει τῆς ἀπορίας καὶ ἐκστροφὴν\footnote{ἰὸν] οἶων Laur.} ὑπομένοντος ὢν ( ? ) τοῦ σώματος. Καὶ νῦν ἀνάστρεψον πρὸς τὰς ἀπορίας καὶ διορατικὸν καὶ διαβλητικὸν δι ᾽ ἀπορίας ῥεύματος καὶ ῥεύσεως\footnote{διορατικῆς Laur. --- διαβλυτικὸν διαπορίας mss Corr. conj.} κατὰ τὴν κριτικὴν τῆς ῥεύσεως φύσιν λαμβάνει τὴν κατεργασθεῖσαν\footnote{F. l. λάμβανε.} διὰ τῆς φιλοσοφίας μαγνησίαν καίουν ἢ διὰ πυρὸς ἢ διὰ τῆς\footnote{F. l. καὶ οὐκ ( ? ). --- ἢ διὰ τὰς mss.} ἑαυτοῦ ἐκπυρώσεως, ἀλλὰ διὰ τῆς ἀπορίας, ἵνα φυλαχθῇ τὸ πνεῦμα, καὶ μὴ ἐκπνεύσῃ τῇ βίᾳ τῆς ἐκπυρώσεως.

10. Οὕτω νόησον, ὥς φησιν Ὁστάνης, βάλλων τὴν χεῖρά σου\footnote{Réd. de Lc : Οὕτω δέ φησι καὶ ὁ καίων. Ὀστ., βάλε.} εἰς τὰ ἐντὸς τοῦ λίθου, καὶ ἔκβαλε τὴν καρδίαν αὐτοῦ, ὅτι ἡ ψυχὴ αὐτοῦ ἐν τῇ καρδίᾳ ἐστίν. Οὐκοῦν διὰ τῆς τοιαύτης ἀπορίας, πάντα\footnote{ἀπορροίας Lc, f. mel.} τὰ ἐντὸς ἀποβάλλει (f. 172, r.) ὁ τοιοῦτος λίθος καὶ ἐξερεύγεται\footnote{ὁ τοιοῦτος ὁ λίθος] ὅτι οὗτος ὁ λ. Laur., f. mel.} τὰ βάθη τῆς καρδίας, καθὼς ἔστι τὸ πνεῦμα, ὅς ἐστιν ὁ ἰὸς ξανθὸς ὡς στίγμα χρυσοῦν δογματιζόμενον · περὶ τούτων γὰρ συναπτόμενα\footnote{ὡς στίγμα χρυσοῦν] F. l. ὡς τῆγμα χρυσοῦν \emph{vel} χρυσοῦ (ici et plus loin). Cp. p. 119, l. 12. --- δογματιζόμενος Lc, f. mel., puis : Διὸ καὶ ὁ Δημ. --- τούτων] τοῦτον AK.} < ἃ > πάλιν Δημόκριτός φησιν, « πυρίτην οἰκονόμει ἕως ξανθὸς γένηται ὡς στίγμα χρυσοῦν, καὶ δοκίμαζε εἰ γέγονεν ἄσκιον.\footnote{ἐὰν δὲ μὴ Lc. --- ἄσκιος Laur. Lc, ici et lig. suiv.} Ἐὰν μὴ γέγονεν ἄσκιον, τὸν χαλκὸν μὴ μέμψαι, ἀλλὰ σαυτὸν μέμψαι,\footnote{σεαυτὸν Lc. --- μέμψαι om. Laur. ; ajouté sur la ligne dans A.} ἐπεὶ μὴ καλῶς ᾠκονόμησας. Οἰκονόμει οὖν ἕως ξανθὸς ἄσκιος ὁ\footnote{ἕως ἂν ξ. καὶ ἄσκ. γένηται Lc, puis : τότε γὰρ πᾶν σ. βάπτει εἰς χρυσὸν καὶ γίνεται ...} χαλκὸς γενόμενος πᾶν σῶμα βάπτῃ, χρυσὸς γίνεται ὡς στίγμα χρυσοῦν. » Καὶ χρὴ ἐντεῦθεν ἐπιθεωρεῖν καὶ διασκοπεῖν εἰ γέγονεν ἄσκιον\footnote{ἄσκιος καὶ ξανθὸς Lc.} ξανθὸν ὡς στίγμα χρυσοῦν · εἰ γὰρ μὴ γέγονεν ἄσκιον,\footnote{χρυσοῦ Lc, f. mel. --- γὰρ] δὲ Laur. Lc. --- οὔτε] f. l. οὐδὲ.} οὔτε βάπτειν ξανθὸν ὡς στίγμα χρυσοῦν δύναται. Ἐὰν γὰρ μὴ ἔστι\footnote{βάπτει AK Laur.} χρυσοῦν κατὰ ποιότητα · ἐπειδὴ ποιαὶ αἱ ποιότητες ποιοῦσιν ξανθόν\footnote{ποιαὶ] ποίηαι A ; ποίειαι K. Réd. de Laur. : κατὰ πιότιταν. ἔπιδεί περ ποῖαι αἱ πιότιτες. Après ποιότητα] Réd. de Lc. : Πῶς δύναται βάψαι εἰς χρυσόν ; πᾶσαι γὰρ αἱ ἐνέργειαί εἰσιν ἀσώματοι ποιότητες · ὅθεν καὶ ... (l. 22).} · καὶ γὰρ ποιότης ἀπὸ τοῦ ποιεῖν ἐτυμολογεῖται [ποιεῖν.] Ποιεῖ\footnote{ποιῆ K.} βάψιν κατὰ ποιότητα χρυσῆν · φανερὸν γὰρ ὅτι < αἱ > τῶν ποιοτήτων ἐνέργειαι ὡς ἀσώματοί εἰσιν · ὅθεν καὶ ἡ κατενέργεια χρυσοῦν · ἐπεὶ\footnote{Réd. de Lc : ἡ κατ ᾽ ἐνέργειαν ποιότης τοῦ χρυσοῦ ὅταν μὴ κατὰ π. λ.} μὴ [κατὰ] ποιότητα λευκὴν κατ ᾽ οὐσίαν ἔχει τὸ χρῶμα οὔτε ποιεῖν\footnote{ἔχει] ἔχοι Laur. : ἔχῃ Lc. --- οὔτε ποιεῖν, ἢ ποιοῦν δύναται, οὔτε βάπτειν χρυσοῦν Lc.} δύναται, οὔτε βάπτειν χρυσόν. Ὁ δὲ ἡμέτερος χρυσὸς, ἐπεὶ κατὰ\footnote{ἐπεὶ] ἐπειδὴ Lc.} ποιότητά ἐστιν, ποιεῖν καὶ βάπτειν δύναται, ὃ καὶ μυστήριον τοῦτο\footnote{ποιεῖν καὶ βάπτειν] ποῖος χρυσὸς δύναται καὶ ποιοῦν καὶ βάπτειν Lc.} μέγα ἐστὶν, ὅτι ποιότης γίνεται χρυσὸς, καὶ τότε ποιεῖ τὸν χρυσόν.\footnote{χρυσὸς] F. l. χρυσῆ.}

11. Διὸ καὶ Στέφανος τῶν φιλοσόφων φησὶν ὅτι ποιότης μὲν διαβάσει\footnote{Στέφανος τῶν φιλοσόφων] ὁ Στέφανος ὁ φιλόσοφος Lc. --- ποιότης jusqu'à ἡ ποιότης] (l. 8) Réd. de Lc : ἡ ποιότης διαβάσα ἐποίησε τὸν χρυσὸν, ἤγουν τὸ ζητ. Puis : καὶ πάλιν ὁ αὐτὸς · ἡ ποιότης ...} ἐποίησε τὸ ζητούμενον, καὶ πειθομένας καὶ διερωτᾶν αὐτὸν\footnote{πειθομένας] F. l. πειθομένους.} ἐπάγει · καί φησιν · « Ποία (f. 172 v.) ἐστὶν ποίοτης ; » ἡ συγκρινόμενος\footnote{ἡ συγκρινόμενος] F. l. καὶ ἀποκρινόμενος.} καὶ δίδωσιν λέγειν · « ἡ ποιότης τοῦ ξηρίου κατὰ ποιότητας χρυσᾶς ἐστιν. Καὶ ἡ μὲν οὐ κατὰ ποιότητα γίνεται χρυσῆν, τὸ χρῶμα\footnote{καὶ ἡ μὲν] εἰ μὲν γὰρ Lc, mel. --- χρυσῆν] signe de l'or A ; χρυσοῦν K ; χρυσὸς Lc. Corr. conj.} τέλειον χρυσός ἔχων, οὐ δύναται ποιεῖν χρυσόν. Οὐκοῦν, ὥς φησιν,\footnote{χρυσὸς] χρυσοῦ Lc. --- οὐκοῦν ὥς φησι] Διὸ φ. Lc.} δοκίμαζε εἰ γέγονεν ἄσκιον ξανθὸν, ὅ ἐστιν ἀσώματον, ἰὸς ξανθὸς γινομενος\footnote{ἄσκιος ξανθὸς ὅ ἐστιν ἀσώματος Lc. --- γενόμενος Lc.} ὡς στίγμα χρυσοῦν · ὃ τοίνυν δοκιμαστέον οὖν εἰ γέγονεν ἄσκιον\footnote{ὃ τοίνυν --- παρεξελεάσαμεν (l. 17) om. Lc.} ξανθὸν ὡς στίγμα χρυσοῦν βλεπόμενον.

12. Οὕτω μὲν οὖν αἰτούμενον ἐπικοπτόμενοι τὴν τοῦ λόγου ἔνταξιν,\footnote{ἐπικοπτόμενον K. F. l. ἐπισκεπτόμενοι.} καὶ μέλη ποιεῖν, καὶ εἰ περὶ τῆς ὕλης καὶ τῆς κατ ᾽ αὐτῆς οἰκονομίας\footnote{μέλη] μέρη Laur. --- εἰ περὶ] Leçon de Laur. ; ὑπερὶ A ; ὑπὲρ K. F. l. αἱ περὶ.} ἀποδόσεις, ὡς δεῖ ὑπερτίθεσθαι τὸν τρόπον τῆς δοκιμῆς καὶ\footnote{ἀποδώσεις A ; ἀπόδοσις K.} ἀναστρέφειν ὅθεν παρεξελεάσαμεν. Καὶ λογικώτερον δείκνυται, ὅτι καὶ\footnote{F. l. παρεξηλάσαμεν \emph{νel} παρεξελεύσομεν.} λευκὸς γενόμενος ξανθός ἐστιν εἰς ἄκραν προσφαινόμενον. Διασκοπητέον\footnote{πρὸς τὸ φαινόμενον Lc. --- διασκοπ. τ. κ. σημ. om. Lc.} τοίνυν καὶ σημειωτέον, διὸ αὐτόν φασι, μετὰ τὴν τοῦ χαλκοῦ ἐξίωσιν\footnote{Διὸ καὶ φασὶ πάντες μετὰ ... Lc. --- Mετὰ τὴν τ. χ. jusqu'à ξάνθωσις]. Cette phrase est dans Stephanus, p. 204, éd. Ideler. --- ἐξίωσιν jusqu'à τότε] Réd. de Lc : ἐξίσχνωσιν καὶ μελάνωσιν καὶ λεύκωσιν καὶ ἐξίωσιν, τότε ...} καὶ μελάνωσιν, ἐς ὕστερον λεύκωσιν, τότε ἔσται βεβαία ξάνθωσις\footnote{λεύκωσιν] λεύκοσης A ; λευκώσης Laur. ; λεύκωσις corrigé en λεύκωσης K. Corrigé d'après Stephanus.} · ὡς κἀντεῦθεν τρόπος τοῦ δοκιμάσαι εἰ γέγονεν ἄσκιον ξανθὸν ἀποδέδεικται\footnote{ὡς κἀντεῦθεν τρ. τοῦ δοκ.] καὶ οὗτός ἐστιν ὁ τρ. τοῦ δοκ. Lc. --- ἄσκιος ξανθὸς Lc.} · τοιοῦτον γάρ ἐστιν, ὃ λέγειν μετὰ τήνδε τὴν ἴωσιν\footnote{ἀποδέδεικται jusqu'à Ὀστάνης (l. 7)] Réd. de Lc : ἡ γὰρ μέλανσίς ἐστιν αἰτία τῆς λευκώσεως, ἡ δὲ λεύκωσις τῆς ξανθώσεως τῆς ἐν βάθει ἐγκεκρυμμένης καὶ ἐγκαθαιρομένης. Διὸ καὶ ὁ Ὁστ. φησὶν.} συσταθῆναι τὸ σύστημα, ἤγουν τὸ σύνθημα, καὶ ταῦτα ἐκπλυνθῆναι καὶ ἐξισχνωσθῆναι τὸ σῶμα, καὶ λίαν λεπτότατον καὶ ἀερῶδες γενέσθαι,\footnote{ἐξιχνωσθῆναι AK Laur. Corr. conj.} καὶ πᾶσαν μελάνωσιν ἀποστῆσαι, καὶ ὕστερον τοῦ ταῦτα ἀποτελεσθῆναι, τότε βεβαία ξάνθωσις ἔσται, ἡ ἐν βάθει καθαιρουμένη καὶ ἐνκεκρυμμένη · ἅμα γὰρ, ὥς φησιν Ὀστάνης, ἐλεύκανας, ἐξάνθωσας\footnote{Ὁστάνης A ; ὁ Oστ. Laur. K Lc. --- ἐξανθώσας jusqu'à βλέπε] Réd. de Lc : ἐξανθώσας. Καὶ ὁ Ζώσιμος · Βλέπε.} (f. 173 r.) καὶ πολὺ ἔσται διαμαρτυρούμενον καὶ διὰ Ζωσίμου\footnote{F. l. καὶ < τοῦτο > πολὺ. --- πολὺ] πολλὴ A Laur. K. --- διὰ] λίαν A Laur. K. Corrigé d'après un passage précédent (§ 7) : διὰ Ζώσιμον.} · « Βλέπε μὴ ἀκηδιάσῃς ἐν τῷ καιρῷ τῆς λευκώσεως, » ἀνθ ᾽ ὧν\footnote{ἀνθ ᾽ ὧν jusqu'à γίνεται (l. 13)] om. Lc.} αἴτιον τοῦ ταύτην ταῦτα τὴν ξάνθωσιν γίνεσθαι, ἡ λεύκωσίς ἐστιν.\footnote{ταύτην] ταύτης A Laur. K.} Καὶ εἰ μὲν πρῶτον λευκώσεις, τελεία γενήσεται ξάνθωσις · τελεία καὶ\footnote{Καὶ εἰ μὲν] καὶ εἰ μὴ Laur. ; avec cette leçon, il faudrait lire < οὐ > τελεία γενήσεται.} βεβαία, καὶ ἀκριβὴς οὐκ ἔσται, καὶ μὴ διαγινώσκειν ὅτι πρὸς τὰ μέτρα\footnote{οὐκ] F. l. οὖν. --- μὴ] μὴν Laur. F. l. δεῖ.} τῆς λευκώσεως, ἡ ξάνθωσις γίνεται, καθὰ ἐκλείπει ἡ λεύκωσις,\footnote{καθὰ ἐκλ. jusqu'à ξάνθωσις] Réd. de Lc : ἐκλειπούσης γὰρ τῆς λευκώσεως, ἐκλ. κ. ἡ ξ.} ἐκλείπει καὶ ἡ ξάνθωσις.

13. Καὶ χρεία ἔσται παρατηρεῖσθαι καὶ διασκοπεῖν πρὸς τὴν λεύκωσιν,\footnote{Καὶ χρεία ἐ. παρ. κ. διασκ.] Réd. de Lc : χρὴ τοίνυν παρ. κ. διασκ. καλῶς.} καὶ ταύτην ἐπιτείνειν · ὥσπερ γὰρ καὶ ὁ Ἑρμῆς ἀπὸ μηνὸς\footnote{ἐπιτείνειν] ἐπὶ τίνην A ; ἐπεὶ τοίνυν K ; ἐστι τίμον Laur. Corr. conj. --- καὶ ταύτην jusqu'à Ὀστάνης] Réd. de Lc : Περὶ δὲ τοῦ χρόνου, ὁ μ. Ἑ. λέγει · μῆνας ἓξ δεῖ πλύνειν τὸ σύνθημα, ἀπὸ μ. μεχὶρ, ἤγουν φευρουαρίου, εἰκοστῇ πέμπτῃ, μέχρι μεσωρὶ, ἤγουν αὐγούστου εἰκοστῇ πέμπτῃ. Ὁ δὲ Ὁστ. κ. τ. λ.} μεχεὶρ συνάγει μῆνας πλύνειν ἕξ · καὶ Ὀστάνης διὰ τοῦ κατὰ τὸν ἀετὸν παραδείγματος τέλειον ἐνιαυτὸν διαγράφει. Πρὸς δὲ τούτοις καὶ οἱ οἰκουμενικοὶ φιλόσοφοι καὶ νέοι πάνσοφοι, καὶ ἐξηγηταὶ τοῦ Πλάτωνος\footnote{πανσόφοισται Laur.} καὶ Ἀριστοτέλους τὴν ἐναρίθμησιν τῶν ἀναλύσεων καὶ\footnote{ἀναλύσεων] πλύνσεων Lc.} καύσεων συντέμνοντές φασιν · ἑκατοντάδες δὶς ὀκτὼ, καὶ τρεῖς τρεῖς\footnote{ἑκατοντάδας δ. ὁ. κ. τρὶς τρεῖς δεκάδας καὶ τέσσαρα Lc. F. l. τρεῖς τρισκαιδεκάδας. Cp. Stephanus, p. 227.} καὶ δεκάδες καὶ τέσσαρες, δηλοῦντες ὅτι ἑκδεκάκις ἑκατὸν ἀνακάμπτεται\footnote{ἑκδεκάκις] ἑκκαιδεκάκις Lc.} καὶ ἀναλύεται τὸ σύνθημα, πρὸς τελείαν λεύκωσιν γίνεσθαι\footnote{πρὸς τὸ τελείαν Lc. --- γενέσθαι Lc.} καὶ συντελεσθῆναι κατὰ τὴν τελείαν καὶ βεβαίαν ξάνθωσιν.\footnote{καὶ ἐκφ. Ζώσ.] ἐκφ. δὲ ὁ Ζώσ. Lc.} Καὶ ἐκφαντικώτερον Ζώσιμος ἔλεγεν · « Μὴ φοβεῖσθε τὴν πολλὴν καῦσιν καὶ ἐξυδάτωσιν τῶν σωμάτων, ὅτι αἱ μυρίαι καύσεις τοῦ χαλκοῦ βαπτικώτερον αὐτὸν ποιοῦσιν χαλκόν. » Ὁ δὲ καλῶν ἰὸν τὴν προσηγορίαν\footnote{χαλκὸν om. dans Lc, qui continue ainsi : Ὅτε δὲ καλοῦσι τοῦτον ἰὸν, τὴν προσς. τῆς ὅλης συνθέσεως λέγουσι διὰ τὸ (l. 8). --- καλῶν] καλὸν A Laur. K. Corr. conj. --- ἰὸν] οἷον ἰὸν Laur.} τὴν ὅλην σύνθεσιν, διὰ τὸ κατ ᾽ αὐτὴν πλεονάζειν τὴν συσταθμίαν\footnote{κατ ᾽ αὐτὴν] κατ ᾽ αὐτὸν Laur.} · πρὸς τέσσαρα γὰρ τοῦ χαλκοῦ ἓν μολύβδου διδόντες εὐκρα- (f. 173 v.)\footnote{πρὸς] εἰς Lc. --- τοῦ χαλκοῦ] τὸν χαλκὸν Lc. --- ἐν μολύβδου A Laur. K ; ἐν μολύβδῳ Lc. Corr. conj. --- διδόντες διαιροῦντες Lc. --- εὐκραέστατον AK Lc.} εστάτην τὴν ξάνθωσιν ποιοῦσιν. Διὸ καὶ ἐκστρεφομένη ἡ\footnote{ἐκστρεφομένης τῆς φύσεως Lc.} φύσις τελεία ξάνθωσις γίνεται ὡς στίγμα χρυσοῦν, καὶ τοῦτό φησιν\footnote{χρυσοῦν] χρυσοῦ Lc. --- καὶ τοῦτο] καὶ διὰ τοῦτο Lc, f. mel.} · « Ἔκστρεψον, [φησὶ,] τὴν φύσιν, καὶ εὑρήσεις τὸ ζητούμενον · ἡ γὰρ\footnote{φησι om. Lc.} φύσις ἔνδον κέκρυπται. Ἐκστρεφομένης τοίνυν τῆς φύσεως, οὐκέτι\footnote{ἐκστρεφομένης jusqu'à ξάνθωσιν] om. Lc.} λευκὸν ὁρᾶται κατὰ τὴν προφανηθεῖσαν ἐξυδραργύρωσιν, ἀλλὰ ξανθὸν κατὰ τὴν ἐπηγγελμένην τοῦ ἰοῦ ξάνθωσιν.\footnote{ἐπηγγελμένην jusqu'à προσήκει] om. Laur.} »

14. Καὶ θαυμάσαι προσήκει κατὰ τὴν τῶν ποιοτήτων συνδρομήν · τούτων γὰρ ἀσώματοι ἐνέργειαι συνδραμοῦσαι ἀπετέλεσαν τὴν θαυμαστὴν\footnote{αἱ άσώματοι Lc.} ταύτην χρυσοποιΐαν κατὰ μίαν οὐσίωσιν, τουτέστιν ἡ θερμότης\footnote{τουτέστιν ἡ] ἡ γὰρ Lc.} τοῦ πυρὸς, ἡ ὑγρότης τοῦ ὕδατος, ἡ ψυχρότης τοῦ ἀέρος\footnote{ἡ ὑγρότης] καὶ ἡ ὑγρ. Lc. --- ἡ ψυχρότης jusqu'à ποιότητες] Réd. de Lc : καὶ ἡ ψ. τ. ἀ. αὐταὶ καθ ᾽ ἑαυτὰς αἱ ποιότητες.} · τούτων γὰρ καθ ᾽ ἑνὸς ποιότητες συνδραμοῦσαι, ὡς γῆ τὸ στερεὸν καὶ σῶμα τῆς μαγνησίας εἰς μεταβολὴν καὶ ἀλλοίωσιν μετελθεῖν\footnote{καὶ σῶμα] καὶ om. Lc.} ἐξεβιάσατο. Ποῦ ποτέ εἰσιν οἱ λέγοντες ἀδύνατον μεταβάλλεσθαι φύσιν\footnote{ἐξεβιάσαντο Lc.} ; Ἰδοὺ γὰρ μεταβάλλεται ἡ φύσις τῶν στερεῶν γινομένη, καὶ κατὰ\footnote{γινομένη καὶ κατὰ π.] καὶ γίνεται χρυσὸς βάπτων κατὰ π. καὶ Lc.} ποιότητα χρυσῆν · καὶ ὥσπερ ὧδε μετέβαλλεν ὁ μολυβδόχαλκος εἰς\footnote{Réd. de Lc : ὁ μολ. κατὰ ποιότ. εἰς χρυσὸν καὶ εἰς μέλανσιν, καὶ λεύκωσιν καὶ ξάνθωσιν κατεσπάσθη.} < χρυσὸν > κατὰ ποιότητα χρυσῆν, καὶ εἰς μέλαν κατασπασθήσεται, οὕτω μεταβάλλει εἰς τὴν κατενέργειαν χρυσοῦ ὁ κοινὸς ἄργυρος.\footnote{εἰς τὴν κατ ᾽ ἐνέργειαν χρυσοῦ οὐσίαν ὁ κ. ἄργυρος Lc.}

15. Ἀλλ ᾽ ἐπισκεψώμεθα καὶ ἴδωμεν, ὡς φιλόσοφοι ἐσμὲν,\footnote{Nos §§ 15 à 24 et dernier constituent la partie comprise entre les §§ conventionnels 1 à 9, dans le traité sur l'\emph{Art diνin}, de Jean l'Archiprêtre. Cette reproduction sera supprimée dans le texte de Jean (ci-après, 4, 3). Nous en donnons ici les principales variantes, relevées dans A (A $\star$) et surtout dans Lc (l'astérisque seul). --- ὡς] εἰ Lc.} πρὸς τὴν ἐγκεκρυμμένην ῥῆσιν ταύτην, τί μᾶλλον ὁριζόμενοι ποιῆσαι.\footnote{Réd. de Lc : πρὸς τὸ ἀκριβὲς τῆς ῥήσεως τι μᾶλλον ὀριζόμεθα ποιεῖν ἐνταῦθα.} Ὡς ἄρα οὖν ἀπολείπει τι τῶν ποιοτήτων, εἰς οὐδὲν γίνεται τὸ προσδοκώμενον. Καὶ πρότερον μὲν οὖν, ἐὰν μὴ ἡ σύγκρασις τῶν στερεῶν ἀποτελεσθῇ, εἰς κενὸν καὶ μάταιον πᾶς πόνος καὶ κάματος\footnote{ὡς ἄρα οὖν] εἰ γὰρ Lc.} λογισθήσεται ἡμῖν. Διὸ καὶ καθ ᾽ ἑαυτῶν ἠ σύγκρασις οἰκονομηθεῖσα,\footnote{καλῶς ἀποτελεσθῇ Lc.} ὡς (f. 174, v.) εἴρηται, ἐν τῇ ἀπορίᾳ τῆς ῥεύσεως ἄχρηστος γίνεται,\footnote{Διὸ καθ ᾽ ἑαυτὴν Lc.} καὶ εἰς κενὸν μεταβάλλει, μετὰ δὲ τῆς συμμετρίας τοῦ ὑγροῦ κερασθεὶς\footnote{Réd. de Lc : ὡς εἴρηται ἄχρηστος γίν. ἐν τῇ ἀπορροίᾳ Lc.} εἰς ἄκρα τῶν ξανθῶν ἐπανάγει. Καὶ ἠ αἰτία φανερὰ,\footnote{κερασθεῖσα Lc.} ὅτι τοῦ πυρίτου κατὰ πολὺ στερεοῦ ὄντος, καὶ πρὸς τὸ ξανθὸν ῥέποντος, τὸ κατάλληλον χαῦνον καὶ εἰς ὑγρὸν ἀποσύροντος εὐκρασίαν\footnote{εἰς ἄκρατον ξανθὸν Lc.} ἐποίησεν. Καὶ ἐνταῦθα διαδείκνυται γὰρ τέλειον τὸ χρῶμα. Εἰ δὲ\footnote{καὶ om. Lc.} οὖν ἄρα καὶ πλεονάσει τὸ ὑγρὸν, καὶ νικήσει κατὰ τοῦ στερεοῦ, ποιεῖς τὸ ξηρὸν συνκαιόμενον, μεταβάλλει εἰς μέλι. Οὕτω γὰρ τὸ\footnote{καὶ ἐνταῦθα --- τὸ χρῶμα] om. K Lc. --- Lc, par contre, ajoute : εἰ τοίνυν πλεονάσοι τὸ ξηρὸν, ὡς εἴπομεν, οὐδὲν ποιήσεις. --- γὰρ] δὲ Laur. --- Réd. de Lc : εἰ δὲ πλεονάσειε τὸ ὑγρὸν.} τῶν καθ ᾽ ἡμᾶς φιλοσόφων [μὲν] μυστήριον · συμμετρίῳ μὲν θερμαινόμενον\footnote{ποιεῖς] ποιήσεις Lc. --- καὶ μεταβάλλει Lc. --- εἰς μέλι ici et plus bas] F. l. εἰς μέλαν. --- οὕτω γάρ ἐστι τὸ τῶν φιλ. μυστ. Lc.} κατὰ τὴν ἀπλότητα τοῦ πυρίτου μένει ἐρυθραῖον αἷμα\footnote{συμμετρίῳ] μετρίως Laur. et $\star$. F. l. συμμέτρως. Réd. de Lc : συμμετρίῳ μὲν γὰρ πυρὶ θερμ.} · περισσῶς συγκαιόμενον, τῇ τοῦ ὑγροῦ συνουσίᾳ, μεταβάλλει εἰς ξανθὸν, ἐπιπλέον δὲ κατὰ πολὺ συνκαιόμενον ῥεῦσαι εἰς μέλι ποιεῖ, ἃ ποιεῖ\footnote{ἐρυθρὸν Lc. --- περισσῶς δὲ Lc ; περισσὸς AK ; περισὸς Laur.} · τὸ πᾶν ὅπερ καὶ δαίμονα ἄνθρωπον ἡ μέλαινα ποιεῖ.\footnote{ἃ ποιεῖ --- ἐφορώσης (l. 6)] om. Lc.}

16. Διανοητέον οὖν καὶ περιφυλακτέον τὴν αἰτιολογίαν, ἵνα καὶ\footnote{δαιμονὰ K ; δαιμονὰν Laur. Réd. de $\star$ : ποιεῖ ὡς ποιεῖ τὸ πᾶν, ὥσπερ καὶ δαιμονᾶν ἄνθρωπον ἡ μέλαινα χολὴ ποιεῖ.} ἡμεῖς δαίμονα παραδοθείημεν τῆς θείας δίκης, ἐπὶ πάντας ἐφορώσης\footnote{F. l. παραφυλακτέον. --- Réd. de $\star$ : ἵνα μὴ καὶ ἡμεῖς δαιμονᾶν παραδοθ.} · κατὰ ποιότητα δὲ μελετήσωμεν, ἵνα μηδὲν διαφύγῃ. Ἐὰν γὰρ μὴ\footnote{ἐφορώσης] ἐφορίσης mss. Corr. d'après $\star$.} ἡ ὑγρότης τῆς ἐξυδραργυρώσεως περιελθοῦσα κατὰ τὴν γεώδη\footnote{κατὰ ποιότητα κ. τ. λ.] Réd. de Lc : ἡμεῖς δὲ κατὰ ποιότητα μελετ. ἵνα μηδὲν διαφύγῃ (dernier mot). A la ligne au-dessous : Tέλος.} < οὐσίαν > τοῦ στερεοῦ σώματος, καὶ τὸ ξηρίον διαλύσῃ καὶ ἐξυδατώσῃ κατὰ τὴν οὐσιώδη τῆς ἐξυδραργυρώσεως ποιότητα, εἰς οὐδὲν ἔσται τὸ προσδοκώμενον. Ἐαν μὴ καὶ διαλυθῇ καὶ ἐξυδατωθῇ μὲν καὶ θερμανθῇ, εἰς οὐδὲν ἔσται τὸ προσδοκώμενον. Ἐὰν δὲ καὶ μὴ διαλυθῇ καὶ θερ- (f. 174 v.) μανθῇ, περιψυχθῇ δὲ, εἰς οὐδὲν ἔσται τὸ προσδοκώμενον. Ἐὰν δὲ καὶ μὴ διαλυθῇ πάντα κατὰ τὴν τάξιν καὶ ὁμοῦ κατὰ ἀκολουθίαν γένηται, ἐλπίζῃς τῆς ἐκβάσεως, σὺν τῇ θείᾳ προνοίᾳ,\footnote{οὐσίαν ajouté d'après $\star$.} τυχεῖν.

17. Οὐκοῦν ἐπαινετέον καὶ τὸν φιλόσοφον, ὡς ἔνθεν οὐσιώσεις\footnote{ἀκολουθίαν] ἀκολούθως A Laur. K ; ὁμοῦ καὶ κατακολούθως $\star$. Corr. conj. --- ἐλπίζεις Laur. ; ἐλπίς ἐστι τῆς ἐκβ. $\star$.} καὶ ἐν ἐκστάσει γινόμενον, καὶ ἐν μεγάλῳ θαύματι ἀναβοήσαντα\footnote{τῶν φιλοσόφων mss. Corr. d'après $\star$. --- ὡς ἔνθεν οὐσιώσεις] ἔ. οὐσ. καὶ om. $\star$. F. l. ὡς ἐν ἐνθουσιάσει.} · Ὦ φύσεις οὐράνιαι, φύσεων δημιουργοί ! Οὐράνιαι < δὲ > φύσεις αὐταὶ ἀνακαλοῦνται αἱ ἀσώματοι ποιότητες. Αὗται γὰρ ἀσώματοι οὖσαι, ἀσωμάτων\footnote{ἀναβοήσας mss. Corr. d'après $\star$. --- ὦ φύσεις κ. τ. λ.] Même phrase dans Stephanus, p. 215. --- Réd. de Laur. : ὦ φύσεις (pour φύσις) οὐρανίων φύσεων δημιουργὸς. Puis (note intercalée dans le texte) : Ἄχρις δὲ τούτου ἔντος ἀλλαχοῦ (lire ἐν τῷ ἄλλω ? ) τὸν λόγον ὁ Ζώσιμος ἔφη περὶ τῆς ἀσβέστου (Titre du morceau 3, 2, dans A, f. 8 r.). Fin du texte dans Laur. (f. 259 v.)} ἐνέργειαν δημιουργοῦσιν · < καὶ > τὰς ἐπὶ γῆς φύσεις τῶν στερεῶν\footnote{δὲ ajouté d'après $\star$.} καὶ ποιοῦσιν πάλιν ἀσωμάτων ποιότητα, ἀκωλύτως ἐνεργοῦσι κατὰ τὸ\footnote{καὶ add. $\star$.} πνευματικὸν ἀποτέλεσμα τῆς χρυσοποιΐας. Ἀσωμάτου τινὰ ποιότητα\footnote{καὶ ἀκολούθως $\star$, f. mel.} ἡ ἐξυδραργύρωσις κατὰ τὸ ποιοῦν αὐτῆς κανονίζεται · ἀσωμάτων\footnote{ἀσώματον τ. π. $\star$.} ποιότης, ἢ τοῦ ἀέρος περίψυξις ἥτις μετὰ τὴν θερμασίαν ἐγγινομένην\footnote{ἀσώματος δὲ ποιότης.} διὰ ψυχῆς καὶ τὰ ἀπὸ τοῦ πυρὸς ἐγκαύσεως. Διὸ καὶ νοητέον\footnote{ἐγγίνεται $\star$.} τοῦ θερμοῦ καὶ ψυχροῦ τὰς ἀσωμάτους ἐνεργείας [ποιοῦσιν,] τί\footnote{καὶ τὰ] F. l. καὶ τῆς.} ποιοῦσι καὶ πόσην δύνανται, καὶ θετέον μεγάλην θεωρίαν.\footnote{[ποιοῦσιν] om. $\star$ mel.} Αἱ τοιαῦται [καὶ] δραστικαὶ ποιότητες διορίζονται, ὡς κατ ᾽ αὐτὰς\footnote{πόσον δύνανται $\star$.} αὐξήσεις καὶ συντηρήσεις τῶν τοιούτων γίνεται · θερμότητες γὰρ\footnote{[καὶ] om. $\star$.} καὶ ψυχρότητες ὧδε αὐτίκα συντηροῦνται, αἱ δὲ ἄλλαι ποιότητες παθητικαὶ\footnote{γίνονται Lc $\star$.} ποιότητες ἀνακαλοῦνται · ἀνθ ᾽ ὧν τὸ ὑγρὸν καὶ < τὸ > ξηρὸν πάσχειν\footnote{ποιότητες] ποιότης A ; ποιόται K. Corrigé d'après $\star$.} ἐοίκασι παρά τινι συνθέματι. Καὶ ὡς γὰρ ἂν τὸ σῶμα τῶν στερεῶν εἰς ξηρὸν ἐπανάγον, τὸ λεγόμενον ἀσώματον θεῖον διὰ τοῦ\footnote{ποιότητες] ποιότης A ; ποιότοις K. Corr. d'après $\star$. --- τὸ add. $\star$. --- πάσχει mss. Corr. d'après $\star$.} ὑγροῦ εἰς χαῦνον καὶ ὀλισθη- (f. 175 r.) ρὸν ἀποτρέχει · συνελθόντων\footnote{ἐπανάγων mss. Corr. d'après $\star$.} τοίνυν ἔπαθον · καὶ τὸ μὲν στερεὸν διελύθη, τὸ δὲ ὑγρὸν συνεπάγη\footnote{F. l. συνελθόντα.} · αἱ γοῦν δραστικαὶ ποιότητες κατὰ μὲν τὸ θερμὸν ἐζώωσαν, κατὰ δὲ τὸ ψυχρὸν ἐψύχωσαν · καὶ ἐντεῦθεν ζῶον ἔμψυχον λέγεται τῷ θεωρητικωτάτῳ Ἑρμῇ.

18. Τὸ παρὸν σύνθημα κινούμενον ἀπὸ μονάδος καὶ μέχρι τριάδος\footnote{συνεπάγει mss. Corr. d'après $\star$.} τῆς ἐξυδραργυρώσεως ἕστηκεν · καὶ μονὰς συστάσεως ἐπὶ τριάδα ἀδιάστατόν < ἐστι > · καὶ ἔτι πάλιν τριὰς συνισταμένη ἐπὶ τριάδα διαιρουμένην,\footnote{A mg. : Une main, d'une écriture plus récente.} κόσμον συνίστησι προνοίᾳ τοῦ πρωτοποιητικοῦ αἰτίου καὶ\footnote{ἔστι add. $\star$. --- συνισταμένης. Corr. d'après $\star$.} δημιουργοῦ τῆς κτίσεως, ἔνθεν καὶ Τρισμέγιστος καλεῖται, ὡς τριαδικῶς ἐπιθεωρήσας τὸ πεποιημένον καὶ τὸ ποιοῦν. Καὶ ποιούμενος μέν ἐστιν ὁ χαλκὸς μόλυβδος ἐτήσιος λίθος · ποιοῦν δὲ θερμὸν, ψυχρὸν καὶ ῥευστὸν, τριὰς μία ἀδιαίρετος, ὡς μονὰς δευτέρα διαιρουμένη.\footnote{κόσμον συν. πρόνοιαν τοῦ πρ. αἴτιον mss. Corr. d'après $\star$.}

19. Ἀλλ ᾽ ἐπαναληψώμεθα τῶν κατ ᾽ ἐνέργειαν θεωρημάτων, ἐπὶ τοῦ φυσιολογικοῦ καὶ πρακτικοῦ < τῆς > κατ ᾽ ἐπίβασιν θεωρίας.\footnote{ἀδιαρέτη A. Corr. d'après $\star$.} Ἐπιλελυμένως δὲ κατέστη τὰς ἀνακαύσεις καὶ ἀναλύσεις · καὶ ἔτι ἐπαναλαμβανόμενος\footnote{ἐπιλυμένως mss. Corr. d'après $\star$.} Ζώσιμός φησι · « Καύσατε τὸν χαλκὸν ἐν τῷ λευκῷ συνθέματι τῷ καίοντι τὰ σώματα, καὶ πάλιν ἰοῦντι, ὁμοῦ [δὲ]\footnote{κατὰ τὰς ἀνακ. $\star$. --- ἐπαναλαμβάνων $\star$.} καὶ λευκαίνοντι. Οἱ ἐρχόμενοι γὰρ διὰ τούτων τῶν φιλοσοφικῶν θεωρμάτων,\footnote{δὲ om. $\star$} ἀνεπιλαμβανόμενοι κατ ᾽ αὐτῶν < τῆς > μυστικῆς θεωρίας\footnote{φιλοσόφων mss. Corr. d'après $\star$.} · (f. 175 v.) ἐπείπερ ἡ τούτων ἄνοια σκοτασμὸς καὶ πάσης ἀποτυχίας\footnote{ἀνεπιλαμβανόμενοι] ἀναπιμπλάμενοι A$\star$ ; ἀναπίμπλανται $\star$. --- τῆς add. $\star$.} πρόξενος ἐγένετο. Διὰ γοῦν τῶν ἐνταῦθα λέγει · « Καύσατε τὸν χαλκὸν ἐν τῷ λευκῷ συνθέματι, » ἵνα ἀπαγάγῃ ὑμᾶς ἀπὸ πάσης\footnote{ἀνοίας A. ἄγνοια $\star$.} ἄλλης καύσεως · Διελέγχεσθαι δὲ τοὺς διὰ θείου, ἢ ἀρσενίκου,\footnote{ὑμᾶς] ἡμᾶς $\star$, f. mel.} ἢ σανδαράχης καίοντας, ὡς οὐδὲν κατ ᾽ αὐτάς · οὐδὲν γὰρ λευκὸν\footnote{διελέγξῃ $\star$. --- θεῖον ἢ ἀρσένικον mss. Corr. d'après $\star$.} γίνεται ἐν τούτοις καιόμενος ὁ πυρίτης, ἀλλὰ μέλας, μηδὲν τὸ λευκαίνεσθαι ἔτι δυνάμενος, < ἐν δὲ τῷ λευκῷ συνθέματί καιούμενος >\footnote{κατ ᾽ αὐτὰ $\star$. --- οὐδὲν γὰρ λευκὸν] οὐδὲν γὰρ λευκὸς (οὐδὲν corrigé en οὐδὲ) $\star$.} ἀπολευκαίνεται, καὶ ἐξιοῦται πλυνόμενος, ὥσπερ γέγραπται.\footnote{< ἐν δὲ τῷ λ. σ. καιούμ. > restitué d'après $\star$.}

20. Λοιπὸν ἐλευκάνθη καὶ ἐξανθώθη, ὡς εἶπεν Ὀστάνης. « Ἅμα γὰρ, φησὶν, ἐλεύκανας, ἐξάνθωσας. » Καὶ Ζώσιμος λέγει · « Βλέπε μὴ ἀκηδιάσῃς ἐν τῷ καιρῷ τῆς λευκώσεως · δύο γὰρ ἅμα κατ ᾽ αὐτὸν\footnote{ὡς προγέγραπται $\star$.} γίνονται, λεύκωσις καὶ ξάνθωσις · οὐδὲν γὰρ πρῶτον λευκαίνεται καὶ ξανθοῦται ὕστερον, ἀλλ ᾽ ἅμα λευκαίνεται καὶ ξανθοῦται ἀδιαστάτως κατὰ μίαν μονάδα τῆς τρισυποστάτου ταύτης συνθέσεως. Καὶ νῦν\footnote{κατ ᾽ αὐτὸν corrigé en κατ ᾽ αὐτὸ $\star$.} δὲ ἱσταμένης τῆς τριαδικῆς ἐπιδιαιρέσεως · καὶ γὰρ κατὰ μὲν τὴν\footnote{Après συνθέσεως] Réd. de $\star$ : ἥτις καὶ τριαδικὴ ἐπιδιαίρεσις λέγεται · καὶ γὰρ ...} λεύκωσιν, κατὰ μίαν μονάδα συστάσεως, τὰ τρία λευκαίνονται καὶ ξανθοῦνται, κατὰ δὲ τὴν διαιρουμένην τριάδα διΐστανται καὶ ἀποχέονται.\footnote{καὶ γὰρ κάτω mss. Corr. d'après $\star$ qui donne : κ. γ. κατὰ μίαν λεύκ. καὶ κατὰ μ. μ. σ.} Οὕτω γὰρ ἔλεγεν τὸ κατὰ Δημοκρίτου · « Οἰκονόμει δὲ\footnote{διΐσταται καὶ ἀπόχυται mss. Corr. d'après $\star$, qui donne ensuite : οὕτω γάρ φησι καὶ ὁ Δημόκριτος.} ἅλμῃ, ἢ ὀξάλμῃ, ἢ ὡς ἐπινοεῖς. » Καὶ πρῶτον ὑποφωνῶν ὅτι ὁ χαλκὸς\footnote{οἰκονόμοι mss. Corr. d'après $\star$.} οὐ βάπτει, καὶ ὅτι ὁ (f. 176 r.) χαλκὸς νιτρελαίῳ ἀνακαυθεὶς, καὶ τοῦτο πολλακὶς παθὼν, χρυσοῦ καλλίων γίνεται, καὶ ὅπερ ὁ χαλκὸς\footnote{καὶ ὅτι] ἢ ὅτι $\star$.} οὐ βάπτει κατ ᾽ οὐσίαν ἁπλῆν ἐκ τοῦ μένειν, ἀλλὰ βάπτεσθαι\footnote{καλλίων] κάλιον mss. Corr. d'après $\star$. --- ὅπερ] εἴπερ $\star$.} κατὰ σύνθεσιν γινόμενος · πῶς ἢ ἄνευ τῆς συνθέσεως ταύτης,\footnote{Réd. de $\star$ : ἐκ τοῦ μένειν ἀδιαστάτως, ἀλλὰ βάπτεται κατὰ σύνθ. ὀπτώμενος.} καὶ πρὸ τοῦ βαφῆναι τὸν χαλκὸν διὰ τῆς ἐν πυρὶ συνεργείας πυρόντας βάπτειν ; Ἀλλ ᾽ ἐκεῖνος μὲν ἀρκεῖ πρὸς ἔλεγχον, καὶ τὴν\footnote{γινόμενος] F. l. δυνάμενος. --- ἢ] οἱ $\star$. F. l. καὶ.} πρώτην ἐγχείρησιν ἀποτυχία.\footnote{πυρόντας] πειρῶνται $\star$, f. mel. --- ἐκεῖνος] ἐκείνοις $\star$. F. l. ἐκεῖνο.}

21. Ἡμεῖς δὲ κἂν ἐντεῦθεν σημειωσόμεθα ὅτι ἡ διὰ νιτρελαίου\footnote{ἀποτυχία] F. l. ἀποτεύχει, effectue (verbe supposé).} ἀνάκαυσις τῷ φιλοσόφῳ κατ ᾽ ἀντίθεσιν καὶ ἀπόθεσιν καὶ ὑπέμφασιν εἴρηται. Ὥσπερ γὰρ ὁ ἐν κατόπτρῳ διαβλεπόμενος, οὐ σκιὰς βλέπει, ἀλλ ᾽ ὑπεμφάσεις, διὰ τοῦ φαινομένου ψευδοῦς τὸ ἀληθὲς κατανοῶν,\footnote{κἂν ἐντεῦθεν] καὶ ἐντεῦθεν $\star$.} ὅτι < τῷ > διὰ τοῦ νιτρελαίου καθ ᾽ ὑπέμφασιν κεχρημένος ὑποτίθεται\footnote{A mg. σῆ. --- ἐμφάσεις $\star$.} νοεῖν τὸ ἀληθές · ἀντὶ γὰρ τοῦ « ὄξει νίτρου, » τὸ « νιτρελαίῳ » παραλαμβάνεσθαι προσηγορίαν. Καίεται τοίνυν ἐν τῷ λευκῷ συνθέματι\footnote{ὅτι] οὕτως καὶ ὁ διὰ τοῦ νιτρελαίου $\star$.} καὶ ἐξιοῦται καὶ λευκαίνεται, ὄξει νίτρου πλυνόμενον, καὶ ἅμα ἐν τούτῳ ξανθοῦται, ἔξωθεν μὲν λευκαινόμενον, ἔσωθεν δὲ ξανθούμενον.

22. Οὐκοῦν δεῖ καῆναι ἕως μόνον θερμανθῇ, καὶ ἀσφαλίζεσθαι προσήκει, ἵνα μὴ καπνισθῇ · ἐὰν γὰρ καπνισθῆ, ἠφανίσθη.\footnote{περιλαμβάνεται $\star$. F. l. παραλαμβάνεται.} Οὕτως γὰρ ἄφθονος καὶ ἀγαθώτατος ὁ Δημόκριτος πρὸς μὲν ἑκάστην\footnote{καπνισθῇ] καπτισθῆ AKA$\star$. Corr. d'après $\star$.} ἐπιστέλλων φησὶ τὸν σάλλον ( ? ) περὶ τοῦ χαλκοῦ · « Μὴ σφόδρα\footnote{ἀφθώνος A ; ἀφθόνως K. Corr. conj. Réd. de $\star$ : Οὕτω γὰρ ὁ Δημ. ἀφθόνως καὶ ἀγαθῶς πρὸς ἑκάστην ἀποστέλλων φύσιν, τὸν σάλον περὶ τοῦ χαλκοῦ προλέγει καὶ συνίστησι · βλέπε ἵνα μὴ σφόδρα καύσῃς ... --- A mg. σῆ. --- ἀγαθότητος mss.} καύσῃς, ὦ φίλε, ἵνα μὴ τὸ τούτου (f. 176 v.) κάλλος ἀπολέσῃς, < καὶ > εἰς φλόγα πυρὸς μηδέποτε τοῦτο θήσῃς, οὐ συμφέρει γὰρ, ἀλλὰ φεύγει · ἀλλ ᾽ εἰσάγαγε τῷ πυρὶ ὡς ἐν ἡλίῳ σφοδρῷ, καὶ σῶσον αὐτοῦ πᾶσαν τὴν αἰθάλην, καὶ ποιῆσον ὡς λέκιθον ὠοῦ.\footnote{σᾶλλον K.} » Ἐνσημειώμεθα < δὲ > ὅτι διὰ τοῦ λέγειν « μὴ σφόδρα καύσῃς, καὶ εἰς\footnote{πᾶσαν αὐτὴν τὴν αἰθ. $\star$ --- λέκυνθον mss. Corr. d'après $\star$.} φλόγα πυρὸς μηδέποτε θήσῃς, » ὡς ἐξέβαλλεν ἀπὸ τῆς πνοῆς ταύτης\footnote{ἐνσημειούμεθα mss. Corr. conj. --- δὲ add. $\star$. --- διὰ τὸ λέγειν $\star$, f. mel.} πᾶσαν ἐκπύρωσιν καὶ πᾶσαν ἐκφλόγωσιν. Τούτου ἕνεκεν κατασοφιζόμενοι\footnote{θῇς $\star$.} τοῦ πυρὸς καὶ τοῦ πνεύματος, μήποτε γένηται λελυθώτον\footnote{κατασοφιζόμενοι] F. l. κατασφαλιζόμενοι.} ἐκπύρωσις, πηλῷ ὡς λίαν πυριμάχῳ καὶ τετριμμένῳ,\footnote{λελύθωτον A ; λελυθότων K ; λεληθότως $\star$. f. mel.} περιδεύουσιν ἔξωθεν τὰ ὄργανα ἐκ δευτέρου καὶ τρίτου, ἵνα τὴν μὲν πύρωσιν ἐκστρέψωνται, τὴν δὲ θερμασίαν ἐπισπάσωνται · οὐ μόνον < δὲ τῇ >\footnote{τετριμμένῳ] τετρυχομένῳ $\star$. F. l. τετριχωμένῳ.} περιπηλώσει ταύτῃ κέχρηται, ἀλλὰ καὶ διαστάσεις καὶ χώρας,\footnote{ἐκστρέψονται, puis ἐπισπάσονται mss. Corr. d'après $\star$. --- δὲ τῇ add. $\star$.} κατὰ τὰ ὄργανα ἐπιτηδεύει. Οἷον γὰρ ὁ Δημιουργὸς τὸ στερέωμα\footnote{κέχρηνται, et l. suiv. ἐπιτηδεύουσιν $\star$. --- A mg. : Une main.} ἐξ ὑγροῦ ποιήσας διαχωρίζει τὸ ὕδωρ ὑποκάτω τοῦ στερεώματος,\footnote{ὥσπερ γὰρ ὁ Δημ. $\star$.} διάστασιν ἐπιτηδεύει, ἵνα κατὰ τὰ ὄργανα μὴ ἐκπυρωθῇ τὸ σύνθεμα καὶ ἐξαφανισθῇ · Καὶ ἐπείπερ πάλιν τὸν ἥλιον διατρέχειν καὶ\footnote{Après στερεώματος] Réd. de $\star$ : οὕτω καὶ οὖτοι διάστασιν ἐπρτηδεύουσιν ἵνα ...} ὰναβαίνειν πάντα τὰ τρυφερὰ < καὶ > διακαίειν ὡς τὰ τῶν ἐμψύχων σωμάτων, καὶ μυελοὺς καὶ τὰ ἐπιπολάζοντα σώματα,\footnote{Καὶ ἐπείπερ --- διακαίειν] Réd. de $\star$ : ὥσπερ δὲ πάλιν ὁ Δημιουργὸς τὸν ἥλιον διετάξατο πρὸς τὸ διατρέχειν κ. ἀναβ. καὶ π. τὰ τρυφ. διακαίειν.} ἐκπίνειν καὶ διαπνεῖν τὸν ἀέρα διετάξατο, ἵνα διαψυχούμενα διασώζηται τῆς ἐνκαύσεως · καὶ οὕτως ὁ δημιουργὸς νοῦς διανοηθεὶς\footnote{καὶ τὰ ἐπιπολ.] Réd. de $\star$ : καὶ τὸν ἐπιπολ. τοῖς σώμασιν ἀέρα ἐκπίνειν δὲ καὶ διεκπνεῖν, ἵνα διαψ. διασώζωνται ἐκ τῆς ἔγκαύσεως.} ἐν μέσῳ τοῦ ὑπερκειμένου συνθέματος ἡ τοῦ ὑποκειμένου πυρὸς\footnote{καὶ οὗτος mss. Corr. d'après A$\star$. Réd. de $\star$ : οὕτω καὶ ὁ ἀνθρώπινος νοῦς ἐκ τούτων διανοηθεὶς ...} χώρα μεταλαμβάνει, εὐκρασίας τὰ ὑπερκειμένα · ἑκατοντάδες δὶς\footnote{... πυρός χ. μεταλαμβάνει] πυρὸς διετάξατο χώρας ὥστε μεταλαμβάνειν $\star$.} ὀ- (f. 177 r.) κτώ, καὶ τρεῖς τρεῖς δεκάδες καὶ τέσσαρες, πάλιν τὴν\footnote{τὰ ὐπερκείμενα ἀνάρτησιν (l. 3). Réd. de $\star$ : τὰ δὲ ὑπερκ. τοῦ συνθήματος ἑκ. εἰσὶ, δὶς ὀκτὼ καὶ τρὶς τρεῖς δεκάδες καὶ τέσσαρα, ἃ συναριθμούμενα τὴν ἀνάρτησιν κ. τ. λ.} ἀνάρτησιν τοῦ πυρὸς ποιοῦσιν. Διὰ τοῦτο πολλῆς δεῖται τῆς εὐκρασίας,\footnote{καὶ τρεῖς καὶ τρεῖς καὶ δεκάδες A$\star$. F. l. τρεῖς τρισδεκάδες. Cp. ci-dessus, p. 129, l. 1.} ἵνα μὴ καῇ, καὶ τὸ πᾶν ὑγρὸν ἐξαναλωθῇ. Φησὶν γὰρ · « πᾶν ὑγρὸν τῇ βίᾳ τῆς ἐκπυρώσεως ἐξανάλωται.\footnote{διὰ τοῦτο τοίνυν $\star$.} »

23. Σωζομένης τοίνυν πάσης τῆς αἰθάλης τῆς κατὰ τὸ σύνθεμα, καὶ ὡς λέκυνθον γινόμενον, ἐπὶ τὴν μεγάλην καὶ δευτέραν ταριχείαν\footnote{ἐξαναλούται A (ει au-dessus de ου, d'une encre plus pâle) ; ἐξαναλειοῦνται K ; ἐξαναλοῦται A$\star$ et $\star$. Corr. conj.} μετερχώμεθα · τότε γὰρ ἐκστρέφει τὴν φύσιν καὶ τὴν ἐνκεκρυμμένην\footnote{ὡς λέκυθος γινομένης $\star$. F. l. ὡς λεκίθου γινομένου. Cp. p. précédente, l. 5.} ἐντεριώνην ἀποκαλύπτει. Πρὸς τὸν τόπον γὰρ τοῦτον\footnote{τότε γὰρ] ἐν ᾗ $\star$.} διασυνάπτει καὶ ὃ λέγει Στέφανος, « ὅρος φιλοσοφίας ἐστὶν κατάλυσις σώματος, καὶ χωρισμὸς ψυχῆς ἀπὸ σώματος. » Ἀπὸ τούτων τοίνυν ἄγε < καὶ > τὸν Δημόκριτον [δεῖ] λέγοντα · « Οὐδὲν ὑπολέλειπται,\footnote{πρὸς γὰρ τὸν τόπον $\star$.} οὐδὲν ὑστερεῖ, πλὴν τῆς νεφέλης καὶ τοῦ ὕδατος ἡ ἄρσις. » Καὶ Στέφανος πάλιν λέγει · « Οὐδὲν δεῖ γὰρ αὐτῆν ἀφείην ( ? ) ἔνυγρον,\footnote{< καὶ > add. $\star$. --- δεῖ om. $\star$. mel. Cp. Stephanus, p. 205 et 206 : οὐδὲν ἀπολέλειπται jusqu'à ἡ ἄρσις. \emph{Ibid.} p. 217 : οὐδὲν ὑπολείπεται κ. τ. λ.} ἵνα μὴ ἀποφρενωθῇ καὶ δύνῃ ἀφ ᾽ ἡμῶν. Ἀλλὰ αἱροῦμεν ἀπ ᾽ αὐτῆς τὰ ἐπιπολάζοντα ὕδατα, ἵνα ἴδωμεν αὐτῆς τὸ κάλλος, ἵνα θεασώμεθα\footnote{λέγων mss. Corr. d'après $\star$. Cp. Stephanus, p. 207. --- οὐδὲν δεῖ γὰρ] οὐ γὰρ δεῖ $\star$. --- ἀφείην] ἀφεῖναι $\star$ (ἐᾶν Stephanus).} τὴν εὐμορφίαν τοῦ ἀρρήτου κάλλους, τὴν χρυσόθρονον χάριν.\footnote{ἐπιπολάζοντα] περιπολεύοντα Stephanus.} Τί οὖν ἔχει ποιῆσαι ; πῶς ἄρσιν ποιήσομεν τοῦ ὕδατος ; » Εἰ γὰρ τὸ\footnote{A mg. σῆ. --- χρυσόθρονον] mss. Réd. de $\star$ : τοῦ ἀρρήτου κάλλους αὐτῆς, τὴν χρυσόθρονον χάριν φημί. Cp. Stephanus, \emph{ibid.} : ἵνα ἴδωμεν ἡλιόδωρον νεφέλην. (Variantes produites sans doute par l'emploi, dans les manuscrits antérieurs aux nôtres, du signe commun au soleil et à l'or.)} πῦρ ἐναντίον ἐστίν τῇ οἰκονομίᾳ τῶν εἰδῶν · ὡς ἄλλος δὴ, φησὶν,\footnote{ἔχει] ἔχωμεν $\star$ F. l. ἔχοιμεν. --- Εἰ] ἢ mss. Corr. d'après $\star$.} καὶ < εἰ > χωρὶς πυρὸς οὐ καίεται, τί ποιήσομεν ; ἄπυρον τὸ πράγμα καταλειψόμεθα ; Καὶ τίς ἔσται ἀρχὴ [καὶ] τέλος μὴ ἔχουσα, κατὰ\footnote{δὴ] δεῖ mss. Corr. d'après A$\star$. --- ὡς ἄλλοι φασὶ $\star$, f. mel.} τὰς πρακτικὰς ἐνεργείας, μνησθησόμεθα. Τί λοιπὸν (f. 177 v.)\footnote{[καὶ] om. $\star$.} ἔλεγεν ὁ ἡμέτερος φιλόσοφος, ὁ εἰς πάντα πληρέστατος διδάσκαλος, ὁ εὔφρων καθηγητής ; Οὐδὲν γὰρ ἐλλειπές τι τῶν εἰς χρείαν συντεινόντων,\footnote{Mνησθῶμεν οὖν τι λοιπὸν $\star$.} ὃ οὐκ ἐπεκρότησεν τῶν συμπληρούντων αὐτοῦ τὴν ἐπαγγελίαν.\footnote{ἔμφρων $\star$. --- ἔλιπε $\star$.} Διὸ καὶ ἐνταῦθά φησι · « Λαβὼν μόλυβδον, οὐχ ἁπλῶς\footnote{Après ἐπεκρότησεν] κατὰ τὴν add. $\star$.} λέγω, ἀλλὰ τὸν ἡμέτερον, στῆσον αὐτὸν εἰς πλάτος τὸ διπλοῦν,\footnote{διὸ καὶ ἐντ. φησι] φησὶ γὰρ $\star$.} καὶ πρότερον ὅτε εἰς ἔργον λαβόμενος, καὶ δι ᾽ ἐργαλείου ὑποτιθέμενος τὴν ἄρσιν τοῦ ὕδατος ποίει, καὶ σημείωσαι, φησίν · εἰ διαπορεῖς, πορεύου\footnote{Réd. de $\star$ : στῆσον α. εἰς πλ., κατὰ τὸ διπλοῦν · καὶ πρ. μὲν, ἢ ὅταν εἰς ἕργον λάβῃς, καὶ δι ᾽ ἐργ. ὑποτιθῇς τὴν ἄ. τ. ὕ. ποίει, καὶ σημείου ἀεὶ τὰς ἐνεργείας, φησὶν · εἰ δὲ διαπορεῖς.} εἰς Αἴγυπτον, καὶ λαβὼν ἱμάτιον πυκνὸν, πλύνον, ἔκθλιψον τὴν σταφυλήν.\footnote{ποιεῖν mss. Corr. d'après $\star$. --- ἢ διαπορὶς mss. Corr. d'après $\star$.} » Καὶ ἑρμηνεύων Ζώσιμος καὶ αὐτὸς φησίν · « Καὶ λαβὼν\footnote{εἰς Αἴγ. φησι $\star$.} ἅλας, τὸ θεῖον τὸ λευκὸν ἐξιὸν νότισον ὀξεῖ ζώμῳ. » Καὶ Στέφανος\footnote{καὶ ἐρμ. --- φησιν] ὅπερ ἑρμ. ὁ Ζώσ. φ. · λαβὼν $\star$.} λέγει · « Ὅταν ἐν ὕλῃ ποιῇς τὸ σύνθεμα, ὑπερδαπανᾶται.\footnote{ἐξιὼν mss. Corr. d'après $\star$. --- ὄξει mss. Corr. d'après $\star$.} »

24. Ὁ ἄφθονος καὶ ἀνελλιπὴς ἐμὸς Στέφανος ὁ τῶν μυστηρίων ἀποκαλυπτὴς, πρὸς δὲ νεκρὰν τὴν φύσιν · « Λαβὼν τὴν αἰθάλην,\footnote{λέγει] λέγ A : λέγων K. Corr. d'après $\star$. --- ἐν ὕλης ποιεῖς mss. Corr. d'après $\star$. Cp Stephanus, p. 216, l. 23 : ὅτε καὶ τὴν διὰ τοῦ ὕδατος ἄρσιν ἔναυλον ποιήσῃς τὸ σύνθεμα. --- ὑπερδαπανώτας mss. Corr. d'après $\star$.} ἐπίθες ἐν σάκκῳ λινῷ καὶ λίαν πυκνοτάτῳ, καὶ σινίασον ὅλου τοῦ ὕδατος\footnote{Réd. de $\star$ : ... ἀποκαλυπτής · πρὸς δὲ ν. φ. φησίν · λαβὼν ...} · ἡ γὰρ περιουσία θἄττον κατασπασθήσεται · καὶ στήσας ἅλας καππαδοκικὸν\footnote{ἐν σακῇ λίνῳ mss. Corr. d'après $\star$. --- σιν. αὐτὴν ἐξ ὅ. τ. ὕδ.$\star$ F. l. ὅλον τὸ ὕδωρ.} ἴσον νότισον ὀξεῖ ζωμῷ, ἕως γένηται ὡς πηλός · καὶ ἀναξήρανον\footnote{αὐτῆς περιουσία $\star$.} ἀνατρίβων ὀξεῖ νίτρῳ · οὕτω γὰρ ὁ ποιῶν ἐστιν ἀνὴρ τέλειος,\footnote{ὄξει mss. Corr. d'après $\star$. ἕως ἂν $\star$. --- καὶ ἀναξηραίνων, ἀνάτριβε $\star$.} τηρῶν τὰς ὁδοὺς τῶν γραφῶν τὰς καμπύλους, τὰς λοξάς. » Εἴ τι ἄρα\footnote{ὄξει νίτρῳ mss. ; ὄξει νίτρου $\star$. Corr. conj. --- ἔσται $\star$.} λαμβάνοντας αὐτῶν χαριέντους, χαριεστάτας καὶ ἀπλόκους πλάνας,\footnote{εἴ τι] ἥ τι A. Réd. de $\star$ : εἶτα ἐπιλαμβάνων τὰς αὐτῶν χαριέσσας καὶ χαριεστάτας ...} φησίν · « Λαβὼν νίτρον μέρη βʹ, στυπτηρίας στρογγύλης < μέρος >\footnote{F. l. χαριέντως.} αʹ, μίσεως μέρη βʹ, ἅλατος καππαδοκικοῦ μέρη δʹ, βάλλε ἐν ὄξει\footnote{νίτρου $\star$ --- ajouté μέρος avec $\star$.} λίαν δριμυτάτῳ, καὶ ποιῆσον ζωμόν · ἐν τούτοις γὰρ ἀποσκιάσεις τὰ\footnote{εἶτα βάλε $\star$.} πέταλα. Οὗτος ὁ ζωμὸς ἀρχὴ καὶ τέλος ἐδοκιμάσθη.\footnote{λίαν om. $\star$. --- ἀποσκιάσῃς ἂν $\star$ (pour ἀποσκιάσοις ἂν ? ).} »

\bigskip
\centerline{\EightStarTaper}
\centerline{\EightStarTaper\EightStarTaper}
\bigskip

\subsubsection[3. --- 7. ΠΕΡΙ ΤΗΣ ΕΞΑΤΜΙΣΕΩΣ ΥΔΑΤΟΣ ΘΕΙΟΥ.]{3. --- 7. ΠΕΡΙ ΤΗΣ ΕΞΑΤΜΙΣΕΩΣ ΥΔΑΤΟΣ ΘΕΙΟΥ.\footnote{οὗτος δὲ ὁ ζ. $\star$.}}
\paragraph{}
\emph{Transcrit sur} M, f. 112 r. --- \emph{Collationné sur} B, f. 84 v. ; --- \emph{sur} A, f. 82 r.

\bigskip

1. Ἐν τοῖς ὑμετέροις οἴκοις, ὦ γύναι, διὰ τὴν σὴν ἀκοήν ποτε\footnote{Après θείου] BA aj. : τοῦ πήξοντος (πήσοντος A ; πήσσοντος B. Corr. conj.) τὴν ὑδράργυρον.} διατρίβων, ἐθαύμαζον μὲν πᾶσαν τὴν τοῦ παρὰ σοὶ καλουμένου στρούκτορος ἐργασίαν, ἔκπληξιν δέ με ἱκανὴν ἐνέβαλεν ἀντὶ τῶν ἔργων\footnote{ἡμετέροις A.} αὐτοῦ, παρῆν μοι δὲ καὶ τὸν πόξαμον ἐκθείαζειν · καὶ ᾦμεν καὶ τὸν\footnote{με] F. l. μοι. --- θ. πάξαμον BA.} ἴδιον νοῦν ἑκάστου τεχνίτου, ὅτιπερ ὀλίγας ἀφορμὰς παρὰ τῶν προγενεστέρων λαβόντες, κάλλιον αὐτοὶ ἐπετήδευσαν. Ἦν οὖν τὸ εἰς ἔκπληξίν με ὄξαν τοῦτο · ἡ τοῦ ἰθμητοῦ ὀρνιθίου ἕψησις, πῶς πεποσμενον\footnote{ὄξαν] ἄξαν BA. --- ἰθμητοῦ ὀρνιθίου] F. l. ἠθμοῦ τοῦ ὀρνιθείου. --- Cp. l'\emph{Introduction} de M. Berthelot, p. 150, fig. 26. --- πεπωμασμένον BA, f. mel.} ἐκ τῆς αἰθάλης καὶ θέρμης ἑψεῖται, καὶ τῆς τοῦ ζωμοῦ ποιότητος · εἰ καὶ βαφῆς οὐκ ἀμοιρεῖ. Καὶ τοῦτο θαυμάζων ἐπὶ τὸ ἡμέτερον σπούδασμα ὁ νοῦς μὲν ἡνιοχεῖ. Εἰ ἄρα ἐκ τῆς ἀναδόσεως [καὶ] αἰθάλης τοῦ θείου ὕδατος δύναται ἑψεῖσθαι, καὶ χροΐζεσθαι τὸ ἡμέτερον σύνθεμα. Ἐζήτουν δὲ εἴ πού τις (f. 112 v.) ἄρα καὶ τῶν ἀρχαίων τοῦ τοιούτου ὀργάνου μέμνηται · καὶ οὐ παρῆν μοι κατὰ τὸν νοῦν. Ἔνθεν ἀθυμῶν καὶ τὰς σὰς περιβλεπόμενος βίβλους, εὗρον ἐν ταῖς ἰουδαϊκαῖς πλησίον τοῦ τεκνοπαραδότου ὀργάνου καλουμένου τριβίκου,\footnote{τεκνοπαραδότου] τεκνοδότου BA. F. l. τεχνοπαραδότου. Les trois formes sont également inconnues.} καὶ ταύτην τὴν τοῦ ὀργάνου διαγραφὴν. Ἔχει δὲ οὕτως ὡς πρόκειται. Λαβὼν ἀρσένικον, λεύκανον οὕτως · πηλὸν λιπαρὸν ποίησον πλατὺν ὡς σπεκλαρίου σχῆμα λεπτότατον · καὶ τρῆσον λεπταῖς τρώγλαις κοσκινοειδῶς · καὶ ἐπίθες προσαρηρὸς λοπάδιον, εἰς ὃ ἔστω τοῦ θείου μέρος ἕν · εἰς δὲ τὸ κόσκινον, ἀρσένικον ὅσον βούλει · καὶ ἐπιπωμάσας ἑτέρῳ λοπαδίῳ, καὶ περιπηλώσας τὰς συμβολὰς, < μετὰ > νυχθήμερα δύο εὑρήσεις ψιμύθιον. Τούτου ἐπίβαλλε τῇ μνᾷ τὸ τέταρτον,\footnote{F. l. τῆς μνᾶς.} καὶ ἐκφύσα ὅλην ἡμέραν, ἐκ μικροῦ ἐπιβάλλων ἄσφαλτον, καὶ < τὰ > ἑξῆς. Καὶ αὕτη μὲν ἡ τοῦ ὁργάνου κατασκευή.

2. Ἐγὼ δὲ ἐπὶ τὸ ἡμέτερον ἐλεύσομαι, δεικνὺς ἐξ αὐτῆς τῆς γραφῆς ὡς οὐκ ἔστιν [ἐξ αὐτῆς τῆς γραφῆς] λεύκωσις · ἐπεί πως δύο νυχθήμερα ἑψεῖσθαι παρακελεύεται, δυναμένης ὥρας μιᾶς πολὺ θεῖον ἐξατμίσαι. Ἀλλ ᾽ ἐκ τούτου ἀφορμήν σοι δίδωσι νοημάτων\footnote{δίδωμι B ; δίδω μοι A.} · ἐμνημόνευσε δὲ καὶ Ἀγαθοδαίμων ὅτι περ τὸ ἀρσένικον ὅλον\footnote{δὲ] γὰρ BA.} ἐστὶ τὸ σύνθεμα, περὶ οὗ ἐν τῷ ἕκτῳ τῆς ἑψήσεως τῶν κατ ᾽ ἐνέργειαν\footnote{τῶν om. BA, f. mel.} ἰσχυρῶς διέλαβον. Ἐμνημόνευσαν δὲ καὶ ἄλλοι πολλοὶ ἀρχαῖοι τῇδε βουλῇ πολλῇ ἔσω. Πότε ἡ ἀρχὴ τῆς γραφῆς περὶ τοῦ παρόντος\footnote{πολλῇ] πολλῆι B ; πολλὺ A. F. l. πολὺ.} διδάσκει ; φησὶ γὰρ · (f. 113 r.) « Λεύκωσις ἀρσενίκου ποιοῦσα ἐν ἐκτάσει < εἰς > τὸ ἀρσένικον μὴ λευκαινόμενον ἐκτείνεται. » Οὐ δῆτα\footnote{τὸ ἀρσένικον μὴ] τὸ μὴ ἀρσένικον mss. --- οὐ δῆτα ... ] F. l. οὐ δῆτα μὲν Δημοκρίτου < ἤκουσας > εἰπόντος ... Cp. 2, 1, 24.} μὲν Δημόκριτον εἰπόντα ὅτι « ἐὰν πλεονάσῃ τὰ φῶτα, γίνεται ξανθόν · ἀλλ ᾽ οὐ χρησιμεύσει σοι νῦν · λευκάναι γὰρ βούλει τὰ σώματα. »

3. Πῶς δὲ ἄρα ἡλίθιός ἐστίν τις ἀνὴρ ὁ μὴ τὸ πᾶν ἐννοῶν εἶδος τοῦ ἀρσενίκου ; ἢ αἱ τούτου λάμναι, καθὼς ἡ προκειμένη γραφὴ φάσκει,\footnote{τὸ ἀρσίνικον M.} ἐὰν λευκανθῶσιν οὕτως, οὐχὶ κατὰ τὴν ἐπιφάνειαν, ἔσται μόνον λευκὸν, πυρὸς δὲ ὡς μηθὲν φεύξεται · καὶ αὐτὸ καὶ ἡ τούτου ἐπιφάνεια\footnote{φθέγξεται BA.} λευκή. Πῶς δὲ οὐκ ἔστιν ἠλίθιον ἀρσένικον ἐννοεῖν τὸ λευκαινόμενον, ὅπου καὶ ἐπιβάλλειν αὐτὸ ἐκέλευσεν ἡ γραφὴ καὶ ἐκφυσᾶσθαι, οὐδὲν μολύβδου ἔχοντος τοῦ ἀρσενίκου, ἀλλ ᾽ αὐτοῦ διὰ τῆς πυρᾶς ἐξατμιζομένου ; Ὅτι δὲ σύνθεμά ἐστιν μολιβώδη ἔχον, οὐ μόνον\footnote{μολιβδώδη A.} ἐκφυσᾶν παρακελεύεται, ἀλλὰ γὰρ καὶ ἄσφαλτον ἐπιβάλλειν, ἵνα τρόπον\footnote{γὰρ om. BA, f. mel.} τινὰ μολιβώσῃ, καὶ καθάρῃ καὶ λιπάνῃ τὸ πᾶν.\footnote{μολιβώδη M ; μολιβδώση BA. Corr. conj.}

4. Καὶ ὅσα μὲν οὖν ἔνεστι μοι λέγειν εἰς τοῦτο, λέγειν ὑμᾶς\footnote{ἕνεστι] εἰ ἐστι M ; οὗν εἰ om. BA. Corr. conj. --- εἰ τοῦτο M. --- F. l. ἡμᾶς.} ἔστε μάρτυρες. Ἀλλ ᾽ ἐπειδὴ λοιπὸν πολλὰς ἀφορμὰς λαβόντες λοιπὸν\footnote{ἔσται MA. --- ἐπειδὴ λοιπὸν] λοιπὸν om. BA, f. mel.} ἔστε καὶ διδάσκαλοι. Ἀλλὰ τὸ εἰς ἐμὲ ταυτὸν μέχρις ὧδε\footnote{ὧδε] ἐνταῦθα A.} παρακελεύομαι, ἐκδεχόμενος κἀγὼ τοὺς παρ ᾽ ὑμῶν τοῦ τέλους καρπούς. Φησὶν οὖν ἡ γραφὴ ὅτι καὶ εἰς νομίσματα ποιεῖ. Ἔστιν δὲ ὁ τρόπος οὗτος καρκινοειδής.

5. Ὅτι ἐπὶ τοῦ συνθέματος ὀπὴν ἔχει τὸ ὀστράκινον ἄγγος ἀποκαλύπτον\footnote{Transcrit sur A (f.83 r., l. 8 et suiv.) tout notre § 5, qui manque dans MB. --- Ce paragraphe est reproduit dans le morceau 3, 29, 23. Les principales variantes sont rapportées ici et désignées par un astérisque. --- τὸ ἀποκαλύπτον Lb$\star$.} τὴν φιάλην τὴν ἐπὶ τὴν κηροτακίδα, ἵνα περιβλέπων εἰ λευκανθῇ,\footnote{περιβλέπεις Lb$\star$ --- F. l. ἵνα περιβλέπῃς. --- εἰ] ἢ A. Corr. conj.} ἢ ξανθωθῇ. Ἡ δὲ ὀπὴ τοῦ ὀστρακίνου ἄγγους ἐπιπωμάζεται φιάλῃ ἑτέρᾳ, ἵνα μὴ δι ᾽ αὐτῆς ἐκπνεύσῃ καὶ τὸ καρκινοειδὲς αὐτοῦ ἐκφύγῃ, ὅ ἐστι μονοήμερον. Ἐὰν γὰρ ἄλλη ἡ ἕψησις, καὶ ἄλλη ἡ\footnote{ἐὰν γὰρ] F. l. ἐὰν δὲ.} ὄπτησις, δύο καμίνων χρεία, πρῶτον φανῶν ληκυθίων, ἔπειτα κηροτακίδων,\footnote{φανῶ ALb$\star$.} ἢ πηξάδων, ἢ βουκλῶν · ἐὰν καρκινοειδὴς ἡ ὁμοία αὐτῶν\footnote{ἡ ὁμοία] ᾖ ἡ ὁμ. Lb$\star$, mel.} ἑψηθῆναι, ἐπιτιθέντα κηροτακίδων ἐκτείνων, τὰ δὲ ποιοῦν ὡς ἄρρευστον.\footnote{ὥστε ἑψηθῆναι Lb$\star$, mel. --- ἐπιτιθέντα jusqu'à ἄρρευστον (l. suiv.)] Réd. de Lb$\star$ : ἐπιτεθέντα ἐπὶ κηροτ., ἐκτεινόμενα δὲ ποιεῖν ἄρρευστα.} Ἔλεγεν ὁ ἀρχαῖος Ζώσιμος. « Μίαν τάξιν οἶδα ἐγὼ δύο ἔργα\footnote{Mίαν τάξιν οἶδα κ. τ. λ.] Même citation dans Pélage, ci-après, 4, 1, 6.} ἔχουσαν · μίαν μὲν ἵνα ῥεύσῃ διὰ τῆς ῥυτῆς, καὶ δευτέραν ἵνα\footnote{μίαν] πρῶτον Lb$\star$, mel. --- ῥυτῆς] ῥιτῆς A. --- δευτέρα mss. ; δεύτερον Lb$\star$, f. mel. --- Réd. de Lb$\star$ : ξηρανθῇ καὶ ξανθωθῇ ἡ ὑγρότης τοῦ μολ. σῶα καὶ ἀκεραία καὶ ἀκένωτος < ᾖ. >} ξηρανθῇ ὑγρότης μολύβδου ἀκενώτην · πηχθήσεται γὰρ καὶ ξηρανθήσεται αὕτη.\footnote{Ce passage explique le jeu de mots de 3, 6, 2, p. 119 (\emph{M. B.}). --- Après αὕτη, M et B reprennent la suite du texte avec le morceau suivant.} »

\bigskip
\centerline{\EightStarTaper}
\centerline{\EightStarTaper\EightStarTaper}
\bigskip

\subsubsection{3. --- 8. ΠΕΡΙ ΤΟΥ ΑΥΤΟΥ ΘΕΙΟΥ ΥΔΑΤΟΣ.}
\paragraph{}
\emph{Transcrit sur} M, f. 113 v. ;--- \emph{Collationné sur} B, f. 86 r. ;--- \emph{sur} A, f. 83 r. --- \emph{Consulté} E, f. 183 v.

\bigskip

1. Λαβὼν ὠὰ ὅσα βούλει, ἔκζεσον, καὶ κλάσας αὐτὰ, ἔξελε ἅπαν αὐτῶν τὸ λευκόν · τὰ δὲ ὄστρακα αὐτῶν μὴ χρήσῃ. Λαβὼν δὲ ἀγγεῖον\footnote{τὸ λευκόν --- λαβὼν] Réd. de BA : τὸ λευκὸν διὰ τῶν ὀστρακίνων ἀγγείων καὶ τὸ ξανθόν. Λαβὼν ...} ὑελοῦν ἀρσενόθηλυ τὸν καλούμενον ἄμβικα, βάλλε ἐν αὐτῷ τοὺς κρόκους\footnote{ἐν αὐτῷ τὰ λευκὰ ἢ τὰ ξανθὰ σταθμῷ BA.} τῶν ὠῶν σταθμῷ χρώμενος τοιῷδε, τῇ γ° τῶν κρόκων · ἐπίβαλλε ἐκ τοῦ ὀστράκου τῶν ὠῶν κεκαυμένου ὑπάρχοντος κεράτια δύο, μὴ πλεῖον ἢ ἔλαττον, ἀλλὰ καθὼς γέγραπται · εἶτα λειώσας, καὶ λαβὼν ἕτερα ὠὰ, καὶ κλάσας τὰ ὠὰ, βάλλε ἐν τῷ βικίῳ ἅμα καὶ < μετὰ > τῶν\footnote{μετὰ add. BA.} κρόκων τῶν λελειωμένων, ἵνα τὰ ἀκέραια ὠὰ χωννύωνται εἰς τὰ κρόκα · καὶ περιπηλώσας τὸν ἄμβικα καὶ τὸ μαστάριον σὺν τῷ ῥογίῳ ἀσφαλείᾳ πολλῇ, οἰκονομήσας στέατι, ἢ γύψῳ, ἢ προπόλει, ἢ ἐλαιοκονίᾳ, ἢ ὡς\footnote{πρόπολι (tripoli) E.} βούλει, δὸς ὀπτᾶσθαι ἐν ἱππείᾳ κόπρῳ ἢ ὀνείᾳ, ἢ πρισματοκαύστου, ἢ κουκουμοκανδήλης, ἢ οἵᾳ δήποτε συμμέτρῳ θερμασίᾳ, εἴ τι βαστάζει ἡ χεὶρ ἀνθρώπου. Ἔστω δὲ καὶ ὁ τόπος ὅπου δ ᾽ ἂν τὰ ἐργαλεῖα κεῖνται ἀπήνεμος, ἔχων τὰ φῶτα ἀνατολικὰ ἢ νότια, < καὶ > μὴ δυτικὰ, ἢ ἀρκτικὰ,\footnote{καὶ add. E.} ἢ βόρεια, ἢ θρασκικὰ, διὰ τὴν διάψυξιν. Καὶ δὸς ὀπτᾶσθαι ἡμέρας ιδʹ ἢ καʹ, ἕως δ ᾽ ἂν τῶν αἰθαλῶν παύσηται ἡ ἀναγωγή · περιφίμου δὲ τὰς ἁρμογὰς τοῦ ἐργαλείου ἀσφαλῶς, ὅπως ἡ ὀσμὴ φυλαχθῇ · ἐπὰν γὰρ ἐκβῇ, ἀπώλετο ἡ τέχνη · δυσώδης γάρ ἐστιν ἡ ὀσμὴ πάνυ, καὶ αὐτὴ ἡ ὀσμὴ ὑπάρχει ἡ τέχνη.

2. Τὸ μὲν οὖν πρῶτον ἀνερχόμενον ὕδωρ ἐστίν · δεύτερον τάξει\footnote{δεύτερον] le signe de λευκὸν BA ; ἐστι λευκὸν ὡς δάκρυον E. Corr. conj. (\emph{M. B.}).} δακρύου, δύσοσμον, ἄσβεστος μόνη · εἶτα, παυσαμένης τῆς (f. 114 r.) ἀναγωγῆς τοῦ ὕδατος, αἴρεις τὸ ῥογίον ἐν ᾦ ἦλθε τὸ ὕδωρ · καὶ\footnote{Après τὸ ὕδωρ] (ce recipiant [\emph{sic}] ῥοϊον) E 1\textsuperscript{re} main. --- ἐν οἷς M.} περιφιμοῖς ἀσφαλῶς φυλάττων αὐτό. Τὸν δὲ ἄμβικα ἀνακαλύψας φράσσεις τὰς ῥῖνας διὰ τὴν ὀσμὴν, καὶ εὑρήσεις τὰς ἐν τῷ θηλυκῷ πατελλίῳ οὔσας σκωρίας νεκράς. Μὴ ἀπείπῃς δὲ τὸν νεκρὸν εἰς ἀνάστασιν\footnote{πάτῳ σκωρίαν, καὶ μὴ ἀπ. E.} ἐλθεῖν, ἀλλὰ προσδόκα τοῦ ἀπεγνωσμένου τὴν ἀνάστασιν. Εἶτα πρόσμιξον τῇ σποδῷ κρόκα ἕτερα ὠῶν, ὡς ἐπὶ τῆς σαπωναρικῆς τέχνης, καὶ συλλείου τὰ ὑγρὰ μετὰ τῶν ξηρῶν, καὶ βάλλε ἐν ἄμβικι,\footnote{συλλείου τὰ ξηρὰ μ. τῶν ὑγρῶν BA.} καὶ ποίησον ὠς προτέτακται, ἀλλάσσων τὸ δοχεῖον τοῦ ὕδατος, τουτέστιν τὸ ῥογίον. Τοῦτο ποίει ἐπὶ τρὶς, καὶ ὄψει τὸ μὲν πρῶτον\footnote{ἐπὶ τρὶς] ἐκ τρίτου BA.} ὕδωρ λευκὸν ὡς προγέγραπται, ὃ οἱ ἀρχαῖοι ὄμβριον ὕδωρ ἐκάλεσαν, τὸ δὲ δεύτερον ὕδωρ ξανθόχλωρον, ὃ καὶ ῥαφάνινον ἔλαιον εἰρήκασι, τὸ δὲ τρίτον ὕδωρ μελάγχλωρον. Ὀμοίως καὶ αἱ σκωρίαι αἱ ἐν τῷ\footnote{Après μελάγχλωρον] ὃ καὶ κίκινον ἔλαιον ἐκάλεσαν add. A ; ὃ κ. κ. ἔλ. εἰρήκασιν add. E.} πατελλίῳ οὖσαι · εἰς μὲν τὴν πρώτην ἀποκάλυψιν εὑρήσεις τὴν σκωρίαν μελαντέραν, εἰς δὲ τὴν δευτέραν, λευκὴν, εἰς δὲ τὴν τρίτην, ξανθήν. Μετὰ οὖν τὴν πρώτην καὶ δευτέραν καὶ τρίτην ἀνάσπασίν τε καὶ ἀποκάλυψιν, συνενοῖς τῶν τριῶν ἀνασπάσεων τὰ ὕδατα, τουτέστι τὰ ἐν αὐτοῖς ὄντα θεῖα ὕδατα ἐν τῇ σκωρίᾳ τῇ ὑπολιμπανομένῃ\footnote{ἐν τῇ σκωρίᾳ.} ἐν τῇ θηλείᾳ. Καὶ μετὰ ταῦτα, λαβὼν βίκον ὑελοῦν, χάλασον\footnote{θηλείᾳ] Réd. de BA : ἐν τῆ ἐναπολειφθείση τρυγία ἐν τῆ θυεία.} τὰ ὄντα ἐν τῷ ἄμβικι ἐν αὐτῷ, καὶ πωμάσας τὸν βίκον ὄστρακον\footnote{ὀστράκω γεγανομένω ἰσομέτρω BA.} γεγανωμένον ἰσόμετρον τὸ χείλος τῷ βίκῳ, περι- (f. 114 v.) φίμου ἐν ἀσ\footnote{τοῦ χείλους τοῦ βήκου mss. Corr. conj.} φαλείᾳ οἵᾳ βούλει, μάλιστα δὲ πυριμάχῳ πηλῷ τὸ ἄγγος περιχρίων · καὶ ἔασον τοῦτο ἐν βολβίτοις καμίνου ἡμέρας μαʹ, ἵνα, σήψεως γενομένης,\footnote{ἔασον αὐτὸ παρὰ τῷ ἐν β. κ. E.} ἐξομοιωθῇ τῷ βάπτοντι τὸ βαπτόμενον, καὶ κρατήσῃ ἡ φύσις τὴν φύσιν · οὕτως γὰρ τὰ θειώδη ὑπὸ τῶν θειωδῶν κρατοῦνται, καὶ τὰ ὑγρὰ ὑπὸ τῶν καταλλήλων ὑγρῶν.

3. Καὶ μηκέτι φρόντιζε σταθμοῦ, μήτε νεαρὰ ὠὰ ἢ τοὺς κρόκους αὐτῶν, πλὴν τὰ ὑγρὰ μετὰ τῶν ξηρῶν, ὡς προγέγραπται, συλλειώσας, ἔγκρυβε ἐν τῷ βίκῳ. Καὶ μετὰ τὴν μαʹ ἡμέραν ἀποκάλυψον τὸν βίκον, καὶ εὑρήσεις ἐν αὐτῷ σύνθεμα ὁλοπράσινον, τουτέστιν εἰς ἰὸν μετατραπέν. Ὁ γὰρ ἰὸν ποιῶν οἶδεν τί ποιεῖ, καὶ ὁ μὴ ποιῶν < ἰὸν > οὐδὲν ποιεῖ.\footnote{ἰὸν add. BAE.} Μετὰ δὲ τὴν μαʹ ἡμέραν ἄρον τὸν βίκον ἐκ τῆς θέρμης, καὶ ἔασον αὐτὸν ἡμέρας πέντε χωρὶς θέρμης ὁποίας οὖν · καὶ μετὰ τὰς πέντε ἡμέρας ἀνάσπα διὰ τῶν ἀμβίκων ἐπὶ πρισματοκαύστων ἀνθράκων τὸ θειότατον ὕδωρ, ὃ καὶ δεξάμενος οὐ χειρὶ, ἀλλά τινι ὑελίνῳ σκεύει, εἶτα λαβὼν ὕδωρ, βάλλε εἰς τὸν βίκον, ὡς προγέγραπται, καὶ ὄπτα ἡμέρας δύο ἢ τρεῖς · καὶ ἐξελὼν λείωσον, καὶ τίθει ἐν ἡλίῳ διὰ μύακος.\footnote{F. l. δι ᾽ ἄμβικος.} Ἐπὰν δὲ πήξῃ ὥσπερ σαπώνιον, πυρώσας ἀργύρου γ° αʹ, βάλε ἐκ τοῦ πηχθέντος ὕδατος, τουτέστιν τοῦ ξηρίου κεράτια δύο · καὶ ἔσται σοι χρυσός. Ἡ δὲ ποσότης πασῶν τῶν ἡμερῶν τῆς τέχνης εἰσὶν ἡμέραι ριʹ,\footnote{εἰσιν] περιΐσταται εἰς BE ; περίσταται εἰς A.} καθὼς Ζώσιμος καὶ Χριστιανὸς καὶ Στέφανος ἔφασαν.\footnote{Χριστιανὸς]. L'absence de l'article devant ce mot, dans nos mss., donnerait à croire que c'est un nom propre : « Chrétien. »} Ἐγὼ δὲ ἐκ πάντων, ὡς ἡ μέλισσα, καλῶς ἀναλεξά- (f. 115 r.) μενος, καὶ ἐκ πολλῶν ἀνθέων στέφανον πλέξας, ἀνεθέμην τῷ δεσπότῃ μου · ἑξῆς σοι\footnote{ἑξῆς δέ σοι BAE, f. mel.} καὶ τὰ ἐργαλεῖα ὑποθήσομαι οἷά πέρ εἰσιν. Ἔρρωσθε ἐν Χριστῷ τῷ Θεῷ Ἰησοῦ.\footnote{Réd. de BE : ἔρρ. ἐν Χω ιυ τῶ θω ἡμῶν (ἀμήν om. B) ; réd. de A : comme B, puis : πάντοτε, νῦν καὶ εἰς τοὺς αἰῶνας τῶν αἰώνων · ἀμήν.} Ἀμήν.

Suit dans M (f. 115 r.) et dans B (f. 188 r.) une copie du texte 3, 1, 1 (ci-dessus, p. 107). On a donné les variantes de M (M\textsuperscript{2}) ; celles de B sont sans importance, sauf p. 107, l. 4 : μετὰ] ἀπὸ. Titre de ce texte dans MB : περὶ συνθέσεως ὑδάτων.

\bigskip
\centerline{\EightStarTaper}
\centerline{\EightStarTaper\EightStarTaper}
\bigskip

\subsubsection[3. --- 9. ΠΕΡΙ ΤΟΥ ΘΕΙΟΥ ΥΔΑΤΟΣ.]{3. --- 9. ΠΕΡΙ ΤΟΥ ΘΕΙΟΥ ΥΔΑΤΟΣ.\footnote{Titre dans BA\textsuperscript{1. 2.}. : Ζωσίμου τοῦ Πανοπολίτου γνήσια ὑπομνήματα περὶ τοῦ θείου ὕδατος.}}
\paragraph{}
\emph{Transcrit sur} M, f. 188 r. --- \emph{Collationné sur} B, f. 82 r. ;--- \emph{sur} A, f. 80 r. ; (= A ou A\textsuperscript{1}). --- \emph{sur} A, f. 220 r. (= A\textsuperscript{2}) ;--- \emph{sur} K, f. 96 r. ;--- \emph{sur} Lc, page 219.

\bigskip

1. Τοῦτό ἐστι τὸ θεῖον καὶ μέγα μυστήριον, τὸ ζητούμενον · τοῦτο γάρ ἐστι τὸ πᾶν · καὶ ἐξ αὐτοῦ τὸ πᾶν · δύο\footnote{ἐστι τὸ πᾶν] Cp. l'\emph{Introduction} de M. Berthelot, p. 132 et suiv.} φύσεις, μία οὐσία · ἡ δὲ μία τὴν μίαν ἕλκει · καὶ ἡ μία τὴν μίαν\footnote{δὲ] γὰρ BA.} κρατεῖ. Τοῦτο τὸ ἀργύριον ὕδωρ, τὸ ἀρσενόθηλυ, τὸ φεῦγον ἀεὶ, τὸ ἐπειγόμενον εἰς τὰ ἴδια, τὸ θεῖον ὕδωρ, ὃ πάντες ἠγνοήκασιν, οὗ ἡ φύσις δυσθεώρητος · οὔτε γὰρ μέταλλόν ἐστιν, οὔτε ὕδωρ ἀεικίνητον, οὔτε σῶμα · οὐ γὰρ κρατεῖται.

2. Τοῦτό ἐστι τὸ πᾶν ἐν πᾶσι · καὶ γὰρ ζωὴν ἔχει καὶ πνεῦμα, καὶ ἀναιρετικόν ἐστι. Τοῦτο ὁ νοῶν καὶ χρυσὸν καὶ ἄργυρον ἔχει. Ἡ μὲν δύναμις κέκρυπται · ἀνάκειται δὲ τῷ ἐρωτύλῳ.\footnote{ἐρωτύλῳ] Cp. Leemans, \emph{Pap. gr. mus. Lugd. Bat.}, t. 2, p. 155 (pag. 21, l. 34). Voir \emph{Introduction} de M. Berthelot, p. 17.}

\bigskip
\centerline{\EightStarTaper}
\centerline{\EightStarTaper\EightStarTaper}
\bigskip

\subsubsection[3. --- 10. ΠΑΡΑΙΝΕΣΕΙΣ ΣΥΣΤΑΤΙΚΑΙ ΤΩΝ ΕΓΧΕΙΡΟΥΝΤΩΝ ΤΗΝ ΤΕΧΝΗΝ.]{3. --- 10. ΠΑΡΑΙΝΕΣΕΙΣ ΣΥΣΤΑΤΙΚΑΙ ΤΩΝ ΕΓΧΕΙΡΟΥΝΤΩΝ ΤΗΝ ΤΕΧΝΗΝ.\footnote{Dans MB, on trouve, avant ce morceau, le titre : Περὶ φώτων et la phrase : Ἐλαφρὰ φῶτα πᾶσαν τὴν τέχνην ἀναφέρει. Cp. le titre de 3, 52, et son § 2.}}
\paragraph{}
\emph{Transcrit sur} M, f. 115 r. --- \emph{Collationné sur} B, f. 88 r. ;--- \emph{sur} A, f. 89 r. ;--- \emph{sur} K, f. 3 v. ;--- \emph{sur} Lc, p. 223.

\bigskip

1. Παρεγγυῶ τοίνυν ὑμίν τοῖς σοφοῖς, ὅτι ἄνευ τοῦ ὀργάνου τοῦ τὸν χαλκὸν ἀνασπῶντος μετὰ τὸν τεταγμένον τῆς ἰώσεως χαλκὸν πολὺν ὄντα ἢ ὀλίγον, καὶ τῆς μίξεως τῶν λεγομένων δέκα εἰδῶν, ξηρῶν ἢ ὑγρῶν ὅντων, τουτέστι τῶν ὁμοτεριζόντων, μὴ ἐλπίζετέ τι ποιεῖν, ὦ ἄνθρωποι\footnote{ὁμοαιτεριζόντων Lc.} οἵ τινες ἂν εἴητε τοῦ χρυσοῦ χοροῦ, ἢ χρυσέου γένους, ἢ χρυσέας κεφαλῆς\footnote{ἴητε mss. Corr. conj.} παίδων, τουτέστιν ἐρασταὶ τῆς σοφίας, καὶ τῆς λεκιθώδους (f. 115 v.) ὕλης μεθοδευταί. Ἀλλ ᾽ ὅσοι τοῦ ὀστρακίνου χοροῦ ὑμεῖς ἐαυτοὺς μωμήσασθε, καὶ οὐκ ἐμὲ τὸν τοῖς διδασκάλοις ἀκολουθεῖν ἐπειγόμενον\footnote{μωμήσασθε] μιμεῖσθαι BAK ; μιμεῖσθε Lc.} καὶ ταῖς αὐτῶν συγγραφαῖς, καὶ τὰς ἐκείνων δόξας γνωρίσαντα ὑμῖν, καθὼς ἂν ἡ τοῦ θείου λόγου ἡμῖν ἐνήχσεν δύναμις.

2. Τοῦτο τὸ ὕδωρ τὸ δίχρωμον, τὸ λευκὸν καὶ ξανθὸν, μυρίοις\footnote{τοῦτο οὖν τὸ θεῖον ὕδωρ BAK Lc.} κεκλήκασιν ὀνόμασιν. Ἄνευ οὖν τοῦ θείου ὕδατος οὐδέν ἐστιν.\footnote{ἄνευ οὖν ... ] Cp. 3, 21, 1.} Τὸ γὰρ ὅλον σύνθεμα δι ᾽ αὐτοῦ ἀναλαμβάνεται, καὶ δι ' αὐτοῦ ὀπτᾶται, καὶ δι ᾽ αὐτοῦ καίεται, καὶ δι ᾽ αὐτοῦ πήνυται, καὶ δι ᾽ αὐτοῦ ξανθοῦται, καὶ δι ᾽ αὐτοῦ σήπεται, καὶ δι ᾽ αὐτοῦ βάπτεται, καὶ δι ᾽ αὐτοῦ ἰοῦται καὶ ἐξιοῦται καὶ ἑψεῖται. Φησὶ γάρ · « Ἐπιβάλλων ὕδωρ θείου ἄθικτον καὶ κόμμι ὀλίγον, πᾶν σῶμα βάψεις. Ὅσα γὰρ\footnote{ἀθίκτου Lc, f. mel.} ἀπὸ ὕδατος ἔσχον γέννησιν, ταῦτα τοῖς ἀπὸ πυρὸς ἀντιπάσχει\footnote{γένεσιν B etc., f. mel.} · ὥστε ἄνευ τοῦ καταλόγου τῶν ὑγρῶν πάντων, οὐδέν ἐστιν ἀσφαλές. »

3. Ἐμνημόνευσαν δέ τινες, τάχα δὲ καὶ οἱ ὅλοι, ὅτι δεῖ τοῦτο τὸ ὕδωρ ζύμης χάριν καταφθεῖραι τῷ ὁμοίῳ τὸ ὅμοιον τοῦ μέλλοντος βάπτεσθαι σώματος. Ὡς γὰρ ἡ ζύμη τοῦ ἄρτου, ὀλίγη οὖσα,\footnote{ὡς γὰρ ... ] Cp. 3, 21, 3.} τοσοῦτον φύραμα ζυμοῖ, οὕτω καὶ τὸ μικρὸν χρυσίον τὸ πᾶν μέλλει\footnote{χρυσίον] signe pur et simple de l'or et du soleil MAK ; signe avec l'esprit rude et la finale ου (ἡλίου ? ) B ; τοῦ χρυσοῦ Lc. Corr. conj.} ξηρίον ζυμοῦν.

4. Ἄλλοι δὲ, ἀμφότερα μίξαντες τοῖς ὑπολείμμασι τῶν θειωδῶν, χρύσεα χρυσέοις προσέπλεξαν, καὶ τούτων οἱ μὲν τοῖς ὠμοῖς καὶ ἀσήπτοις, οἱ δὲ τοῖς συνεψηθεῖσι τῷ ὕδατι τῆς ἴώσεως.

\bigskip
\centerline{\EightStarTaper}
\centerline{\EightStarTaper\EightStarTaper}
\bigskip

\emph{Après ce morceau, on lit dans} A Lc :

Ἄνω τὰ οὐράνια καὶ κάτω τὰ ἐπίγεια · δι ᾽ ἄρρενος καὶ θήλεως\footnote{ἐποίηα A.} συμπληρούμενον τὸ ἔργον.

\bigskip
\centerline{\EightStarTaper}
\centerline{\EightStarTaper\EightStarTaper}
\bigskip

\subsubsection[3. --- 11. ΣΩΣΙΜΟΥ ΤΟΥ ΠΑΝΟΠΟΛΙΤΟΥ ΓΝΗΣΙΑ ΓΡΑΦΗ ΠΕΡΙ ΤΗΣ ΙΕΡΑΣ ΚΑΙ ΘΕΙΑΣ ΤΕΧΝΗΣ ΤΗΣ ΤΟΥ ΧΡΥΣΟΥ ΚΑΙ ΑΡΓΥΡΟΥ ΠΟΙΗΣΕΩΣ, ΚΑΤ ᾽ ΕΠΙΤΟΜΗΝ ΚΕΦΑΛΑΙΩΔΗ.]{3. --- 11. ΣΩΣΙΜΟΥ ΤΟΥ ΠΑΝΟΠΟΛΙΤΟΥ ΓΝΗΣΙΑ ΓΡΑΦΗ ΠΕΡΙ ΤΗΣ ΙΕΡΑΣ ΚΑΙ ΘΕΙΑΣ ΤΕΧΝΗΣ ΤΗΣ ΤΟΥ ΧΡΥΣΟΥ ΚΑΙ ΑΡΓΥΡΟΥ ΠΟΙΗΣΕΩΣ,\footnote{ἀργύρου] signe du mercure BAK ; signe de l'argent E ; ἀργύρου en toutes lettres Lb.} ΚΑΤ ᾽ ΕΠΙΤΟΜΗΝ ΚΕΦΑΛΑΙΩΔΗ.}
\paragraph{}
\emph{Transcrit sur} A, f. 112 r. --- \emph{Collationné sur} B, f. 118 r. ;--- \emph{sur} K, f. 18 r. ;--- \emph{sur} E, f. 41 r. ;--- \emph{sur} Lb (copie de E), p. 145. --- \emph{Chap. 3 de la compilation du Chrétien dans} E Lb. --- \emph{Sauf indication spéciale, les νariantes de} Lb \emph{peuνent être considérées comme étant communes à ce manuscrit et à son original} E, \emph{dans tous les morceaux que renferment ces deux manuscrits}.

\bigskip

1. Λαβὼν τὴν ψυχὴν τοῦ χαλκοῦ τὴν οὖσαν ἐπάνω τοῦ ὕδατος τῆς ὑδραργύρου, ποίησον σῶμα πνευματικόν · ἀνα- (f. 112 v.) βαίνει γὰρ ἐπάνω ἡ ψυχὴ τοῦ χαλκοῦ ἡ κεκολλημένη ἐν τῇ χώνῃ. Τὸ δὲ ὕδωρ μένει κάτω ἐν τῇ κηροτακίδι, ἵνα παγῇ μετὰ τοῦ κόμμεως χρυσάνθιον, χρυσοζώμιον, καὶ τὰ ἑξῆς. Ἄλλοι δέ φασι περὶ χρώματος καὶ ἑψήσεως\footnote{φησι A. --- ἄλλοι δὲ jusqu'à καὶ τὰ ἑξῆς A mg., E mg. de 1\textsuperscript{re} main, Lb ; om. BK.} καὶ ἔργου μυστικῆς θεωρίας. Ἀρχὴ μέν · ὁ χαλκὸς ἐμβαλλόμενος μετὰ τῆς οἰκονομίας ἐν τῷ ἐργαλείῳ τῆς πράξεως ἐπιδείκνυται ὀμμάτων τέρψιν · ἐν δὲ τῷ χρονίζειν γινομένης ἀπομαυρούσθ < ω ? > μετὰ τοῦ κόμμεως\footnote{ἀπομαυρώσεως Lb.} χρυσῷ σύνθετον, χρυσοζώμιον, καὶ τὰ ἑξῆς. Περὶ εἰσποιήσεως\footnote{χρυσῷ σύνθετον] χρυσάνθιον Lb., f. mel. --- περὶ γὰρ εἰσποιήσεως Lb.} ἔγραφεν ἐν ᾗ καὶ περὶ τῆς πήξεως κηρύττουσι. Καὶ πάλιν ἡ Μαρία\footnote{κηρ. πάντες Lb.} · « Βάλλων ὕδωρ θείου καὶ κόμμι ὀλίγον, θὲς ἐν θερμοσποδιᾷ · οὕτω γάρ φασι παρ ᾽ αὐτοῖς τὸ ὕδωρ πήγνυσθαι. » Καὶ πάλιν ἡ Μαρία · « Ἐν τῷ\footnote{F. l. φησι.} σκευαστῷ χρυσάνθιον · καὶ ἐν τῷ πετάλῳ τῆς κηροτακίδος ἐχέτω, φησὶ, τὸ ὕδωρ τοῦ θείου, κόμμι ὀλίγον, ὅταν παρ ᾽ αὐτοῖς πήγνυται · τούτῳ ἐπ ᾽ ὀλίγον βολβίτοις · μετὰ γὰρ τὸ « ἐπ ᾽ ὀλίγον, » ταῦτα πάλιν ἡ Μαρία\footnote{καὶ τοῦτο ἐπ ᾽ ὀλίγοις βολβ. Lb. F. l. καίε τοῦτο. --- γὰρ] F. l. δὲ.} · « Χαλκοῦ τοῦ ἡμῶν μέρος ἓν, χρυσοῦ μέρος ἓν, ποίει δίχυτον πέταλον καὶ ὑπόθες ἐπὶ τῷ κρεμαστῷ θείῳ καὶ ἔα νυχθήμερα γʹ, ἕως ὀπτηθῇ.\footnote{Interrompu ici la collation suivie de E, ms. corrigé souvent par le copiste de La, Lb, Lc.} »

2. Τοῦτο καὶ ὁ φιλόσοφος διηγεῖται · μετὰ γὰρ τὸ πῆξαι ἐπ ᾽ ὀλίγον βολβίτοις ὀπτοῦμεν τῇ τοῦ θείου ἀγωγῇ αὐτὸ ἡμέρας βʹ ἢ γʹ, ἕως οὗ γένηται ξανθὸν φάρμακον εἰς ὑπερβολὴν, μεταβάλλοντες εἰς ἕτερον ἄγγος, δηλονότι τὸ σύνθεμα. Μετὰ γὰρ τὴν τοῦ ὕδατος τοῦ θείου παρ ᾽ αὐτοῖς πῆξιν ἐν βουκλανίῳ, βαλόντες εἰς ἀγγεῖον, ὀπτοῦσι λαβρῶς ἡμέρας βʹ ἢ γʹ.

3. Πᾶσαι αἱ γραφαὶ ἐκ προβάσεως τὰ φῶτα βούλονται · πρῶτον\footnote{πᾶσαι δὶ αἱ γρ. Lb.} ἐν θερμοσποδιᾷ, ἢ βολβίτοις, ἕως οὗ τὸ ὕδωρ τοῦ θείου παγῇ. Καὶ οὕτως μεταβάλλοντες ἐπὶ τὰς ἡμῶν ὀπτήσεις · πῆξον γὰρ, φησὶ,\footnote{μεταβάλλουσι Lb, f. mel.} καὶ στρέψον καὶ μετάβαλλε βούκλας, καὶ ὄπτα εἱλικτοῖς ἢ διαφόροις\footnote{βούκλας] E mg. : βοκάλι. --- εἱλικτοῖς] ἑλικτοῖς E ; ἐλ. Lb. F. l. ἀλήκτοις. Cp. p. 123, l. 6.} φωσίν. Ἔγωγε κατείληφα ἐν τῷ λευκῷ · ἡμέραν μίαν ὀπτοῦ- (f. 1132) σι\footnote{ἔγ. δὲ κατ. ὅτι Lb.} πρότερον, καὶ τοῦτο πήξαντες ἐπ ᾽ ὀλίγον, οὐ μόνον μετὰ τῆς νεφέλης, ἀλλὰ καὶ ὕδατος θείου.

4. Διὰ τοῦτο καὶ ὁ φιλόσοφος ἐν τῷ καταλόγῳ τῶν ζωμῶν μετὰ παρατηρήσεως εἴρηκεν νεφέλην · καὶ πάλιν θεῖον. Μετὰ οὖν τὸ πῆξαι\footnote{θεῖον] θείου A ; ὕδατος θείου K.} αὐτὸ ἐπ ᾽ ὀλίγον τὴν νεφέλην, καὶ τὸ ὕδωρ τοῦ θείου τὸ ἀπολελυμένον μεταβάλοντες, ὀπτοῦμεν ἡμέραν αʹ, ὡς ἔχει ἐν τῇ λιθαργύρῳ, ἵνα γένηται ψιμμυθίῳ παρεμφερὲς, τοῦτο καθεὶς μετὰ τοῦ φαρμάκου\footnote{καθεὶς] καὶ τοῦτο καθίεμεν Lh.} λείψανον εἰ χρεία χρυσοῦ · εἰ δὲ οὐκ ἐκφυσήσαντες ἠρέμα τὸν μόλυβδον\footnote{εἰ δὲ οὔ Lb.} · δηλαδὴ λειώσαντες τὸ σύνθεμα, καὶ νιτρελαίῳ ἀναλαβόντες, ἢ, ὡς δοκεῖ, ἄρρευστον · ἐκφυσοῦσι μὲν ἔστ ᾽ ἂν ἐκφύγωσι μετὰ τῆς\footnote{ἐκφυσῶμεν Lb.} σκιᾶς τὰ θειώδη. Εἰ δὲ ἐξ ἐλαίου ἐκθειουμένης ἕψοντες ἕως ἄρρευστον,\footnote{Réd. de Lb : Τὴν δὲ ἐξ. ἐλ. ἐκθειουμένην ἕψ. ἕως ἂν ἄρρ. ποιήσωμεν, καὶ ἐκφ. ἔχομεν ...} καὶ ἐκφυσήσαντες ἔχουσι. Καὶ οὕτως φέρομεν ἐπὶ τὴν ξάνθωσιν, λειώσαντες αὐτὴν, καὶ βάλοντες τὰ ξανθῶσαι δυνάμενα ὕδωρ θείου\footnote{αὐτὴνον K.} καὶ κόμμι, καὶ πήγνυμεν μικρὸν τοῖς βολβίτοις. Καὶ πάλιν ὀπτοῦμεν\footnote{δηλαδὴ καὶ κόμμι Lb.} ἡμέρας βʹ ἢ γʹ, ἕως οὗ γένηται ξανθὸν εἰς ὑπερβολὴν, τοῦτο καθιέμενον εἰς τὸ τοῦ φαρμάκου λείψανον ἡμέρας γʹ ἢ εʹ ἢ ζʹ, ἕως οὗ ἰωθῇ. Καὶ ἐπιβάλλομεν ἀργύρῳ, καὶ βάπτομεν χρυσόν. Οὕτως ἔγνωμεν τὴν τῶν φώτων ποσότητα, ὀλίγον ἕως οὗ παγῇ ἡ νεφέλη.

5. Καὶ τὸ ὕδωρ τοῦ θείου τὸ ἀπολελυμένον μετὰ τοῦ μολυβδοχαλκοῦ\footnote{μολύβδου Lb.} μεταβαλόντες ὀπτοῦμεν ἡμέραν αʹ, καθὼς ἔχει ἐν τῇ πρώτῃ τάξει τῶν λευκῶν ζωμῶν, ἀλλὰ καὶ εἱλικτοῖς, καθὼς ἔχει ἐν τῇ\footnote{εἱλικτοῖς] mêmes variantes que l. 3.} λιθαργύρῳ. Τοῦτον εἰ μὲν βουλόμεθα λευκοῦν, οὕτως ἰῶμεν · εἰ\footnote{τοῦτον δὲ Lb. --- εἰ δ ᾽ οὔ Lc.} δ ᾽ οὖν ἐκφυσήσαντες ἐπὶ τὴν ξάνθωσιν, πάλιν φέρομεν τὴν διὰ ὕδατος\footnote{διὰ ὕδ. τοῦ θ. ἀθ. Lb.} θείου ἀθίκτου, καὶ κόμμεως, καὶ πήξαντες τοῖς βολβίτοις μεταβαλόντες, ὀπτοῦμεν ἡμέρας βʹ ἢ γʹ, ἕως οὗ γένηται ξανθὸν εἰς ὑπερβολήν. Καὶ ἐξενέγκαντες, ἰοῦμεν εἰς τὸ τοῦ φαρμάκου λείψανον. Ταύτην κατείληφα τὴν τῶν φώτων ποσότητα.

\bigskip
\centerline{\EightStarTaper}
\centerline{\EightStarTaper\EightStarTaper}
\bigskip

\subsubsection[3. --- 12. ΠΕΡΙ ΤΑ ΥΠΟΣΤΑΤΑ ΚΑΙ ΤΑ Δ ΣΩΜΑΤΑ ΚΑΤΑ ΤΟΝ ΔΗΜΟΚΡΙΤΟΝ ΤΟΝ ΕΙΠΟΝΤΑ.]{3. --- 12. ΠΕΡΙ ΤΑ ΥΠΟΣΤΑΤΑ ΚΑΙ ΤΑ Δ ΣΩΜΑΤΑ ΚΑΤΑ ΤΟΝ ΔΗΜΟΚΡΙΤΟΝ ΤΟΝ ΕΙΠΟΝΤΑ.\footnote{Titre dans BAK : περὶ τῶν ὑποστατῶν καὶ δ ʹ σωμάτων κ. τ. λ. --- Titre dans E Lb : περὶ τῶν ὑποστατῶν δ ʹ σωμάτων κατὰ Δημόκριτον. (accent reporté partout sur la dernière syllabe de ὑπόστατα dans les mss.)}}
\paragraph{}
\emph{Transcrit sur} M, f. 141 v. ;--- \emph{Collationné sur} B, f. 119 v. ;--- \emph{sur} A, f. 113 v. ;--- \emph{sur} K, f. 18 v. ;--- \emph{sur} E, f. 43 (le § 1 seulement) ;--- \emph{sur} Lb, (copie de E), p. 153 ;--- \emph{Plusieurs leçons de} M \emph{sont rapportées en marge de} K. --- \emph{Chap.} 34 \emph{de la compilation du Chrétien dans} E Lb.

\bigskip

1. Τὰ τέσσαρα σώματα ὑπόστατά εἰσιν, καὶ οὐδὲν αὐτῶν φεύγει\footnote{τὰ ὑποστατά (τὰ gratté) M. --- Après σώματα] φησὶν ὁ Δημόκριτος add. Lb.} · ἔνθεν οὐδὲ ἐκφυσᾶν τὸ σύνθεμα ἐμνημόνευσεν. Εἰ γὰρ ἦν χρήσιμον, πάντως ἂν ἐμνημόνευσεν · φησὶ γάρ · « Οὐδὲν ὑπολέλειπται, οὐδὲν ὑστερεῖ. Τοῦτο καὶ εἰς τὸ χρυσοζώμιον « πᾶν σῶμα βάπτει, » τὰ τέσσαρα σώματα λέγων. Διὰ τοῦτο καὶ τὸν διδάσκαλον φάσκει\footnote{φάσκειν M.} λέγοντα · « πάσας τὰς οὐσίας βάπτοντα, » δεικνύων ὅτι οὐδὲν ἐκφυ- (f. 142 r.) σᾶν τάχα οὐδὲ δύναται, ὅτι δὲ καὶ τὰ τέσσαρα ὑπόστατα\footnote{ἐκφ. δεῖ Lb.} καὶ βάπτονται καὶ βάπτουσιν · τὸν Παμμένην εἰσάγει μετὰ τοῦ μολύβδου πεπραχότα ὡς οὐ χρεία αὐτὸν ἐκφυσᾶν. Ἑαυτὸν γὰρ\footnote{πεπρικότα M.} ἐν ταῖς ἑψήσεσιν ἐξατμίζεται, ὅτι αὐτὸς βάπτει, φησὶν ἡ Μαρία, τὴν μολιβδίνην τοῦ μολύβδου. Ἄρον, φησίν · ὅπου ἂν ἔμβῃ βάπτει · ἐμφῆναι καὶ αὐτὴ ἠθέλησεν ὡς οὐ καλῶς τὸν μόλυβδον ἐκφυσῶμεν.\footnote{ἕως οὖ Lb., f. mel.} Τοῖς γὰρ ὀνόμασιν τοῖς ἔξωθεν τῶν τεχνῶν ἐχρήσατο ἐν τῇ αὐτῶν ἐργασία. Οὐχ οὕτως αὐτοὶ ἐργαζόμενοι, ὅταν λέγωσι τὸν ἡμῶν χαλκὸν, ἢ οἱονδήποτε σῶμα ποιεῖ πέταλον, καὶ ποιεῖ δίχυτον. Καὶ ὁ φιλόσοφος\footnote{διάχυτον B, etc. --- (= BAKELb), f. mel.} τοῦτον καθεὶς γενόμενον πέταλον · καὶ δεξάμενον πετάλου τομὴν.\footnote{πέταλον] Le signe de πέταλον partout MA. --- τομήν] · τὸ μήνης BAK.} Καὶ ἑὰν ῥεύσῃ, βέλτιον. Ταῦτα μὲν οὖν λέγουσιν · « Οὐ διὰ πετάλου, ἀλλὰ διὰ ξάνθωσιν ὡς ἀποτεινόμενοι περὶ τῶν ξ ...\footnote{ἀλλὰ διὰ ξ MBAKE. Lu comme Lb. (\emph{M. B.}). --- ξ est un signe inconnu. E Lb ont lu, la première fois : ξάνθωσιν, leçon que nous adoptons, et la seconde fois : τῶν ὑδάτων θαλασσίων, confondant ce signe avec celui de la planche 6, l. 6 (\emph{Introd.} de M. Berthelot, p. 116), et de plus Lb a ajouté τῶν ξανθῶν. --- La seconde fois, lire peut-être περὶ τῶν ξανθῶν (\emph{M. B.})}

2. Οὕτως καὶ ἐὰν λέγωσιν ἐκφυσᾶν, οὐ τὸν ἔξω λέγουσιν,\footnote{Interrompu ici la collation suivie de E.} ἀλλ ᾽ ἐν τῇ ἑαυτῶν ἐργασίᾳ · ἑαυτοῖς γὰρ ἐκφυσῶνται ἑψόμενα, καταλείψαντα τὸ εἰλικρινὲς αὐτῶν καὶ τὸ βαπτικὸν, ἅπερ ἑψόμενα, ἀποβάλλουσι καὶ ἐξατμίζουσι τὰ ἄχρηστα, καὶ ἕτερα ὀνόματα\footnote{ἑτέροις ὀνόμασι Lb.} καλοῦνται καθαρθέντα, ὥστε καὶ ἐκφυσῶνται, καὶ ἕως ᾖ τὸ εἰλικρινὲς αὐτῶν καὶ βαπτικὸν, καίονται ἐν ταῖς ἑψήσεσι καὶ τὰ ἐν ἑαυτοῖς ἐκφυσῶνται πάντα, καταλείψαντα τὸ χρήσιμον καὶ βαπτικὸν πνεῦμα.

3. ΠΕΡΙ ΤΩΝ ΑΥΤΩΝ ΣΤΑΘΜΩΝ ΩΜΩΝ ΤΕ ΚΑΙ ΕΦΘΩΝ.\footnote{Titre du chapitre 35 de la compilation du Chrétien dans E Lb. --- Réd. de Lb : Αἱ γραφαὶ παρεγγυῶσιν ὅτι ὁ μόλ. (d'après les corr. portées dans E).} --- Τῶν γραφῶν περὶ τούτων παρεγγυουσῶν, ἀμέλει οὖν ὁ μόλυβδος ἐκφυσηθεὶς\footnote{παραγνουσων (\emph{sic}) M K mg.} ἀπολείπεται · καὶ τοῦτο ἠνίξατο ἡ Μαρία λέγουσα · « Εὑρήσεις γὰρ μέρη εʹ ὑστεροῦντα μέρους ἑνὸς, δηλονότι τοῦ ἐκφυσηθέντος\footnote{μέρους] μέρος M.} μολύβδου. Ὁμοίως καὶ ἐν τῇ τελείᾳ τῆς ἐκδόσεως τὸν χαλκόν φησιν\footnote{ἐν τῇ τελείᾳ ἐκδόσει Lb.} κατ ᾽ ἐξίωσιν, καὶ χώνευσιν, τὸ τρίτον τοῦ σταθμοῦ ἐλαττοῦται.\footnote{ἐλαττοῦσθαι Lb.} » Τελείας δὲ εἴρηκεν αὐτὰς ὁμοῦ λευκαινούσας καὶ ξανθούσας · τὰ γὰρ θειώδη βάπτουσιν, ἀλλὰ (f. 142 v.), φεύγουσιν. Ὑστερούμεθα γοῦν καὶ τῶν θειωδῶν διὰ τὴν φυγὴν, τάχα δὲ καὶ τῶν βοτανῶν, εἴπερ ὅλως συλλειοῦνται. Τινὲς γὰρ σὺν τῷ ὕδατι τοῦ θείου ἥψησαν αὐτὰ,\footnote{F. l. αὐτὰς.} τὸ ξυλῶδες ἀποβάλλοντες.

4. Οὐ μάτην ὁ Ἀγαθοδαίμων φησὶ « καὶ ἑνούμενα. » ἀλλ ᾽ ἵνα τῷ βάθει τοῦ σώματος τοῦ ἀργύρου προσομιλήσαντα τὴν ἀπὸ τοῦ πυρὸς φθορὰν φυγεῖν δυνηθῶσιν. Στερούμεθα οὖν καὶ τῶν βοτανῶν,\footnote{F. l. ὑστερούμεθα. Cp. p. précédente, l. 20. --- οὖν] δὲ B etc.} μαθόντες τὴν ἀπ ᾽ αὐτῶν ποιότητα, καὶ βαφὴν οὐ λαμβάνοντες.\footnote{καὶ βαφὴν καὶ λαμβ. M ; καὶ οὐ λαμβ. τὴν βαφὴν B, etc.} Αἱ γὰρ ποιότητες μόναι ἐνεργοῦσι · σῶμα γὰρ διὰ σώματος παρελθεῖν ἀδυνατεῖ. Ὁ Ἀριστοτέλης · αἱ ποιότητες δι ᾽ ἀλλήλων παρέρχονται\footnote{Réd. de Lb : Διὸ καὶ Ἀρ. φησίν. --- παρέχονται M.} · καὶ Ἀγαθοδαίμων ὃ καὶ κάτω ἀσώματα τὰ σώματα\footnote{καὶ Ἀγ.] ὁ Ἀγ. δὲ καὶ ὁ Κώμαρις ἀσώματα Lb. --- ὃ καὶ κάτω] F. l. ἄνω καὶ κάτω.} λαμβάνει χρῆσαι πνεύματι χρυσοκόλλης · πνεῦμα δὲ πᾶσι κατάδηλον\footnote{πνεῦμα MBAK. --- Réd. de Lb : χρῆσαι γάρ φασι Lb. --- κατάδηλόν ἐστι ὅτι ὡς ἀσώμ. λαμβάνουσι Lb.} ὡς ἀσώματον λαμβάνων · αἱ αἰθάλαι αὗται πνεύματι ἐοίκασιν\footnote{αἱ αἰθ. δὲ Lb.} · αἰθάλη λευκὴ, ἡ τῆς κινναβάρεως νεφέλη,\footnote{E a traduit par σήψεως le signe de κινναβάρεως ; Lb l'a suivi. De même, l. 17.}
\begin{quotation}
... καὶ πνεῦμα μελάντερον, ύγρὸν, ἄχραντον.\footnote{Vers cité ailleurs (3, 19, 3) comme oracle d'Apollon.}
\end{quotation}
\paragraph{}
Πᾶσα γὰρ αἰθάλη πνεῦμα, καὶ αὐταὶ αἱ ποιότητες αἱ βαπτικαί. Καὶ ὁ θεῖος Δημόκριτος λέγει τὴν λεύκωσιν, καὶ ὁ Ἑρμῆς τὸν καπνὸν εἴρηκεν. Οἱ γὰρ χρήσιμοι αὐτοὶ ἦσαν · παρέλαβον αὐτὰς ἐν\footnote{χρήσιμοι] F. l. χρησμοὶ. Réd. de Lb : εἰ γὰρ χρ. αὗται ἦσαν.} ταῖς οἰκονομίαις, ἀλλὰ δι ᾽ αἰνιγμάτων · διὰ τοῦτο καὶ μυστήριον.\footnote{Après οἰκονομίαις] ἀλλ ᾽ οὐχ οὕτως add. B, etc. --- Réd. de Lb : Διὰ τοῦτο κ. μυστήρια ταῦτα ἔγραψεν εἰς τ. κ. τὸ Ἐὰν.} Ταῦτα ἔγραψα εἰς τὸ κεφάλαιον τοῦ « Ἐὰν ᾖς νοήμων. » Αἰθάλη\footnote{Réd. de Lb : ἡ αἰθάλη δὲ τὸ θεῖον τῶν ἀρσενικῶν καὶ ἡ αἰθ. δὲ ἡ λευκὴ ἐστιν ἡ τῆς σήψεως.} θείου ἀθίκτου, ἀρσενίκου, σανδαράχης, καὶ αἰθάλη λευκὴ κινναβάρεως.\footnote{ἀρσενίκου σανδαράχης] signe de l'arsenic redoublé, dans M, et ἀρσενίκου d'une main du 15\textsuperscript{e} siècle au-dessus du second signe, que nous lisons σανδαράχης comme BAK. Lb a lu ce double signe ἀρσενικῶν} Ὁ Ἀγαθοδαίμων · « Ἀρσενίκου τῷ χρυσίζοντι τοῦτο ψυχῆς\footnote{Réd. de Lb : Ἀγ. δὲ ἀρσενικόν φησι τὸ χρυσίζον τοῦτο εἶναι τὴν ψυχήν.} · δίχα τοῦ παχυτάτου αὐτοῦ καὶ καυστικοῦ, καὶ θἐιῶδες σῶμα ἐάσας, λάμβανε ποιότητα. »

5. Αἰθάλη δὲ πνεῦμα, πνεύματι διὰ τὰ σώματα. Διενήνοχεν οὖν\footnote{Le texte commençant avec notre § 5, et finissant sur les mots ὁ χαλκὸς ὁ ἡμῶν παρ ᾽ αὐτοῖς αἰθάλη, cinquième ligne du § 7, reparaît dans M seul (= M\textsuperscript{2}), à partir de cette ligne, avec des variantes nombreuses, mais sans importance. Le texte des mss. B etc. est généralement conforme à celui de cette reproduction ; toutefois il est plus complet (Cp. l. 21). --- Αἰθάλη δὲ πνεῦμά ἐστι Lb. Cp. p. suiv., l. 4. --- οὖν] δὲ Lb.} ψυχὴ πνεύματος. Ψυχὴν καλεῖ τὴν ἀπ ᾽ ἀρχῆς θειώδη καὶ καυστικὴν\footnote{ψ. δὲ καλεῖ Lb.} φύσιν, ταύτην διὰ πυρὸς προσομιλοῦν τε καὶ καθαιρόμενον τὸ\footnote{ταύτην --- προσομιλοῦν τε] αὕτη γὰρ διὰ π. προσομιλοῦσα Lb.} πνεῦμα σώζει, ἐὰν τεχνικῶς τηρηθῇ · ἀπολέσθαι γὰρ οὐ δύναται. Τοῦτο τὸ χρήσιμον τὸ βαπτικόν · τοιούτῳ δὲ χρὴ εἶναι ἀνθρώπῳ\footnote{Réd. de Lb : τοιοῦτον δὲ χρὴ εἶναι τὸν ἄνθρωπον λεπτὸν τῷ νοΐ.} λεπτῷ τῷ νοῒ, ἵνα ἐπιγνῷ πνεῦμα ἀπὸ σώματος ἐξερχόμενον, κἀκείνῳ\footnote{εἶτα κἀκ. χρήσεται. et l. 7 : ἐπιτεύξεται B, etc.} χρήσηται, καὶ ἐξ ἐκείνου διατηρή- (f. 143 r.) σας ἐπιτεύξηται τοῦ\footnote{καὶ] ἢ M\textsuperscript{2}.} σκοποῦ, δηλαδὴ τοῦ σώματος ἀπολομένου, καὶ τὸ πνεῦμα συναπολέσθαι.\footnote{συναπολεῖται Lb.} Οὐκ ἀπώλετο δὲ, ἀλλὰ τῷ βάθει διέδυ, ποιήσαντος τὸ\footnote{ποιήσ. τινος αὐτὸ τὸ πρ. Lb.} πρᾶγμα.

6. Οἱ δὲ μὴ ἐπιγνῶντες τὸ καλῶς γεγονὸς, κακῶς ὑπέλαβον · οὐδὲν γὰρ ἄλλο ὁρῶσιν, εἰ μὴ σώματα, καὶ ταῦτα καέντα, ἢ τεφρωθέντα · καὶ ὑπολαβόντες τούτων μόνον τὸ ὁρώμενον, ὥσπερ ζημιωθέντες οἱ\footnote{καὶ om. M\textsuperscript{2} B. etc. --- ζημ. τι M\textsuperscript{2}.} ἀποτυχόντες τὰ πάντα σφετερίζουσιν · καὶ οὐδ ᾽ οὕτω φεύγουσιν τοῦ\footnote{καὶ om. M\textsuperscript{2} --- F. l. φεύγουσιν καὶ τεφροῦνται (leçon de M\textsuperscript{2}).} τεφροῦντος · οὐδαμοῦ γὰρ τῶν γραφῶν εἴρηταί τι ὑπόστατον, εἰ μὴ\footnote{Après τεφροῦντος] Addition de M\textsuperscript{2} B, etc. : ἡ δὲ ποιότης μόνη μετὰ τοῦ χαλκοῦ παραμένει · ἐκεῖνος γὰρ μόνος ἄφευκτος < καὶ add. L. > ὑπόστατος. --- εἰ μὴ μόνον τὸν χαλκὸν Lb.} ἐκεῖ μόνος ὁ χαλκὸς ὃν ἡ Μαρία λέγει οἰκονομεῖσθαι χαλκὸν καὶ\footnote{ἐκεῖ om. M\textsuperscript{2} B, etc. --- χαλκὸν om. M\textsuperscript{2} B etc., f. mel.} ὕστερον καίεσθαι · καὶ ἔσται ὑποστατικός. Οὕτως ὁ τῆς ἐργασίας ἡμῶν χαλκὸς ἢ ἄργυρος · οὔτε γε ποιότητα ἐξ αὐτῶν βουλόμεθα λαβεῖν · τὸ δὲ σῶμα αὐτῶν θνητὸν ἄχρηστον · οὔτε γὰρ βοτάναι · πυρὶ γὰρ\footnote{δὲ] γὰρ M\textsuperscript{2} B, etc. ---  γὰρ om. M\textsuperscript{2}.} εἰώθασιν δαπανᾶσθαι.

7. Ὁ Ἀγαθοδαίμων λέγει · « Μαγνησία καὶ στίμι καὶ λιθάργυρος\footnote{λέγει] φησὶ M\textsuperscript{2} B, etc. --- μαγνησία jusqu'à αἰθάλας om. M\textsuperscript{2} seul.} φεύγουσιν, τὸ εἰλικρινὲς καταλείψαντα. » Ἡ Μαρία · « Ἐκφύσα, φησὶν, αἰθάλας ἕως ἐκφύγωσιν μετὰ τῆς σκιᾶς τὰ θειώδη, καὶ γένηται χαλκὸς ἀσκίαστος. » Οὕτως ὁ χαλκὸς ὁ ἡμῶν παρ ᾽ αὐτοῖς, αἰθάλη · αἰθάλη δὲ πνεῦμα · πνεῦμα δ ᾽ ἐστὶ τὸ τοῦ σώματος. Διενήνοχεν\footnote{Αἰθάλη δὲ πνεῦμα, κ. τ. λ. (lignes 4 à 25)] Voir la note, p. 151, l. 1.} οὖν ψυχὴ πνεύματος. Ψυχὴν καλεῖ τὴν ἀπ ᾽ ἀρχῆς θειώδη καὶ καυστικὴν φύσιν, ταύτην διὰ πυρὸς προσομιλοῦν τε καὶ καθαιρόμενον τὸ πνεῦμα σώζει, ἐὰν τεχνικῶς τηρηθῇ · ἀπολέσθαι γὰρ οὐ δύναται. Τοῦτο τὸ χρήσιμον τὸ βαπτικόν. Τοιούτῳ δὲ χρὴ εἶναι ἀνθρώπῳ λεπτῷ τῷ νοΐ, ἵνα ἐπιγνῷ πνεῦμα ἀπὸ σώματος ἐξερχόμενον, κἀκείνῳ χρήσηται, ἢ ἐκεῖνο διατηρήσας ἐπιτεύξηται τοῦ σκοποῦ, δηλαδὴ τοῦ σώματος ἀπολλομένου, καὶ τὸ πνεῦμα συναπολέσθαι. Οὐκ ἀπώλετο δὲ, ἀλλὰ τῷ βάθει διέδυ, ποιήσαντος τὸ πρᾶγμα.

 

8. Οἱ δὲ μὴ ἐπιγνῶντες τὸ καλῶς γεγονὸς, κακῶς ὑπέλαβον · οὐδὲν γὰρ ἄλλο ὀρῶσιν, ἢ μὴ σώματα, καὶ ταῦτα καέντα, καὶ τεφρωθέντα ὑπολαβόντες τούτων (f. 143 v.) μόνον τὸ ὁρώμενον, ὥσπερ ζημιωθέντες τι οἱ ἀποτυχόντες τὰ πάντα σφετερίζουσιν · οὐδ ᾽ οὕτω γὰρ φεύγουσιν τε καὶ τεφροῦνται · ἡ δὲ ποιότης μόνη μετὰ τοῦ χαλκοῦ παραμένει · ἐκεῖνος γὰρ μόνος ἄφευκτος ὑπόστατος · οὐδαμοῦ γὰρ τῶν γραφῶν εἴρηταί τι ὑπόστατον, εἰ μὴ μόνος ὁ χαλκός · Μαρία λέγει οἰκονομεῖσθαι καὶ ὕστερον καίεσθαι · καὶ ἔσται ὑποστατικός. Οὗτος ὁ τῆς ἐργασίας ἡμῶν χαλκὸς ἢ ἄργυρος · οὔτε γὰρ ποιότητα ἐξ αὐτῶν βουλόμεθα λαβεῖν · τὸ γὰρ σῶμα αὐτῶν θνητὸν ἄχρηστον, οὔτε βοτανῶν ποιότητα · πυρὶ γὰρ εἰώθασι δαπανᾶσθαι. Ἀγαθοδαίμων φησὶν --- --- ἕως οὗ ἐκφύγωσιν μετὰ τῆς σκιᾶς τὰ θειώδη,\footnote{Cp. p. 151, l. 21.} καὶ γένηται ὁ χαλκὸς ἀσκίαστος. Οὕτως ὁ χαλκὸς ὁ ἡμῶν αἰθάλη.\footnote{Fin de la répétition dans M.}

9. Τὰ σταθμὰ ἀπεσιώπησεν ὁ Δημόκριτος · φησίν · « Οὺδὲν ὑπολέλειπται, οὐδὲν ὑστερεῖ πλὴν τῆς νεφέλης καὶ τοῦ ὕδατος ἡ ἄρσις. Εἰ δὲ ὅπερ ἔλεγεν καὶ περὶ σταθμῶν · καὶ θείου σταθμὸν πεποίηται\footnote{εἰ δὲ ὅπερ ἔλεγεν] τοῦτο δὲ ἔλεγε Lb. F. l. ἔλεγον. --- περὶ σταθμῶν gratté dans M et corrigé par le copiste en περισταθμὸν. --- καὶ θείου] και θείων BAK ; καὶ ἐκ τῶν θείων Lb. --- πεποίηνται B, etc.} ἐν τῇ ὑστέρᾳ τάξει · καὶ τὸν λευκὸν ζωμὸν ἀρσενίκου γ° αʹ, » καὶ τὰ ἑξῆς. Δύο γὰρ συνθέματα θείων καὶ οὐσίαι τῶν οὐσιῶν · καὶ ἄλλαι αἱ οὐσίαι καὶ τὰ μέταλλα ἐν τῷ θείῳ, καί γε καὶ τὰ ὅμοια,\footnote{αἱ om. B etc., f. mel. --- καί γε] καὶ νεφέλῃ (γε lu νε ? ) E, corrigé en ἐν τῇ νεφέλῃ, (leçon de Lb.)} πλὴν πάντα ἀπολειφθέντα, χαλκὸς εὑρεθήσεται ποιωθεὶς, ὡς φύσιν\footnote{πάντα · ἀπολειφθὲν, χαλκὸς M.} ἔχων συγγαμεῖσθαι, καὶ συγκρατεῖται, καὶ συντέρπεται · καὶ τοῦτο\footnote{συγκρατεῖσθαι καὶ συντέρπεσθαι B etc.} · « Η φύσις τὴν φύσιν τέρπει. » Πάντα γὰρ τὰ σώματα λαβὼν ὁ ἄργυρος οὐκ ἐλαύνεται, εἰ μὴ ὁ χαλκὸς, καὶ τοῦτο μόνον δέχεται,\footnote{καὶ τοῦτο] οὗτος γὰρ Lb.} ὥσπερ ἵππος ὄνον, καὶ κύων λύκον, καὶ ὅσα κατὰ τὸν αὐτὸν καιρὸν\footnote{αὐτὸν] ταὐτὸν M.} τὰ ὅμοια φυσικὰ πάσχουσιν. Καὶ γὰρ ἰώθη ὁ χαλκὸς, καὶ ἀνεξιώθη\footnote{ἀνεξιώθη] ἐξιώθη B etc.} · καὶ οὐκ ἀπαλλάττεται τῆς ἑαυτοῦ φύσεως. Ὁ Δημόκριτος\footnote{ὁ Δημ. δὲ Lb.} ἐν τῇ τάξει τῆς μαγνησίας · « Ἡ γὰρ μαγνησία λευκανθεῖσα οὐκ ἐᾷ ῥήγνυσθαι τὰ σώματα, οὐδὲ τῇ σκιᾷ τοῦ χαλκοῦ ἐπιφαίνεσθαι. » Καὶ\footnote{καὶ οὕτως ἀπεδ. Lb.} ἀπεδώκαμεν τὸν περὶ σταθμῶν λόγον. Ἔρρωσο.

\bigskip
\centerline{\EightStarTaper}
\centerline{\EightStarTaper\EightStarTaper}
\bigskip

\subsubsection{3. --- 13. ΠΕΡΙ ΔΙΑΦΟΡΑΣ ΧΑΛΚΟΥ ΚΕΚΑΥΜΕΝΟΥ.}
\paragraph{}
\emph{Transcrit sur} M, f. 144 r. --- \emph{Collationné sur} B, f. 123 r. ;--- \emph{sur} A, f. 115 v. ;--- \emph{sur} K, f. 20 r. ;--- \emph{sur} E, f. 47 r. ;--- \emph{sur} Lb (\emph{copie de} E), p. 169. --- \emph{Chap.} 36 \emph{de la compilation du Chrétien dans} E Lb. --- \emph{Ce texte, dans son entier, forme le} § 1 \emph{du morceau} 3, 46. \emph{Nous le donnons ici aνec les principales νariantes de ce morceau, désignées par un astérisque}.

\bigskip

Χαλκὸν κεκαυμένον ποιοῦσιν πολλοὶ διὰ θείου, ὡς αἱ τάξεις τῶν\footnote{πολλοι] τινὲς $\star$.} ἄλλων λέγουσιν ἀσαφῶς · μόνος δὲ Δημόκριτος ἀφθόνως. Τῷ χαλκῶ\footnote{Réd. de Lb : μόνος δὲ ὁ Δημ. ἀφθ. τῷ χαλκῷ ἐπιβάλλει τὴν λευκὴν λιθάργυρον θειωθεῖσαν τουτέστι χωνευθεῖσαν μετὰ τοῦ τετάρτου τοῦ μαγν. ἢ θείου τοῦ ἡμίσεως.} ἐπιβάλλειν τὸν δʹ σίδηρον θειωθέντα, τουτέστιν χωνευθέντα μετὰ τοῦ\footnote{ἐπιβαλλ ʹ B ; ἐπίβαλλον A ; ἐπιβάλλον (ω sur ο) K ; ἐπίβαλον $\star$ ἐπιβάλλων corr. en ἐπιβάλλει E. F. l. ἐπίβαλλε. --- τὸ τέταρτον ἢ σιδ. $\star$. F. l. τὸν Δ (= λευκὸν) σίδ.} μαγνήτου τὸ δʹ, ἢ θείου ἀθίκτου ἥμισυ, ἵνα ῥεύσῃ ἐπὶ τὸν μολύβδον\footnote{τοῦ μολύβδου τοῦ $\star$.} τὸν ἀπὸ στίμμεως καὶ λιθαργύρου · ἔπειτα πυρίτην, χαλκὸν, σίδηρον\footnote{Réd. de Lb : ἔπειτα σὺν τῷ πυρίτῃ ἡ χαλκολιθάργυρος.} κάῃς, ἵνα πρεπόντως γένηται σκωρίδιον. Τούτῳ ἐπίβαλλε νεφέλην\footnote{κάῃς] καίεται B etc. --- τοῦτο M ; τούτου $\star$ ; ταύτῃ Lb.} τὴν ἀπὸ ἀρσενίκου. Λευκαίνεται δὲ διὰ τοῦ θείου ἀθίκτου ἡ νεφέλη.\footnote{τὴν νεφέλην M.} Ὅταν δὲ λέγῃ ψιμύθιον ἅμα θείῳ ὀπτηθὲν, τὸ θεῖον ἄθικτον δηλοῖ,\footnote{ψιμύθιον --- λέγῃ om. $\star$.} ἵνα γένηται χαλκὸς, μόλυβδος, ἐτήσιος · ὅταν δὲ λέγῃ · « τὸ δὲ\footnote{χαλκομόλυβδος Lb. --- αἰτήσιος E (par correction) Lb.} αὐτὸ ποιεῖ καὶ μαγνησία λευκανθεῖσα, » κιννάβαριν συνοικονομηθεῖσαν\footnote{κιννάβαριν] σῆψιν Lb. Cp. p. 150, l. 10. note.} ἔλεγεν. Ἀλλ ᾽ ἐρεῖ τις · μαγνησίαν πρῶτον εἴρηκεν, καὶ πυρίτην.\footnote{μαγνησίαν] μέγα $\star$ (Confusion causée par le signe commun μ\textsuperscript{γ} de mss. antérieurs).} Ναὶ, ἵνα μάθῃς ὅτι ἅμα τῷ χαλκῷ σίδηρος καὶ ὁ μόλυβδος βάλλεται,\footnote{ναὶ om. $\star$. --- σίδηρος] ἡ λιθάργυρος Lb.} καὶ οἱ λίθοι, ἵνα γένηται χαλκὸς, μόλυβδος, ἐτήσιος χαλκός.\footnote{Add. de $\star$ : ὅ τι τὸ ἀπ ᾽ αἰῶνος ζητούμενον ὠόν.}

\bigskip
\centerline{\EightStarTaper}
\centerline{\EightStarTaper\EightStarTaper}
\bigskip

\subsubsection{3. --- 14. ΠΕΡΙ ΤΟΥ ΟΤΙ ΠΑΝΤΩΝ ΤΩΝ ΥΓΡΩΝ ΤΟ ΘΕΙΟΝ ΥΔΩΡ ΚΑΛΟΥΣΙΝ · ΚΑΙ ΤΟΥΤΟ ΣΥΝΘΕΤΟΝ ΕΣΤΙΝ, ΚΑΙ ΟΥΧ ΑΠΛΟΥΝ.}
\paragraph{}
\emph{Transcrit sur} M, f. 144 r. --- \emph{Collationné sur} B, f. 123 r. ;--- \emph{sur} A, f. 116 r. ; (A ou A\textsuperscript{1}) ;--- \emph{sur} A, f. 242 v. (A\textsuperscript{2}) ;--- \emph{sur} E, f. 47 v. ;--- \emph{sur} Lb (\emph{copie de} E), p. 173. --- A\textsuperscript{2} \emph{ne contient que le} § 1 \emph{jusqu'à la ligne} 5 \emph{de la} page 155. --- \emph{Chap.} 37 \emph{de la compilation du Chrétien dans} E Lb (\emph{non numéroté dans} E).

\bigskip

1. Τὴν προγεγραμμένην νεφέλην ἕψει ἐλαίῳ · ἡ προγεγραμμένη νεφέλη ὅλον τὸ σύνθεμα · ἔοικεν γὰρ τὸ ὕδωρ τοῦ θείου καὶ ἔλαιον\footnote{Après τὸ σύνθεμα] ἐστὶ A\textsuperscript{2} ; ἔχει Lb. --- τὸ θεῖον A\textsuperscript{2}.} λαμβάνειν. Μετὰ ὅλων δὲ τῶν ὑγρῶν οἰκονομοῦσιν, ἔνυγρον αἰνισσόμενοι · πρῶτον γὰρ ὀξάλμῃ, εἶτα ἐλαίῳ, εἶτα μέλιτι καὶ γάλακτι, ὕδωρ θεῖον αἰνίσσονται · ἀλλὰ καὶ ὁ κρόκος καθ ᾽ ἑαυτὸν ἀδυναμεῖ, εἰ μὴ διὰ τοῦ σκεύους τοῦ θείου ὕδατος · καὶ οἱ βαφεῖς οὕτω χρῶνται\footnote{οὕτω] αὐτὸ A\textsuperscript{2}, f. mel.} · Καὶ Μαρία · « λύσιν κομάρεως καὶ ἐλυδρίου. » Καὶ Δημόκριτος ἐν\footnote{καὶ ἡ Mαρία B etc. ; καὶ μηούσιν λείωσιν κωμ. A\textsuperscript{2}. --- Après ἐλυδρίου] καλεῖ add. Lb.} τῇ ὑστέρᾳ τάξει τῶν λευκῶν ζωμῶν · « Ὕδωρ ἀσβέστου στακτικῆς διὰ τοῦ ῥυτοῦ στάζον, ἢ δι ᾽ ὑλιστῆρος. » Ταριχεύονται τὰ εἴδη πάντα διὰ τῶν ἁπλῶν ὑγρῶν · καὶ τὰ ἐνδεχόμενα πλύνεται · πλύνονται δὲ οἷον τὰ\footnote{Après ὑγρῶν] Réd. de A\textsuperscript{2} : ψήνεται ἐν χωνεία πλυνόμενα. πλύναι τὰ στερεὰ σώματα καὶ ταριχεύονται (Fin dans A\textsuperscript{2}). --- οἷον] ὥσπερ Lb.} στερεὰ σώματα · ταριχέυονται δὲ, ἢ λειούμενα, ἢ βρεχόμενα, καὶ τὰ ἐνδεχόμενα (f. 144 v.) ἡλίῳ καὶ δρόσῳ λειοῦνται, ὡς τὸ λευκὸν θεῖον ἡ λιθάργυρος · ταριχεύονται περὶ τὸν ἀριθμὸν οἷα ἡμέραν αʹ ἢ γʹ ἢ εʹ ἢ ζʹ,\footnote{ἡ] F. l. ἢ. --- ταριχ. δὲ Lb. --- περὶ τὸν ἀριθμὸν [en toutes lettres dans les mss.]. F. l. περὶ τοῦ ὄξους. Les signes de ἀριθμός et l'un de ceux de ὄξος sont presque semblables. Voir dans l'Introduction de M. Berthelot, p. 110 et 116, les notations alchimiques, pl. 3, l. 4 et pl. 6, l. 5.} [ἕως] τοῦτο ἐπὶ πάσης λειώσεως.\footnote{ἕως] καὶ E, f. mel. ; om. B etc. --- τοῦτο δὲ ποίει ἐ. π. λ. Lb.}

2. Ταριχευθέντων οὖν αὐτῶν, συμμίξεις ποιήσεις καὶ συλλειοῖς ἐν δρόσῳ καὶ ἡλίῳ. Καὶ ἀναξηράνας καὶ συλλειώσας αὐτοῖς νιτρελαίῳ\footnote{αὐτὰ Lb, f. mel.} κατάσπα, καὶ εὑρήσεις μέλανα μόλυβδον. Τοῦτον λύε, ἀναλάμβανε\footnote{λύει M ; λείου B etc. Corr. conj.} ὑδράργυρον καὶ ὕδωρ θεῖον καὶ κόμμι, καὶ ὄπτησον ἐλαφροῖς φωσὶν,\footnote{κόμμι] κομίδι M.} ἕως ἄν ἀπόθηται τὸ ὕδωρ, καὶ λύεις ἐν ἡλίῳ, ἕως οὗ λευκανθῇ καλῶς.\footnote{λείεις A ; λειώσεις Lb. F. l. λειοῖς.}

3. Τοῦτο πολλάκις ποιοῦσιν βαπτίζοντες τὸ σκωρίδιον. Καὶ Πηβίχιος · « Κατάβαπτε δὶς ζʹ καὶ δὶς ὀκτὼ ἐπὶ ὀκτὼ καὶ ἐπιπλέω.\footnote{ἐπιπλέον B etc.} » Καὶ Δημόκριτος, τὸ αὐτὸ ποιῶν ἐν τῇ ὑστέρᾳ τάξει τῶν\footnote{ποιεῖ B etc.} λευκῶν ζωμῶν, εἰς τοῦτο πόρον καταβάπτει καὶ τὰ ἔνσκια πέταλα,\footnote{πόρον] γὰρ Lb ; om. B etc. F. l. εἰς τοῦτον πόρον.} καὶ ἀποσκιώσεις ποιεῖ. Καὶ ἀναξηράνας εἰ ἔστιν ἀσκίαστος, ἀναλάμβανε\footnote{καὶ ἀναξηράναντες MBAK ; σὺ δὲ ἀναξηράνας Lb. F. l. ἀνεξερευνήσας. --- εἰ] ἥ M.} νεφέλην, βάλλε τὰ ξανθῶσαι δυνάμενα ὕδωρ θεῖον, καὶ κόμμι,\footnote{ὕδωρ θεῖον] en signe M ; μετὰ ὕδατος θείου B etc.} πῆξον ἐλαφροῖς φῶσιν · Ὅταν πήξῃς, μεταβαλὼν\footnote{καὶ πῆξον B etc. --- [ἡμ. --- γʹ] om. B etc.} [ἡμέρας βʹ ἢ γʹ] καταρρεῦσαι ποίησον εἰς τὸ τοῦ φαρμάκου λείψανον ἡμέρας βʹ ἢ γʹ ἢ ζʹ μαʹ. Τούτῳ ἐπιβάλλεις ἄργυρον κοινὸν, καὶ βάπτεις. Ἑξῆς δὲ\footnote{τοῦτο M ; καὶ τοῦτο Lb. --- ἐπιβαλείς Lb. --- ἄργυρον] en signe M ; ἀργύριον B etc.} καὶ περὶ τῶν καιρῶν ζητήσωμεν.

\bigskip
\centerline{\EightStarTaper}
\centerline{\EightStarTaper\EightStarTaper}
\bigskip

\subsubsection{3. --- 15. ΠΕΡΙ ΤΟΥ ΕΝ ΠΑΝΤΙ ΚΑΙΡΩ ΑΡΚΤΕΟΝ.}
\paragraph{}
\emph{Transcrit sur} M, f. 144 v. --- \emph{Collationné sur} B, f. 124 r. ;--- \emph{sur} A, f. 116 v. ;--- \emph{sur} K, f. 20 v. ;--- \emph{sur} E, f. 48 v. ;--- \emph{sur} Lb, p. 177. --- \emph{Les variantes de} M, \emph{par rapport à} BAK, \emph{ont été reportées en marge de} K. --- \emph{Chap.} 38 \emph{de la compilation du Chrétien dans} E Lb.

\bigskip

1. Ἀναγκαῖον καὶ περὶ καιρῶν ζητήσωμεν. Τὸ πνεῦμα ἔλεγεν, φησὶν, ἀπὸ ἄνθους ἡλιοῦσθαι καὶ ταριχεύεσθαι ἕως τοῦ ἔαρος · καὶ τότε λοιπὸν ἐν παντὶ καιρῷ πυρὸς, ὁ χρυσὸς εἰς τὸ χρῆσθαι. Ὁ γὰρ\footnote{χρυσὸς] signe de l'or ou du soleil MBAKE ; ἥλιος en toutes lettres Lb. --- Lu χρυσὸς (\emph{M. B.}). --- F. l. πυρρὸς ὁ χρ. --- χρᾶσθαι M.} μέγας, φησὶν, ἥλιος ποιεῖ τοῦτο, ὅτι δι ᾽ αὐτοῦ, φησὶν, γίνεται. Ἄκουε τοῦ Ἑρμοῦ λέγοντος ὅτι ἡ μάλαξις τῶν ἀλαξίμων γίνεται ἐν ψυχροῖς.\footnote{ἀλλαξίμων AKELb.} Περὶ τούτου ἰσχυρῶς διέλαβεν ἐν τῷ τέλει τῆς λευκώσεως τοῦ μολύβδου\footnote{τῆς λειώσεως (λευκώσεως E) καὶ τῆς σκευάσεως τοῦ μολ. Lb.} · ἐκεῖ καὶ περὶ τοῦ χρυσοῦ λέγει · οὕτως πως ὁ ποιῶν τὸ πᾶν · ἐκεῖ καὶ περὶ τοῦ ἠθμῆσαι τὸ πᾶν διέλαβεν ὅν τινα ἠθμόν · οὔτε Ἀγαθοδαίμονα λέληθε, καὶ ταύτην ἄμμου πλύσιν ἔφη\footnote{Ἀγαθοδαίμων M. --- Réd. de Lb. : ὅστις ἡθμὸς οὔτε τὸν Ἀγ. λέληθεν, οὔτε τοὺς ἄλλους · ταύτην γὰρ ἐργασίαν πλύνσιν ἄμμου καὶ κάθαρσιν ὠνόμασαν.} καὶ κάθαρσιν,\footnote{ἔφη.] ὀνόμασαν A.} ὅτε τὸ πᾶν λειω- (f. 145 r.) θὲν καὶ γενόμενον ὕδωρ ἔλθῃ\footnote{τὸ πᾶν] τὸ πνεῦμα A (en sigle) K.} διὰ ἠθμοῦ ἢ ὑλιστῆρος. Καὶ ὁ Ἑρμῆς φησιν · « Γίνεται ὡς ἡ στάκτη ἀκακία. » Ἐὰν μὲν γὰρ ὑποστάθμην, δῆλον γέγονεν ὡς αἱ\footnote{ἀκακία] καὶ ἡ ἀκακία B ; καὶ ἡ ἀκαγία A (2\textsuperscript{e} κ corrigé en γ par le copiste) ; καὶ ἡ ἀκαγία K (κ sur γ, d'une autre main) ; καὶ ἡ ἀκαγία E, et en mg. : ἀκαΐα ; καὶ ἡ ἀκαΐα Lb. --- ὑποσταθμὴ ᾖ Lb, f. mel.} οὐσίαι καὶ τὰ μέταλλα οὐδαμῶς λειοῦνται ·

2. Καὶ περὶ τούτων αὐτὸς ὁ Ἑρμῆς ἐν τοῖς κοσκίνοις ἰσχυρῶς διέλαβεν, λέγων ἄνω καὶ κάτω · « Ἐὰν καταβῇ τὰ ὕδατα,\footnote{κατάβῃ mss.} αὐτὸ τὸ κοσκινον ὡς ἔοικε ῥοῦν. » Ὅλα ὁμοῦ καταβαίνοντα αὐτὰ κατὰ τὸν μέγαν Ἑρμῆν · τάχα καὶ ἀναβαίνοντα δι ᾽ ὀργάνου, εἰς ὃ καὶ ἕψεσθαι δοκοῦσι. Ταῦτα δὲ εἰρήκαμεν τῷ λόγῳ, πλὴν ὁ λόγος περὶ καιροῦ. Καιρὸς\footnote{F. l. ταὐτὰ.} γὰρ ὁ θερινὸς, ὅτε ὁ ἥλιος φύσιν ἔχει πρὸς τὸ πρᾶγμα. Ἀμέλει οὖν ἡ Μαρία ἐν ταῖς ποιήσεσιν τοῦ προσωπιδίου · « Ὕδωρ θεῖον ληφθήσεται\footnote{προεισοδίου B etc. ὕδωρ θεῖον (en signes) M ; πρὸ ὕδατος τοῦ θείου Lb.} τοῖς μὴ νοοῦσιν, ὡς γέγραπται, ὃ διὰ τῆς λωπάδος καὶ τοῦ σωλῆνος εἰς ὕψος ἀναπέμπεται. » Ἀλλ ᾽ ἔθος τοῦτο λέγειν ὕδωρ τὴν αἰθάλην\footnote{θείου ἀθ.] en signe MBAK ; τοῦ θείου Lb.} θείου ἀθίκτου, ἀρσενίκων · οὗ ἕνεκεν ἐμυκτήρισάς με ὅτι περ δι ᾽ ἑνὸς\footnote{ἀρσενίκων] signe de l'arsenic redoublé MBAKE ; τῶν ἀρσενίκων Lb.} λόγου, τοσοῦτόν σοι τὸ μυστήριον ἐξέφρασα.

3. Τοῦτο μὲν τὸ ὕδωρ τοῦ θείου λευκαινόμενον διὰ τῶν λευκαινόντων,\footnote{M mg. : \emph{nřm} (nostrum ? ), d'une main du 16\textsuperscript{e} siècle.} λευκαίνει, καὶ ξανθούμενον διὰ τῶν ξανθούντων, ξανθοῖ, [καὶ ποιῶν] καὶ μελαινόμενον διὰ χαλκάνθου καὶ κικίδου, μελανοῖ\footnote{ποιῶν] ποιούμενον (ajouté) ποιὸν E ; καὶ ποιούμενον Lb.} · εἰς μέλανσιν ἀργύρου εἰς τὸν ἡμῶν μολυβδόχαλκον, περὶ οὗ μολυβδοχάλκου ἐν τῷ πατροπαραδότῳ ἀργύρῳ σοι προσεφώνησα.\footnote{προπαραδότου Lb seul.} Μελαινόμενον οὖν καὶ τὸ ὕδωρ ἀναλαμβάνοντα < τὸν > μολυβδόχαλκον ἡμῶν\footnote{ἀναλαμβάνον BAK ; ἀναλαμβανόμενον τὸν μ. Lb.} βάπτει ἄφευκτον μέλανσιν, ἥν τινα, μηδὲν οὖσαν, μέγα ἐπιθυμοῦσιν\footnote{βʹ ἅπτει M.} οἱ μύσται πάντες εἰδέναι · τὸ δὲ αὐτὸ ὕδωρ οἷον λαμβάνει τοιοῦτον,\footnote{χρῶμα τοιοῦτον BAK. χρῶμα τοιοῦτον λειοῦται καὶ β. Lb.} καὶ βάπτει ἄφευκτον, ὑφεξαιρουμένου τοῦ ἐλαίου καὶ τοῦ μέλιτος.\footnote{ὑπεξ. Lb.}

4. Καὶ ὁ φιλόσοφός φησιν ὅτι ὀλίγον θεῖον ἄθικτον οἶδε πολλὰ εἴδη καῦσαι, ἀλλὰ καὶ τοὺς λίθους καὶ τὰ μέταλλα μαλάσσει. Ἐν τούτῳ τῷ ὕδατι λειοῦται τὸ σύνθεμα τὸ θεῖον, ὡς εἰς τὸν ἀνδροδάμαντα φησίν · « Ἐὰν ἄπυρον θεῖον προβάλλῃς, ποι- (f. 145 v.) εῖς χρυσοζώμιον,\footnote{Signe de θεῖον ἄθ. M. --- προσβάλλεις BAK ; προσβάλῃς Lb.} ὁμοῦ σὺν τῷ συνθέματι τῶν οὐσιῶν · καὶ τὸ σύνθεμα τῶν θειωδῶν λειοῦται. » Καὶ οὕτως ἕψειται ἢ ὀπτᾶται, ἵνα ὁ νοῦς σωθῇ.\footnote{καὶ οὕτως ὁμοῦ Lb.} « Ἐὰν, φησὶν, θεῖον ἄπυρον προσβάλλῃς, ποιεῖς χρυσοζώμιον διὰ πρίσματος,\footnote{διὰ πρήσματα M.} ἢ κηροτακίδος, τὸ θεῖον ὕδωρ, ἕως σχῇ χρυσόν · ἕψει ἐλαφρῶς\footnote{τὸ δὲ θεῖον Lb.} κινῶν, ἐπιβάλλων τὰ μωτάρια τῆς ξανθῆς σανδαράχης. » Μωτάρια δὲ\footnote{παχὺν M ; παχέα εἶναι αὐτὰ Lb. --- τότε λοιπὸν B etc.} εἰρήκασι διὰ τὸ παχὺ εἶναι αὐτὸ ὡς αἷμα · τὸ λοιπὸν ὄπτα σφοδροτέρως ἡμέρας βʹ ἢ γʹ, καὶ κατενέγκας, ἔκχεε εἰς τὸ τοῦ φαρμάκου λείψανον ἐν ἑκάστῳ, καὶ γίνεται ἰός. Περὶ τούτου ἔλεγεν ὁ Πηβίχιος\footnote{Πηβήχιος B etc.} · « Διαμερίσατε τὸ φάρμακον εἰς μέρη δύο, καὶ τὸ ἤμισυ ἔχετε ἐν\footnote{καὶ τὸ μὲν ἓν (sur grattage) ἒχ. Lb seul} ὀστρακίνῳ ἀγγείῳ, τὸ δὲ ἕτερον εἰς χαλκοῦν. » Τοῦτο αἰνιττόμενος δι ᾽ ἑνὸς, ἀπὸ μέν [τοι] τοῦ ὀστρακίνου τὴν ὄπτησιν, ἀπὸ δὲ τοῦ χαλκοῦ τὴν ἴωσιν. Προεῖπε δὲ καὶ τὴν λεύκωσιν ἀπὸ τοῦ εἰρηκέναι ἐν δαφνίνοις ξύλοις καίεσθαι τὸν χαλκὸν, τουτέστιν τὸ θεῖον ἄθικτον\footnote{τὸ θεῖον ἄθικτον (en signe) M ; τῷ et le même signe B etc. sauf Lb, qui écrit θείῳ en toutes lettres.} τῷ ἔχοντι φύλλα δάφνης, ἵνα ἔχῃς εἰδέναι τὴν τῶν ἀρχαίων ἀρετὴν, πῶς φανερῶς πάντα εἰρήκασιν · δοκοῦντες πάντα κρῦψαι, φανερῶς\footnote{δοκοῦντες τισὶν ἅπαντα κρ. B etc. --- M mg. : \emph{nota} (main du 16\textsuperscript{e} siècle).} εἰρήκασι · « Πρῶτον ἐλαφροῖς φωσὶν, ἵνα συμπίῃ τὸ ὕδωρ τοῦ θείου\footnote{δεῖ δὲ πρῶτον Lb} ἀθίκτου. » Περὶ ὧν φώτων ἡ Μαρία ἔλεγεν ἐκ προβάσεως τὰ φῶτα,\footnote{προσβάσεως M} καὶ πάλιν ἐκ προσαγωγῆς τὸ πῦρ, ὅταν ἀρκούντως ποιῇ, προοδωτέρως, ἵνα σωθῇ ὁ νοῦς, ἐκ προβάσεως τὰ φῶτα. Ὁ δὲ καιρὸς ὁ θερινὸς, καὶ ἡ πορφύρα καιρὸν ἴδιον ἔχει διὰ τὰς λύσεις καὶ ψύξεις τὸ ἁλιστέον,\footnote{ψύξεις, τοῦ ἁλιστέου Lb.} ὅ τι καὶ τὸ κόμμι δάκρυον αὐτομάτως προερχόμενον, ἀπὸ τῆς ἰδίας\footnote{τὸ κόμμι ἐστὶ δάκρυον Lb.} φύσεως, θέρος. Ἤκουσα δέ τινων ὅτι ἐν παντὶ καιρῷ γίνεται ἡ ἡμῶν\footnote{κατὰ τὸ θέρος Lb. --- τινων οἳ λέγουσιν Lb.} ἐργασία, καὶ ἀμφιβάλλω.\footnote{A mg. : Βλέπε ἔμπροσθεν εἰς φύλλ ` κβ ʹ τὴν ῥῆσιν τοῦ λόγου ὅπου τὸ σημεῖον τοῦτο, puis un signe de renvoi, reproduit en rouge 21 ff. plus loin (f. 139 r.) en regard des mots : Καὶ ὁ Ζώσιμος ... ἀμφιβαλλόμενος (3, 29, 21).}

\bigskip
\centerline{\EightStarTaper}
\centerline{\EightStarTaper\EightStarTaper}
\bigskip

\subsubsection[3. --- 16. ΠΕΡΙ ΤΗΣ ΚΑΤΑ ΠΛΑΤΟΣ ΕΚΔΟΣΕΩΣ ΤΟ ΕΡΓΟΝ.]{3. --- 16. ΠΕΡΙ ΤΗΣ ΚΑΤΑ ΠΛΑΤΟΣ ΕΚΔΟΣΕΩΣ ΤΟ ΕΡΓΟΝ.\footnote{Pas de titre dans B ; titre dans AKELb : περὶ τῆς κατὰ πλάτος ἐκδ. τοῦ λόγου πρὸς Φιλάρετον.}}
\paragraph{}
\emph{Transcrit sur} M, f. 145 v. --- \emph{Collationné sur} B, f. 126 r. ;--- \emph{sur} A, f. 118 r. ;--- \emph{sur} K, f. 21 v. (\emph{suite} f. 113 v.) ;--- \emph{sur} E, f. 51 r. ;--- \emph{sur} Lb (\emph{copie de} E), p. 187. --- \emph{Les variantes et restitutions de} M, \emph{par rapport à} BAK, \emph{ont été reportées en marge de} K. --- \emph{Chap.} 39 \emph{de la compilation du Chrétien dans} E Lb.

\bigskip

1. Καὶ ταῦτα μὲν οὕτως πρὸς τοὺς Αἰγυπτίους προφήτας ὁ (f. 146 r.) Δημόκριτος γράφει. « Ἐγὼ δὲ πρὸς σὲ, ὦ Φιλάρετε, πρὸς ὃν ἡ δύναμις, τὴν κατὰ πλάτος σοι γράφω τέχνην. Ὁ μὲν τῶν εἰδῶν κατάλογος οὕτως ἔχει. Ὑδράργυρος ἡ ἀπὸ κινναβάρεως,\footnote{εἰδῶν] ἰδίων corrigé par une main assez récente M.} μαγνησία, καὶ στίμμι κοπτικὸν, χαλκηδόνιον, ἰταλικὸν, λιθάργυρος, ψιμμίθιον, μόλυβδος, κασσίτερος, σίδηρος, χαλκὸς, χρυσόκολλα κλαυδιανὸν, καδμεία, πυρίτης, ἀνδροδάμας, θεῖον ἄθικτον, ἀρσένικον, σανδαράχη, κιννάβαρις.

2. Ταῦτα τὰ εἴδη ἐπίκοινα εἰς χρυσὸν καὶ ἄργυρον · λευκαινόμενα γὰρ λευκαίνουσι, καὶ ξανθούμενα ξανθοῦσιν. Τὰ οὖν λευκαίνοντα αὐτὰ ταῦτα · γῆ χεία, καὶ ἀστερίτης, γῆ σαμία, γῆ κιμωλία, καὶ ἀφροσέληνον.

3. Τὰ δὲ λειούμενα, αὐτὰ · θεῖον ἄθικτον, ἅλας καππαδοκικὸν,\footnote{F. l. ταῦτα. --- θεῖον ἄθικτον en signe M. F. l. θεῖον.} ἅλες παντοῖοι, ἀλὸς ἄνθη, τίτανος, ὃς προσκέκληται ὀπὸς συκαμίνου,\footnote{ἅλας παντοῖον, ἁλὸς ἄνθος B etc.} συκῆς, στυπτηρία σχιστὴ, μύσι, χάλκανθος, φύλλα περσέας,\footnote{μίσυ Lb, mel. --- ὀπὸς] dernier mot du f. 21 de K ; la suite est au f. 113 ; le f. 22 doit être lu après le f. 115.} φύλλα δάφνης.

4. Τὰ δὲ ξανθοῦντα, ταῦτα · γῆ ποντικὴ, ὅ ἐστιν ὀπτὴ, γῆ ἀττικὴ,\footnote{ὀπὴ M.} ὅ ἐστιν ὁ κυανὸς, καὶ ἡ κυανὸς ἡ ἐπὶ τῶν δύο βαφῶν ἐπίκοινος · καὶ\footnote{ὁ κυανὸς] signe de κυανὸς dans MBAKE ; χαλκὸς en toutes lettres Lb. --- Le bleu mâle et femelle. (\emph{M. B.}) Cp. l'Introduction de M. Berthelot, p. 245.} ἐν βοτάναις, κικίδιον, καὶ κνηκάνθιον, ἐλύδριον καὶ οἰχούμενον\footnote{χιμένιον BAΚ ; χυμένιον Lb. Cp. 2, 1, 18, texte et traduction.} · καὶ ἐν ὀποῖς, κόμμι · Ἔλεγεν δὲ ἀντὶ τοῦ κόμμεως, εἰς γὰρ τὸ λευκὸν σύνθεμα τοὺς ὀποὺς βάλλουσι.

5. Φανερὰ δὲ ἔστω τὰ τῇ ἰώσει ὕστερον συλλειούμενα καὶ ὧδε ἁρμῶσαι ἡ μαρτυρία ἡ λέγουσα ὅτι τὰ ἀνούσια σώματα καλῶς ἐνεργοῦσιν\footnote{ἁρμόσαι BE ; ἁρμόγαι AK ; ἁρμόσει Lc, f. mel.} χωρὶς πυρός. Τινὲς βούλονται δεύτερον καὶ τρίτον ἐν τῇ ἱώσει βαλεῖν βοτάνας, ἄνθος ἀναγαλλίδος, καὶ ῥὰ, καὶ τὰ ὅμοια · καὶ κρόκον τινὲς χρῶνται καὶ ῥίζαν μανδραγόρου τὴν τὰ σφαιρία ἔχουσαν.\footnote{τινὲς δὲ χρ. καὶ ῥίζῃ μ. τῇ τὰς σφαίρας ἐχούσῃ. --- τὰς σφαίρας AK ; ἔχουσιν A ; ἔχουσιν (α sur ι) K.} Ἐγὼ δὲ προσθήσω ὅτι χωρὶς αὐτῆς οὐδὲν βάπτεται · καὶ ταύτῃ πάντα συλλειοῦται ἐν τῇ ἰώσει μετὰ κόμμεως. Ἐμνημόνευσαν δὲ\footnote{M mg : σῆ (main du 16\textsuperscript{e} siècle).} πάντες ὅτι οὐ δεῖ εἰς τοῦτο τὸ ὕδωρ ζύμην καταφθείρειν · καὶ\footnote{F. l. καταφέρειν.} ὁμοιοῦται τῷ μέλλοντι βάπτεσθαι σώματι.

6. Ἐὰν ἀργύρεον μέλλῃς βάπτειν, ἀργύρου πέταλον συνσήπειν\footnote{ἐὰν --- συνσήπειν om. B etc.} · ἐὰν δὲ χρυσοῦν, χρυσοῦ πέταλον συνσήπειν · ὁ γὰρ σῖτος σῖτον\footnote{συσσήπτειν et plus loin σήπτειν BAK. --- δεῖ συσσήπειν Lb.} γεν- (f. 146 v.) νᾷ, καὶ ὁ λέων λέοντα, καὶ χρυσὸς χρυσόν.\footnote{καὶ ὁ χρυσὸς χρυσὸν B etc.} Ἐπίβαλλε γὰρ, φησὶν, ἄργυρον κοινὸν, καὶ βάπτεις. Ὁ γὰρ εἷς ζωμὸς κατὰ τῶν ἀμφοτέρων σήπειν κατηγορεῖται · ὅτι τοιοῦτος λόγος ἐν τῷ παρόντι περὶ τῆς τοῦ φαρμάκου βαφῆς · τὸ γὰρ θεῖον ὕδωρ σκευασθὲν κατὰ ἀλήθειαν, καὶ τὸ καλῶς συγκραθὲν τὰ φάρμακα βάπτει, καὶ ὅταν\footnote{κακῶς M. Corr. d'après ELb.} βαφῇ τὸ φάρμακον, τότε καὶ αὐτὸ βάπτει. Διὰ τοῦτο ζύμας καὶ προζύμια καὶ ὀξυζύμια, καὶ χρυσοζύμια, καὶ ὅσα κέκρυπται · ἐν δὲ\footnote{ἐν δὲ (γὰρ Lb) --- νοοῦσι] E aj. cette phrase en marge, et Lb la transporte après Mαρίαν (p. suiv., l. 1).} πᾶσι τὸ πᾶν εὑρίσκεται τοῖς νοοῦσιν.

7. Ἰδοὺ οὖν τὰ δʹ σώματα πυρίμαχα, ὑπόστατα, τουτέστιν τὸ\footnote{ὑποστατὰ mss. ici et plus loin, comme p. 148. --- M mg. : ὧδε ἀληθὲς (main du 16\textsuperscript{e} siècle).} ὕστερον σύνθεμα, οὗ καὶ αὐτοῦ συντεθέντος, παραλαμβάνομεν μέρος ἓν, ἐπιβάλλοντες ὕδωρ θεῖον ὁμοῦ, ἕως γένηται τὸ χρῶμα καὶ ὁ τόνος τοῦ ὁμοίου κατὰ Μαρίαν. Ἐπειδὴ οὖν κατείληπται τὸ ὕστερον σύνθεμα τὰ ὑπόστατα δʹ σώματα, οἷς οὐ μόνον τὸ σύνθεμα τοῦ χρυσοζυμίου, ἀλλὰ καὶ τὸ σύνθεμα ἐπιβάλλει τοῦ ὕδατος τοῦ θείου\footnote{χρυσοζωμίου Lb.} · ἐπιβάλλειν γὰρ δεῖ τὰ προδεχθέντα, σίδηρον, ἢ κασσίτερον,\footnote{δεῖ] χρὴ B etc. --- προλεχθέντα B etc.} ἢ μόλυβδον, ἢ χαλκὸν, καὶ τὰ ἑξῆς, πάντα τούτοις ἐπιβάλλεται. Ἄκουε αὐτοῦ λέγοντος ἐν τῷ κεφαλαίῳ τῶν δύο συνθεμάτων · « Ἐὰν εἰς σίδηρον προβαλών · ἐὰν εἰς χαλκὸν, προεξιοῖ · ἐὰν εἰς μόλυβδον,\footnote{προσβάλῃς, ἐξιοῖ · ἐὰν ... Lb.} ποιεῖ ἄρρευστον, ἄτριστον, τὸν κασσίτερον προεργάζου,\footnote{ἄτρυτον B etc. F. l. ἄτρητον. (Cp. ci-dessus, p. 45, l. 26).} καὶ οὕτως ἐπίβαλλε, φησὶν, καὶ οὐ μὴ σφάλῃς, τουτέστιν προλεύκαναι.

8. Περὶ δὲ ἐξιώσεως καὶ χαλκοῦ διαλάβωμεν · πάντα τὰ τοιαῦτα εἴδη ἔχουσιν φύλλα περσέας καὶ δάφνης, καὶ γαῖ λευκαὶ, καὶ σύκαμίνου καὶ συκῆς καὶ τιθυμάλλου ὀπὸς, καὶ νίτρον πυρρὸν καὶ ἅλας\footnote{τηθυμ. M. --- πυρὸν M.} καππαδοκικὸν, καὶ τὰ ὅμοια · εἰς τοῦτον, φησὶν, τὸν ζωμὸν καθίενται αἱ λεπίδες τοῦ χαλκοῦ ἡμέρας ιεʹ, καὶ εὑρήσεις ἐξιωθέντα, τουτέστιν\footnote{καὶ εὑρ. ἅπαντα ἐξ. Lb.} λευκανθέντα. Αὕτη οὖν ἡ σύνθεσις τοῦ ζωμοῦ λευκοῦ θείου. Ἐν τῇ\footnote{ἣν ἐν τῇ ὑστ. Lb, f. mel.} ὑστέρᾳ < τάξει > τῶν ζωμῶν ὁ φιλόσοφος ἐξέδωκεν. Εἰ τοίνυν λευκὸν θεῖον, ἆρα (f. 147 r.) τὸν χαλκὸν λευκαίνει ; θεῖον γὰρ ξανθὸν οἰκονομήσας ὁ χαλκὸς διὰ χαλκάνθου καὶ σώρεως, καὶ ἐπιβαλὼν χαλκὸν, ξανθώσας αὐτὸν, τοῦτον τὸν χαλκὸν ἅμα τῷ θείῳ ἀποτίθεται\footnote{ξανθώσι A ; ξανθώσεις Lb.} εἰς ὄξος, καὶ τὰ ἑξῆς, ἵνα ἰωθῇ. Καὶ γάρ φησιν χάλκανθον ποιεῖν τὸ χρυσὸν ίνιον · εἰ δὲ χάλκανθος τῷ θείῳ, τῷ πυρίτῃ συνελειώθη μετὰ\footnote{χρυσὸν ίνιον] χρυσάνθιον B Lb. --- τὸ puis le signe de πυρίτης M ; om. B etc.} σώρεως, τὸ δὲ θεῖον τὸ ξανθὸν, ἔν τε τούτῳ τῷ ξανθῷ · ἐὰν ἐαθῇ\footnote{τε] δὲ Lb. --- ἐὰν ἐαθῇ] ἐάσεις Lb.} κάτω ἵνα ἐσθίῃ · ἤγουν τὸ θεῖον τὸ ξανθόν.

9. Καὶ τί ἄρα ἐξίωσις ἢ ξάνθωσις ; ἐξίωσις οὖν καὶ ξάνθωσις χρώματι μόνον διενηνόχασιν ἀλλήλων · ἤγουν τὸ ἐξίωσις θείου λεύκωσις,\footnote{μόνω BAK.} ἡ δὲ ἴωσις, ξάνθωσις. Φέρε καὶ τὰ ἄλλα < ἃ > εἶπεν · ἐὰν εἰς σίδηρον προμάλαξιν ποιήσας τὸν σίδηρον λεπίδας λεπτὰς, ἐπίστρωσον γῆν σαμίαν, καὶ στυπτηρίαν σχιστὴν διπλώσας, ἔλασον, καὶ ἔσται\footnote{διπλ. ἔλασον] καὶ διπλ. ἄλλασσεν Lb.} μαλακὸς καὶ λευκὸς. Τὰ δὲ τοιαῦτα εἴδη μέρη εἰσὶν τοῦ λευκοῦ θείου. Ὁ Ἑρμῆς μάλαξιν προθέμενος, ὔστερον ἔλεγεν · « Καὶ λευκανθήσεται. » Διὰ τοῦτο ὁ φιλόσοφος ἔλεγεν · « Ἐπίβαλε τοῦ λευκοῦ φαρμάκου τὸ ἥμισυ, καὶ ἔσται πρῶτον τοῦτο τοῦ λευκοῦ θείου. »

10. Φέρε καὶ τὸ τί ἀτριστώσῃς ζητήσωμεν · Ὁ φιλόσοφος\footnote{ἀτριστώσῃς] ἀτρυτώσεις B ; ἀτρυτόσει AK ; ἀτρυπτώσεις corrigé en ἀτρυττώσει E ; ἀτρύτωσις Lb. (Variantes analogues plus loin.)} · « Λαβὼν μόλυβδον λευκὸν τὸν γενόμενον ἄρρευστον διὰ γῆς χείας,\footnote{γενάμενον M, ici et presque partout.} καὶ στυπτηρίας σχιστῆς. » Τὰ δὲ εἴδη ταῦτα μέρη εἰσὶ τοῦ λευκοῦ θείου. Τὸ δὲ λευκὸν θεῖον, λευκαινόμενον, λευκαίνει. Δημόκριτος δέ · « Ἐπειδ ᾽ ἂν ἐξιώσῃς, καὶ μαλάξῃς, καὶ ἀτριστώσῃς, καὶ ἀρρευστώσῃς, ἢ λευκώσῃς. » Ἡ δὲ λεύκωσις ἐκ τοῦ λευκοῦ θείου. Ὅρα τὸν φιλόσοφον περὶ τούτου τοῦ θείου τοῦ λευκοῦ ἐκβακχεύοντα · « Ἐὰν γὰρ, φησὶν, γένηται τὸ φάρμακον μαρμάρῳ παρεμφερὲς, μέγα ἐστὶ μυστήριον · τὸν γὰρ χαλκὸν λευκαίνει, τουτέστιν ἐξιοῖ, μαλάσσει τὸν σίδηρον, ἄτριστον ποιεῖ τὸν κασσίτερον, ἄρρευστον τὸν μολυβδον,\footnote{ἄτριστον] ἄτρυτον B etc.} ἀρρήκτους τὰς οὐσίας, ἀφεύκτους τὰς βαφάς. Αὗται αἱ βαφαὶ τὰ εἴδη ἀπὸ ὑδραργύρου ἕως χρυσοκόλλης καλούμενα χρυσάνθιον\footnote{ὑδραργύρου] signe du mercure M ; signe de l'argent BAKE (E ajoute ἔχουσι) ; ἀρύρου en toutes lettres Lb. --- καλούμενον BAΚ ; κολλώμενον δὲ χρ. Lb. F. l. καλοῦμεν.} · εἰκότως εἴρηται παρά τινων τοῦτο τὸ θεῖον διὰ πάντων. Στέφανος (f. 147 v.) γὰρ, ὅταν ἔλεγεν · « Ἀρρήκτους τὰς οὐσίας, » τὰ τέσσαρα σώματα ἔλεγεν · ἄλλοι δὲ · « τοῦτο τὸ θεῖον ὕδωρ τὸ κατὰ πάντα μέγα μυστήριον, τὸ γενόμενον μαρμάρῳ παρεμφερὲς,\footnote{μέγα μυστ. καλοῦσι Lb.} τὸ λευκαῖνον πᾶσαν οὐσίαν, τὸ λευκαῖνον τὸ σῶμα τῆς μολυβδοχάλκου\footnote{μολυβδοχ.] signe du molybdochalque M ; signe de la magnésie B etc., f. mel.} · τοῦτό ἐστιν ὁ τῶν κωβαθίων καπνός. Τοῦτο ὁ τὰς βαφὰς\footnote{κοβαφίων M. --- M mg. : ὕδ. θείου ἀπύρου (avec renvoi à κοβαφίων), main du 15\textsuperscript{e} siècle (celle de Bessarion ? ) --- τοῦτό ἐστι τὸ τ. β. ποιοῦν Lb.} ἀρρήκτους τηρῶν, τοῦτο ὁ τὰς οὐσίας ἀρρήκτους διατηρῶν. Τὸ δὲ\footnote{F. l. τὰς βaφὰς ἀφεύκτους. --- ἀρρήκτους διατηρῶν om. B etc.} ἀρρήκτους ἐὰν ἀκούσῃς, οὐχ ἵνα ἐλαιούμεναι αἱ οὐσίαι μὴ ῥαγῶσιν,\footnote{ἐλεούμεναι M.} ἀλλ ᾽ ἵνα μὴ ἀπορρήξωσι τὰ εἰωθότα τε τῷ πυρὶ ἀφαντοῦσθαι ἀπὸ νεφέλης ἕως χρυσοκόλλης, ὅτι βαφὰς βούλεται αὐτὰς εἶναι. Ἄκουε αὐτοῦ λέγοντος περὶ αὐτῶν · « Ἐπιβάλλειν οὖν δεῖ σίδηρον,\footnote{A mg. : σῆσαι. --- ἐπίβαλε οὗν σιδήρῳ etc. (datif partout) Lb ; simple signe dans les autres mss.} ἢ χαλκὸν, ἢ κασσίτερον, ἢ μόλυβδον. » Τοίνυν ταύτας βαφὰς καλεῖ · τὰ δὲ βαπτόμενα δʹ σώματα ἅτινα βαφέντα βάπτουσιν\footnote{M mg. : τὸ ὅλον τῶν ἀληθῶν ( ? ) avec renvoi à βάπτουσιν, au moyen du signe zodiacal de la Vierge \virgo (main du 15\textsuperscript{e} siècle).} · τὸ δὲ βάπτον τὰς βαφὰς καὶ τὰ βαπτόμενα ὕδατα θείου, τὸ μέγα\footnote{βαπτόμενα καὶ τῆι υδ puis le signe du θεῖον ἄθικτον M (καὶ et ῆι d'une écriture plus récente ; βαπτόμενα ὕδωρ θεῖόν ἐστι Lb. F. l. ὕδωρ θείου.} μυστήριον, τὸ μαρμάρῳ παρεμφερὲς, τὸ τὰ πάντα ποιοῦν ἐπιτήδεια, τὸ καῖον τὸν χαλκὸν καὶ λευκαῖνον, τὸ τὴν ὑδράργυρον πηγνύον, τὸ ἐξιοῦν · τοῦτό ἐστι τὸ τῆς ὅλης τέχνης μέγα μυστήριον · τὸ γὰρ ξανθὸν ὕδωρ ἐμφανὲς μυστήριον.

11. Ἐπίβαλε λοιπὸν καὶ κόμμι μικρὸν, καὶ πᾶν σῶμα βάπτεις\footnote{καὶ πᾶν] καὶ om. M.} · τοῦτο αἴτιον . καύσεως, λευκώσεως, ξανθώσεως, ὑδραργύρου πήξεως,\footnote{F. l. ὑδραργυροπήξεως.} ἰώσεως · τοίνυν ὅταν λέγῃ « ἀρρήκτους τὰς οὐσίας, » περὶ τοῦ ἀπορρηγνύναι τὰς οὐσίας, τὰ εἴδη τὰ φευκτὰ λέγει. Τοῦτο δὲ τὸ\footnote{M mg. : ἀπορρηγνύντας. --- τὰς om. M.} λευκὸν θεῖον ἀνακεφαλαιοῦται ἐν τοῖς δυσὶ συνθέμασι. Λέγει γάρ · « Ἐὰν εἰς σίδηρον, προμαλάσσει, » καὶ τὰ ἑξῆς, τουτέστιν πάντα προλεύκαναι, καθὼς ἀποδέδεικται · ὅταν ἐξιώσῃς καὶ μαλάξῃς καὶ ἀτριστώσῃς καὶ ἀρρευστώσῃς, τουτέστιν λευκάνῃς τὸ πᾶν, τὰ τεσσαρα\footnote{ἀτριστώσῃς] ἀτρυτώσῃς B etc. --- τὸ πᾶν] τουτέστι E ; om. Lb. --- F. l. τὸ πᾶν, τουτέστι ...} σώματα ὑπόστατα · αὕτη γὰρ ἡ ἀρχὴ κατὰ μίαν τάξιν τὸ\footnote{τὸ] τοῦ Lb, f. mel.} λευκάναι. Ἡ δὲ λεύκωσις ἐκ θείου λευκοῦ · ὁ δὲ τῶν λευκῶν θείων\footnote{F. l. τοῦ λευκοῦ θείου.} σταθμὸς ἐν (f. 148 r.) τῇ ὑστέρᾳ < τάξει > τῶν λευκῶν ζωμῶν κεῖται, ἔχων τὸ ἀρσενίκου χρυσίζοντος γ° αʹ, καὶ νίτρου καὶ τῶν ὁμοίων,\footnote{τὸ] F. l. τοῦ.} καὶ φλοιῶν φύλλων περσέας καὶ δάφνης γ° αʹ, καὶ συκαμίνου χυλοῦ,\footnote{καὶ φλ. κ. φ. Lb.} καὶ ἅλατος, καὶ τὰ ἑξῆς. Πάντα πρὸς ἀνάλογον τοῦ οὐγκιασμοῦ δεῖ\footnote{ὀγκιασμοῦ M.} σε προσπλέξαι. Ἡ γὰρ ὑδράργυρος κατὰ τῶν δύο συνθεμάτων τὰ\footnote{προσεπιπλέξαι Lb.} πάντα μέλλουσα ἀναλαμβάνειν ἤτοι μαλαγματίζειν, περὶ ἧς καὶ ἐν\footnote{μέλλει Lb.} τῷ περὶ κινναβάρεως μηνύσω. Εἰ μὲν οὖν ἀναλήψεται, δεῖ μὴ ὠῶν\footnote{κινναβάρεως] signe lunaire couché BAKE ; ἀργύρου Lb ; K mg. et E mg. (d'après K ? ) : signe du cinabre. --- Réd. de Lb. : ... ἀναλήψεται, καλῶς ἔχει, εἰ δὲ μὴ, ὠῶν λευκοῖς.} λευκοῖς καὶ ὑγρῷ κομμίῳ λευκῷ λειοῦσθαι μετὰ τῶν δύο συνθεμάτων.\footnote{καὶ λευκῷ ὑδραργύρῳ συλλειοῦσθαι μετὰ τῶν τοιούτων συνθημάτων B etc.} Ἐν γὰρ τούτοις εἴωθεν ἡ ὑδράργυρος μολύνειν καὶ ἀναλαμβάνειν,\footnote{ὑδράργυρος] ἄργυρος BAK ; τὸν ἄργυρον Lb.} καὶ πάντα μαλαγματίζειν, περὶ ὧν ἐν τοῖς μολυβδοχάλκοις προσεφώνησα.

12. Τινὲς δὲ ὕδωρ θεῖον ἐλείωσαν παχύτερον ποιήσαντες, καὶ ἀνέλαβον τὰ συνθέματα τὴν ὑδράργυρον. Καὶ γὰρ τὸ λευκὸν σύνθεμα\footnote{τά συνθήματα μετὰ τῶν συνθημάτων B etc. --- M mg. : ὧδε (en lettres retournées).} καὶ ὠὰ ἔχει καὶ κόμμι. Ἄλλοι ἐν τρούλλῳ μεγάλῳ ὑελίνῳ\footnote{κόμμι] κόμεως M.} περιπηλώσαντες, ἔβαλλον τὰ πάντα καὶ ἀσθενεῖ πυρὶ ὤπτησαν, ἐπιβάλλοντες ὕδωρ θεῖον, ἑψήσαντες ὡς τὴν πορφύραν. Δεῖ δὲ προσέχειν ἐν τῇ μεταβολῇ, πῶς ἐκ θαλάττης οὖσα καὶ ἐκλύσματος,\footnote{ἐκλύσματος et en surcharge à l'encre noire : ἑλκύσματος M ; ἐκ κλύσματος B ; ἐκκλείσματος AK ; καὶ κλύσματος ELb.} εἰς πορφύραν μετατρέπεται ἀληθινήν. Ἐντεῦθεν καὶ ὁ φιλόσοφος · « Τὸ γὰρ ψιμύθιον ἄλλην ἔχει δύναμιν παρὰ τὸ ἕλκυσμα, τουτέστι παρὰ τὸ χρυσίζον, ἢ πορφυρίζον, παρὰ τὸ λευκὸν ἢ ἀργυρῶδες. » Τὸ δὲ αὐτὸ σύνθεμα λειωθὲν ἕξει καὶ τὰς ἐνεργείας · τὰ ὅλα ἐκ μιᾶς, φησὶν, ὕλης τοῦ μολύβδου · ὁ δὲ χαλκὸς οἶδας λοιπὸν ὡς ὅλον σύνθετον · ὅθεν ἐν τῷ ἑλκύσματι μεταβολὴν ὠνόμασεν αὐτῷ ἐν ὑποδείγμασιν · ἑψήσαντες\footnote{ὠνόμασαν B etc. --- αὐτὸ BAK ; αὐτὸν Lb. --- ὑποδείγματι B etc.} γὰρ ὕδωρ θεῖον · τῷ γὰρ « ἐψήσαντες » χρῶμα ἀνέδειξαν · καὶ οὐ μόνον\footnote{τῷ] τὸ mss. Corr. conj. --- καὶ οὐ θείῳ μόνον Lb. --- M mg. : abréviation probable de χρῶμα.} ἥνωσαν τὴν ὑδράργυρον, ἀλλὰ καὶ ἐλεύκαναν καὶ ἐξάνθωσαν τὸ σύνθεμα ἑψοῦντες λεπτῷ πυρί, καὶ οὐκ ἐῶντες καπνὸν διὰ τοῦ τρούλλου ἀναδοθῆναι. Μετ ᾽ αὐτοῦ γὰρ (f. 148 v.) τὸ πνεῦμα τὸ βαπτικὸν συναφίσταται. Ἑψοῦσι δὲ ἕως ἂν ἀραιώσῃ τὸ χρῶμα, οἱ μὲν ὥρας θʹ,\footnote{θʹ] δώδεκα Lb.} οἱ δὲ ἡμέρας. Ὅταν δὲ οὕτως γένηται, περισκεπάζουσιν τὸν τρούλλον φιάλῃ,\footnote{ἡμέρας] νυχθήμερον (en signe) B ; νυχθ. αʹ AKE Lb.} καὶ τιθέασιν ἐν κηροτακίδι ἢ ἐν βωταρίῳ, ἐπάνω τῆς καμίνου, καὶ καίουσι τὴν κάμινον ἐκ προβάσεως ἡμέραν αʹ, ἄλλοι δύο · καὶ\footnote{προσβάσεως M. --- νυχθήμερον αʹ B etc.} θεωροῦσι διὰ τῆς φιάλης πότε γίνεται ψιμμύθιον, καὶ κατασπῶσιν ὑπόφιμον.

13. Τινὲς χρόνον ποιοῦσι, καὶ τὸ μέσον τρήσαντες εὑρίσκουσιν\footnote{χρόνον] χ traversé verticalement par un ρ dans BAK ; ce signe et au-dessous : ὑάλῳ E ; Τινὲς δὲ ἐν ὑάλῳ π. Lb. (Les signes de χρόνος et de ὕελος [= 	extbf{Χ}] ont pu être confondus.) « ? ? χρόνος, pour Κρόνος, plomb » (\emph{M. B.}). --- Signe attribué au κρόκος dans BA ; \emph{Nοt. alch.}, pl. 5, l. 8 (\emph{C. E. R.})} ὑποκάτω μόνα τῆς σκωρίας ἄνω ὑπολειφθείσης · εἰς γὰρ τὸ δίχρωμον\footnote{ἄνωθεν τῆς νεφέλης B etc. --- δίχρονον M.} ἡ σκωρία μετὰ τοῦ μολύβδου εὑρίσκεται · καὶ ἀποτινάξαντες τὴν σκωρίαν, ἔχουσι τὸ σῶμα · τοῦτον τὸν λίθον λειοῦσιν ἐν ἡλίῳ ἕως λευκανθῇ, σὺν τούτῳ στήσαντες ὑδραργύρου τὸ ἥμισυ τοῦ σταθμοῦ καὶ θεῖον πρὸς τὸ ὑπερέχειν, καὶ κόμμι λευκὸν, πήσσουσιν ἐν θερμοσποδιᾷ\footnote{καὶ ὕδωρ θεῖον BA ; καὶ ὕδατος θείου KELb.} ἡμέραν ὅλην, ἕως τὸ ὕδωρ τοῦ θείου, πρὸς ὃ ἀναξηραίνει\footnote{πρὸς ὃ] F. l. πρόσω.} · προσβάλλουσιν ὕδωρ θεῖον · καὶ ὅτε τὸ πᾶν ὕδωρ ἀναλωθῇ,\footnote{προσβάλλωσιν ὕδατι θείῳ Lb.} μεταβαλόντες ὀπτοῦσιν βούκλας ἡμέραν μίαν εἱλικτῇ, καὶ εὑρίσκουσι ψιμύθιον.\footnote{ἐν βούκλῃ ἑλικτῇ ἡμ. μίαν Lb.} Τοῦτο ἔτι ζέον μεταβάλλουσιν εἰς θεῖον ἄπυρον, καὶ τὸ ὕδωρ τοῦ θείου, τὸ ἄλλο ἥμισυ τοῦ σταθμοῦ, καὶ ἐῶσι κάτω ἡμέρας,\footnote{ἡμέρας δύο Lb.} ἕως οὗ ἰωθῇ.

14. Τινὲς καὶ εἰς ίππείαν κόπρον χωννύουσιν τὰς αὐτὰς ἠμέρας ἐκεῖ\footnote{F. l. τοσαύτας.} · τούτῳ ἐπιβάλλουσιν χαλκὸν προσλαβόν τι μετὰ τὴν βαφὴν τοῦ λευκοῦ σιδήρου, ἐὰν θέλωσι ποιῆσαι ἄργυρον · ἐὰν δὲ χρυσὸν, συλλειοῦσι\footnote{λευκοῦ σιδήρου en signes M ; λευκοῦ λιθαργύρου Lb seul. F. l. τὸ Δ (= δον) σιδήρου. (même signe pour τέταρτος et pour λευκός.) --- Cp. p. 153, l. 17.} πάλιν ὑδράργυρον τὸ ἥμισυ τοῦ σταθμοῦ, καὶ θείου τὸ ἥμισυ, ξανθοῦ λέγω, καὶ ὕδωρ θείου ἀθίκτου καὶ κόμμεως · καὶ πήσσουσι, καθὼς καὶ τὸ πρῶτον, καὶ ὀπτῶσιν νυχθήμερα βʹ · καὶ ἐξενέγκαντες ζέον, βάλλουσιν εἰς τὸ λείψανον τοῦ θείου καὶ ὕδωρ θείου · καὶ καίουσιν ἡμέρας, καὶ ἕως οὗ καιῶσι τοῦτο · ἐπιβάλλουσιν ἄργυρον κοινόν.\footnote{ἡμέρας δύο Lb. --- καὶ om. B etc.}

15. Ἡ δὲ τοῦ λευκοῦ σκευὴ αὕτη · θεῖον, ἀρσένικον, σανδαράχη, κιννάβαρις, ἐξ ἰσότητος προτεταριχευμένα, ἅλατος καππαδοκικοῦ τὸ ἴσον, ἁλὸς ἄνθους, στυπτηρίας σχιστῆς, φέκλης ὀπτῆς, τιτάνου ὀπτοῦ, ἀφροσελήνου, μίσεως ὠμοῦ καὶ ὀπτοῦ, καὶ νίτρου καὶ ἁλὸς πρὸς (f. 149 r.) τὸ ἥμισυ ἑκάστου ἁλὶ θαλασσίῳ ( ? ) ἐν ἡλίῳ ἡμέρας\footnote{M, à la marge inf. : κατάβασιν : κατάσπασιν (main du 15\textsuperscript{e} siècle). --- ἁλὶ suivi du signe de l'eau de mer (f. l. θαλασσίων ὑδάτων, \emph{M. B.}). --- ἡλίῳ] signe du soleil, M, devenu un θ dans E ; ἐν ἐννέα ἡμέραις ἀνίσοις Lb seul.} ἀνίσους, ἕως γένηται ἄκαυστον. Ἔπειτα λύσον αὐτὰ ὕδατι θείῳ, ἕως ἀκαυστωθῇ, λευκῷ λέγω τῷ δι ᾽ ἀσβέστου ἀπολελυμένου · καὶ ποιήσας ἄκαυστον ἔχεις, ἐκ τούτου μίσγεις τῇ μνᾷ μνᾶν ἡμίσειαν, ὕδατος θείου τὸ ἀρκοῦν.

16. Τὸ δὲ ὕδωρ τοῦ θείου τὸ δι ᾽ ἀσβέστου οὕτω γίνεται. Πάντα τὰ ὕδατα τοῦ καταλόγου ἐξ ἴσου συμμίξας, πρόσβαλε γᾶς λευκὰς,\footnote{μίξας B etc.} ἵνα σφοδρὸν λευκὸν γένηται · καὶ βαλὼν ἐν χύτρᾳ, ἐπίθες τὸ ὄργανον ὑποκαίων, καὶ λάμβανε τὸ στάζον · ἐκ τούτου χρῶ εἰς τὴν λείωσιν τοῦ θείου καὶ εἰς τὴν ἕψησιν τοῦ συνθέματος.

17. Τὸ δὲ ξανθὸν θεῖον οὕτω ποίησον. Θείου, ἀρσενίκου, σανδαράχης,\footnote{θείου] Λάβε τὸ ἴσον θείου Lb. --- ἀρσ., σανδαρ.] en signe dans M ; signe de l'arsenic redoublé BAKE ; ἀρσενικοῦ ἑκατέρου Lb.} κιννάβαρεως, σώρεως, χαλκάνθου, χαλκίτου, μίσεως, στυπτηρίας, νίτρου, ἅλατος, κυανοῦ ἀρμενίου · τοῦτο προταριχευθὲν\footnote{ἀρμενείου M, et plus loin ἀρμένειον ; ἀρμενικοῦ B etc. ; κυανοῦ, ἀρμ. Lb.} λείου ὄξει ἐν ἡλίῳ ἀνίσους ἡμέρας. Ἐκ τούτου τοῦ θείου βάλλεις\footnote{ἐν ὄξι ἐν ἐννέα ἡμέραις ἀνίσοις Lb seul. --- βάλλεις] μίσγεις E ; βαλεῖς Lb.} τῇ μνᾷ μνᾶν ἡμίσειαν.

18. Τὸ δὲ ὕδωρ τοῦ θείου τὸ ἄθικτον οὕτω γίνεται · τὰ δὲ ὕδατα τοῦ καταλόγου ἐξίσου · καὶ γῆ ποντικὴ καὶ ἀττικὴ καὶ ἀρμένιον καὶ\footnote{ἐξ ἴσου γίνεται · λάμβανε γῆν etc. (accusatifs) Lb. --- καὶ] τοῦ Lb.} βοτάναι, δηλονότι τοῦ κρόκου καὶ ἐλυδρίου τὸ διπλοῦν · ἐπίθες εἰς χύτραν, καὶ ἑνώσας τὸ ὄργανον, λάμβανε τὸ ὕδωρ ἐκ τούτου καὶ τὸ θεῖον\footnote{τὸ ὕδωρ αὐτοῦ < ἕως > οὗ τὸ θεῖον B etc.} ἀκαυστοῖς · καὶ ποτίζεις τὸ σύνθεμα μετὰ κόμμεως, καὶ ὑδραργύρου καὶ θείου ὕδατος, ὡς προεῖπον, πρὸς ἥμισυ · καὶ πήξας ἐν θερμοσποδιᾷ ἕως τὸ ὕδωρ ὅλον ἀναλωθῇ, ὄπτα ἡμέρας βʹ ἢ γʹ, ἕως ξανθήσῃ εἰς ὑπερβολήν\footnote{ἀναδοθῇ B etc. --- ξανθήσει M ; γένηται ξανθὸν B etc.} · καὶ ἐξενέγκας ἔτι ζέον, κατάβαπτε εἰς τὸ τοῦ φαρμάκου λείψανον, καὶ ἔα κάτω ἡμέρας ἀνίσους, ἕως ἰωθῇ. Καὶ οὕτω ξηράναντες\footnote{ξηράνας κ. λειώσας ἔχεις Lb. --- Lb seul omet la suite jusqu'à βάπτουσιν.} καὶ λείωσαντες, ἔχουσιν · ἐκ τούτου μίσγουσιν ἀργύρῳ κοινῷ, καὶ βάπτουσιν. Τινὲς δὲ ἰώσαντες καὶ εἰς ἱππείαν κόπρον χωννύουσιν.

19. Ἀποδέδεικται οὖν πάντα τὰ εἴδη κοινὰ ἅμα τοῖς ζωμοῖς, πλὴν < ὅτι > λευκαινόμενα, λευκαίνουσιν, καὶ ξανθούμενα, ξανθοῦσιν. Ἰστέον μὲν ὅτι μετὰ τὸ τελειωθῆναι τῷ συνθέματι συμμίσγεις · τάχα οὐ\footnote{Réd. de Lb : μετὰ τὸ τελ. τὰ συνθήματα συμμιγέντα, τάχα τοῦ θείου τούτου τὸ βαπτ. ποιοῦσι, περὶ οὗ ...} τοῦτο τὸ βαπτικώτερον, περὶ οὗ θείου οὐδεὶς ἀπεσιώπησεν. Μάλιστα ὁ (f. 149 v.) Ἀγαθοδαίμων ἔλεγεν · « Λάμβανε θεῖον ποτὲ μὲν\footnote{Signe de θεῖον M ; om. B etc.} λευκὴν, καὶ ἄλλοτε ξανθὴν, καὶ ἄλλοτε μέλαιναν, καὶ ἄλλοτε λευκὴν\footnote{λευκὴν etc. (féminin partout). Il faudrait le neutre.} ἀμετάτρεπτον, καὶ ἄλλοτε ξανθὴν ἀμετάτρεπτον. » Ἀποδέδεικται οὖν, ὡς εἴρηται, πάντα τὰ εἴδη κοινὰ ἅμα τοῖς ζωμοῖς, πλὴν ὅτι λευκαινόμενα, λευκαίνουσι, καὶ ξανθούμενα, ξανθοῦσιν.

\bigskip
\centerline{\EightStarTaper}
\centerline{\EightStarTaper\EightStarTaper}
\bigskip

\subsubsection[3. --- 17. ΠΕΡΙ ΤΟΥ ΤΙ ΕΣΤΙΝ ΚΑΤΑ ΤΗΝ ΤΕΧΝΗΝ ΟΥΣΙΑ ΚΑΙ ΑΝΟΥΣΙΑ.]{3. --- 17. ΠΕΡΙ ΤΟΥ ΤΙ ΕΣΤΙΝ ΚΑΤΑ ΤΗΝ ΤΕΧΝΗΝ ΟΥΣΙΑ ΚΑΙ ΑΝΟΥΣΙΑ.\footnote{καὶ ἀνούσια] καὶ τίνα ἀνούσια B etc.}}
\paragraph{}
\emph{Transcrit sur} M, f. 149 v. --- \emph{Collationné sur} B, f. 132 v. ;--- \emph{sur} A, f. 122 r. ;--- \emph{sur} K, f. 115 v., \emph{puis} 22 r. ;--- \emph{sur} E, f. 57 v. ;--- \emph{sur} Lb, p. 213. --- \emph{Les variantes de} M, \emph{par rapport à} BAK, \emph{ont été reportées en marge de} K. --- \emph{Chapitre} 40 \emph{de la compilation du Chrétien dans} E Lb.

\bigskip

1. Οὐσίας ἐκάλεσεν ὁ Δημόκριτος τὰ τέσσαρα σώματα · χαλκὸν ἔλεγε καὶ σίδηρον καὶ κασσίτερον καὶ μόλυβδον. Πάντες ἐπιβάλλουσιν ἐν ταῖς δυσὶ βαφαῖς. Πᾶσαι αἱ οὐσίαι ἐν ταῖς δυσὶ βαφαῖς.\footnote{Πᾶσαι αἱ οὐσίαι ἐν τ. δ. --- βαφαῖς om. B etc. F. del.} Πᾶσαι αἱ οὐσίαι κατεγνώσθησαν παρ ᾽ Αἰγυπτίοις ἀπὸ μόνου τοῦ μολύβδου πεποιημέναι · ἐκ γὰρ τοῦ μολύβδου καὶ τὰ ἄλλα τρία σώματα\footnote{ἐκάλεσαν B etc.} γεγόνασιν. Οὐσίας οὖν ἐκάλεσεν τὰ σώματα τὰ ὑφιστάμενα πυρὶ, τὰ δὲ μὴ ὑφιστάμενα, ἀνούσια. Τὰ γὰρ ἀνούσια καλῶς ἐνεργοῦσι χωρὶς πυρός. Ἔλεγε γὰρ δι ᾽ ἄγγους καὶ πρίσματος γίνεσθαι,\footnote{πρίσματα M.} τὸ δὲ ἀληθὲς λείψανον τοῦ φαρμάκου, χωρὶς πυρὸς, ἐκεῖ καὶ βεβαιώσει\footnote{βεβαιώσει] F. l. βεβαίως.} λευκαίνωσι, ξανθῶσι. Ἡ γὰρ τοῦ πυρὸς [τίει] εἴσκρισις τοῦ φθαρτοῦ\footnote{τίει om. B etc. --- εἰσκρίσεις M.} φαρμάκου ἐκ τῶν φώτων διαμαρτάνει μολυβδοχάλκου ξάνθωσις · ὅτι\footnote{φθαρτικὴ τῷ φαρμάκῳ B etc. --- διαμαρτάνη M.} ὃν ἀναιρεῖ · ἐκεῖ δὲ οὐ δεῖ ἀμαρτῆσαι. Ὅτι δὲ ἐπὶ τούτου εἴρηκεν,\footnote{ὅτι ὃν ἀν.] ἐπεὶ ἀναιρεῖ B etc.} βλέπε πῶς αὐτὸς εἶπε · « Ποίησον γλοιῶδες · χρῖσον τοῦ φαρμάκου τὸ\footnote{γλυῶδες M. --- Dernier mot du f. 115 de K ; la suite est au f. 22.} ἥμισυ ὑποκαίεσθαι, καὶ καταβάπτεις τὸ τοῦ φαρμάκου λείψανον · ὅτε\footnote{ὥστι ἀποκαίεσθαι Lb. --- ὅτι K, f. mel.} χωρὶς πυρὸς μένειν εἰάθη.\footnote{ὅταν ... ἐαθῇ Lb. F. l. ὅτι ... εἰώθει.}

2. Καὶ ἀνούσια τὰ θειώδη τὰ μὴ ὑφιστάμενα τῷ πυρί · οἱ δὲ ζωμοὶ ποιοῦσιν αὐτὰ ὑφίστασθαι τῷ πυρὶ καὶ πυρομαχεῖν · ὕδωρ γὰρ\footnote{περιμαχεῖν, corrigé en πυριμαχεῖν E ; ὑφ. καὶ πυριμαχεῖν Lb, f. mel.} ἐναντίον πυρός. Διὰ τοῦτό φησιν. « Ἡ φύσις λαβοῦσα τὸ ἴδιον ὡς τοὐναντίον, ἰσχυρὰ καὶ ἀδίωκτος γίνεται, κρατοῦσα καὶ κρατουμένη. Διὰ τοῦτο οὖν ὡς ἴδιον μὲν καὶ αὐτὸ (f. 150 r.) θειῶδες ἀφ ᾽ οὗ\footnote{M, à la marge supérieure du f. 150 r. : ἤγουν θερμου ( ? ), d'une main du 15\textsuperscript{e} siècle.} καὶ ὕδωρ θείου ἀθίκτου κέκληται · διατί καὶ τοὐναντίον, ἐπειδήπερ\footnote{διότι Lb, f. mel.} τοὐναντίον ὕδωρ πυρός ; ἐπιρρέον γὰρ ὡς ὕδωρ οὐκ ἐᾷ ἐκεῖνα πυρώδη ὄντα ἐξῃθαλῶσθαι καὶ φεύγειν · ἀλλὰ θάπτει αὐτὰ τῇ ὑγρότητι, καὶ κατέχει\footnote{θάπτει] βάπτειν AK θάπτει καὶ βάπτει E ; βάπτει Lb.} ἕως βάπτωσιν. Καὶ ὕδωρ μὲν κατέρχεται διὰ τὸ ὑγρὸν εἶναι.\footnote{καὶ ὕδωρ θεῖον μὲν Lb. --- τακέρχεται] F. l. κατέχεται.} Διὰ τοῦτο γάρ φησιν · « Ἡ φύσις λαβοῦσα τὸ ἴδιον ὡς τοὐναντίον, » καὶ τὰ ἑξῆς. Ἐρρέθη πῶς ὑφίστανται τῷ πυρὶ διὰ τῶν ζωμῶν · οἱ δὲ ζωμοὶ ὕδωρ θεῖόν εἰσιν.

\bigskip
\centerline{\EightStarTaper}
\centerline{\EightStarTaper\EightStarTaper}
\bigskip

\subsubsection{3. --- 18. ΠΕΡΙ ΤΟΥ ΟΤΙ ΠΑΝΤΑ ΠΕΡΙ ΜΙΑΣ ΒΑΦΗΣ Η ΤΕΧΝΗ ΛΕΛΑΛΗΚΕΝ.}
\paragraph{}
\emph{Transcrit sur} M, f. 150 r. --- \emph{Collationné sur} B, f. 133 v. ;--- \emph{sur} A, f. 122 v. ;--- \emph{sur} K, f. 22 r. ;--- \emph{sur} E, f. 58 v. ;--- \emph{sur} Lb, p. 217. --- \emph{Les variantes et restitutions de} M \emph{ont été reportées en marge de} K. --- \emph{Chapitre} 41 \emph{de la compilation du Chretien dans} E Lb.

\bigskip

1. Ἑρμῆς καὶ Δημόκριτος ἀπὸ τοῦ καταλόγου γινώσκονται ὅτι περ πάντα περὶ [ἑνὸς καὶ] μιᾶς βαφῆς εἰρήκασιν διὰ συντόμου,\footnote{ἑνὸς καὶ om. B etc. --- εἰρήκασι] λελαλήκασι B etc.} καὶ οἱ ἄλλοι ῇνίξαντο. Ἀμέλει γοῦν καὶ Ἀφρικανός φησι · « Τὰ ὑπάγοντα εἰς τὴν βαφὴν μέταλλα, καὶ ὑγρὰ καὶ γαῖ καὶ βοτάναι. » Χύμης δὲ καλῶς ἀπεφήνατο · « Ἓν γὰρ τὸ πᾶν · καὶ δι ᾽ αὐτοῦ τὸ πᾶν γέγονεν · ἓν τὸ πᾶν · καὶ εἰ μὴ πᾶν ἔχοι τὸ πᾶν, οὐ γέγονε τὸ\footnote{οὐ γέγονε --- βάλλειν τὸ πᾶν om. AK, hab. BELb.} πᾶν · δεῖ σε οὖν τοῦτο βάλλειν τὸ πᾶν, ἵνα ποιήσῃς τὸ πᾶν. » Πηβίχιος διὰ τῶν τεσσάρων σωμάτων. Μαρία διὰ τοῦ πετάλου\footnote{Πηβήχιος B etc. --- σωμάτων om. M.} τῆς κηροτακίδος. Ἀγαθοδαίμων · « Μετὰ τὴν τοῦ χαλκοῦ ἐξίωσίν τε καὶ ἐξίσχνωσιν καὶ μέλανσιν, εἶτα λεύκωσιν, τότε ἔσται βεβαία ξάνθωσις. » Ὁμοίως καὶ τὰ ἄλλα πάντα παρ ᾽ αὐτοῖς φημιζόμενα.

2. Ὅταν οὖν λέγῃ Μαρία περὶ τοῦ αὐτοῦ, φησί · « Πολλὰ γὰρ\footnote{λέγῃ] λέγωσι M. --- Réd. de B etc. : ὅταν οὖν καὶ ἡ Μαρία λέγῃ περὶ τούτου, φησί.} ἔχει σώματα ἀπὸ μολύβδου ἕως χαλκοῦ. » Ὅταν δὲ λέγῃ διπλωσίδια, περὶ τούτου λέγει · « Δύο γὰρ αὐτὰ βολαί εἰσιν, ποτὲ ἀργυροχαλκοῦ\footnote{ἐπιβολαὶ B etc. Réd. de Lb seul : ποτὲ μὲν ἄργυρος, π. δὲ χρυσὸς καὶ ἄργυρος, π. δὲ μολυβδόχαλκος.} ποτὲ χρυσαργύρου, ποτὲ μολυβδοχάλκου, ὁμοίως καὶ ἄλλα πάντα νοοῦνται. Περὶ δὲ ἄρσεως ἀργύρου εἰ πάντροπον, ἤ μελάνσεως\footnote{εἰ πάντροπον] εἶπον (ὡς εἶπον ELb) πρότερον B etc.} · ὅτι διὰ πάντα παρ ᾽ αὐτοῖς ἔλεγεν, ἡ Μαρία μόνη ἀπέκραξεν, λέγουσα\footnote{μόνη γὰρ ἀπέκραξεν Lb.} « Ὅτι ἐὰν λέγω χαλκὸν, ἢ μόλυβδον, ἢ σίδηρον, τὸν ἰὸν λέγω. »

\bigskip
\centerline{\EightStarTaper}
\centerline{\EightStarTaper\EightStarTaper}
\bigskip

\subsubsection[3. --- 19. ΠΕΡΙ ΤΟΥ ΤΡΟΦΗΝ ΕΙΝΑΙ ΤΑ Δʹ ΣΩΜΑΤΑ ΤΩΝ ΒΑΦΩΝ · ΕΙΣΙΝ ΔΕ.]{3. --- 19. ΠΕΡΙ ΤΟΥ ΤΡΟΦΗΝ ΕΙΝΑΙ ΤΑ Δʹ ΣΩΜΑΤΑ ΤΩΝ ΒΑΦΩΝ · ΕΙΣΙΝ ΔΕ.\footnote{εἰσὶν δὲ οὕτως AKE ; om. Lb.}}
\paragraph{}
\emph{Transcrit sur} M, f. 150 v. --- \emph{Collationné sur} B, f. 134 r. ;--- \emph{sur} A, f. 123 r. ;--- \emph{sur} K, f. 22 v. ;--- \emph{sur} E, f. 59 r. ;--- \emph{sur} Lb, p. 227. --- \emph{Les variantes et restitutions de} M \emph{ont été reportées en marge de} K. --- \emph{Chapitre} 42 \emph{de la compilation du Chrétien dans} E Lb.

\bigskip

1. Τὸν χαλκὸν ἡ Μαρία φάσκει βάπτεσθαι πρῶτον, καὶ οὕτω\footnote{φάσκει] φησὶ B etc.} βάπτειν. Ὁ χαλκὸς αὐτῶν τὰ δʹ σώματα. Αἱ οὖν βαφαὶ αὗται · εἴδη δὲ τοῦ καταλόγου στερεὰ καὶ ὕγρα, βοτάναι · στερεὰ μὲν ἀπὸ νεφέλης ἕως χρυσοκόλλης, ὑγρὰ δὲ πάντα τοῦ καταλόγου · τὸ δὲ\footnote{πάντα τὰ εἴδη τοῦ καταλόγου Lb.} ἀληθὲς, ὕδωρ θεῖον.

2. Ὥσπερ οὖν ἡμεῖς ἀπὸ στερεῶν καὶ ὑγρῶν τρεφόμεθα καὶ βαπτόμεθα\footnote{καὶ ὑγρῶν --- ἀπὸ στερεῶν om. B etc.} ποιότητι μόνον, οὕτω καὶ ὁ χαλκὸς αὐτῶν · καὶ καθάπερ ἀπὸ στερεῶν μόνων οὐ τρεφόμεθα ἢ ὑγρῶν, οὕτως οὐδὲ ὁ χαλκός.\footnote{ἢ] ἀλλὰ καὶ L.} Καθάπερ γὰρ ἡμεῖς τὸ στερεὸν μόνον δεξάμενοι φλεγόμεθα καὶ ἐκκαιομεθα, καὶ φαρμακευόμεθα, καὶ ὁ χαλκὸς αὐτῶν. Πάλιν ἐὰν ἀπὸ\footnote{καὶ πάλιν ὥσπερ Lb.} τοῦ μόνου δεξώμεθα ποτοῦ, μεθύομεν καὶ καρηβαροῦμεν, καὶ περὶ τὰς\footnote{δεξάμενοι Lb. --- M mg. : groupe de points noirs --- F. l. δαισώμεθα.} παρείας βαπτόμεθα, καὶ ἐμοῦμεν. Καὶ ὁ χαλκός · χρωσθεὶς γὰρ καθάπερ\footnote{οὕτω καὶ ὁ χαλκός Lb. --- F. l. χρωϊσθεὶς.} ὁ χρυσὸς ἐκ τοῦ ὕδατος τοῦ θείου, καρηβαρεῖ καὶ ἐμεῖ, καὶ εὐθέως φεύγει. Ὥσπερ οὖν ἡμεῖς δεξάμενοι συμμέτρως ἀμφοῖν τὴν\footnote{ἀμφὺν. M} τῶν στερεῶν καὶ ὑγρῶν τροφὴν, τρεφόμεθα κατὰ λόγον, καὶ αἱ παρείαι\footnote{καταλόγον A.} βάπτονται κατὰ λόγον, καὶ ἡ θρεπτικὴ ἡ δύναμις διανέμει ἐν τῷ στομάχῳ τὴν τροφὴν διὰ τῆς καθεκτικῆς δυνάμεως, οὕτως καὶ ὁ χαλκὸς λαβὼν τὰ στερεὰ ἀντὶ τροφῆς τοῦ ὕδατος θείου μετὰ κομμεως,\footnote{τοῦ θείου Lb seul.} ἀντὶ οἴνου τρέφεται καὶ χρωΐζεται διὰ τῆς ἐν αὐτῷ καθεκτικῆς δυνάμεως. Καὶ ὧδε δὲ τῷ ῥηθέντι εἶπε · « τὰ θειώδη ὑπὸ τῶν\footnote{Καὶ ὧδε δὲ] Οὕτω δὴ καὶ ἐνταῦθα B etc.} θειωδῶν κατέχεται · » τὸ δὲ ἀληθὲς · « Ἡ φύσις τὴν φύσιν τέρπει,\footnote{κατέχεται] κρατοῦνται καὶ κατέχονται Lb. --- F. l. τόδε.} καὶ νικᾷ, καὶ κρατεῖ. »

3. Καθάπερ, φησὶν, ὁ ἄνθρωπος ἐκ τῶν δʹ στοιχείων, οὕτω καὶ ὁ χαλκός · καὶ ὥσπερ οὗτος ἐξ ὑγρῶν καὶ στερεῶν καὶ πνεύματος\footnote{οὗτος ... ] ὁ ἄνθρωπος ἐκ στερεῶν καὶ ὑγρῶν καὶ πνεύματος σύγκειται Lb.} σύγκειται · οὕτω καὶ ὁ χαλκός · πνεῦμα δὲ τὴν νεφέλην ὁ Ἀπόλλων ἐν τοῖς χρησμοῖς λέγει ·
\begin{quotation}
... καὶ πνεῦμα μελάντερον, ὑγρὸν, ἄχραντον.\footnote{ἄχρανθον M. Cp. p. 150, l. 11.}
\end{quotation}
\paragraph{}
4. Περὶ τῆς νεφέλης καλῶς ἐρρέθη παρὰ τῆς Μαρίας · « Ὁ χαλκὸς\footnote{Περὶ δὲ τῆς νεφέλης Lb.} οὐ βάπτει, ἀλλὰ βάπτεται, καὶ ὅ- (f. 151 r.) ταν βαφῇ, τότε βάπτει · καὶ τρεφόμενος τρέφει, καὶ τελειωθεὶς τελεοῖ. » Ἔρρωσο.\footnote{ἔρρωσο om. B etc.}

\bigskip
\centerline{\EightStarTaper}
\centerline{\EightStarTaper\EightStarTaper}
\bigskip

\subsubsection[3. --- 20. ΠΕΡΙ ΤΟΥ ΧΡΗΣΤΕΟΝ ΣΤΥΠΤΗΡΙᾼ ΣΤΡΟΓΓΥΛῌ ΑΝΤΙΛΟΓΟΣ.]{3. --- 20. ΠΕΡΙ ΤΟΥ ΧΡΗΣΤΕΟΝ ΣΤΥΠΤΗΡΙᾼ ΣΤΡΟΓΓΥΛῌ ΑΝΤΙΛΟΓΟΣ.\footnote{Titre dans Lb seul : περὶ τοῦ χρηστέον τῇ κινναβάρει.}}
\paragraph{}
\emph{Transcrit sur} M. f. 151 r. --- \emph{Collationné sur} B, f. 135 r. ;--- \emph{sur} A, f. 123 v. ;--- \emph{sur} K, f. 22 v. ;--- \emph{sur} E, f. 60 r. ;--- \emph{sur} Lb, p. 225. --- \emph{Les νariantes de} M \emph{ont été reportées en marge de} K. --- \emph{Chap.} 43 \emph{de la compilation du Chrétien dans} E Lb.

\bigskip

1. Ἔγνως ὅτι ἓν τὸ πᾶν, καὶ τοῦ παντὸς γέγονεν τὸ πᾶν. Ἰστέον δὲ καθὼς ἀπεδείξαμεν ἐν τοῖς προτέροις μου ὑπομνήμασιν ὅτι πάντα\footnote{καὶ ἐκ τοῦ παντὸς Lb.} ὑφ ᾽ ἓν γενόμενα ἕν τι τῶν σωμάτων καλοῦσι, μάλιστα τὸν χαλκὸν · καὶ σῶμα μαγνησίας φάσκουσιν οἱ φιλόσοφοι. Οὐ μόνον δὲ νεφέλη ποιεῖ τὸν χαλκὸν ἀσκίαστον, ἀλλὰ καὶ ὁ χαλκὸς ἀπεδείχθη τὰ ὅλα · ὥσπερ καὶ σῶμα τῆς μαγνησίας ἄρα μετὰ ὅλων πήγνυται. Λαβὼν γὰρ, φησὶν, ὑδράργυρον, πῆξον τῷ τῆς μαγνησίας σώματι. Ἆρα οὖν τὴν νεφέλην ζητοῦμεν ἀναλαβεῖν τὸ πᾶν, ἵνα οὕτως πήξωμεν ; Πᾶσαι γὰρ αἱ γραφαὶ ἄνω καὶ κάτω · « ἀναλαβὼν νεφέλην. » Ἐμάθομεν δὲ ἐκ τῆς πείρας ὅτι εἰ μὴ χρυσὸς, καὶ ἄργυρος, καὶ κασσίτερος, καὶ μόλυβδος, καὶ ἡ νεφέλη οὐκ ἀναλαμβάνει. Καὶ λοιπὸν τί ποιοῦμεν τοὺς λίθους καὶ τὸν σίδηρον ;

2. Αἱ ἄλλαι γραφαὶ λέγουσιν · « Φάκινον δεῖ ποιεῖν τὸ πᾶν καὶ ἀναλαμβάνειν ὑδροκομίῳ. » Ἄλλοι δὲ οὕτως τὴν νεφέλην περιγίνονται\footnote{ὑδροκομμίῳ Lb. --- ἄλλαι BA.} · Ἔγωγε νομίζω βέλτιον εἶναι κιννάβαριν συλλειοῦν · πλὴν, ὡς οἶδε τις αὐτὴν γεννῶσαν δι ᾽ ἑψήσεως ἧς χρῄζει νεφέλην, καὶ οὕτως κατεργάζεται. » Καὶ γὰρ οἰκονομούμενα ἐν τῷ ἡλίῳ τὰ εἴδη ὕδατι ἢ\footnote{Après κατεργάζεται] ξ redoublé à l'encre noire M.} ὄξει νεφέλην ἀποτίκτουσιν · καὶ τοῦτο διὰ πείρας ἐπιστάμεθα. Καὶ πᾶσαι αἱ γραφαὶ καὶ Χύμης καὶ ἡ Μαρία φησίν · θυεία μολιβδίνη καὶ\footnote{Réd. de Lb, d'après les corr. de E : θυείᾳ μολυβδίνῃ καὶ δοίδυκι μολυβδίνῳ τὴν ἄσβεστον καὶ τὴν κιννάβαριν καὶ ὄξος λείου. --- Χήμης E.} δοίδυξ μολίβδινος · κιννάβαριν ὄξος λύει ἐν ἡλίῳ ἕως γένηται νεφέλη\footnote{λύει] λείου B etc.} · ὁμοίως καὶ ἐπὶ κασσιτέρου πάλιν τὸ αὐτό · πάλιν δὲ ἑψόμενα ἤτοι\footnote{κασσιτέρου] ὑδραργύρου Lb seul. --- M mg. : θʹ ὅλον sur une ligne verticale.} καιόμενα ἢ πηγνύμενα ἢ βαπτόμενα, εἰώθασιν ἀναδιδόναι μάλιστα τὴν νεφέλην, ἐὰν τεχνικῶς ἑψηθῇ · καὶ ὅπερ κάμνει τις τῇ τῶν ὅλων ἀναλήψει, ταῦτα ἡ κιννάβαρις δυνάμει οὖσα, νεφέλην δρᾷ, καὶ διαβαίνει, μετὰ πάντων λειωθεῖσα.\footnote{μετὰ] κατὰ E et mg. : \emph{alias} μετὰ. --- τελειωθεῖσα Lb.}

3. Ἀλλ ᾽ ἴσως ἐρεῖ τις ὅτι βέλτιον τὴν νῦν πεπηγμένην συλλειοῦν (f. 151 v.) νεφέλῃ ἰουμένῃ, ὅτι ἁπλῆν πῆξιν αἱ γραφαὶ οὐ λέγουσιν,\footnote{νεφέλην ἰωμένην Lb.} ἀλλὰ τὴν κατὰ πάντων λευκὴν ἐπιβληθεῖσαν τῷ ἡμετέρῳ χαλκῳ\footnote{πάντα Lb, ici et l. suiv.} ποιεῖν αὐτὸν ἄσκιον ἄργυρον. Οὕτως ὁ κατὰ παντων Στέφανος, τουτέστιν καθ ᾽ ὅλων τῶν εἰδῶν τὴν ἁπλῆν φαντάζεται · εἰ δὲ καὶ\footnote{φαντάζονται M.} ἁπλῆν λέγουσιν, ἴστε πάντες ὡς οὐδὲν δρῶσιν · προσεκπνεύσασα γὰρ\footnote{λέγουσιν] ἄγουσιν M (et en marge de E, mais biffé).} διὰ τῆς πήξεως εἰς τὸ πῦρ, καὶ ἀπολέσασα τὸ πνεῦμα τὸ βαπτικὸν, οὐδὲν δρᾷ. Ἡ δὲ κιννάβαρις ἑψομένη μετὰ τῶν εἰδῶν οὐκ ἀπολεῖται\footnote{M mg. : καλὸν avec renvoi à δρᾷ. --- οὐδὲν δρᾷ καλῶς Lb, f. mel.} τὸ πνεῦμα · διωκόμενον γὰρ αὐτῆς τὸ πνεῦμα, τουτέστιν ἡ νεφέλη ὑπὸ τοῦ πυρὸς, καὶ ἀναδιδομένη εἰς φυγὴν κατέχεται ὑπὸ τῶν συγγενῶν καὶ διωκόντων αὐτὴν σωμάτων, μάλιστα τοῦ κασσιτέρου.\footnote{κασσιτέρου] Ἐρμοῦ B etc. --- μάλιστα τοῦ κασσιτέρου. Ἔχομεν] Μάλιστα δὲ τοῦ Ἑρμοῦ · ἐχ. Lb seul.}

4. Ἔχομέν τινα συνηγοροῦντα, ἃ δεῖ χρήσασθαι στυπτηρίᾳ στρογγύλῃ\footnote{M mg. : ἐψ. avec renvoi à στυπτ. στρ. --- ἃ] F. l. ὅτι.} ἀντι νεφέλης. Καὶ ἡ Μαρία συνηγορεῖ λέγουσα · « Αἱ δὲ χύσεις τῶν καταβαφῶν γίνονται ἐν ληκυθίοις χλωροῖς, τὸ πῦρ ἐκ προσαγωγῆς. » Ἡ δὲ κάμινος φουρνοειδὴς, ἔχουσα ἄνω τοὺς μαζούς. Ἐὰν δὲ μὴ εὐπορήσῃς, βάλε στυπτηρίας στρογγύλης τὸ διπλοῦν, ἤγουν κινναβάρει χρωϊσάμενον,\footnote{M mg. : ·\emph{b}· en noir, et πψ puis le signe de στυπτ. στρ. en gris.} τὸ αὐτὸ δρᾶσαι κάλλιον · ἐπειδὴ μετὰ ἄλλων φακινίνων καὶ\footnote{φακίνων Lb.} εὐεργές. Ἡ γὰρ νεφέλη ἀναλαβοῦσα μόνον τὰ δʹ σώματα. Λέγουσι γὰρ\footnote{ἐνεργὲς AΚELb ; ἐνεργὲς γίνεται Lb.} τινες ὅτι καὶ ἐκ τῶν ἄλλων σωμάτων ἀναλαμβάνεται, καὶ μάλιστα τῆς χρυσοκόλλης · ἐγὼ δὲ οἶδα ὅτι μόνον χρυσόκολλα οὐκ ἀναλαμβάνει, ἀλλὰ τάχα οὐδὲ ζῶντα καὶ ἐκλειωθέντα τὰ σώματα πάντα φέρουσι τὴν νεφέλην.

5. Ὅτι παρὰ Ἀγαθοδαίμονος εἴρηται ὅτι ἡ χρυσόκολλα καὶ ἡ νεφέλη φίλαι ἀλλήλων εἰσίν · καὶ ἀναλαμβάνει αὐτήν. Καὶ ἡ μὲν ὡς τὰ ῥινίσματα [φίλαι ἀλλήλων], ἡ δὲ οὐδὲ διὰ τῆς συλλειώσεως τῆς κιννιβάρεως\footnote{φίλα E. --- M mg. : ÷. --- λειώσεως Lb. --- κινναβαρώσεως BAK ; κινναβαριωσέως Lb.} ἔχει τὴν φιλίαν. Ἀμφότερα γὰρ ξηρὰ ὄντα συλλειοῦνται, καὶ κατὰ τοῦτο φίλαι εἰσίν. Πάλιν δὲ, δυνάμει οὖσα, νεφέλη τὸν δυνάμει χαλκὸν ἀπεργάζει · καὶ εὑρίσκονται φίλαι.\footnote{ἀπεργάζεται B, etc.}

6. Δεῖ δὲ ζητεῖν ὅπως τὰ (f. 152 r.) πάντα ἀναλήψεται ἡ νεφέλη, οὐ μόνον ζῶντα λελησμένα σώματα, ἀλλὰ καὶ κεκαυμένα. Καὶ γὰρ\footnote{ἀλλελησμένα B etc. F. l. λελειωμένα, délayés, dissous (\emph{M. B.}).} τῇ ἀληθείᾳ καὶ μέταλλα ἀναλαμβάνει, μάλιστα ὅσα χαλκοῦ γένεσιν ἔχουσι. Εἰ δὲ οὐκ εὐπορεῖς, βάλλε κινναβάρεως τὸ διπλοῦν · πάντων δὲ εὐπορία. Τὸν δὲ καὶ νοῦν ὁ φιλόσοφος αἰνίττεται. Δεῖ σε οὖν πάντα\footnote{τὸν δὲ νοῦν καὶ B etc.} ἐπινοεῖν, ἐν πρώτοις δὲ μὴ ἀργεῖν ἀπὸ τῆς τέχνης · ἡ γὰρ μελέτη ἐπὶ τὴν ἀληθινὴν ὁδὸν ἄγει. Ταῦτα δέ μοι λέλεκται, δεῖξαι βουλομένῳ\footnote{λέλεχθαι (\emph{sic}) M.} ὅτι τάχα καὶ ἡ στυπτηρία στρογγύλη ὁμοίως δρᾷ, καθὼς εἶπε μάλιστα\footnote{ἡ] ὁ M.} καὶ ἡ θεία Μαρία.

\bigskip
\centerline{\EightStarTaper}
\centerline{\EightStarTaper\EightStarTaper}
\bigskip

\subsubsection[3. --- 21. ΠΕΡΙ ΘΕΙΩΝ.]{3. --- 21. ΠΕΡΙ ΘΕΙΩΝ.\footnote{Titre dans B etc. : περὶ τῶν θείων ὑδάτων.}}
\paragraph{}
\emph{Transcrit sur} M, f. 152 r. --- \emph{Collationné sur} B, f. 136 v. ;--- \emph{sur} A, f. 125 r. ;--- \emph{sur} K. f. 23 ;--- \emph{sur} E, f. 62 v. ;--- \emph{sur} Lb, p. 233. --- \emph{Les variantes et restitutions de} M \emph{ont été reportées en marge de} K. --- \emph{Chap.} 44 \emph{de la compilation du Chrétien dans} E Lb.

\bigskip

1. Οὐκ ἐμὲ ἐπηρώτησας τὸν περὶ θείων λόγον, μέχρι τῆς σήμερον εὐορκοῦσα ; Σοὶ οὖν ὁ λόγος καιρίως λεχθήσεται · ἔγνως γὰρ ὡς οὐ\footnote{ἐνορκοῦσα Lb. --- κυρίως AKELb.} μόνον ὁ φιλόσοφος θείων ἐμνημόνευσεν, ἀλλὰ καὶ πάντες οἱ προφῆται · ἄνευ γὰρ αὐτῶν οὐδὲν ἔσται, τουτέστιν ἄνευ τοῦ θείου ὕδατος.\footnote{Τὸ γὰρ ὅλον σύνθεμα --- πᾶν σῶμα βάπτεις] Cp. 3, 10, 2, p. 144. --- Ligne verticale en marge de Lb jusqu'à πᾶν σῶμα βάπτεις.} Τὸ γὰρ ὅλον σύνθεμα δι ᾽ αὐτοῦ ἀναλαμβάνεται, καὶ δι ᾽ αὐτοῦ ὀπτᾶται, καὶ δι ᾽ αὐτοῦ καίεται, καὶ δι ᾽ αὐτοῦ πήγνυται, καὶ δι ᾽ αὐτοῦ βάπτεται, καὶ δι ᾽ αὐτοῦ ἰοῦται, καὶ δι ᾽ αὐτοῦ ἐξιοῦται. Φησὶν γὰρ · « Ἐπίβαλλε ὕδωρ θείου ἀθίκτου καὶ κόμμι ὀλίγον, πᾶν σῶμα βάπτεις.\footnote{F. l. Ἐπιβάλλων. Cp. p. 145, l. 3.} » Τὸ δὲ αὐτὸ ἄκουε · « Ἔα κάτω καὶ γίνεται · τὸ γὰρ ἐμφανὲς\footnote{Ἔα κάτω. καὶ γίνεται.] Cp. Stephanus, p. 247, éd. Ideler.} μυστήριον τοῦτο. » Ἀλλ ᾽ ἐρεῖ τις · τί ὅμοιον ὕδατι θείῳ θειωδῶν,\footnote{τοῦτό ἐστιν Lb. --- Réd. de Lb : Τί ὅμοιον θεῖα καὶ θειώδη καὶ ὕδατι θείου.} καὶ ὕδωρ θεῖον ; πρὸς ὃν πρῶτον ἐροῦμεν ὅτι ποτέ τίς ἀπὸ θείων\footnote{πότε τίς B etc., f. mel.} ἐποίησέν τι μετὰ ἄλλων · ἐπειδὴ δὲ οὐδὲν ἐποίησεν, δικαίως ὁ ἐμὸς\footnote{μετ ᾽ ἄλλων (ἄ sur grattage) A ; μέταλλον Lb.} φιλόσοφος αὐτὰς οὐ παρέλαβεν, καθὼς ἡμῖν νοεῖται.\footnote{αὐτὰ Lb, f. mel. --- καθ ᾽ ὧν MBAK ; καθ ᾽ ὃν ELb. Corr. conj.}

2. Λοιπὸν θεῖον καλεῖται τὸ ὕδωρ τοῦ θείου · ἄκουε. Λέγεται θείον ἡ κάτωθεν ἄνω ἀναπεμπομένη αἰθάλη, ἔνθεν καὶ τὴν τέφραν τὴν\footnote{αἰθάλη om. M ; souspointillé dans K.} γινομένην ἐν τοῖς τοίχοις τοῖς καπνιζομένοις θεῖον καλοῦσιν · ὁμοίως\footnote{στοίχοις M.} τὰς ῥαθάμιγγας τὰς ἀποπιπτούσας ἀπὸ τῶν λοετρῶν, καὶ τὰς εἰς τὰ πώματα τῶν λεβήτων ἑστη- (f. 152 v.) κυίας σταγόνας, θεῖα καλοῦσιν\footnote{λεκήτων M (confusion du β et du κ, fréquente aux 10\textsuperscript{e} et 11\textsuperscript{e} siècles).} · καὶ πάλιν ἀπὸ πυρὸς κάτωθεν ἀναπεμπόμενον ἄνω, θεῖα καλοῦσιν\footnote{τὰ ἀπὸ πυρὸς ἀναπεμπόμενα Lb.} · καὶ τὴν ὑδράργυρον λευκὴν θεῖα καλοῦσιν, διὰ τὸ καὶ αὐτὸ\footnote{M mg. : ὡρ (ὡραίοτατον ? ) ἁπάντων, sur une ligne verticale et en lettres retournées. --- αὐτὸ] αὐτὴν Lb.} ἀναπέμπεσθαι.

3. Εἰώθασιν δὲ οἱ ἀρχαῖοι λεπτῷ πυρὶ καὶ λευκανθίοις ἀκαυστοῦν\footnote{λευκανθίαις Lb. F. l. ληκυθίοις (\emph{M. B.}).} τὰ θειώδη · ὅπερ δὲ τὸ πῦρ ποιεῖ χωρὶς φύσεως, τοῦτο ὁ ἥλιος ποιεῖ\footnote{δὲ] γὰρ B etc.} μετὰ θείας φύσεως. Καὶ ὁ Ἑρμῆς ὁ μέγας φησί · « Ἥλιος ὁ πάντα ποιῶν. » Πάλιν ὁ Ἑρμῆς πανταχοῦ ἔλεγεν · « Θὲς ἐν τῷ ἡλίῳ, καὶ · τρίβε νεφέλην ἐν τῷ ἡλίῳ · καὶ ἄνω καὶ κάτω τὸν ἥλιον σημαίνει · πάντα που δρᾷ, καὶ πῦρ ἡλιακὸν, ὡς προείπαμεν ἐν τοῖς λευκανθίοις · εἰκότως καὶ τὸ ἄλλο σύνθεμα οὕτως ζώννυται ἅλμῃ, ἕως οὗ λευκανθῇ. Καὶ κατὰ τοῦ εἰπόντος εἰς τὰ ὑπὸ κύνα καὶ τὰς ἡλιακὰς, τῶν ἀμφοτέρων πεῖρα διδάσκει. Ὥσπερ γὰρ ἡ ζύμη τοῦ\footnote{Cp. une phrase semblable, p. 145, l. 9-11.} ἄρτου ὀλίγη οὖσα τοσοῦτον φύραμα ζυμοῖ, οὕτως καὶ τὸ μικρὸν χρυσοῦ ἢ ἀργύρου πέταλον τὸ πᾶν τέλειον γίνεται ξηρίον,\footnote{τέλειον] μελισὸν M. F. l. μελλῆσον. Cp. l. c., l. 10 : μέλλει.} < καὶ > ἅπαντα ζυμοῖ. Καὶ ἐὰν ἀκούσωμεν γʹ ἢ εʹ ἢ ζʹ, τὰς ὅλας ιεʹ · οὕτως\footnote{Réd. de E : καὶ οὕτω γὰρ ποιοῦντες δοκοῦσι μὲν ἀναμαλάξαντες ... --- Réd. de Lb : οὕτω γὰρ ποιεῖν δοκοῦσι (corrigé en δοκοῦμεν) ἀναμαλάξαντες ...} ποιοῦντες δοκοῦσιν, καὶ ἀναμαλάξαντες πάντα ἐν ὑαλίνοις σκεύεσιν · τὰ γὰρ ὄστρακα καὶ παρατηρούμεθα ἐν τῇ ἰώσει, ἵνα μὴ πίῃ τὴν βαφὴν καὶ τὸ τῆς βαφῆς ἄνθος · ἅπαξ γὰρ φθάσασα κορεσθῆναι καὶ\footnote{κορευθῆναι M.} βαφῆναι ἡ δεκτικὴ αὕτη φύσις τοῦ χρυσανθίου, ἢ σκωρία χαλκοῦ\footnote{χρυσανθίου] χρυσοῦ τοῦ θείου E ; τοῦ χρυσοῦ Lb. --- σκωρίας mss.} οὐκέτι πίνει τὸ ἄνθος τῆς ἰώσεως.

4. Τῆς βαφῆς ἐκεῖ ἐν ὑάλίνοις ποιοῦμεν (ἐπειδὴ συμπάσχει τῇ\footnote{Cp. 3, 29, tout le § 15 (= $\star$).} ἰώσει), οὐ ψηλαφοῦντες χερσίν · θανατηφόρος γάρ ἐστιν, ὅτε καὶ ὁ\footnote{ψηλαφῶντες Lb, mel. --- Réd. de $\star$ : ὅτε ὑδράργυρος καὶ ἐν αὐτῷ χρυσὸς σαπῇ · ὅτι πάντων ...} χρυσὸς ἐν αὐτῷ σαπῇ, ὁ πάντων τῶν μετάλλων δηλητηριωδέστερος. Οἱ μὲν συλλειοῦσι τῷ ἰῷ ὃ μεμάθηκας, θείῳ λέγω, χρίουσιν πέταλον\footnote{ὃ μεμαθ. θεῖον · λέγω δὲ χρ. Lb.} ἀργύρου. Καὶ οὕτως ἐκ προβάσεως ὀπτοῦσιν τὸ τεχνικὸν ὄργανον καμίνῳ τῷ ἐοικότι δινιχεῖ καὶ τῷ χωνίῳ τῷ βαθμοειδεῖ, καὶ γίνεται\footnote{δίνυχι B etc. F. l. δοίδυκι.} χρυσός.

5. Τινὲς δὲ, καὶ Μαρία τῶν ὑποκάτω τοῦ ζωδίου ἐμνημόνευσαν · καὶ\footnote{τοῦ ὑποκάτω Lb. --- ζωμίου B E mg. Lb.} οὕτως ἐποίησαν, ὑδράργυρον, φησὶν, καὶ θεῖον καὶ ἰὸν λειοῦντες ὅλα ὁμοῦ ἐν ἡλίῳ, ἕως οὗ (f. 153 r.) γένηται ὅλον ὄλβιος. Καὶ λέγουσιν\footnote{ὄλβιος] ὅλον ἰὸς B etc.} ὅτι οὗτος εἰσακτικώτερός ἐστιν. Τινὲς αὐτὴν τὴν ἴωσιν μόνην ἐλείωσαν\footnote{ἴωσιν] λείωσιν B etc.} ἐν ἡλίῳ, ὡς μηδὲν βάλλοντες, ἀλλὰ φάσκοντες ἔχειν αὐτῶν τὰ\footnote{αὐτῶν] F. l. αὐτοῖς.} ζητούμενα · ἄλλοι τὸν ἰὸν ὕδατι θείῳ ἐλείωσαν, φάσκοντες αὐτὸ εἶναι θεῖον · αὐτὸ καὶ ὑδράργυρον. Καὶ μᾶλλον αὐτοὺς τῶν ἄλλων ἀπεδεξάμην. Ἄλλοι ὑδράργυρον ἔβαλον, οἱ μὲν ὠμὴν, οἱ δὲ παγεῖσαν ξανθήν.\footnote{ἄλλοι δὲ Lb.} Τινὲς δὲ μετὰ τὴν ἴωσιν οὐδὲν περαιτέρω περιειργάσαντο.

6. Οἱ φιλόσοφοι δὲ ᾐνίξαντο μετὰ τὴν ἴωσιν, λέγοντες · « Καὶ χρυσὸν καταβάπτεις, » ὥστε κάλλιον μετὰ τὴν ἴωσιν ἐνεργεῖν. Ἕτεροι δὲ τῶν ἱερογραμματέων τῶν συγγραψαμένων περὶ μόνην τὴν τέχνην, ἀσχολουμένων ἐν τῇ λειώσει, μόνην ἔφασαν τὴν ἴωσιν τὰ πάντα\footnote{ἔφησαν M.} ποιεῖν, μάλιστα καὶ ἰόν. Καὶ οὕτως αὐτοῖς ἤρεσεν. Ἄλλοι δὲ ἑψήσαντες, ὤπτησαν καὶ ἥψησαν ἐκ χώνης, οἷς τὸ πᾶν τῆς λειώσεως\footnote{ἐν χωνείοις l.b.} ἤρεσεν · οἷς οὖν λείωσις μόνη ἤρεσεν, πέταλα ἀργύρου χρίοντες ὤπτησαν καὶ ἥψησαν. Εἰς τοσοῦτον δὲ ἐλείουν ὥστε πάντα μιμεῖσθαι τὸ λειούμενον, καὶ ὕδατι καὶ ὑδραργύρῳ, καὶ εἴ τινι τοιούτῳ.

7. Καὶ ὥσπερ ἐν τῇ ἑψήσει τῇ τεχνικῇ διάφορα γράμματα ἀναδείκνυνται,\footnote{γράμματα] χρώματα BAK Lb ; χρωμάτων E. --- ἀναδείκνυται B etc.} οὕτως καὶ ὁ Ἀγαθοδαίμων ὅτι μᾶλλον οὗτος πλέω πάντων περὶ τῶν λείωσεων ἐφρόντισεν. Εἰς τοῦτο συνηγοροῦσιν ἐν τῇ λειώσει τοῦ προσωπιδίου θείου μετὰ χρυσοκόλλης, καὶ ἀλὸς ἀνθίου.\footnote{προσωποπιδίου BE (πο surpointillé E). --- τοῦ θείου Lb.} Ἐὰν δοκιμάσῃς, φησὶν, διάφορα καίεται, ἕψει, φησὶ, λειῶν ἐν ἡλίῳ\footnote{διαφόρως Lb seul.} ἕως γένηται · ἐκ τούτου μᾶλλον τὴν ἕψησιν, λείωσιν ἐτεκμήραντο\footnote{F. l. ἕως γένηται < ἰός. > Cp. p. précéd. l. 14.} · τοῦτο ποιοῦσιν, βουλόμενοι ἐπιδείξασθαι τὴν τοῦ φαρμάκου δύναμιν, σκεύη τὰ ἀργύρου λαμβάνοντες, καὶ τὸ ἥμισυ χρίσαντες, τὸ φάρμακον\footnote{τὰ ἀπὸ ἀργύρου B etc.} ὀπτοῦσι καὶ ἐκφέρουσι τὸ σκεῦος κεχρυσωμένον τὸ μέρος τὸ χρισθέν · Τὸ δὲ ἕτερον ἀκέραιον μένει. Καὶ οὕτως μὲν ὁ περί θείου ὕδατος λόγος.\footnote{F. l. οὗτος.}

\bigskip
\centerline{\EightStarTaper}
\centerline{\EightStarTaper\EightStarTaper}
\bigskip

\subsubsection{3. --- 22. ΠΕΡΙ ΣΤΑΘΜΩΝ.}
\paragraph{}
\emph{Transcrit sur} M, f. 153 r. ;--- \emph{Collationné sur} B, f. 139 r. ;--- \emph{sur} A, f. 127 r. ;--- \emph{sur} K, f. 24 v. ;--- \emph{sur} E, f. 65 r. ;--- \emph{sur} Lb, p. 243. --- \emph{Les variantes et restitutions de} M \emph{ont été reportées en marge de} K. --- \emph{Chap.} 45 \emph{de la compilation du Chrétien dans} E Lb.

\bigskip

1. (f. 153 v.) Ὁ περὶ σταθμῶν λόγος τὸ πᾶν τῆς ἑψήσεως φαίνεται συνέχων μυστήριον · αὐτὸ γὰρ σύνθεσις, αὐτὸ σταθμὸς, αὐτὸ λεύκωσις, αὐτὸ ξάνθωσις. Ἠρέμα δέ πως ἐν τῷ περὶ συνθέσεως λόγῳ, ταῦτα\footnote{Ἠρέμα δ. π. ὅσα ... Lb.} πάλιν περὶ χαλκοῦ καὶ ἰώσεως. Φαίνεται δὲ καὶ αὐτὸς τοιοῦτον\footnote{καὶ ἰώσεως εἴρηκε Lb.} μόλυβδον λαμβάνων, ἀφ ᾽ οὗ καὶ αὐτὸς · Σκόρπισον μολύβδῳ · οὐχ\footnote{μόλυβδον] μολύβδῳ Lb seul. Le signe du plomb dans les autres mss.} ἁπλῶς ἔλεγεν, ἀλλὰ τὸ ἀπὸ κοπτικοῦ καὶ λιθαργύρου μέλανι τῷ\footnote{τὸ] τοῦ B ; τῷ AKE Lb, mel.} ἡμῶν. Ἡ δὲ σκόρπισις ἐμοὶ λείωσις φαίνεται, ὡς ἀποδείξω ἐκ πασῶν τῶν γραφῶν ἐν τῇ ἐμῇ κατενεργείᾳ περὶ τοῦ σταθμοῦ. Εἰώθασιν γὰρ\footnote{ἐν τῇ ἐ. κατ ᾽ ἐνέργειαν συνθέσει Lb.} δι ᾽ ὧν καίουσιν ἢ σκορπίζουσιν ἢ ἐπιβάλλουσιν, διὰ τούτων συσταθμίζειν κεκρυμμένως · σταθμίζουσι τὸν μόλυβδον · ὃς καὶ διὰ τῆς\footnote{ὃς] ὁ M.} διασκορπίσεως, καὶ συσταθμίζεται λεύκωσις καὶ ἴωσις διὰ τῆς ἐπιβολῆς.\footnote{γὰρ, φησὶ Lb.} « Ἐπίβαλλε γὰρ τοῦ λευκοῦ φαρμάκου τὸ ἥμισυ, » καὶ τὰ ἑξῆς.\footnote{πάσῃ Lb seul, f. mel. --- συσταθμιώοεως M ; συσταθμήσεως B.}

2. Πάντα οὖν ἐν πᾶσι κέκρυπται τῇ τέχνῃ ἀπὸ συσταθμίσεως καὶ ἰώσεως, ὁμοῦ πάντα · ἐπειδὴ ἐκ τῆς προϊζανούσης θείου τῇ φιάλῃ,\footnote{Après προϊζανούσης, le signe, ou de νεφέλης, ou de θεῖον dans M ; signe de θεῖον dans BAKE ; signe de θεῖον surmonté de celui de ὑδράργυρος dans E ; ὑδραργύρου en toutes lettres Lb. Cp. p. 167, l. 13.} οὐχ ὁρᾶται τὸ ὑποκείμενον σύνθεμα πότε λευκανθῇ ἐξ αὐτῆς θείου\footnote{ὁρᾶται] ὅρα M ; ὁρᾶ BAK. --- θείου] signe de θεῖον MBAK. A mg. : λοιπὸν τῆς puis le signe de θεῖον. Réd. de E Lb : ἐξ αὐτῆς λοιπὸν τῆς ὑδραργύρου (en toutes lettres Lb) γινώσκεται.} γινώσκουσιν. Ὅταν γὰρ λευκὴ γένηται, τὸ τηνικαῦτα γινώσκεται\footnote{λευκὴ ὑδράργυρος γένηται Lb.} καὶ τὸ ὑποκείμενον λευκανθέν. Ἔνθεν ὁ Ἀγαθοδαίμων καθ ᾽ ἑκάστην λαμβάνειν θεῖον ἔλεγεν, ἢ λευκὴν ἢ οἵαν δήποτε. Ἐκείνη γὰρ ἡ\footnote{θεῖον] mêmes variantes que ligne 9 ; ὑδράργυρον (en toutes lettres) Lb.} μηνύουσα τὴν ὄπτησιν, ἣν ἁρπάζουσι καὶ κατακαίουσιν εἰς τὸ λείψανον τοῦ θείου, καὶ ἐκκρίνουσιν μᾶλλον ἢ ἐξίουσιν · λευκανθὲν γὰρ\footnote{εἰσκρίνουσιν BE Lb.} ἁρπάζουσιν. Ἐὰν γὰρ ἐάσωσιν, ἐπὶ τὸ ξανθὸν τρέπεται. Διὸ τοίνυν\footnote{M mg. : περὶ ὕδατος θείου, 1\textsuperscript{re} main. --- Διὸ τοίνυν] Διὸ πῶς E. Réd. de Lb : Διὸ πῶς ἔχει οὖν καὶ τοῦ θείου τοῦ λευκ. τὸ πᾶν, τοῦ σταθμοῦ ...} καὶ τοῦ θείου τοῦ λευκαίνοντος, τὸ πᾶν τοῦ σταθμοῦ παρὰ τῶν φιλοσόφων ζητήσωμεν. Ἔχει οὖν ἐν τῇ ὑστέρᾳ τῶν ζωμῶν ἀρσενίκου γ° αʹ, καὶ νίτρου ἥμισυ, καὶ φλοιῶν φύλλων περσέας ἁπαλῶν γ° βʹ, καὶ\footnote{φλωὸν M ; φλοιὸν E. --- ἀπλὸν M.} ἅλας ἥμισυ, καὶ συκαμίνου χυλοῦ γ° αʹ, καὶ στυπτηρίας σχιστῆς · τούτοις\footnote{τούτοις] τοῖς M.} συλλειώσας ὅλα ὁμοῦ ἐν ὄξει ἢ οὔρῳ, ἢ ἀσβέστου στάκτῃ,\footnote{σταλακτῇ ἕνωσον ἕως B, etc.} ἕως (154 ρ.) γένηται ζωμός. Εἶτα ἐν σκιᾷ [πυρὸς] καταβάπτει πέταλα\footnote{πυρο M.} καὶ ἀποσκιώσεις ποιεῖ. Δεῖ οὖν τὰ λείποντα πάντα βάλλειν, πρό γε πάντων, ἀσβέστου μέρη βʹ πρὸς θείου καὶ ἀρσενίκου, καὶ σανδαράχης\footnote{σανδαράχη M.} μέρος αʹ, καὶ τὰ ὕδατα · καὶ ποιήσαντες ὕδωρ λευκὸν μαρμάρῳ παρεμφερὲς,\footnote{καὶ τὸ ὕδ M. --- ποιήσαντας Lb, mel.} ἐν αὐτῷ ποτίζειν ἢ ἑψεῖν τρούλλῳ τὸ προειρημένον σύνθεμα.\footnote{ἢ] καὶ E. --- τρούλλου M.}

\bigskip
\centerline{\EightStarTaper}
\centerline{\EightStarTaper\EightStarTaper}
\bigskip

\subsubsection{3. --- 23. ΠΕΡΙ ΚΑΥΣΕΩΣ ΣΩΜΑΤΩΝ.}
\paragraph{}
\emph{Transcrit sur} M, f. 154 r. --- \emph{Collationné sur} B, f. 139 v. ;--- \emph{sur} A. f. 127 v. ;--- \emph{sur} K, f. 25 r. ;--- \emph{sur} E, f. 66 v. ;--- \emph{sur} Lb, p. 249 --- \emph{Les νariantes et restitutions de} M \emph{ont été reportées en marge de} K. --- \emph{Chap.} 46 \emph{de la compilation du Chrétien dans} E Lb.

\bigskip

1. Φέρε τοίνυν ἐκ τῶν φιλοσόφων καὶ τί ἐστιν καῦσις σωμάτων ζητήσωμεν. Ὁ λόγος γὰρ ὁ περὶ σταθμῶν ἀνῆκεν · ἀλλὰ μὴν καὶ τὸ ὅλον συνέχει. Ἄγαγε τὸν φιλόσοφον λέγοντα · « Λαβὼν νεφέλην τὴν ἀπὸ ἀρσενίκου, πῆξον ὡς ἔθος, καὶ ἐπίβαλλε χαλκῷ ἢ σιδήρῳ θειωθέντι,\footnote{λειωθέντι K.} καὶ λευκανθήσεται. Τινὲς τὸ θειωθέντι καέντι λέγουσιν · μὴ ἀγνοοῦντες\footnote{τὸ] τῶ M ; τῷ E Lb. --- μὴ om. B. etc., f. mel.} γὰρ οὗτοι τὸν χαλκὸν καίουσι τῷ θείῳ, καὶ τὸν σίδηρον μαγνησίᾳ.\footnote{μαγνησίᾳ om. M. --- A mg. : σῆ.} Οὐκ ἔστιν δὲ αὕτη καῦσις, ἀλλὰ φθορά. Ἡ δὲ τοῦ φιλοσόφου καῦσις αὕτη λεύκωσις ὀνομάζεται. Ὥσπερ ἡ ἐξίωσις καὶ τὰ ἄλλα ἀποδέδεικται λεύκωσις,\footnote{M mg. : Λεξ, à l'encre noire. (Cp. \emph{Lexique}, ci-dessus, p. 10. l. 4). --- ἡ ἐξίωσις καὶ ἡ λεύκωσις Lb (λεύκωσις biffé dans E).} οὕτως καὶ ἡ καῦσις ἡ παρ ᾽ αὐτῷ ἐν τούτῳ τῷ προκειμένῳ λεύκωσις\footnote{Après λεύκωσις] γίνεται et au-dessus : ὀνομάζεται E. --- λεύκ. ὀνομάζεται Lb.} · ἐν γὰρ δευτέρῳ, ξάνθωσις.\footnote{γὰρ] F. l. δὲ. --- ξάνθωσίς ἐστι E.}

2. Αὐτὸς οὖν ὁ φιλόσοφος καίει τὸν χαλκὸν διὰ τοῦ ὕδατος τοῦ θείου, ἑψῶν καθὰ προλέλεκται. « Ἐπίβαλλε γὰρ, φησὶν, τοῦ λευκοῦ φαρμάκου τὸ ἥμισυ · καὶ ἔσται πρῶτον · τοῦτο ἕψει · τὸ γὰρ ἄλλο ἥμισυ ἐν τῇ ἰώσει τηροῦμεν. » Διὰ τοῦτο καὶ Πιβήχιος ἄνω καὶ κάτω · « Διαμερίσατε\footnote{Πηβήχιός φησιν Lb.} εἰς δύο μοίρας τὸ φάρμακον. » Ἔλεγεν · « Καύσατε τὸν χαλκὸν\footnote{M mg. : ὧδε N° (sc. νόει ? ), à l'encre noire.} ἐν δαφνίνοις ξύλοις, τουτέστιν ἐν τῷ λευκῷ συνθέματι · φύλλα γὰρ δάφνης οὕτως καίονται τὰ σώματα ἑψόμενα διὰ τοῦ ὕδατος τοῦ\footnote{τὰ δὲ σώμ. ἕψονται Lb.} θείου, ὀμοῦ δὲ καὶ λευκαίνονται · τὸ γὰρ « ἐπίβαλλε χαλκῷ ἢ σίδήρῳ θειωθέντι · τοῦτῳ καὶ λευκανθήσεται. Καὶ ὁ Ἀγαθοδαίμων οὕτως\footnote{θειοθέντα M.} παρεγγυᾷ, ἵνα ζῶσιν τὰ σώματα καὶ ἑψῶνται μετὰ τῆς νεφέλης τῷ θείῳ\footnote{ἐν τῷ θ. ὕ. Lb.} ὕδατι. Καὶ οὕτως ἐστὶν καῦσις καὶ λεύκωσις · ἐν γὰρ τῷ κασσιτέρῳ ὁ\footnote{ἐν γὰρ τ. κ.] ἐὰν γ. τῷ κ. E ; ἐὰν γ. τῇ ὑδραργύρῳ Lb. (même variante plus loin, l. 9 et 12).} φιλόσοφος (f. 154 v.) τὴν ἕψησιν ὑπέθετο · « τὴν προγεγραμμένην νεφέλην ἕψει ἐλαίῳ κικίνῳ ἢ ῥαφανίνῳ προσμίξας βραχὺ στυπτηρίας.\footnote{στυπτ. σχιστῆς Lb seul.} Εἶτά φησιν · « Ποίει μίγματα τοῦ κασσιτέρου, » καὶ τὰ ἑξῆς · πάντα\footnote{καὶ τὰ ἑξῆς --- κασσιτέρου om. BAK.} τέλεια διὰ μιᾶς τάξεως. Ἀπὸ γὰρ τῶν ἡμερῶν τὰ ὅλα ἐμνημόνευσεν · ἀπὸ τῶν ἐλαίων τοῦ ὕδατος τοῦ θείου · ἀπὸ τῆς στυπτηρίας, τὸ θεῖον\footnote{τὸ ὕδωρ Lb mel.} · ἀπὸ τοῦ κασσιτέρου, τὰ δύο συνθέματα · ἡ γὰρ νεφέλη κατ ᾽ αὐτὸν\footnote{καθ ᾽ ἑαυτὴν Lb.} δύει.

3. Αἱ γοῦν ἐπιβολαὶ κατὰ τῶν τοῦ θείου πάλιν ζωμῶν · ἡ δὲ ὄπτησις κατὰ τοῦ ὅλου, ἥτις καῦσις ἢ ἕψησις καὶ λεύκωσις.\footnote{καὶ λεύκωσις καλεῖται, καὶ ἐν ταύταις καίονται Lb.} Ἐν τούτῳ καίουσιν καὶ ἑψοῦνται τὰ σώματα. Αὕτη ἡ καῦσις ἡ ἀπ ᾽ αἰῶνος κηρυττομένη, τοῦτον ὃν πᾶσαι αἱ γραφαὶ μυστικῶς διδάσκουσιν τὸν\footnote{τοῦτον ὃν] τοῦτο οὖν B etc., mel.} χαλκὸν θείῳ καίειν. Αἱ δὲ ἄλλαι καύσεις φθοραί εἰσιν μᾶλλον ἢ καύσεις. Οὗτος ἐὰν καῇ, εὔχρηστος χαλκὸς εἰς πάντα καὶ ἔτοιμος εἰς καταβαφὴν,\footnote{Réd. de Lb : Οὕτως οὖν ἐὰν καῇ, καλὸς καὶ εὔχρηστος χαλκὸς εἰς πάντα γίνεται, καὶ ἔτ. --- χαλκὸς om. M.} ὡς καὶ ἐκταθεὶς ἠλεκτροῦται. Καὶ ἐὰν πλεονάσῃς τὰ φῶτα, γίνεται\footnote{ὃς καὶ ἐκτ. Lb. --- πλέον ἐάσης M. F. l. πλεονάσῃ.} ξανθὸν τὸ ἥμισυ τὸ θεῖον καιόμενον · τῆς γὰρ μαγνησίας τὸ τέταρτον\footnote{καιόμενος M. --- τὸ τέταρτόν ἐστι Lb.} · καὶ οὕτως χρώμεθα ἐν τῷ χαλκῷ γ° δʹ, σιδήρου γ° αʹ, καὶ μαγνησίας γ\textsuperscript{α}\textsubscript{ρ} ΣΤʹ,\footnote{[ΣΤ, ΣΤʹ = Stigma]. Réd. de Lb (d'après les corr. et add. de E) : ἐκ τοῦ χαλκοῦ ὀγγίαις τίσσαρσι, καὶ ἐκ τοῦ σιδήρου ὀγγίᾳ μιᾷ, καὶ ἐκ τῆς μαγνησίας γραμμαρίοις ἓξ, ἐκ τῆς ὑδραργύρου δὲ καὶ μολ. καὶ χαλκίων, καὶ καδμίας ...} κασσιτέρου δὲ καὶ μολύβδου χαλκία < βʹ, > καὶ καδμίας, καὶ κλαυδιανοῦ, καὶ χρυσοκόλλης, καὶ κινναβάρεως πρὸς ἀνάλογον τούτων τῶν οὐγγιῶν. Κἄν τε γὰρ ἐξ ἴσου ποιήσῃς ἢ πλέον ἢ ἔλασσον, ἐπιτυγχάνεις · οὕτως οἰκονομεῖσθαι ἐργῶδές ἐστι καὶ εὐηθές. Δεῖ δὲ μετὰ\footnote{οὕτως οὖν οἰκονομεῖν Lb.} σταθμοῦ ἐκθέσθαι, Δημοκρίτου εἰρηκότος · « Οὐδὲν ὑπολέλειπται,\footnote{τὰ πάντα ἐκθέσθαι Lb.} οὐδὲν ὑστερεῖ. » Καὶ μὰ τὴν Δημοκρίτου ἀρετὴν, οὐδὲν ὑπολείπει.\footnote{ματὴν MA.} Ἡ γὰρ σύνθεσις τοῦ ἀπολελυμένου, λέγω δὲ ὕδατος θείου καὶ νεφέλης ἄρσις, ἀφθόνως ὑμῖν ἐξεδόθη · ἡ δὲ ἔκδοσις αὕτη ἡ τῆς βίβλου ἑρμηνεία.\footnote{ἡμῖν Lb seul.} Ἐπειδὴ τοίνυν περὶ σταθμοῦ καὶ καύσεως ἀποδέδεικται, φέρε καὶ περὶ σταθμῶν ξανθώσεως ζητήσωμεν.

\bigskip
\centerline{\EightStarTaper}
\centerline{\EightStarTaper\EightStarTaper}
\bigskip

\subsubsection{3. --- 24. ΠΕΡΙ ΣΤΑΘΜΟΥ ΞΑΝΘΩΣΕΩΣ.}
\paragraph{}
\emph{Transcrit sur} M, f. 154 v. --- \emph{Collationné sur} B, f. 141 r. ;--- \emph{sur} A, f. 128 v. ;--- \emph{sur} K, f. 26 r. ;--- \emph{sur} E, f. 68 r. ;--- \emph{sur} Lb, p. 257. --- \emph{Les νariantes et restitutions de} M \emph{ont été reportées en marge de} K. --- \emph{Chap.} 47 \emph{de la compilation du Chrétien dans} E Lb.

\bigskip

1. Διατί ὁ Ἀγαθοδαίμων ἐμνημόνευσεν ; οὐχ ἵνα σταθμὸν\footnote{ἐμν. τοῦ σταθμοῦ Lb.} (f. 155 r.) διδάξῃ, ἀλλ ᾽ ἵνα κρόκου καὶ ἐλυδρίου τὸ διπλάσιον τῶν ἄλλων ποῶν βάλλῃ · αὗται γάρ εἰσι βαπτικώτεραι · τὸν γὰρ σταθμὸν κατὰ ἀνάλογον τοῦ λευκοῦ θείου ποιεῖ · ἔκ τε θείων καὶ ὑδάτων καὶ\footnote{θεῖον ποιεῖ Lb seul.} ποῶν ὕδωρ θεῖον, ὃ καλεῖται παρ ᾽ αὐτοῖς ὕδωρ ἄθικτον. Ἐκ τούτου\footnote{παρ ᾽ αὐτ M.} ποτίζουσιν ἑψοῦντες τὸ λευκὸν σύνθεμα καὶ ξανθοῦται. Καὶ ὄπτα ὡς\footnote{Réd. de Lb : ποτίζουσιν ἑψ. καὶ ξανθοῦντες καὶ ὀπτῶντες, καὶ πάλιν ἁρπάζοντες ἕως ...} ἤκουσας πρότερον, ἁρπάζων πάλιν ἕως οὗ ξανθωθῇ. Ὁμοίως δέ ἐστιν τοῦτο σταθμὸς καὶ ξάνθωσις. Οὗτος ὁ περὶ σταθμῶν καθὼς προεῖπεν ὁ λόγος.

2. Δεῖ δὲ εἰδέναι ὅτι ἐν τῷ ἐπιχειρεῖν τὸ πρᾶγμα πολλὰ αἴτια συμβαίνει · τὰ μὲν ὀφθαλμοφανῶς, τὰ δὲ οὔ. Ἔστι δὲ τὰ πρῶτα πλυνόμενα ἢ μιγνύμενα, μολυβδόχαλκος, καὶ τὰ ὅμοια, πυρίτης καὶ τὰ ὅμοια. Δεῖ δὲ καὶ τὸν πυρίτην καὶ τὸν ἀνδροδάμαντα, μὴ ὄξει πρῶτον οἰκονομεῖσθαι, καθὼς ἔχουσιν αἱ γραφαὶ, ἵνα μὴ τὸ χαλκῶδες αὐτοῦ ἰωθῇ, τὰ δὲ ὕστερον συμμισγόμενα κινναβάρει, καὶ τὰ ὅμοια\footnote{συμμιγνύμενα B etc.} · ἢ ἐγχωρεῖ καὶ ἐν ἡλίῳ, καὶ τὰ ὅμοια.\footnote{ἐν om. M. ἡλίῳ en signe M ; ἐν χρυσῷ (en toutes lettres) Lb seul.}

3. Μαρία γὰρ πρὸ πάντων μολυβδόχαλκον καὶ τὰς ποιήσεις\footnote{γὰρ] F. l. δὲ. --- Réd. de Lb : καὶ τὰς π. λέγει · τὴν γὰρ καῦσιν ...} · ἢ γὰρ καῦσις ἣν πάντες οἱ ἀρχαῖοι κηρύττουσιν, Μαρία πρώτη φησίν\footnote{φησίν] εἶπεν Lb.} · « Ὁ χαλκὸς καεὶς θείῳ καὶ ἀνακαμφθεὶς νιτρελαίῳ καὶ ἐκτιναχθεὶς,\footnote{ἀνακαυθεὶς B etc.} καὶ πολλάκις τὰ αὐτὰ παθῶν, χρυσὸς κρεῖττον ἀσκίαστος γίνεται.\footnote{χρυσοῦ Lb seul. --- κρείττων B etc. --- καὶ ἀσκ. γίν. E.} » Καὶ τοῦτο ὁ θεὸς εἶπεν · « Ἴστε πάντες ἀπὸ τῆς πείρας ὅτι καύσαντες τὸν χαλκὸν θείῳ οὐδὲν ἐποιήσατε · ἐπὰν δὲ καύσῃ τοῦτο τὸ θεῖον,\footnote{καύσητε τούτῳ τῷ θείῳ B etc., f. mel.} τότε οὐ μόνον ἀσκίαστον ποιεῖ, ἀλλὰ καὶ ἐπὶ τὸν χρυσὸν βαδίζοντα. » Ἔνθεν καὶ Μαρία ἐν τοῖς ὑποκάτω τοῦ ζωδίου καὶ δεύτερον αὐτὸ ἐβόα, καί φησιν · « Καὶ τοῦτό μοι ὁ θεὸς ἐχαρίσατο · ὅτι χαλκὸς πρῶτον καίεται θείῳ, εἶτα σῶμα τῆς μαγνησίας · καὶ ἐκφυσᾶτε ἕως\footnote{ἐκφυσᾶται MB Lb ; φυσσᾶται AKE. Corr. conj.} ἐκφύγωσιν ἀπ ᾽ αὐτοῦ μετὰ τὴς σκιᾶς τὰ θειώδη. Καὶ γίνεται χαλκὸς ἀσκίαστος.

4. Οὕτως οὖν πάντες καίουσιν. Ἥ Μωσέως μάζα · « Οὕτως καίεται θείῳ, καὶ ἀλὶ καὶ στυπτηρίᾳ, θείῳ (f. 155 v.) λευκῷ λέγω.\footnote{στυπτ. σχιστῇ (en toutes lettres) Lb.} Οὕτως καὶ Χίμης εἰς πολλοὺς τόπους καίει μάλιστα τὴν δι ᾽ ἐλυδρίου.\footnote{Χύμης Lb seul. --- καίει] καὶ B etc. --- τὴν] ἡ Lb.} Οὕτως καὶ Πηβίχιος · « Καὶ ἡ ἐν δαφνίνοις ξύλοις. » Περιφραστικῶς\footnote{Πηβήχιος BAKE ; Πιβήχιος Lb. --- καὶ ἡ] καίει B etc. F. l. καίε. Cp. p. 179, l. 20.} τῷ λευκῷ θείῳ ἀπὸ τῶν φύλλων δάφνης αἰνίττεται.\footnote{τὸ λευκὸν θεῖον Lb seul, mieux. --- ὑπὸ Lb.} Οὕτος ὁ περὶ σταθμῶν λόγος.

5. Τοῦτο οὖν ἄνω καὶ κάτω Μαρία εἰς μυρίας τάξεις ἔλεγεν. « Τὸν ἡμέτερον χαλκὸν καῦσον τῷ θείῳ, καὶ ἐκτιναχθεὶς, ἔσται\footnote{καῦσον] καύσατε B etc. --- καὶ] ὃς Lb.} ἀσκίαστος. » Οὐ γὰρ μόνον οἶδεν καίειν τῷ λευκῷ αὐτῷ θείῳ, ἀλλὰ\footnote{καίειν αὐτὸν τῷ λ. θ. Lb, f. mel. --- αὐτὸν] αὐτὴν BA ; αὐτὴνῶ K.} καὶ λευκαίνειν καὶ ἄσκιον ποιεῖν. Ἐν τούτῳ Δημόκριτος καίει, καὶ λευκαίνει καὶ ἄσκιον ποιεῖ. Πάλιν τὸ ξανθὸν θεῖον οὐ μόνον καίουσιν, ἀλλὰ καὶ ἀσκιαστοῦσιν καὶ ξανθοῦσιν. Τοῦτο ὁ Δημόκριτος λέγει · « Τὴν γὰρ αὐτὴν ἐνέργειαν ἔχει ὁ κρόκος τῇ νεφέλῃ,\footnote{τῆς νεφέλης (en signe) M.} ὡς ἡ κασία τῷ κινναμώμῳ. » Καὶ ἐν τῇ μάζᾳ Μοϋσέως ἐπὶ τέλει\footnote{τοῦ κιναμώμου M. --- Mώσεως B etc. --- ἐπιτελείτω M ; ἐπιτέλει BAKE ; om. Lb. Corr. conj.} ὁμοίως κεῖται · « Πότιζε ὕδατι θείου ἀθίκτου, καὶ ἔσται ξανθὸν,\footnote{M mg. : ·)(· avec renvoi à ἄθικτον. --- ξανθὸς, ἀσκίαστος Lb, f. mel.} ἀσκίαστον. » Δηλονότι καείς.

6. Αὕτη οὖν καῦσις, αὕτη λεύκωσις ἢ ξάνθωσις, αὕτη ἐν τοῖς\footnote{M mg. : ξα (à l'encre noire).} δυσὶν ἀσκίαστος · Οὕτως καίονται καὶ ἐκτινάσσονται τὸν χαλκὸν,\footnote{F. l. ἀσκιάστωσις. (Cp. l. 19). --- F. l. καίοντες καὶ ἐκτινάσσοντες.} χρυσῷ ἶσον, ἀσκίαστον ποιήσετε, καὶ πρὸς δίπλωσιν ἀργύρου καὶ χρυσοῦ ἔτοιμον. Οὐδεὶς δὲ < πλὴν > τὴν πᾶσαν ὁδὸν ἐπιστάμενος δίπλωσιν κατεργάζεται · ἐπεὶ ὅμοιος τῷ τὰς σταφυλὰς ὄμφακας ὄντας\footnote{ὅμοιος ἔσται Lb seul. --- ὄντας] οὔσας Lb.} ἔτι τρύγοντι. Τινὲς τῷ παντὶ ὀστράκῳ ἐν ὑαλοῖς κύβοις ἑψοῦσιν καὶ\footnote{ὑέλοις M ; ὑαλίνοις Lb.} ὀπτῶσιν ἐπὶ τῆς κηροτακίδος · καὶ ταῦτα καλοῦσιν ληκύθια.\footnote{ὄπτουσι M, ici et presque partout. --- λεκύθια mss. excepté Lb.} Ὁ Ἀγαθοδαίμων ἐν ταῖς λειώσεσιν ἰσχυρῶς καὶ ἰατρικῶς κολλούρια\footnote{F. l. ἰσχυρᾷ καὶ ἰατρικῇ. --- Lire κολλύρια.} ἀγωγῇ εἶπεν λειοῦσθαι.

7. Αὕτη οὖν ἐστιν καῦσις σώματων · οὗτος ὁ περὶ σταθμῶν λόγος · αὕτη καλεῖται καῦσις, λεύκωσις. Ἡ δὲ τοῦ θείου αὕτη καλεῖται λεύκωσις καὶ ἀσκίαστος · ἡ λεύκωσις αὕτη καλεῖται ἴωσις,\footnote{ἀσκιάστωσις B etc., f. mel. --- M mg. : ἰω.} (f. 156 r.) καὶ ἐξίωσις καὶ λεύκωσις. Πάλιν δὲ καὶ εἰς δεύτερον καλεῖται λεύκωσις ξάνθωσις, καὶ ἀσκίαστος ξάνθωσις, καὶ ἴωσις,\footnote{λεύκωσις] F. l. καῦσις (\emph{M. B.}). --- Lire ἀσκιάτωσις (\emph{M. B.}).} ξάνθωσις. Καὶ ὁ προφήτης Χίμης χορεύων, μετὰ ἐξεπιβολὰς ἔλεγεν\footnote{Καὶ ὁ πρ. δὲ Χύμης Lb. --- χορεύων] F. l. ἀγορεύων. --- ἓξ ἐπιβολὰς BAKE.} · δεὶς [ἔλεγεν] αὐτὸν ἄσκιον ξανθὸν. Ἐξῆς δέ σοι ὁ περὶ θείου\footnote{ἔλεγεν · δεὶς ἔλεγεν αὐτὸν ἄ. ξ.] δὶς αὐτὸν ἄ. ξ. BAK ; ἔλεγεν δὶς (mot biffé) αὐτὸν ἄ. ξ. E ; ἔλεγεν αὐτὸν ἄ. ξ. Lb.} ὕδατος καὶ ἰώσεως ἤτοι σήψεως λαληθήσεται τρόπος.

\bigskip
\centerline{\EightStarTaper}
\centerline{\EightStarTaper\EightStarTaper}
\bigskip

\subsubsection[3. --- 25. ΠΕΡΙ ΘΕΙΟΥ ΥΔΑΤΟΣ.]{3. --- 25. ΠΕΡΙ ΘΕΙΟΥ ΥΔΑΤΟΣ.\footnote{Titre dans BAK : περὶ θείου ἀθίκτου ὕδατος ;--- dans Lb : περὶ ὕδατος θείου ἀθίκτου.}}
\paragraph{}
\emph{Transcrit sur} M. f. 156 r. --- \emph{Collationné sur} B, f. 143 r. ;--- \emph{sur} A, f. 129 v. ;--- \emph{sur} K, f. 26 v. ;--- \emph{sur} E, f. 70 v. ;--- \emph{sur} Lb, p. 267. --- \emph{Les variantes et restitutions de} M \emph{ont été reportées en marge de} K. --- \emph{Chap.} 48 \emph{de la compilation du Chrétien dans} E Lb.

\bigskip

1. Πρῶτον δεῖξαι δεῖ ὅτι σύνθετον τὸ ὕδωρ τοῦ θείου ἐκ πάντων τῶν ὑγρῶν, ἔχον τὴν σύγκρασιν, καὶ διὰ πάντων τῶν ὑγρῶν ὀνομάζεται.\footnote{M mg. : ὥ (pour ὧδε), à l'encre noire.} Καθάπερ τὸ στερεὸν σύνθεμα δι ᾽ ἑνὸς ἑκάστου αὐτῶν εἴδους ἐκάλεσεν,\footnote{F. l. ἐκάλεσαν. --- ὕδωρ θεῖον] ὕδωρ et le signe de θεῖον M ; ὕδατος θείου E. ὕδωρ θείου Lb. --- ὑγροῦ ὑδ. θ. om. BAK. --- ὕδωρ θεῖον jusqu'à σικερίτου καὶ ζύθου (l. 16)] Cp. 3, 29, 14 (= $\star$).} οὕτως καὶ τὸ ὑγρὸν δι ᾽ ἑνὸς ἑκάστου ὑγροῦ ὕδωρ θεῖον, διὰ δὲ μυρίων\footnote{διὰ δὲ μυρ. ὀν. κ. τ. λ.] ὅτι τὰ δύο συνθέματα καλ. πολλοῖς ὀνόμασιν οἷον ὕδωρ ἅλμης $\star$.} ὀνομάτων τὰ δύο συνθέματα καλοῦσιν. Καλεῖται ὕδωρ θεῖον δι ᾽ ἅλμης, διὰ ὕδατος θαλασσίου, διὰ οὔρου ἀφθόρου, δι ᾽ ὄξους, δι ᾽ ὀξάλμης,\footnote{διὰ ὕδ. θαλ. om. $\star$.} δι ᾽ ἐλαίου κικίνου, ῥεφανίκου, βαλσάμου, γάλακτος γυναικὸς ἀρρενοτόκου, καὶ γάλακτος βοὸς μελαίνης, καὶ δι ᾽ οὔρου δαμάλεως, καὶ προβάτου θηλείας · τινὲς οὔρου ὀνείου · ἄλλοι καὶ ὕδατος ἀσβέστου, καὶ μαρμάρου,\footnote{θήλεος Lb. --- τινὲς] καὶ $\star$. --- ἄλλοι καὶ om. $\star$. --- ὕδατος (en signe)] ὕδωρ $\star$.} καὶ φέκλης, καὶ θείου, καὶ ἀρσενίκου, καὶ σανδαράχης, καὶ νίτρου,\footnote{καὶ φ. κ. σανδ. κ. στ. σχ. κ. νίτρου $\star$.} καὶ στυπτηρίας σχιστῆς, καὶ γάλακτος πάλιν ὀνείου, καὶ αἰγείου, καὶ κυνίνου · καὶ ὕδατος σποδοκράμβης, καὶ ἄλλων ὑδάτων ἀπὸ σποδοῦ\footnote{κυκίνου B etc. F. l. κυνικοῦ.} γινομένων · ἄλλοι καὶ μέλιτος, καὶ ὀξυμέλιτος, καὶ ὄξους, καὶ νίτρου,\footnote{ἄλλα καὶ M ; ἄλλοι BAK ; ἀλλὰ καὶ Lb ; om. $\star$.} καὶ ὕδατος ἀερίου, καὶ Νείλου, καὶ ἄρκτου, καὶ οἴνου ἀμηναίου,\footnote{καὶ ἄρκτου καὶ σαπφείρου $\star$.} καὶ ῥοϊτοῦ, καὶ μορίτου, καὶ σικερίτου καὶ ζύθου · καὶ ἵνα μὴ τὰ πάντα ἀναγινώσκω, διὰ παντὸς ὑγροῦ.\footnote{F. l. ἀναμιμνήσκω.}

2. Καὶ τὸ λευκὸν καὶ τὸ ξανθὸν πολλάκις ἐκάλεσαν οἱ παλαιοὶ διαφόρως. Δοκεῖ μοι ὅπως ὁ φιλόσοφος Πηβίχιος διέσταλκε τῷ φιλοσόφῳ ἐπὶ\footnote{Δοκεῖ --- Πηβίχιος] Ἀπορῶ δὲ πῶς ὁ Πηβήχιος ὁ φιλ. Lb. --- διέσταλκε] επ au-dessus de δι E. --- ἐπὶ] περὶ Lb.} τῶν ξανθῶν ζωμῶν · « ἄ- (f. 156 v.) νες οἴνῳ ἀμηναίῳ. » Ὅπερ οἴνῳ νέῳ\footnote{ὅπερ ὡς BAKE.} πάσαις ταῖς λευκώσεσιν οὐ κατέλεξαν ζωμόν. Πηβίχιος δὲ · « Σίκερα\footnote{ἐν πάσαις δὲ τ. λ. E.} καὶ μορίτην καὶ ῥοΐτην, πλὴν οὕτω διαστείλαντες οὐδὲν ὠφέλησαν τοὺς\footnote{ῥοΐτην εἶπον Lb.} ἀκροατὰς, πάνυ δυσνοήτως οὕτως · ἓν γὰρ ἕκαστον εἶδος οἰκονομῶν\footnote{πάνυ γὰρ δυσν. ἐλάλησαν Lb. --- ἐν M.} ὁ φιλόσοφος διὰ λευκώσεως καὶ ξανθώσεως οἰκονομεῖ, καὶ διὰ τῶν δύο ὧν προήκουσας, καύσεων ἢ ἑψήσεων. Φησὶν οὖν ἐπὶ τοῦ πυρίτου\footnote{F. l. καύσεως ἢ ἑψήσεως.} · « Λαβὼν πυρίτην, οἰκονόμει, λείου ἢ ὀξάλμῃ καὶ τοῖς ἑξῆς » ὃ αἰνίττεται ὕδωρ θεῖον λευκόν. Εἶτα ἐπὶ τῆς κινναβάρεως · « Τὴν κιννάβαριν\footnote{θείου Lb. --- ἐ. τ. κινναβ. φησίν Lb.} ποίει λευκὴν δι ᾽ ἐλαίου ἢ ὄξους καὶ μέλιτος καὶ τῶν ἑξῆς. » Ἐπὶ δὲ τοῦ\footnote{ἢ] καὶ Lb.} ἀνδροδάμαντος · « Ὁμοίως πάλιν, ἅλμῃ ἢ ὀξάλμῃ. » Εἶτα ἐπιφέρει · « Ἕψει ὕδωρ θείου ἀθίκτου, ἵνα γνῷς ὅτι ὕδατα θαλάσσια, καὶ οὖρον,\footnote{ὕδωρ] ὕδατι Lb.} καὶ ὄξος, καὶ τὸ ἐν τῇ κινναβάρει ἔλαιον, καὶ μέλιτος, ὕδωρ θεῖόν ἐστιν. Δι ᾽ ἑνὸς γὰρ εἴδους τὸ ὅλον αἰνίττεται.

3. Ὕστερον ἐν τῷ ἀνδροδάμαντι κηρῦξαι θέλων ἔλεγεν · « Ἕψει ὕδωρ θείου ἀθίκτου · τὰ γὰρ αὐτὰ ὑγρὰ καὶ ὕδατά εἰσιν ἀθίκτων · καὶ τῶν δι ᾽ ἀσβέστου ἐπιβολῶν ἀμειβουσῶν καὶ τὸ χρῶμα καὶ τὸ ὄνομα, ἐν μὲν τῷ θείῳ τῷ λευκῷ, « γῆ χεία καὶ ἀστερίτης καὶ ἀφροσέληνον\footnote{γῆν etc. (accusatif partout) Lb.} ἐν τῇ τάξει τοῦ χαλκοῦ · » ἐν δὲ τῷ ξαντῷ · « ἐπίβαλλε ὤχραν\footnote{τοῦ χαλκοῦ λέγει Lb. --- ἐπίβαλε, φησὶν Lb.} ἀττικὴν, σινώπην ὀπτὴν ποντικὴν καὶ τὰ ὅμοια. » Πάλιν τε ἐπὶ τῆς\footnote{τε] δὲ Lb ; om. BAK.} χρυσοκόλλης · « Πυρῶν καὶ ποτίζων αὐτὴν ἐλαίῳ ἕως ἑπτάκις.\footnote{ἐλαίων M ; ἔλαιον B etc. Corr. conj.} » Καὶ ἐν χρυσοποιΐᾳ ἕκαστον αὐτῶν προελεύκανεν. Ὁμοίως καὶ τὴν λιθάργυρον ἐν τοῖς ἀμφοτέροις συνθέμασιν · πλέω γὰρ δύο ἑψήσεων\footnote{M mg. : ἕψησις sur une ligne verticale, en lettres retournées.} οὐ γίνεται ἐν τῇ κατενεργεία · ἀλλὰ καὶ τὴν νεφέλην καὶ τὴν λιθάργυρον\footnote{ἐνεργείᾳ B etc.} ἐν τοῖς ζωμοῖς μέλιτι λευκοτάτῳ ἀναλαμβάνει. Καὶ οὐ παρέλειψέν τι τῶν ὑγρῶν, ἀλλ ᾽ ἐν τοῖς ἀμφοτέροις συνθέμασιν · συνέθετο γὰρ λύσιν κομάρεως καὶ ῥάκινον, (f. 157 r.) καὶ δι ᾽ ἐλυδρίου σκευαστοῦ γίνεσθαι ἔλεγε σύνθετον τὸ ὕδωρ τοῦ θείου · καὶ τὴν\footnote{ἔλεγε γὰρ (om. E) τὸ ὕδ. τοῦ θ. συνθ. ἐστι Lb.} χρυσόκολλαν κελεύει ζέννυσθαι ὕδωρ μαρμαρικῆς ἀσβέστου ἐλαίῳ\footnote{ζευγνύσθαι B etc. --- σὺν ἐλαίῳ B etc.} · καὶ τὸν πυρίτην σὺν μέλιτι ὕδωρ θεῖον διὰ τῶν τεσσάρων βιβλίων\footnote{τὸ δὲ ὕδωρ τοῦ θείου Lb seul.} διαφόρως διέρχεται οἰκονομῶν, ἐν μὲν τῷ ἀργύρῳ « γῆν χείαν, ἀστερίτην\footnote{ἐν μὲν τῇ, puis le signe de l'argent MBAΚE ; ἐν μὲν τῷ λευκῷ Lb. F. l. ἐν μὲν τῇ άργύρου < βίβλῳ. > --- « C'est le livre de l'argent, c'est-à-dire du blanc. » (\emph{M. B.})} καὶ ἀφροσέληνον, καὶ τῆς ἰδίας αὐτοῦ ἐπιβολῆς » · ἐν δὲ τῷ\footnote{καὶ τὴν etc. (accusatif partout) Lb.} ξανθῷ, « σινώπην, ὤχραν ἀττικὴν, καὶ λιθοφρύγιον, ἐὰν εὕρῃς » · ἐν δὲ τοῖς λίθοις, « αἷμα τράγου καὶ χυλὸν ἁλικακάβου · » ὕστερον δέ · « εἴ πώ\footnote{εἴπω MBAKE ; λέγω Lb.} τι χρήσιμον · τὰ θειώδη ὑπὸ τῶν θειωδῶν κρατεῖται, καὶ τὰ ὑγρὰ ὑπὸ\footnote{τὰ θεώδη --- ὑγρῶν] Cp. p. 142, l. 21. --- κρατεῖται] κατέχεται B etc.} τῶν καταλλήλων ὑγρῶν · τὰ γὰρ θειώδη ὑπὸ τῶν θειωδῶν κατέχεται.\footnote{κατέχεται] κρατεῖται B etc.} »

\bigskip
\centerline{\EightStarTaper}
\centerline{\EightStarTaper\EightStarTaper}
\bigskip

\subsubsection[3. --- 26. ΠΕΡΙ ΣΚΕΥΑΣΙΑΣ ΩΧΡΑΣ.]{3. --- 26. ΠΕΡΙ ΣΚΕΥΑΣΙΑΣ ΩΧΡΑΣ.\footnote{σκευασίας] σημασίας mss. Corr. conj.}}
\paragraph{}
\emph{Transcrit sur} M, f. 157 r. --- \emph{Collationné sur} B, f. 144 v. ;--- \emph{sur} A, f. 131 r. ;--- \emph{sur} K. f. 27 v., \emph{puis} 108 r ;--- \emph{sur} E, f. 73 r. ;--- \emph{sur} Lb, p. 277. --- \emph{Les νariantes et restitutions de} M \emph{ont été reportées en marge de} K. --- \emph{Chap.} 49 \emph{de la compilation du Chrétien dans} E Lb.

\bigskip

1. Σκευασία ὤχρας γίνεται ἐν τῷ ὄρει τῆς Ἁδριανοῦ πλαγίας\footnote{σκευασία] σημασία mss. Corr. conj. --- Réd. de E Lb : ἡ σημασία καὶ ἡ συλλογὴ τῆς ὤχρας γίν. ἐν τῷ ὄ. τοῦ Ἀδριατικοῦ (Lb seul) πελάγους · συλλεγομένη (Lb seul) ἐκεῖ κατὰ λακκήματα τοῦ ὄρους.} λεγομένης. Ἐκεῖ λακήματα τοῦ ὄρους · καὶ διὰ τῶν ῥαγάδων θεωρήσεις ζώνας ὤχρας πλακώδεις. Γίνεται δὲ καὶ εἰς Βαβυλωνίαν εἰς\footnote{Bαβυλῶνα B etc. --- εἴς τι ὄρος Lb seul.} τὸ ὄρος. Θεωρεῖς διὰ τῶν ῥαγάδων · ἀρῶνται καὶ ὀπτῶσιν, καὶ γίνεται\footnote{ἣ θεωρεῖται Lb. --- αἴρουσι δὲ ταύτην καὶ ὀπτ. Lb.} μίλτος, ὅντινα καὶ σινώπην καλοῦσιν. Ἡμεῖς δὲ οὐδὲ αὐτῇ τῇ ὤχρᾳ χρώμεθα, οὐδὲ ταύτῃ τῇ σινώπῃ, ἀλλὰ ὤχρα μὲν ἡ ἀληθὴς\footnote{M mg. ωχ (à l'encre noire). --- ταύτης τῆς σινώπης M.} βαφὴ ἔσται · πλὴν τὸ προκείμενον ἤτοι σῶμα μαγνησίας, ἤτοι μέλας μόλυβδος.

2. Καὶ οἵαν τάξιν λέγουσιν χωρὶς τῶν βαφικῶν, περὶ αὐτῆς λέγουσιν\footnote{οἵαν δήποτε Lb mel.} πᾶσαι αἱ γραφαί. Εἴ ποτε οὖν ἀναγιγνώσκεις οἱανδήποτε τάξιν, ἐν τούτῳ τοίνυν ἔχε, καὶ θηράσεις πρᾶγμα τὸ ζητούμενον, μάλιστα\footnote{ἐν τούτῳ ἔχε τὸν νοῦν Lb.} ἐὰν Μαρίᾳ καὶ τῷ φιλοσόφῳ ἀκολουθήσῃς. Καὶ γὰρ πυρίτας, κιννάβαριν ὁ φιλόσοφος, ἢ κλαυδιανὸν, ἢ καδμίαν, ἢ ἀνδροδάμαντα, ἢ χρυσόκολλαν · ἢ ὅτι δεῖ ὑπὸ τὸν μολυβδόχαλκον, κιννάβαριν, σῶμα\footnote{ὅ τι δεῖ Lb.} μαγνησίας ὃ λέγεται μέλας μόλυβδος. Κἄν τε πάλιν ἐν τῇ χρυσοποιΐᾳ ἀπέλθῃς καὶ εὑρήσῃς αὐτὰ κασσίτερον σκορπίζοντα ἢ σίδηρον\footnote{ἔλθῃς B etc. --- κασσίτερον] ὑδράργυρον Lb seul.} ἢ χαλκὸν κιννάβαριν ὄντα, ἢ λιθάργυρον λευκὴν, σὺ πάλιν τὸ σὸν\footnote{τὸ σὸν] τὸ ... (lettres effacées) M.} νόει, τῇ μαγνησίᾳ τὸν μολυβδόχαλκον ἢ μόλυβδον τὸν μολυβδόχαλκον.\footnote{νόει, ἤγουν τῆς μαγνησίας τὸν μολυβδόχαλκον ἢ τὸν μόλυβδον Lb seul.} Κἂν γὰρ ἀργυροποιΐαν λέγουσιν, ἢ χρυσοποιΐαν (f. 157 v.) περὶ τοῦ μολυβδοχάλκου λέγουσιν · ὅπερ ἀπαρτίσαντες ἔχουσιν ἀποκείμενον\footnote{ὅνπερ Lb.} · καὶ ὅτε θέλουσιν, σκορπίσαντες πήσσουσιν · καὶ τότε λευκαίνουσιν ἢ ξάνθουσιν ἄρρευστον αὐτοῖς.

3. Λευκαίνουσι δὲ θεῖον, καὶ λειώσαντες ἔχουσιν εἰς τὰ ἑπόμενα\footnote{θείῳ Lb seul.} τοῦ ἀποτελέσματος · ταύτην τὴν μετὰ θείου καὶ ὑδραργύρου καλοῦσιν\footnote{M mg. : κράτει (à l'encre noire, sur une ligne verticale, avec renvoi à ἀποτελέσματος).} καῦσιν · καὶ χαλκὸν κεκαυμένον τὸν αὐτὸν, ὡς καὶ λεύκωσιν αἱμωπὸν,\footnote{ὡς] ὥστε E.} κατὰ τὴν ἐπιφάνειαν, καὶ κατὰ τὸ βάθος ἔχων εὑρίσκεται.\footnote{ἔχων] ἔχον BAK ; ἔχον ουσιν et ειν superposés E ; ἔχειν Lb.} Τοῦτο οὖν λέγουσι καῦσιν · διὰ δὲ τούτου τὸ ὅλον σύνθεμα αἰνιττόμενος, τὰς εἰς ἀμφοῖν αὐτοῦ λειώσεις ἐμήνυσεν, ὀρθῇ ὁδῷ χρησάμενος, πρῶτον τὸ λευκαίνειν εἴρηκεν, ἔπειτα τὸ ξανθῶσαι.

\bigskip
\centerline{\EightStarTaper}
\centerline{\EightStarTaper\EightStarTaper}
\bigskip

\subsubsection{3. --- 27. ΠΕΡΙ ΟΙΚΟΝΟΜΙΑΣ ΤΟΥ ΤΗΣ ΜΑΓΝΗΣΙΑΣ ΣΩΜΑΤΟΣ.}
\paragraph{}
\emph{Transcrit sur} M, f. 157 v. --- \emph{Collationné sur} B, f. 145 v. ;--- \emph{sur} A, f. 131 v. ;--- \emph{sur} K, f. 28 r. ;--- \emph{sur} E, f. 73 v. ;--- \emph{sur} Lb, p. 281. --- \emph{Les νariantes et restitutions de} M \emph{ont été reportées en marge de} K. --- \emph{Chap.} 50 \emph{de la compilation du Chrétien dans} E Lb.

\bigskip

1. Πάλιν\footnote{μέσην M. --- A mg. : Παυφνουτίας καὶ Παυφνουτίου < π > υρίσεις ... ντὸς τοῦ λόγου.} τοὺς ἀρχαίους εἰς μέσον φέρωμεν · κιννάβαριν λέγουσιν\footnote{[καὶ] om. B etc.} λοιπὸν τὴν λεύκωσιν τῆς μαγνησίας · ὡς καὶ τοὺς πρώην λόγους [καὶ]\footnote{ὑποστατὰ mss. (Oxyton.) Cp. p. 148, l. 6 (note). --- γενέσθαι λέγουσι Lb.} οὓς ἔγραψα ἀργοὺς γενέσθαι · περὶ οὗ τὰ ὑπόστατα τέσσαρα σώματα\footnote{λέγουσιν ἀναδέξασθαι Lb seul.} · καὶ ὅτι περὶ αὐτῶν ἔχει σταθμὸν, ὠμὸν καὶ ἑφθὸν τὸ σύνθεμα, καὶ διὰ τὸν λόγον τῆς μαγνησίας πάντα ἐκεῖνα ἀναδέξασθαι. Πῶς οὖν γίνεται\footnote{εἶπον, πάλιν λέγω · ἄφες Lb. --- ἄφες B etc. --- (Cp. p. suiv., l. 1).} τὸ σῶμα τῆς μαγνησίας, εἰ ἔχει διαφορὰν κατὰ τὴν ταριχείαν ἡ λεύκωσις, οὕτως ὡς πρώην σοι εἶπον, ἀφεὶς ἀπέναντι τῆς καμίνου\footnote{κοβαθίων M.} ; Ἡ δὲ κάμινος καιέσθω τοῖς ξύλοις καὶ λεπύροις φοινίκων [καὶ] κωβαθίων. Ὁ γὰρ καπνὸς τῶν λεπύρων πάντα λευκαίνει. Ἐὰν οὖν λάβῃ τὸν καπνὸν, συλλαμβάνει ἡ μαγνησία καὶ λευκαίνεται.

2. Οὐκ ἐμνήσθημεν δὲ ἐν τῷ ἑβδόμῳ λόγῳ περὶ τῶν κωβαθίων τῶν\footnote{ὀφείλομεν δὲ Lb.} φοινίκων < ὅτι > ὀφείλομεν μαθεῖν πρῶτον ποίαν μαγνησίαν λέγουσιν\footnote{φανερὸν δὲ ὅτι Lb seul.} οἱ φιλόσοφοι, τὴν ἁπλῆν τὴν ἀπὸ Κύπρου, ἢ τὴν σύνθετον τὴν ἀπὸ τῆς ἡμῶν τέχνης ; ὅτι τὴν ἁπλῆν λειώσαντες, σύνθετον αἰνίττονται.\footnote{ἐκ τοῦ περὶ αὐτῶν διπλῶς διαλέγεσθαι B etc.} Ἔλεγον δὲ ὁμοῦ καὶ περὶ τῆς (f. 158 r.) ἁπλῆς. Οὕτω γὰρ ἐκρύβη ἡ τέχνη ἐκ τοῦ περὶ διπλῶν διαλέγεσθαι.\footnote{νίτρον en signe M ; signe du molybdochalque BAKE ; μολυβδόχαλκον en toutes lettres Lb.}

3. Ὅτι ὁ φιλόσοφος Ἑρμῆς, μετὰ τὴν θαλασσίαν βάλλει νίτρον\footnote{κνίπιον M, ici et partout.} καὶ ὄξος καὶ κνίπειον αἷμα, χυλὸν στύρακος, καὶ στυπτηρίαν σχιστὴν καὶ\footnote{προεῖπον · ἡ δὲ κάμινος καιέσθω Lb, puis add. de Lb seul : τοῖς ξῦλοις καὶ.} τὰ ὅμοια · καί φησιν · « Ἄφες αὐτὴν ἀπέναντι τῆς καμίνου, ὡς προεῖπεν\footnote{λεπ. τῶν κωβ. τῶν φοιν. Lb seul.} λεπύροις φοινίκων κωβαθίων. Ὁ γὰρ καπνὸς φοινίκων τῶν κωβαθίων,\footnote{λευκὸς ὢν om. M.} λευκὸς ὢν, πάντα λευκαίνει.\footnote{ὀφ. δὲ, B etc. --- Au-dessus de νίτρον et des autres noms : θεῖον en signe MB etc.} »

4. Ταῦτά φησιν ὁ Ἑρμῆς · Ὁφείλομεν εἰδέναι ὅτι τὸ νίτρον καὶ ὁ στύραξ\footnote{ἡ σποδιὰ τῶν αἰθαλῶν Lb. --- M mg. : signes de νεφέλη et de θεῖον.} καὶ ἡ στυπτηρία σχιστὴ καὶ ἡ σποδὸς τῶν θαλλῶν τῶν φοινίκων,\footnote{Signe du cinabre au-dessus de ἀσβέστου MBAK. --- δι ᾽ ἀσβέστου καὶ κινναβάρεως Lb.} τὸ λευκὸν θεῖόν ἐστιν ὃ λευκαίνει πάντα · τὸ δὲ κνίπειον αἷμα καὶ τὸ ὄξος, ὕδωρ θεῖον τὸ δι ᾽ ἀσβέστου · τὰ δὲ λέπυρα τῶν κωβαθίων\footnote{τὸ θειῶδές ἐστι, μάλιστα τῆς σανδαράχης, ὅπερ Lb.} τῶν φοινίκων τὰ θειώδη εἰσὶν, μάλιστα ἀρσένικον, ὅπερ ἔοικεν\footnote{κωβαθίῳ, καὶ χρυσίζει. B etc. --- Au-dessus de κωβαθίων, le signe du soufre B.} κωβαθίοις, τὸ χρυσίζειν. Καί φησιν · « Ὁ καπνὸς τῶν κωβαθίων\footnote{Au-dessus de θείου, le signe du mercure M.} πάντα λευκαίνει, » ἅπερ κωβάθια θέλων διδάξαι ὁ φιλόσοφός φησιν · « Ὁ γὰρ καπνὸς τοῦ θείου λευκαίνει πάντα.\footnote{τὴν σποδὸν B etc.} »

5. Πάλιν δὲ τὸν σποδὸν τῶν θαλασσίων τῶν φοινίκων σε θέλων\footnote{σποδοῦ Lb. F. l. σποδὸν.} διδάξαι, ὁ φιλόσοφος, ὅ ἐστιν ὕδωρ θεῖόν φησιν οὕτως · « Ἀναλύσας ἐν ὕδατι < θείῳ > σποδῷ λευκίνων ξύλων, ἐν τῇ δευτέρᾳ τῶν λευκῶν\footnote{σποδὸς Lb. --- θείου Lb seul.} ζωμῶν, σποδὸν λευκίνων οὐκ ἔστιν ἀπλῶς, ἀλλ ᾽ ὕδωρ θεῖον\footnote{M mg. : μη puis le signe de l'or, avec renvoi à μαρμάρου. --- τῆς τοῦ μαρμάρου γεγονέναι φησὶν ἢ ἀσβέστου Lb.} τὸ δι ᾽ ἀσβέστου, ὅπερ ἀπὸ σποδοῦ λευκῆς τῆς τοῦ μαρμάρου ἢ ἀσβέστου\footnote{μαγνησίας λευκοῦ B etc.} γεγονέναι. Ὥσπερ οὖν τὰ θειώδη ἀπὸ τῶν κωβαθίων τῶν φοινίκων ἐρρήθη, ὡσαύτως καὶ τὸ ὕδωρ τοῦ θείου ἀπὸ θείου ἔχον τὴν σύνθεσιν, τὸ τηνικαῦτα καὶ αὐτὸ ἀπὸ τοῦ φοινικοῦ προσηγορεύθη. Ἔτι οὖν ἡ λεύκωσις τῆς συνθέτου μαγνησίας ἀπὸ θείου συνθέτου λευκοῦ,\footnote{F. l. λευκὸν. --- ὢν M.} καὶ ὕδωρ σύνθετον λευκοῦ τὸ δι ᾽ ἀσβέστου, ὧν τὴν σύνθεσιν ἐν τῷ\footnote{λόγῳ εἰρήκαμεν E. --- M mg. : ερμ (Ἑρμῆς ? ) en lettres retournées.} περὶ συνθέσεως λόγῳ, τὸν δὲ σταθμὸν ἐν τῷ περὶ σταθμῶν λόγῳ,\footnote{καὶ τὴν τῆς καμ. ἀγ. Lb.} τὴν δὲ ὄπτησιν καὶ τῆς καμίνου ἀγωγὴν ἐν τῷ περὶ ὀπτήσεως λόγῳ.\footnote{λευκώσεως] λειώσεως B etc.}

6. Καὶ ταῦτα μὲν περὶ λευκώσεως σώματος μαγνησίας (f. 158 v.).\footnote{ἐπιβαλέσθαι B etc. ; E mg. : \emph{alias} ἐπιβαλέσθαι. --- M mg. : N° (νόει) puis le signe de l'or.} Ἔξεστιν δὲ καὶ ὑμῖν τοῖς ἐχέφροσιν τὸ βέλτιον ἐπιβάλλεσθαι ἡμᾶς\footnote{μᾶλλον δὲ μὴ B etc. F. l. μᾶλλον ἢ.} καὶ ὠφελῆσαι, μᾶλλον δὲ κατ ᾽ ἐκείνου βαράθρου κατακρημνίσαι\footnote{περὶ] παρὰ M. --- οἱ γὰρ jusqu'à la fin du §]. Tous les nominatifs au pluriel Lb.} ἡμᾶς. Ὁ γὰρ περὶ αὐτὴν τὴν διδασκαλίαν ἕτερόν τι λογιζόμενος,\footnote{ματαιοπονούμενος M.} ἐν σκότῳ μεγάλῳ ἀνεχόμενος, ψηλαφᾶν ταῖς χερσὶ τὸν ἀέρα ἔοικε, καὶ τὸν πόντον τοῖς ποσὶν, οἵ κενεμβατοῦντες καὶ εἰς αὐτὸν λαλοῦντες τὸν ἀέρα μάταια, διόλου τὸν τύπον τοῦ σώματος πρὸς τὴν ἰδίαν αὐτῶν ἐνέργειαν ματαιοπονούμενοι.\footnote{Zosime s'adresse à Théosébie.}

7. Σὺ δὲ, ὦ μακαρία, παῦσαι ἀπὸ τῶν ματαίων στοιχείων,\footnote{ταραττουσῶν mss. ;--- όντων au-dessus de ουσῶν E. --- Ταφνουτίης M.} τῶν τὰς ἀκοάς σου ταραττόντων. Ἤκουσα γὰρ ὅτι μετὰ Παφνουτίας τῆς\footnote{ἄλλων om. B etc., f. mel.} παρθένου καὶ ἄλλων τινῶν ἀπαιδεύτων ἀνδρῶν διαλέγῃ · καὶ ἅπερ ἀκούεις\footnote{καὶ ἐκείνους διελεηθῆναι B etc. --- Le mot διελεηθῆναι termine le fol. 28 du ms. K. La suite est à la première ligne du fol. 108.} παρ ᾽ αὐτῶν μάταια καὶ κενὰ λογύδρια, πράττειν ἐπιχειρεῖς. Παῦσαι οὖν ἀπὸ τῶν τε τυφλωμένων τὸν νοῦν καὶ ἄγαν καιομένων.\footnote{χρημάτων] ῥημάτων B etc.} Καὶ γὰρ κἀκείνους ἐλεηθῆναι δεῖ καὶ ἀκοῦσαι τὸν λόγον τῆς ἀληθείας, καθώς εἰσιν ἄξιοι. Ἐπειδὴ καὶ αὐτοὶ ἄνθρωποί εἰσιν, ἀλλ ᾽ οὐ βούλονται ἐλέους ἐπιτυχεῖν, οὐδὲ παρὰ διδασκάλων ἀνέχονται διδάσκεσθαι, καυχώμενοι διδάσκαλοι εἶναι, ἀλλὰ καὶ τιμᾶσθαι βούλονται ἐκ τῶν ματαίων αὐτῶν καὶ κενῶν λογυδρίων. Καὶ διδασκόμενοι βαθμοὺς ἀληθείας, τὴν τέχνην οὐκ ἀνέχονται, οὐδὲ πέπτουσιν, χρυσοῦ μᾶλλον ἢ λόγων ἐπιθυμοῦντες · καὶ ἀπὸ θερμότητος καὶ πολλῆς ἀνοίας, ἄμοιροι γίνονται τῶν λόγων καὶ τῶν χρημάτων. Εἰ γὰρ ἠνιοχοῦντο ὑπὸ\footnote{εἰ μὴ γὰρ BAK. --- M mg. : N° M.} τοῦ λόγου, εἵπετο ἂν αὐτοῖς καὶ ἠκολούθει ὁ χρυσός · ὁ γὰρ λόγος\footnote{ἠκολούθη M.} δεσπότης ἐστὶν τοῦ χρυσοῦ, καὶ ὁ τοῦτον προσπίπτων καὶ ποθῶν καὶ\footnote{τούτῳ Lb.} προσκολλώμενος εὑρήσει τὸν χρυσὸν τὸν ἔμπροσθεν ἡμῶν κείμενον, σκολιῶς διακεκρυμμένον.

8. Ὁ οὖν λόγος δείκτης ἐστὶν πάντων τῶν ἀγαθῶν, ὡς καθώς πού φησιν, ἡ φιλοσοφία γνῶσις ἐστὶν ἀληθείας, εἰ ὄντα εἰσίν · καὶ ἐάν τις\footnote{φησὶν ὁ φιλόσοφος, ἡ φ. Lb. --- ὄ. ἐστὶ B etc. --- Réd. de Lb : ἡ φ. ἐστι γν. ὄντων (biffé E) ᾗ ὄντα ἐστὶ.} τὸν λόγον δέξηται, ἕ- (f. 159 r.) ξει αὐτὸν δεικνύοντα αὐτῷ ἐν τοῖς ὀφθαλμοῖς κείμενον χρυσόν. Οἱ δὲ μὴ ἀνεχόμενοι τῶν λόγων πάντοτε κενεμβατοῦσιν, γέλωτος ἰσχυρότερα ἔργα ἐπιχειροῦντες · οἷόν ποτε γέλωτα ἐκίνησεν Νεῖλος ὁ σὸς ἱερεὺς, μολυβδόχαλκον ἐν κλιβάνῳ\footnote{ὁ Nεῖλος Lb. --- ὅσος MBA. Noter qu'une lettre de l'ascète Nilus (liv. 11, l. 15, éd. Allatius) est adressée à « Théosébius. »} ὀπτῶν · ὥστε ἐὰν βάλῃς ἄρτους καίων κωβαθίοις πανημέριος τύχοις · καὶ τυφλούμενος τοὺς σωματικοὺς ὀφθαλμοὺς, οὐκ ᾤετο τὸ βλαβησόμενον, ἀλλὰ καὶ ἐφυσιοῦτο, καὶ μετὰ τὸ ψυγῆναι ἀνενέγκας, ἐπεδείκνυεν\footnote{F. l. ἐφυσιᾶτο.} τὴν τέφραν. Καὶ ἐπερωτώμενος ποῦ ἡ λεύκωσις, καὶ ἀπορήσας ἔλεγεν ἐν τῷ βάθει αὐτὴν δεδυκέναι. Εἶτα ἐπέβαλεν χαλκὸν, ἔβαπτεν σποδόν. Οὐδὲν γὰρ στερρὸν διατραπεὶς, ἀνέστη καὶ ἔφυγεν αὐτὸς ἐν τῷ βάθει, καθὼς ἡ λεύκωσις τῆς μαγνησίας. Ταῦτα δὲ ἀκούσας\footnote{F. l. ἀκούσασα.} παρὰ τῶν διαφερόντων Παφνουτία, ἀπὸ τοῦ πολλοῦ γέλωτος ἐκακώθη,\footnote{παρὰ] περὶ B etc. --- Tαφνουτίη M ; παφνουτ B ; παφνουτίου A ; πτ αφνουτίου K ; τῆς Παφνουτίας ELb.} ὡς καὶ ὑμεῖς κακοῦσθε ἀπὸ ἀνοίας. Ἄσπασαί μοι Νεῖλον τὸν κωβαθηκαύστην, πλήρης.\footnote{κωβατικαύστην BAK ; κωβαθιοκαύστην ELb. --- πλήρης (pleinement édifiée ? ) περὶ οἰκονομίας < τοῦ > τῆς μαγνησίας σώματος M (signe final après πλήρης, d'une main plus récente).}

\bigskip
\centerline{\EightStarTaper}
\centerline{\EightStarTaper\EightStarTaper}
\bigskip

\subsubsection[3. --- 28. ΠΕΡΙ ΣΩΜΑΤΟΣ ΜΑΓΝΗΣΙΑΣ ΚΑΙ ΟΙΚΟΝΟΜΙΑΣ < ΑΥΤΟΥ. >]{3. --- 28. ΠΕΡΙ ΣΩΜΑΤΟΣ ΜΑΓΝΗΣΙΑΣ ΚΑΙ ΟΙΚΟΝΟΜΙΑΣ < ΑΥΤΟΥ.\footnote{Titre dans Lb seul : Περὶ τοῦ σώμ. τῆς μαγν. καὶ τῆς οἰκ. αὐτῆς.} >}
\paragraph{}
\emph{Transcrit sur} M, f. 159 r. ;--- \emph{Collationné sur} B, f. 148 r. ;--- \emph{sur} A, f. 133 v. ;--- \emph{sur} K, f. 108 r. ;--- \emph{sur} E, f. 76 v. ;--- \emph{sur} Lb, p. 295. --- \emph{Les νariantes et restitutions de} M \emph{ont été reportées en marge de} K. --- \emph{Chap.} 51 \emph{de la compilation du Chrétien dans} E Lb.

\bigskip

1. Ταῦτα μὲν ἡ Μαρία ἄρτους ὄνομα < τὸ σῶμα > τῆς μαγνησίας ἀφθόνως\footnote{ὀνομάσασα B etc. --- < τὸ σῶμα > [Cp. l. 19.} καὶ φανερῶς ἐξέθετο. Ὁ γὰρ πρῶτος βαθμὸς ἀληθὴς τοῦ μυστηρίου\footnote{τοῦ μυστ. ὅλου ELb. (mots placés après διηγ. dans Lb).} ἐν τούτοις διηγόρευται. Μαρία οὖν βούλεται εἶναι τοῦτο τὸ σῶμα τῆς μαγνησίας · καὶ οὐ μόνον εἰς ἕνα τόπον κηρύττει, ἀλλὰ καὶ εἰς πολλούς. Ἀμέλει ἐν ἄλλῳ τόπῳ φησί · « Χωρὶς τοῦ μέλανος μολύβδου, οὐδὲν γίνεται ὃ ἀπηρτίσαμεν καὶ ἐτελειώσαμεν σῶμα μαγνησίας.\footnote{ὃ] Lb seul mg. : « \emph{Puto legendum} ᾧ, h. e., » puis les signes du plomb et du cuivre.} » Λὗταί εἰσιν, φησὶν, αἱ διδασκαλίαι · καὶ οὐκ ἀποκάμνει, δεύτερον γὰρ καὶ τρίτον διδάσκουσα καὶ καλοῦσα σῶμα μαγνησίας, καὶ μέλανα μόλυβδον καὶ μολυβδόχαλκον, περὶ οὗ φησιν « κιννάβαρις ἢ μόλυβδος\footnote{κιννάβαρις en signe M ; signe du cuivre BAK ; ὁ χαλκὸς ELb.} ἐτήσιος λίθος. » Ἑξῆς ὀμορρευστήσαντα ποιεῖ πάντα χρύσοπτα δυνάμει,\footnote{ποιεῖ] ποιοῦσι Lb.} τὰ ὠμὰ ὀπτὰ, ὀπτὰ διπλοῖ · δυνάμει, φησὶν, ποιεῖ πάντα χρύσοπτα\footnote{καὶ τὰ ὠμὰ ὄπτα · τὰ ὀπτὰ διπλῆ δυν. π. φ. BAK ; καὶ τὰ ὠμὰ ὀπτὰ · καὶ διπλῇ δυν. π. φ. ELb. --- F. l. τὰ ὠμὰ όπτᾷ, ὀπτὰ διπλοῖ δυνάμει, φησίν · ποιεῖ π. χρ.} · οὔπω γὰρ ἐνεργείᾳ. Καὶ περὶ (f. 159 v.) μὲν τούτου ἕτερός μοι λόγος ἀναγραφήσεται, ἐν δὲ τῷ παρόντι < ἐπὶ > τοῦ προκειμένου γινώμεθα.

2. Ἐδείχθη οὖν τῇ Μαρίᾳ τὸ πᾶν σῶμα μαγνησίας τοῦτο μολυβδόχαλκος\footnote{Réd. de E Lb : τὸ πᾶν σ. κατὰ τὴν μαγνησίαν εἶναι καὶ τοῦτο μολυβδόχαλκός ἐστι μέλας (μ. ἐ. E).} μέλας · οὔπω γὰρ ἐβάφη, καὶ τοῦτο · καὶ μολυβδόχαλκος\footnote{Réd. de Lb : καὶ οὖτός ἐστιν κ. μολ. ὃν μ. β.} · ὃ μέλλεις βάπτειν καὶ ἐπιβάλλειν αὐτῷ τὰ μωτάρια τῆς ξανθῆς σανδαράχης · ἵνα μηκέτι εἴη δυνάμει, ἀλλ ᾽ ἐνεργεία χρυσὸς ὀπτός. Οὕτως ἡ Μαρία ἄρτους ὀνομάσασα τὸ σῶμα τῆς μαγνησίας · ὀφείλομεν\footnote{Réd. de Lb seul : οὕτως οὖν ὀνομ. ἡ Μαρία τ. σ. τ. μ. ἄρτους φαν. ἐξέθετο τὴν τέχνην · ὀφ. --- ὀφ. δὲ B etc.} πρό γε πάντων δεῖξαι καὶ τὸν φιλόσοφον ταῦτα φρονοῦντα < περὶ > σώμα τῆς μαγνησίας, ὅπερ καὶ ΤΟ ΠΑΝ ἔλεγον · καὶ μέλανα μόλυβδον τοῦτο · μολυβδόχαλκος · Ἀλλ ᾽ ὅταν λέγωσι τὴν\footnote{ταὐτὰ Lb.} ὑδράργυρον πήγνυσθαι μετὰ τοῦ τῆς μαγνησίας σώματος, δι ᾽ ὅλου\footnote{μετὰ (f. l. σὺν) τῷ τ. μ. σώματι M.} τοῦ σώματος ἔλεγον, ὅπερ κατηχήθη ἐν τῷ προτέρῳ μου ὑπομνήματι,\footnote{κατήχθη MBAK ; κατελέχθη Lb seul. (Corr. de E.)} ὥσπερ ἡ Μαρία λέγει ἐν τῷ προλεχθέντι σώματι τῆς μαγνησίας. Καί φησιν · « Εὑρήσεις μόλυβδον μέλανα · τοῦτον ἄρας, χρῶ, μίξας αὐτῷ ὑδράργυρον. Ὃ δὲ καλοῦσιν αἰ τάξεις, τοῦτο ἐν προοιμίοις ὁ φιλοσόφος λέγει · ὑδράργυρον μῖξον τῷ τῆς μαγνησίας σώματι,\footnote{ὑδράργυρον om. M.} ὅτι μέλανα αὐτὸ οἶδεν ὁ φιλόσοφος, μόλυβδον, φησὶν ἐν τῷ πυρίτῃ\footnote{αὐτὸν Lb. --- φησὶ γἀρ Lb.} · οὐκ ἁπλῶς λέγει, ἵνα μὴ πλανηθῇς, ἀλλὰ « μέλανι τῷ ἡμῶν. » Ὁτι δὲ καὶ\footnote{ἀλλὰ τῷ μολύβδῳ τῷ μελ. τ. ἡ. Lb seul.} μολυβδόχαλκον οὐκ ἀγνοεῖς, φησὶν, ὅτι μόνη ὑδράργυρος τὸν χαλκὸν\footnote{μολ. λέγει Lb. --- φησὶ γὰρ Lb.} ἀσκίαστον ποιεῖ. Οὐκέτι σῶμα μαγνησίας πήσσει, ἀλλὰ καὶ χαλκόν.\footnote{ἧς οὐκέτι μόνον τὸ σ. ἡ μαγνησία πήσσει, ἀλλὰ καὶ τὸ τοῦ χαλκοῦ Lb. --- πήσει M. --- καὶ χαλκόν] καὶ om. M.} Οὕτω καὶ ὁ φιλόσοφος τὸ πᾶν οἶδεν σῶμα μαγνησίας καὶ μόλυβδον\footnote{μόλυβδον en signe M ; μολυβδόχαλκον en signes BAKE ; μόλυβδον χαλκοῦ Lb, puis : μέλανα καὶ μολυβδόχαλκον · ἐν δὲ τοῖς βιβλίοις ...} μέλανα · καὶ μολυβδόχαλκος ἐν τοῖς βιβλίοις τῶν ἀρχαίων μεληδὸν\footnote{βίβλοις M.} ἀπεδόθη, κατὰ μίαν τάξιν κηρυττόμενος · διὰ ὑδραργύρου κηρύττεται\footnote{ἀεὶ κηρυττ. · διὰ δὲ ὑδρ. Lb.} διὰ παντὸς λίθου, καθὼς καὶ ἐν τοῖς πρώτοις προσεφώνησα.

3. Τοῦτο οὖν δυνάμει χρυσὸς ὀπτός ἐστιν. Καὶ ἐὰν λευκανθῇ ἢ ξανθωθῇ,\footnote{ἢ] καὶ Lb.} τότε καὶ ἐνέργειαν ἔχει τὰ ὠμὰ μετὰ τῶν ὀπτῶν, τουτέστιν ἐὰν μὲν λευκὸν ἐπιβαλλόμενον χαλκῷ ὠμῷ, κυπρίῳ, ποιεῖ ἄργυρον\footnote{λευκὸν ᾖ Lb.} · ἐὰν δὲ ξανθωθῇ, ἐπιβαλλόμενον ἀργύρῳ ὠμῷ κοινῷ, ποιεῖ χρυσὸν,\footnote{χρυσὸν om. M. --- Réd. de Lb : χρυσόν · καὶ πάλιν λέγω βρίξας χάλκανθον οἴνῳ ἀμινέῳ, τουτέστι ὄξει κ.} χαλκάνθῳ βρέξας οἴνῳ ἀμιναίῳ < ἢ > ὄξει κοινῷ, ἔασον ἡμέρας ιδʹ,\footnote{ἀμοινέῳ BAKE (qui corrige en ἀμηνέῳ).} τοῦτο ἐστὶν τὸ ζητούμενον ἐπὶ τῆς τοῦ ἀργύρου ποιήσεως.

4. Ὡς πολλάκις ἀποτυγχάνουσι τῆς (f. 160 r.) οἰκονομίας,\footnote{ὡς] διὸ Lb.} διὰ τὸ μὴ εἰδέναι τὸ ἀληθὲς τῆς λειώσεως. Τοιούτων οὖν καὶ ἐπὶ τῶν νεφελῶν\footnote{τοιοῦτον B etc., f. mel.} ἐρρήθη ὅτι ἡ χάλκανθος ἐπὶ τὸ χρυσίζον ἄγει τὴν νεφέλην. Ὁμοίως καὶ ὁ Ἀγαθοδαίμων ἐν τῇ διδασκαλίᾳ τοῦ προβαφίου τοῦτο ἔλεγεν · « Ἵνα εἰδέναι ἔχῃς ὃ ἐνεργεῖς, ἐλθὼν εἰς τὴν χάλκανθον ἣν οἶδας,\footnote{ἔχοις M ; ἔχεις AKLb. --- ἐλθὼν γὰρ Lb. --- τὸν χ. ὃν Lb seul.} τὸ βαπτικὸν αὐτῆς τὴν νεφέλην ἐπὶ τὸν χρυσὸν ἄγει. Ἐφάνη οὖν ἡ\footnote{αὐτοῦ Lb seul. --- ἄγει] λέγει M ; ἄγε Lb.} ἀναγραφὴ περὶ ἐξιώσεως, ἐμνήσθη δὲ περὶ τῶν ἀμφοῖν ὅτι περὶ σταθμοῦ\footnote{ἡ ἀναγραφὴ] τῇ γραφῇ B etc. --- ἀμφοτέρων Lb.} ὁ λόγος περὶ τῶν καλλίστων καὶ θεοφιλῶν λίθων καὶ λευκῶν,\footnote{αύτῷ ὁ λόγος καὶ Lb.} καὶ αἱμωπῶν · οὓς οἱ μὲν ἐκάλεσαν πυρίτην, ὡς πολύχροον καὶ πολυώνυμον,\footnote{αἱμοπὸν ὧν M. --- πολυχρόους καὶ πολυωνύμους BAKE ; om. Lb.} οἱ δὲ ἀλάβαστρον · οἱ δὲ καὶ ἀμφοῖν εἶπον πυρίτην ὃ καὶ ἀπεδειξάμην.\footnote{M mg. : $\svgA$, avec renvoi à οἱ δὲ. --- ἄμφω Lb seul.} Ἄλλος γὰρ οὐκ ἂν εἴη κάλλιστος καὶ θεοφιλῆς,\footnote{ἄλλως E.} εἰ μὴ ὁ πυρίτης. »

5. Νῦν δὲ περὶ σώματος μαγνησίας ὁ λόγος πρόκειται · ὅτι περ τὰ πάντα ὑφ ᾽ ἓν γενόμενα μετὰ τοῦ ἀληθοῦς σταθμοῦ τῆς δεούσης ταριχείας · ἡ κιννάβαρις ποιεῖ τὸ ἀληθινὸν σῶμα μαγνησίας.\footnote{ἡ] ἢ MELb. --- κινναβάρεως Lb. --- σῶμα τῆς μαγνησίας B etc.} Καὶ τοῦτο ἀληθῶς μὴ πλανῶν, ἤθελον κἀγὼ τηλικοῦτος εἶναι κατ ᾽ ἐκεῖνον\footnote{κατὰ τοῦτο ἀληθῶς μὴ πλανῶ Lb.} τὸν εἰπόντα · « Ὦ γῦναι, οὐχ ἁπλῶς ἔλεγον, ἵνα μὴ πλανηθῇς. » Ἀλλ ᾽ ἐπειδὴ οὐκ εἰμὶ ὁ Δημόκριτος, ὠμνύω σε κατὰ τῆς ἐκείνου\footnote{ὄμνυμί σοι B etc.} ἀρετῆς τοῦτο, ὅτι μὴ πλανῶ · καὶ αὕτη μετὰ τῶν τὴν ἀνεπίστροφον πλάνην πλανωμένων, καὶ λεγόντων ὅτι ὁ σπόρος ἀσώματος λέλεκται\footnote{ἀσώματον mss. Corr. conj.} τὸ τῆς μαγνησίας σῶμα. Φησὶν αὐτῆς τὸ ἀσώματον ὑδράργυρον εἶναι.\footnote{καὶ τὸ τῆς μαγνησίας σῶμα Lb. φασὶν, Lb seul.} Φημὶ κἀγὼ ὅτι νενόηταί τι αὐτοῖς. Δείξουσιν τοιγαροῦν ἡμῖν τὸ ἀποτέλεσμα,\footnote{τι] τις BA ; τοῖς ELb. --- δεῖξον B etc.} ἐξ οὗπερ ὁ νοῦς αὐτῶν συσταθμίζεται. Ἀλλ ᾽ οὔτε ἀποτέλεσμα ἔχουσιν · οὐ γὰρ σῶμα μαγνησίας ἐλέχθη ὁ σπόρος, ἀλλ ᾽ ἀσώματον.\footnote{ἔχουσιν οὔτε ἄλλο τι · οὐ γὰρ Lb.} Καὶ γὰρ ἡ ὑδράργυρος σῶμα. Κἂν λεπτομερές μοι τοῦτο\footnote{λεπτόμερόν M. F. l. λεπτομερῶς.} εἴπῃς τὰ ὅλα σώματα, ἆρα οὖν ὁ σπόρος τῶν ἀσωμάτων ἐλέχθη σῶμα μαγνησίας ; οὔ, ἀλλὰ τί βούλεται ; ἐπειδήπερ θειώδη ὄντα φεύγουσιν.\footnote{οὐκ B etc., mel.} Τὸ τηνικαῦτα οὖν κρατηθέντα καὶ μηκέτι φεύγοντα, σῶμα (f. 160 v.) προσαγορεύονται · ἀφ ᾽ οὗ καὶ ἡ Μαρία · « τὸ σῶμα τῆς μαγνησίας τὸ ἀπόκρυφον, φησὶν, ἐκ μολύβδου καὶ ἐτησίου καὶ χαλκοῦ γίνεται.\footnote{M mg. : ὧδε, à l'encre noire (15\textsuperscript{e} siècle).} »

6. Λοιπὸν ὅσα ὅμοια τοῖς φεύγουσι συγκραθέντα, σῶμα προσαγορεύονται\footnote{ὅσα εἰσὶν ὅμ. Lb. --- συγκραθέντα] συναχθέντα Lb.} · οἷον ἐπὶ τῆς ὑδραργύρου ἐν τοῖς λευκοῖς ζωμοῖς φησι · « Πρόσμιξον αὐτὴν στυπτηρίαν σχιστὴν ἢ μόλυβδόχαλκον, ἢ ἄσβεστον,\footnote{αὐτῇ Lb, mel.} ἵνα γένηται σῶμα ἡ ἀσώματος. Πάλιν ἐπὶ τῆς χρυσοκόλλης. Καὶ γὰρ καὶ αὐτὴ φεύγει · ἀφ ᾽ οὗ καὶ ὁ Ἀγαθοδαίμων · « Πρόσεχε, φησὶν, ἵνα μὴ τὸ πνεῦμα αὐτῆς τὸ βαπτικὸν φύγῃ. » Καὶ αὐτὴν φευκτὴν οὖσαν σῶμα καλοῦσιν · συγκραθεῖσαν ὁ φιλόσοφός φησιν ἐν\footnote{M mg., sur une ligne verticale : N° ἀλη (νόει ἀληθές ? ).} τῇ τάξει τῆς χρυσοκόλλης. Ἐπιβάπτε πᾶν σῶμα χαλκῷ, ἀργύρῳ,\footnote{χαλκοῦ etc. (génitif partout) Lb seul ; signes dans les autres mss.} χρυσῷ. Ἡ Μαρία περὶ τῆς χρυσοκόλλης, μολυβδόχαλκόν φησι · μονοήμερον οὐγγιάσας, ἢ λαβὼν, φησὶν, χρυσοκόλλαν καὶ κιννάβαριν,\footnote{οὐγκιάσας M. --- φησὶν] μέρος Lb. --- χρυσοκόλλης B etc. --- κινναβάρεως Lb seul ; signe dans les autres mss.} συλλείου αὐτῇ λιθάργυρον λευκὴν καὶ κατάσπα. Καὶ ἐὰν στραφῇ καὶ\footnote{ἐστραφῇ M ; ἐκστραφῇ B etc.} γένηται σῶμα χαλκοῦ, ἐπίβαλλε χρυσάνθιον, καὶ ἔσται χρυσός. Λοιπὸν καὶ ἡ χρυσοκόλλα χρηματίζει συγκραθεῖσα καλῶς, καίτοι καὶ αὕτη φευκτὴ οὖσα, ὅτι καὶ αὐτὴν ποιήσεις σῶμα διὰ τῆς στροφῆς.

7. Οὐκοῦν τὸ στρέψαι ἢ ἐκστρέψαι παρ ᾽ αὐτοῖς ἐστιν, ἵνα τὰ\footnote{ἢ] καὶ Lb.} ἀσώματα, τουτέστιν τὰ φεύγοντα, σωματωθῇ, καὶ κατασπασθεὶς γένηται\footnote{σώματα MBAK. --- κατασπασθέντα Lb.} μολυβδόχαλκος ὁ μέλας μόλυβδος ὁ μέλλων οἰκονομεῖσθαι μετὰ τῆς ὑδραργύρου, καὶ γένηται σῶμα μαγνησίας. Καὶ οὐχ ὥς τινες τὴν\footnote{γίνεται Lb. --- τὸ καυθὲν, σῶμα τῆς μαγνησίας B etc.} ἐκστροφὴν τὸ στρέψαι καὶ ἐκστρέψαι ὑδράργυρον βούλονται · ἀλλ ᾽ ὅταν σωματωθῶσιν τὰ φεύγοντα, ὡς ἐπὶ πάντων τῶν σωμάτων,\footnote{Réd. de Lb : τῶν σωμάτων ἐστίν · ἡ δὲ στροφὴ ...} ἡ στροφὴ εἰς τὸ λευκὸν ἢ εἰς τὸ ξανθόν. Καὶ γὰρ αὕτη ἡ στροφὴ ἐκτροφὴ\footnote{εἰς τὸ ξανθὸν γίνεται Lb.} καλεῖται, μετὰ τὸ σωματωθῆναι τὰ ἀσώματα, ὥσπερ < κατὰ > τὴν τέχνην, ὡς πρὸς τὸ πῦρ ἐν τῇ παλιντροπῇ, τουτέστιν τῇ λευκώσει ἢ ξανθώσει λειούμενα σφόδρα καὶ πυρὶ προσομιλοῦντα πάλιν ἐξαιθαλοῦνται,\footnote{προσομιλ. λέγομεν · πάλιν δὲ Lb.} καὶ γίνονται ἀσώματα. Εἰώθασιν γὰρ πάνυ λελειωμένα εἶναι. Αἰθάλη δὲ, ὠς πρώτη ἀσώματος, (f. 161 r.) ὡς πρώτην τέχνην λέγει.\footnote{αἰθάλην δὲ ὡς πρώτην ἀσώματον Lb. --- F. l. εἰς πρ. τέχνην ἄγει. Cp. § 4, p. 194, l. 2.}

8. Ἀπὸ ἀσωμάτων οὖν καὶ πάλιν σωματοῦνται μετὰ τὴν ὑδράργυρον\footnote{ἀσωματοῦνται mss. Corr. conj. --- μετὰ τῆς ὑδραργύρου B etc., f. mel.} ἐν τῇ ἰώσει, ἵνα γένηται σώματος. Καὶ σαπέντα ἀσωματοῦνται,\footnote{σώματος] ἀσώματα B etc. F. l. σωμάτωσις.} ἔχοντα καλῶς ἐνεργοῦντα χωρὶς πυρός. [αʹ] Ἀλλαχοῦ ἐλέχθη\footnote{F. l. δέχοντα < τι > καλῶς ἐνεργοῦν χ. π. --- καὶ ἐνεργ. Lb. --- αʹ dans M seul. --- ἀλλαχοῦ δὲ Lb.} · χολαὶ καὶ τὰ ὅμοια, ἅπερ καὶ αὐτά εἰσιν < μετὰ > τοῦ θείου ἤγουν μετὰ θείου ὕδατος. Τί δὲ ἄλλο καλῶς ἐνεργεῖ χωρὶς πυρὸς, ἢ ὕδωρ θεῖον\footnote{θείου Lb seul ; signe dans les autres mss.} ; ἀφ ᾽ οὗ καὶ Πηβίχιος ὅτι παντὸς πυρὸς δυναμικώτερον καὶ ἐν τοῖς θείοις, ὅτι χωρὶς πυρὸς δρᾷ. Καὶ Μαρία · « τὸ πύρινον φάρμακον.\footnote{θείοις] signe figuré dans les notations alch. (\emph{Introd.}, p. 112, pl. 4, l. 18), et confondu avec celui de la pl. 5, l. 2, dans ELb, qui écrivent : ἐν δὲ τοῖς πετάλοις σιδηροῖς. --- δρᾷ] δρῶσι Lb.} » Καὶ πάλιν λέγει ὅτι « εἰ μὴ τὰ σώματα ἀσωματωθῇ, καὶ τὰ ἀσώματα σωματωθῇ, οὐδὲν τῶν προσδοκωμένων ἔσται, » τουτέστιν, ἐὰν μὴ τὰ πυρίμαχα συγκραθῶσιν μετὰ τῶν φευγόντων τὸ πῦρ, οὐδὲν ἔσται τῶν προσδοκωμένων.

9. Τί οὖν ἄρα καὶ τὰ σώματα καὶ τὰ ἀσώματα τῆς ἡμῶν τέχνης\footnote{Τί] τίνα Lb. --- ἆρα M.} ; Ἀσώματα μὲν πυρίτης καὶ τὰ ὅμοια, μαγνησία καὶ τὰ ὅμοια, ὑδράργυρος καὶ τὰ ὅμοια, χρυσόκολλα καὶ τὰ ὅμοια, πάντα ἀσώματα\footnote{πάντα] F. l. ταῦτα. --- καὶ τὰ πάντα ὅμοια, ἀσώματα B etc.} · τὰ δὲ σώματα χαλκὸς, σίδηρος, κασσίτερος, μόλυβδος\footnote{M mg., sur une ligne verticale, à l'encre noire : ἀπ ᾽ ὦδε τέλειον.} · ταῦτα οὐ φεύγουσι τὸ πῦρ · ταῦτα σώματα. Ἐπὰν ταῦτα ἐκείνοις\footnote{ταῦτα γὰρ οὐ φεύγ. Lb seul.} συγκραθῶσι, γίνονται τὰ σώματα ἀσώματα, καὶ τὰ ἀσώματα, σώματα. Οὕτως πρόσμισγε ὑδράργυρον ἣν καλοῦσιν αἱ τάξεις, καὶ ποιεῖς πᾶν προσδοκώμενον, περὶ οὗ ἔλεγεν ἡ Μαρία · « Ἐὰν μὴ τὰ δύο γένηται ἓν, τουτέστιν, ἐὰν μὴ τὰ φεύγοντα συγκραθῶσι τοῖς μὴ φεύγουσιν, οὐδὲν ἔσται τῶν προσδοκωμένων · ἐὰν μὴ λευκανθῇ, καὶ γένηται τὰ\footnote{λέγω δὲ, ἐὰν ... Lb.} δύο τρία μετὰ τοῦ λευκοῦ θείου, τοῦ λευκαίνοντος αὐτό. Ἐπειδὰν δὲ\footnote{Après αὐτό] Suppléer οὐδὲν ἔσται τ. προσδ. ?} ξανθωθῇ, γέγονε τὰ τρία τέσσαρα · διὰ γὰρ ξανθοῦ θείου ξανθοῦται. Ἐπειδὰν δὲ ἰωθῇ, γέγονε τὰ ὅλα ἕν.

10. Τί βούλεται Ὁστάνης ; λέγει γὰρ περὶ τῆς συγκράσεως τῶν\footnote{ὁ Ὁστ. λέγειν περὶ τ. συγκρ. ELb ; γὰρ om. BAK. F. l. δὲ.} φευγόντων καὶ τῶν μὴ φευγόντων · « Πάλιν συγγένειαν ἔχει ὁ πυρίτης\footnote{φησὶ γὰρ, πάλιν E.} λίθος πρὸς τὸν χαλκόν. » Ὁ γὰρ Ὁστάνης οὐ περὶ ὑδραργύρου (f. 161 v.) ἔλεγεν, ἀλλὰ περὶ τῆς ἄγαν λειώσεως, ἵνα λειούμενος ὑποστάθμην μὴ ἔχῃ, ἀλλ ᾽ ὅλος ᾖ ὅλον ὕδωρ. Ἤδη δεῖ σε νοεῖν περὶ\footnote{M mg. : ὑποσταθμὴν, à l'encre noire. --- ἢ M ; ἣ B ; ἦ AK. --- ὅλος] ὅλως E.} ὕδατος ἢ ἐξυδατισμοῦ < ἅ τινα > ὁ φιλόσοφος καλῶς ἐν ταῖς πλύσεσιν καὶ λειώσεσιν διέλαβεν περὶ τῆς λειώσεως, καὶ ἔλεγεν · « Ἵνα γένηται ὡς ὕδωρ. Ὁ φιλόσοφος πάλιν · « Συγγένειαν ἔχει ἡ μαγνησία καὶ ὁ μαγνήτης πρὸς τὸν σίδηρον. » Πάλιν ὁ διδάσκαλος · « Πάλιν συγγένειαν\footnote{πάλιν συγγ.] πάλιν om. Lb. F. l. πολλὴν συγγ. Cp. 3, 29, 5.} ἔχει ἡ ὑδράργυρος πρὸς τὸν κασσίτερον. Ὁ φοιτητής φησιν · « Ὑδράργυρος ποιεῖ μίγμα κασσιτέρου. » Φησίν · « Τοῦτο λευκαίνει πᾶν σῶμα. Ὁ μόλυβδος πάλιν συγγένειαν ἔχει ὁ λίθος ὁ ἐτήσιος πρὸς\footnote{πάλιν] F. l. πολλὴν. --- Après ἔχει] πρὸς τὸν πυρίτην add. B etc. --- καὶ ὁ ἐτήσιος λίθος Lb.} τὸν μόλυβδον. » Ταῦτα μιμούμενος ὁ φιλόσοφος ἔλεγεν περὶ τῆς ἡμῶν\footnote{μιμούμενος] λογιζόμενος B etc.} τέχνης ὅτι ἡ φύσις τὴν φύσιν τέρπει.

11. Περὶ δὲ μαγνησίας ὁ λόγος · « Πάντα κατασπάσας εὑρήσεις σῶμα μέλαν ἢ μέλανα μόλυβδον, πολλάκις, καὶ σκωρίαν ἐπάνω πολλὴν,\footnote{μέλαν] μελανὸν M. F. l. μελανοῦν.} ἣν εἴ τις [ἐὰν] γεύσηται, εὑρήσει αὐτὴν δριμεῖαν ὥσπερ σφέκλην.\footnote{M mg. : καλῶς sur une ligne verticale, en lettres retournées.} Ταύτην ἀποκρούσαντες εὑρίσκουσιν ἔσω μέλανα μόλυβδον, τὸν ἐν αὐτῷ χαλκὸν, τὴν ἐν αὐτῷ μαγνησίαν · ταύτην καλοῦσιν μολυβδόχαλκον\footnote{τὴν] τὸν M.} καὶ σῶμα μαγνησίας · αὕτη περὶ ἧς μοι γέγραπται · αὕτη ἐστὶν περὶ ἧς πᾶσαι αἱ γραφαὶ κηρύττουσιν εἶναι ταύτην ἣν πλάζονται ζητοῦντες τοῦτον τὸν μολυβδόχαλκον, τοῦτο ὃ κηρύττουσιν αἱ τῶν προγόνων γραφαί. Ἡ τοῦ Ἀπόλλωνος ἔκδοσις, τουτέστιν τὸ σῶμα τῆς μαγνησίας · τοῦτό ἐστιν ὁ χαλκὸς, ὃν καὶ αὐτὸς Θεόφιλος\footnote{θεόφιλος ὢν ἔλεγεν E.} ἔλεγεν · ἕνα δέξαι χαλκὸν στέφανον. Καὶ ὁ Ἐρμῆς πάλιν ἔλεγεν\footnote{ἵνα δείξῃ χαλκοῦ στέφ. Lb. F. l. χαλκοῦν στέφ.} · « Τὸ σῶμα τῆς μαγνησίας ὃ ἐπεθύμησας μαθεῖν, εἰς τὴν οἰκονομίαν καὶ\footnote{ὃ] τὸ M.} τὸν σταθμὸν, εἴπομεν ὅτι κιννάβαριν λέγουσιν τὴν λεύκωσιν · λοιπὸν\footnote{λοιπὸν biffé E.} ἢ τὴν ξάνθωσιν. Τὰ γὰρ προλευκανθέντα, ἡ οἰκονομία αὕτη ἐστὶν ὡς\footnote{M mg. : προλε avec renvoi à ξάνθωσιν. --- τῶν γὰρ προλευκανθέντων ἡ οἰκ. B etc.} γέγραπται ἡμῖν.

\bigskip
\centerline{\EightStarTaper}
\centerline{\EightStarTaper\EightStarTaper}
\bigskip

\subsubsection[3. --- 29. ΠΕΡΙ ΤΟΥ ΛΙΘΟΥ ΤΗΣ ΦΙΛΟΣΟΦΙΑΣ.]{3. --- 29. ΠΕΡΙ ΤΟΥ ΛΙΘΟΥ ΤΗΣ ΦΙΛΟΣΟΦΙΑΣ.\footnote{Deux titres dans A ; second titre, en marge : λόγος τῆς σοφοτάτης (sic) μαρίας περὶ τοῦ λ. τ. φ. (seul titre de ELb.)}}
\paragraph{}
\emph{Transcrit sur} A, f. 136 v. (= A ou A\textsuperscript{1}). --- \emph{Collationné sur} K, f. 110 v. ;--- \emph{sur des fragments contenus dans} A, f. 9, 10, 11 (= A\textsuperscript{2}), \emph{à partir du} § 18 ;--- \emph{sur} E, f. 82 r. ;--- \emph{sur} Lb. p. 321. --- \emph{Chap.} 52 \emph{de la compilation du Chrétien dans} E Lb.

\bigskip

1. Η Μαρία φησίν · « Ἐὰν ὁ μόλυβδος ἡμῶν μέλας γένηται, ἰδοὺ γεγένηται · ὁ γὰρ μόλυβδος ὁ κοινὸς ἐξ ἀρχῆς μέλας ἐστίν · πῶς γὰρ γένηται ; ἐὰν μὴ τὰ σώματα ἀσωματώσῃς, καὶ τὰ ἀσώματα\footnote{F. l. γεγένηται. --- ἐὰν μὴ ... Cp. Olympiodore, 2, 4, 40.} σωματώσῃς καὶ ποιήσῃς τὰ δύο ἓν, οὐδὲν τὸ προσδοκώμενόν ἐστιν.\footnote{F. l. τῶν προσδοκωμένων. Cp. \emph{ibid.} (p. 93, l. 15).} Καὶ ἐὰν μὴ τὰ πάντα ἐν τῷ πυρὶ ἐκλεπτυνθῇ, καὶ ἡ αἰθάλη πνευματωθεῖσα\footnote{Cp. plus bas le § 11.} βασταχθῇ, οὐδὲν εἰς πέρας βασταχθήσεται. » Καὶ πάλιν\footnote{ἀχθήσεται, dans Olympiodore.} · « οὐχ ἀπλῶς λέγω, φησὶν, ἀλλὰ μολύβδῳ μέλανι τῷ ἡμῶν. Ἰδοὺ\footnote{Ἰδοὺ ... Cp. Ol. § 41.} γὰρ ὅλως σκευάζουσιν μέλανα μόλυβδον · ὡς γὰρ ἠπτημένον μετὰ\footnote{ὅλως ici et dans Ol. F. l. ὅπως. --- ἠπτημένον] ὠπτημένος E par correction, Lb.} κοινὸν μόλυβδόν ἐστιν · Ὁ γὰρ μόλυβδος μὲν ὁ κοινὸς (f. 137 r.) ἐξ ἀρχῆς μέλας ἐστὶν, ὁ δὲ ἡμέτερος γίνεται μέλας, μὴ ὄντος αὐτοῦ πρότερον. »

2. Ὅτι πάντα οἱ φιλόσοφοι τὰ ἔργα τοῦ λίθου εἰς δʹ διῄρουν · πρῶτον μελάνωσιν, δεύτερον λεύκωσιν, τρίτον ξάνθωσιν, καὶ τέταρτον ἴωσιν · μεταξὺ δὲ μελάνσεως καὶ λευκώσεως καὶ ξανθώσεως ἐστιν ἡ χοωποίησις, ἤτοι ἡ ταριχεία, καὶ τῶν εἰδῶν ἡ πλύσις. Ἀδύνατον δὲ ταῦτα γενέσθαι πλὴν διὰ τοῦ ὀργάνου τοῦ μασθωτοῦ\footnote{πλὴν διὰ τῆς τοῦ ὀ. μ. οἰκονομίας E, f. mel.} οἰκονομίᾳ, καὶ τῆς ἑνώσεως τῶν μορίων.

3. Πελάγιος ὁ φιλόσοφός φησιν · « Σημείωσις οὖν ἐστιν ἀρχομένης ἰώσεως, εἰ δὲ ἐντὸς γενομένη ἴωσις, αὕτη ἐστὶν ἡ ἀληθινὴ ἴωσις, ἥτις καὶ ἰὸς χρυσὸς ἑρμηνεύθη, ἐὰν μία τις ποιήσῃ, γίνεται, εἰ δὲ μὴ,\footnote{χρυσοῦ E mg. Lb. --- ἐρμηνεύεται Lb seul, mieux.} οὐ γίνεται. Σκόπει οὖν ἵνα ἐν τῷ βάθει γίνηται · εἰ δὲ μὴ, οὐ γίνεται. »

4. Ἀλάβαστρον τὸν πάνυ λευκότατον λίθον τὸν ἐγκέφαλον τὸν ὡς\footnote{τόν ὡς καὶ (add. Lb) ὄ. ἔχ. καὶ θ. ELb.} ὄζον ἔχοντα ὡς θέρμην. Τοῦτον λαβὼν, λείωσον καὶ ταρίχευσον ὄξει. Καὶ βαλὼν εἰς ὀθόνιον, καὶ μετὰ πάντων ἔγκρυψον εἰς κόπρον ἱππείαν ἢ ὀρνιθείαν ἄχρις εἴκοσιν ἡμερῶν, ὥς φησιν ὁ θεῖος Ζώσιμος.

5. Ὅτι τὰ θεῖα τὰ ὄντα δύο, ἕν ἐστι σύνθημα. Δύο τοίνυν ὄντων\footnote{ὄντων ὑδρ.] εἰσὶν αἱ ὑδράργυροι Lb.} ὑδραργύρων τὸ λευκὸν σύνθημα καί τὸ ὕδωρ τοῦ θείου, κατὰ τὸν Δημόκριτον · τὸ θεῖον θείῳ μιγὲν θείας ποιεῖ τὰς οὐσίας, πολλὴν\footnote{πολλὴν ... ] Cp. 3, 28, 10.} ἔχοντα πρὸς ἄλληλα τὴν συγγένειαν.

6. Συνέσιός φησιν ἐν μὲν τῆς χρυσοποιΐας < λόγῳ > · « Δημόκριτος\footnote{Συνέσιος ...] Cp. 2, 3, 10, 12 et 18. --- τῇ χρυσοποιίᾳ Lb seul.} εἶπεν · ὑδράργυρος ἡ ἀπὸ κινναβάρεως ... ἐν δὲ τῷ λευκῷ εἶπεν · ὑδράργυρον τὴν ἀπὸ σανδαράχης καὶ τὰ ἑξῆς. »

7. (f. 137 v.) Διόσκορος εἶπεν · « Καθὰ ὁ κηρὸς οἷον ἂν χρῶμα\footnote{οἵῳ ἂν χρώματι Lb seul.} προσομιλήσῃ αὐτῷ μεταβάλλεται · τὸ αὐτὸ καὶ ἡ ὑδράργυρος μεταβάλλεται.\footnote{αὐτῷ] εἰς αὐτὸ Lb seul. --- οὕτω καὶ τὸ αὐτὸ E. --- ἡ, puis le signe de l'argent AKE.} »

8. Ὅτι δύο ξανθώσεις εἰσὶν καὶ δύο λευκώσεις, καὶ δύο συνθέματα,\footnote{ὅτι ... ] Cp. Olympiodore, § 50.} ξανθὸν καὶ ὑγρὸν τουτέστιν ἐν τῷ καταλόγῳ τοῦ ξανθοῦ βοτάνας καὶ\footnote{ξανθὸν] Lire ξηρῶν comme dans Ol. --- Après ξανθοῦ] λέγει γὰρ βοτ. Lb.} μέταλλα, καὶ ζωμοὺς δύο, ἕνα ἐν τῷ ξανθῷ, καὶ ἕνα ἐν τῷ λευκῷ · καὶ ἐν μὲν τῷ ξανθῷ ζωμῷ, διὰ ξανθῶν βοτανῶν, οἷον κρόκου, καὶ ἐλυδρίου καὶ τῶν ὁμοίων · ἐν δὲ τῷ λευκῷ πάλιν συνθέματι,\footnote{ἐν δὲ τῷ λευκῷ gratté par le copiste de Lb et corrigé en ἄνευ τοῦ λευκοῦ, puis : ἐν δὲ τῷ λευκῷ.} ἐν μὲν τῷ ξηρῷ πάντα τὰ λευκὰ, οἷον γῆν κρητικὴν, κιμωλίαν, καὶ ὅσα τὰ τοιαῦτα · καὶ ἐν μὲν τῷ ὑγρῷ τοῦ λευκοῦ, ὅσα λευκὰ ὕδατα,\footnote{μὲν] δὲ Lb seul.} οἷον ζύθου < καὶ > χυλὸς καὶ τὰ ὅμοια.\footnote{ζύθος, χυλὸς Lb seul.}

9. Ὀλυμπιόδωρός φησιν · « Γίνεται ἡ ταριχεία ἀπὸ μηνὸς μεχὶρ\footnote{Cp. Ol., § 1.} κεʹ ἕως μετοπωρινῶν κεʹ · ὅσα ἂν δύνῃ ταριχεῦσαι καὶ πλύναι ἕως\footnote{μετοπωρινῶν] F. l. μεσωρὶ. Cp. p. 69, l. 15. --- Réd. de Lb seul (qui omet ὅσα ἂν δύνῃ) : ταρίχευε δὲ καὶ πλύνε, καὶ ἄφες αὐτὰ ἐν ἀγγείοις.} ἀφῇς αὐτὰ ἐν ἄγγεσιν ἀποκείμενα. Γίνεται δὲ ἡ ταριχεία περὶ τῆς\footnote{γίνεται --- καταλήξῃ] Cp. Ol., § 2.} πηλώδους γῆς,\footnote{ἀφεὶς AKE. --- περὶ] ἐπὶ Lb.} μέχρις ἂν τὸ πηλῶδες ἐξέλθῃ, καὶ εἰς ψάμμον καταλήξῃ. Ὅτι ἡ τέχνη αὕτη διὰ πυρὸς οὐ γίνεται.

10. Ὅτι μʹ ἡμερῶν ἐστι τὸ πῦρ τῆς ὅλης τέχνης. Ὅτι οἱ ἀρχαῖοι τὴν τέχνην ἐκάλυψαν τῇ πολυπληθείᾳ τοῦ λόγου, καὶ ὀνόμασι πολλοῖς ἐκάλεσαν τὸ ὕδωρ τὸ θεῖον.\footnote{τοῦ θείου Lb seul.}

11. Ὅτι Μαρία φησίν · « Ἐαν μὴ τὰ πάντα τῷ πυρὶ ἐκλεπτυνθῇ, καὶ ἡ αἰθάλη πνευματωθεῖσα βασταχθῇ, οὐδὲν εἰς πέρας βασταχθήσεται. » Ὅτι ὁ χαλκομόλυβδος ἐτήσιος λίθος ἐστίν. Ὅτι τῆς ὅλης\footnote{χαλκὸς μόλυβδος MBAK.} πραγ- (f. 138 r.) ματείας τὸ σκεύασμα ἐξ ἀρχῆς μέλας ἐστίν.\footnote{μέλαν Lb seul.} Ὅτι ὅταν τὰ πάντα ἴδῃς σποδὸν γινόμενα, τότε νόει ὅτι καλῶς ἐσκεύασας. Τοῦτο οὖν τὸ σκωρίδιον λείωσον καλῶς, καὶ ἐξυδάτωσον καὶ ἀπόπλυνον ἑξάκις καὶ ἑπτάκις ἐν γλυκέοις ὕδασιν καθ ᾽ ἑκάστην χωνείαν ποιῶν · πρὸς γὰρ τὴν δύναμιν τοῦ ψάμμου καὶ ἐν χωνείᾳ γίνονται.\footnote{οὕτω ποιῶν Lb. --- καὶ αἱ χωνείαι Lb.} Διὰ γὰρ ταύτης τῆς ἀγωγῆς, ἤγουν τῆς πλύσεως, φησὶν ἡ Μαρία,\footnote{πλύνσεως mss.} γλυκαίνεται τὸ σύνθημα · καὶ ἰδοὺ ἐπιστοιχειοῦται. Μετὰ γὰρ τὸ τέλος τῆς ἰώσεως, ἐπιβολῆς γίνομένης τῶν ὑγρῶν, γίνεται καὶ βεβαία ξάνθωσις. Τοῦτο δὲ ποιῶν, ἐκφέρει ἔξω τὴν φύσιν τὴν ἔνδον κεκρυμμένην.\footnote{ἐκφέρεις Lb.} Ἔκστρεψον γὰρ, φησὶν, αὐτὴν τὴν φύσιν, καὶ εὑρήσεις τὸ ζητούμενον.

12. Ὅτι τὰ συνθέματα δύο εἰσὶν, λεύκωσις καὶ ξάνθωσις · καὶ δύο μὲν λευκώσεις, καὶ δύο ξανθώσεις, ἤγουν μία διὰ λειώσεως, καὶ ἑτέρα δι ᾽ ἑψήσεως. Οὐ γὰρ ἁπλῶς συλλειοῦται, ἀλλ ᾽ ἐν τῷ δῶματι ἱερατικῷ · καὶ ἐκεῖσε γίνεται λίμνη καὶ κήτη.\footnote{κήτη] κοίτη Lb. Hœfer : « dépôt. »}

13. Ὅτι ἡ Μαρία φησίν · « Ζεύξατε ἄρρενα καὶ θήλειαν, καὶ εὑρήσετε\footnote{Ζεύξεται corrigé en ζεύξαται (sic) E. Cp. Ol., § 53. --- εὑρήσειται AKE.} τὸ ζητούμενον. » Καὶ ἀλλαχοῦ φησιν ἡ Μαρία · « Μὴ θέλετε\footnote{A mg. : une main, d'une encre plus pâle. --- Μὴ θέλετε ... ] Cp. Ol., § 54, et Zosime, 3, 21, 4.} ψηλαφεῖν χερσὶν, ὅτι ἐστὶν πύρινον φάρμακον. »

14. Ὅτι τὰ δύο συνθέματα καλοῦσιν πολλοῖς ὀνόμασιν, οἷον ὕδωρ\footnote{Tout le § 14 est emprunté, à partir de ὕδωρ, au morceau 3, 25, 1. Nous en supprimons le texte. Les variantes de ce § ont été rapportées au passage cité.} δι ᾽ ἅλμης κ. τ. λ.

15. Ὅτι τὰ σκεύη τῶν συνθεμάτων ὑάλινα χρὴ εἶναι, ἐπειδὴ συμπάσχει\footnote{Τὰ δὲ σκεύη Lb. Cp. 3, 21, 4. --- συμπάσχουσιν Lb seul.} [ἐν] τῇ ἰώσει, οὐ ψηλαφῶντες χερσί · θανατηφόρος γάρ ἐστιν\footnote{A mg. : Une main, d'une encre plus pâle.} ὅτε ὑδράργυρος καὶ < ὁ > ἐν αὐτῷ χρυσὸς σαπῇ · ὅτι πάντων τῶν\footnote{ὅτε] ἡ τε Lb. --- καὶ ὁ ἐν αὐτῷ χρυσὸς σαπείς πάντων γὰρ ... Lb.} μετάλλων δηλητηριωδέστερός ἐστι.
\begin{center}
\emph{Chapitre} 53 \emph{de la compilation du Chrétien dans} E Lb.
\end{center}
\paragraph{}
16. Ὅτι προκείμενόν ἐστιν ἐν τῇ καύσει πρῶτον λεύκωσις, δεύτερον\footnote{Titre en marge de E (f. 85 v.) et en vedette dans Lb (p. 337) : ΠΕΡΙ ΣΗΨΕΩΣ.} ξάνθωσις. Ἐπίβαλε, φησὶ, τοῦ λευκοῦ φαρμάκου τὸ ἥμισυ, καὶ ἔσται πρῶτον, καὶ οὕτως ἕψει · τὸ γὰρ ἄλλο ἥμισυ ἐν τῇ ἰώσει τηρούμενον.\footnote{τηροῦμεν Lb.} Διὰ τοῦτο καὶ Ἐπιβήχιός φησιν ἄνω καὶ κάτω · « Διαμερίσατε εἰς\footnote{Πηβήχιος Lb seul, par correction.} δύο μοίρας τὸ φάρμακον. » Ἔλεγεν καί · « Τὸ μὲν ἓν ἔχει ἐν ὀστρακίνῳ\footnote{ἔλεγεν --- λεύκωσιν (p. suiv. l. 3)] Réd. de Lb seul : καὶ τὴν μὲν μοῖραν ἔχε ἐν ὀστρακίνῳ ἀ., τὴν δὲ ἑτέραν ἐ. χ. καὶ τὸ μὲν ὄστρακον δηλοῖ τοῦ ὀστρακίνου (en marge : \emph{legο} λευκοῦ) τὴν ὄπτησιν, ἀπὸ δὲ τοῦ χαλκοῦ τὴν ἴωσιν · προεῖπε καὶ τὴν λεύκωσιν. (τὴν λεύκωσιν reporté, par un trait, après ὄπτησιν.)} ἀγγείῳ, τὸ δὲ ἕτερον εἰς χαλκοῦν · δηλοῖ < ἀπὸ > τοῦ ὀστρακίνου τὴν ὄπτησιν, ἀπὸ δὲ τοῦ χαλκοῦ τὴν ἴωσιν · προεῖπεν καὶ τὴν λεύκωσιν, ἤγουν · « Καύσατε τὸν χαλκὸν ἐν δαφνίνοις ξύλοις, » τουτέστιν ἐν τῷ λευκῷ συνθέματι.

17. Καὶ ὁ Ἀγαθοδαίμων φησίν · « Ἕψει τὸ (f. 139 r.) θεῖον ὕδωρ μετὰ τῆς νεφέλης · καὶ οὕτως ἐστὶν ἡ καῦσις καὶ ἡ λεύκωσις. » Καὶ πάλιν · « Τὴν προγεγραμμένην νεφέλην ἕψει ἐλαίῳ κικίνῳ, ἢ ῥεφανίνῳ, προσμίξας βραχὺ στυπτηρίας.

18. Καὶ ὁ Ζώσιμός φησιν · « ... Χρὴ γὰρ ἀκριβῶς ἐπὶ τῆς\footnote{ἐπὶ τῆς παρ. ἐργ.] Réd. de A\textsuperscript{2} : ἐπὶ τὸν τῆς παρ. ἐργ. ἀφικόμενος οὖν διὰ φιλοσοφίας δι ᾽ ὅλων ...} παρούσης ἐργασίας ἀμφιβαλόμενον δι ᾽ ὅλων τῶν τριακοσίων ἐξηκονταπέντε\footnote{ἀμφιβαλόμενος KE ; ἀναμφιβόλως Lb.} ἡμερῶν λούειν τὸν χαλκοῦν ἀετὸν, καὶ ἀνανεῶν, καὶ ἑξῆς\footnote{χαλκὸν AKE ; χάλκινον Lb. Corr. conj. --- ἀνανέον A\textsuperscript{2} ; ἀνανεύων K ; ἀνανεοῦν ELb. --- ὡς οὖν καὶ ἔξις (\emph{sic}) A\textsuperscript{2} ; καὶ ἑξῆς ἀπὸ τῆς πραγμ. E ; καὶ ἕξεις ὅλην πραγματείαν Lb.} δι ᾽ ὅλης αὐτοῦ τῆς πραγματείας.

19. Φησὶν ὁ θεῖος Σοφάρ · εἶδον κ. τ. λ.\footnote{Le texte du § 19 est emprunté au morceau 3, 4, 5, où l'on a reporté les variantes de ce §.}

20. Μαγνησία ἐτυμολογεῖται ἀπὸ τοῦ μιγνύειν τὰς κράσεις ἐνώσεις\footnote{Réd. de Lb : ... τὰς κράσεις · ἔστι γὰρ ἕνωσις καὶ συμπλοκὴ τῶν δύο. --- Après le contenu de notre § 19, A\textsuperscript{2} continue ainsi : χρὴ γὰρ π. τ. λ. (§ 18).} συμπλοκῇ τῶν δύο.\footnote{συμπλοκὴ mss.}

21. Ὅτι ὁ θεῖος Ζώσιμός φησιν · « Ἐπειδὴ ὁ Δημόκριτος ἐκεῖνος ὁ ἐμὸς ἀγαθῶς λέγει · » Δέξαι κ. τ. λ.\footnote{ἀγαθὸς Lb. --- Le § 21, depuis ἐπειδὴ est emprunté au morceau 3, 6, 6. On en a reporté les principales variantes au passage cité.}

22. Ὅτι ὁ Ζώσιμος ἔλεγεν · « Μὴ φοβηθῇς τὴν πολλὴν καῦσιν καὶ ἐξυδάτωσιν τῶν σωμάτων ... Ὅτι εἰσὶ μυρίαι καύσεις τοῦ χαλκοῦ, βαπτικωτέραι αὐτὸν ποιοῦσιν τὸν χαλκόν. Ἔκστρεψον τὴν φύσιν,\footnote{βαπτικώτερον Lb. --- Réd. de A\textsuperscript{2} (f. 11 r.) : καὶ τοῦτο φησίν · ἔκστρεψον, φησὶν, τ. φ. ; Réd. de ELb : ἔκστρ. δὲ αὐτοῦ τ. φ.} καὶ εὑρήσεις τὸ ζητούμενον · ἡ γὰρ φύσις ἔνδον κέκρυπται · ἐκστρεφομένης δὲ τῆς φύσεως, οὐκέτι λευκὸν ὁρᾶται, κατὰ τὴν προφανεῖσαν\footnote{λευκὸς Lb.} ἐξυδραργύρωσιν, ἀλλὰ ξανθὸν κατὰ τὴν ἐπηγγελμένην τοῦ ἰοῦ ξάνθωσιν.\footnote{ξανθὸς Lb.} Καὶ ποῦ ποτέ εἰσιν οἱ λέγοντες ἀδύνατον μεταβάλλεσθαι φύσιν ; Ἰδοὺ γὰρ μεταβάλλεται ἡ φύσις στερεὸν γενομένη κατὰ\footnote{ἡ φύσις τῶν στερεῶν A\textsuperscript{2} E. --- στερεὰ E (en surcharge) Lb.} τὴν ποίοτητα χρυσοῦ καὶ εἰς μέλαν κατασπασθήσεται. Ἐὰν γὰρ μὴ ὑγρότης τῆς ἐξυδραργυρώσεως περιελθοῦσα κατὰ τὴν γεώδη τοῦ\footnote{γεώδη φύσιν Lb lue.} στερεοῦ σώματος καὶ τὸ ξηρίον διαλύσεις καὶ ἐξυδατώσεις κατὰ τὴν\footnote{διαλύσῃ καὶ ἐξυδατώσῃ Lb.} οὐσίαν τῆς ἐξυδραγυρώσεως ποιότητα, εἰς οὐδὲν ἔσται τὸ προσδοκώμενον\footnote{οὐσίαν] οὐσιώδη A\textsuperscript{2}. --- οὐσίαν καὶ τὴν ποιότ. τῆς ἐξ. Lb.} · ἐὰν μὴ καὶ διαλυθείη καὶ ἐξυδατωθείη, καὶ θερμανθείη δὲ, εἰς οὐδὲν ἔσται τὸ προσδοκώμενον Ἐὰν δὲ καὶ μὴ διαλυθείη καὶ\footnote{ἐὰν δὲ καὶ μὴ --- προσδοκ. (l. suiv.) biffé dans E.} θερμανθείη, περιψυχθῇ δὲ, εἰς οὐδὲν ἔσται τὸ προσδοκώμενον Ἐὰν\footnote{ἐὰν μὴ δὲ E, f. mel.} δὲ πάντα τὰ κατὰ τὴν τάξιν ὁμοῦ κατακολούθως γένηται, ἐλπὶς καὶ ἑκβάσεως, σὺν τῇ θείᾳ προνοίᾳ, τυχεῖν [εἰς οὐδὲν ἔσται (f. 140 r.) τὸ προσδοκώμενον].

23. Βλέπε καλῶς τὸν μὲν τῆς κυοφορίας καιρὸν μὴ ἐλάττονα τῶν ἐννέα\footnote{βλέπε δὲ (om. E) καλῶς τὸν μὲν (om. Lb) τῆς κ. κ. ELb.} μηνῶν, ἐπεὶ ὡς ἔκτρωμα συμβήσεται, τὸν δὲ τῆς ὁπτήσεως κατὰ πάντα,\footnote{τὸ δὲ mss.} κατὰ τὰ πέταλα μὴ ἔλαττον ὡρῶν ἐννέα, ἡ τῆς κυοφορίας γὰρ τρόπος,\footnote{μηνῶν au-dessus de ὡρῶν E ; μηνῶν Lb. --- ἡ] ὁ Lb.} καὶ οὕτως ἐστίν · τὸν δὲ κατὰ τὴν ἄσκησιν τοῦ φιαλοβωμοῦ καιρὸν\footnote{τὸ δὲ AK.} συγκρίνῃ κατὰ τὴν ταριχείαν. Ἐπιθεωρῆσαι γὰρ ὅτι τρεῖς τρόποι\footnote{σύγκρινε Lb seul. --- F. l. ἐπιθεώρησαι.} τῆς ἐργασίας · εἰ μὲν ὅτι τῆς συγκράσεως πρῶτος τρόπος, καὶ κατανοήσῃς μου, ἔχει καταφυρώμενα καὶ ζυμούμενα ἐπιτεύχωος καὶ\footnote{ἐπὶ τεύχωος corrigé en τέφρας E. F. l. ἐπὶ στάχυος.} ἀλεύρου. Ὥσπερ γὰρ τὸ ὑγρὸν οὐ κατὰ τὰ μέτρα αἰθάλεται, ἀλλὰ καθόσον ἡ χρεία ἐπιζητεῖ, οὕτω καὶ ἐπὶ τοῦ συνθέματος ὀπὴν ἔχει\footnote{ἐπὶ τοῦ συνθέματος] Le texte compris depuis ces mots jusqu'à la fin du § est emprunté au morceau 3, 7, 5. On en a reporté les principales variantes au passage cité.} τὸ ὀστράκινον ἄγγος, κ. τ. λ.

24. Ὅτι αὐτός ἐστιν ὁ ἐτήσιος λίθος. Γλυκάνῃς οὖν τὸ ξηρίον,\footnote{αἰτήσιος Lb.} καὶ ξήρανον, στῆσον καὶ ἐξίωσον τὸ ξηρίον τοῦ χαλκάνθου μέρη γʹ, μαγνησίας μέρος ἓν, χαλκοῦ μέρος ἕν. Ἐξίωσον τὸ ξηρίον μέρος ἕν · λείωσον ὁμοῦ ποτίζων ἐν ἡλίῳ ἀπὸ τοῦ ὄξους τοῦ λευκοῦ ἡμέρας ἑπτὰ, καὶ ὕστερον ὀπτᾶσθαι ἡμέρας δύο ἢ τρεῖς, καὶ ἐξενεγκὼν εὑρήσεις\footnote{λεῖπε ὀπτᾶσθαι Lb.} βαφέντα τὸν χρυσὸν πυρρὸν ὡς τὸ αἷμα. Αὕτη ἐστὶν ἡ κιννάβαρις\footnote{πυρὸν mss.} τῶν φιλοσόφων, καὶ ὁ χαλκάνθρωπος χρυσός · ἀλλὰ καὶ αὐτὸ\footnote{χαλκάνθρ. ὁ χρυσοῦς Lb.} ξηρίον ποτιζόμενον ἀπεστύφη ἐν τοῖς ζωμοῖς · ἐὰν γὰρ πλεονάσῃ τὰ\footnote{ποτιζ. καὶ ἀποστυφόμενον Lb seul. --- γὰρ biffé dans E ; om. Lb. F. l. δὲ.} φῶτα, γίνεται ξανθὸν, ἀλλ ᾽ οὐ χρησιμεύει.\footnote{Après χρησιμεύει] Τέλος τοῦ Χριστιανοῦ Lb seul.}

\bigskip
\centerline{\EightStarTaper}
\centerline{\EightStarTaper\EightStarTaper}
\bigskip

\subsubsection{3. --- 30. ΠΕΡΙ ΑΦΟΡΜΩΝ ΣΥΝΘΕΣΕΩΣ.}
\paragraph{}
\emph{Transcrit sur} M, f. 161 v. (\emph{ms. unique}).

\bigskip

Ἡ περὶ ἀφορμῶν σύνθεσις, ὦ Θεοσέβεια, τὰς κατὰ μέ- (f. 162 r.) ρος\footnote{ὡς θεοσέβειαν M. Corr. conj.} τῶν ἀρχαίων συνθέσεις εἰς ἕνα νοῦν συνῆξεν · ἔτι γε μὴν καὶ ὀνόματα σύνθετα ἐν ταῖς αὐτῶν συντάξεσιν ἀγνοούμενα διὰ τοῦ πράγματος δηλοῖ, ὡς τὴν σποδὸν καὶ τὰ ὁμοιότροπα. Εἰδέναι δὲ δεῖ τίνα κατὰ τοῦ φιλοσόφου ποιεῖ τὸ < μὲν > πυρίμαχον, τὸ δὲ προσπλακὲν\footnote{M mg. : ·)(· puis le signe du mercure (lire Ἑρμοῦ ? ).} ποιεῖ πυρίμαχον, καὶ τὰ ἑξῆς. Ὁ γὰρ σοφὸς, ἀφορμὰς λαβὼν, πάντως ἀπὸ τῆς ἀρχῆς ἐπὶ τὸ πέρας ἀφίξεται. Ἔνθεν ἐγὼ τὰ τέλεια ἐνθεῖναι οὐκ ἠδυνήθην, ἐπείπερ οὐδὲ παρ ᾽ αὐτοῖς εὗρον · οὔτε μὴν\footnote{F. l. ἐκθεῖναι.} τοῦτο προσεθέμην ὅπερ οὐδὲ τοσοῦτος, εἰ μὴ μόνον καθὼς δυνατὸν ὡς εἰκότα τὰ σκορπισθέντα συνάξαι, καὶ τὰ ἀλληγορικὰ εἶναι ἑρμηνεῦσαι · καὶ ὅσα ἐγχωρεῖ ὑπομνήμασιν γενέσθαι ἐπόιησα. Ἔρρωσο.

\bigskip
\centerline{\EightStarTaper}
\centerline{\EightStarTaper\EightStarTaper}
\bigskip

\subsubsection{3. --- 31. ΠΕΡΙ ΞΗΡΙΟΥ.}
\paragraph{}
\emph{Transcrit sur} M, f. 136 v. --- \emph{Collationné sur} A, f. 110 r. ;--- \emph{sur} E, f. 37 r. ;--- \emph{sur} Lb, p. 129. --- \emph{Chap.} 28 \emph{dans} E, 29 \emph{dans} Lb, \emph{de la compilation du Chrétien}.

\bigskip

Τρεῖς δυνάμεις εἰσὶ τοῦ ἀληθεστάτου ξηρίου, καὶ τρεῖς ἐνέργειαι ἐκ τούτων προιοῦσαι τῶν δυνάμεων · βαφὴ, εἴσκρισις, κάτοχον. Καὶ τὸ\footnote{κάτοχος M. Réd. de E : καὶ τὸ μαθ. δὲ καὶ φυσικὸν τρ. διαστ. ἔχει ... --- Réd. de Lb : καὶ τὸ μαθ. δὲ κ. φυσ. σῶμα τὸ τριχῆ διάστατον κ. ἀντίτ. τρεῖς διαστάσεις ἔχει γ. πλ. κ. βάθος. Puis, d'après E corrigé : διὸ καὶ τούτου τοῦ εἴδους λέγομεν βαφὴν εἴσκρ. κάτοχον ...} μαθηματικὸν τρεῖς διαστάσεις ἔχει, μῆκος, πλάτος καὶ βάθος. Καὶ τὸ φυσικὸν σῶμα τὸ τριχῇ διάστατον καὶ ἀντίτυπον, ὃ μῆκος ἔχει,\footnote{ὁ M. --- ὃ μῆκος --- ἀντιτυπίαν om. AE Lb.} καὶ πλάτος, βάθος τε καὶ ἀντιτυπίαν · οὕτω καὶ εἴδους ἐροῦμεν βαφὴν\footnote{εἴδους] τὸ εἶδος AE avant les corrections.} εἴσκρισιν, κάτοχον καὶ στίλψιν. Καὶ τὸ τριχῇ διάστατον προσαγορεύομεν ἴδεον καὶ ἀνίδεον καὶ πανίδεον ὕλην τὴν ὑποδεχομένην τὰς\footnote{εἴδεον καὶ (om. E) ἀνείδεον κ. πανείδεον AELb, mel. --- ἡ ὑποδεχομένη M.} δυνάμεις καὶ ἐνεργείας.

\bigskip
\centerline{\EightStarTaper}
\centerline{\EightStarTaper\EightStarTaper}
\bigskip

\subsubsection{3. --- 32. ΠΕΡΙ ΙΟΥ.}
\paragraph{}
\emph{Suite du texte précédent}. --- \emph{Chap.} 29 \emph{dans} E, 30 \emph{dans} Lb, \emph{de la compilation du Chrétien}.

\bigskip

Ἡ μὲν γὰρ ἰώδης δύναμις συμπληρωτική ἐστιν τῆς ὅλης ὑποκειμένης οὐσίας ἀτόμου, καὶ μέρος αὐτῆς, καὶ ἄνευ ταύτης ἀτελὴς ἡ ὅλη\footnote{διὸ καὶ μέρος αὐτῆς καλεῖται Lb.} οὐσία καθέστηκεν. Τὰ γὰρ μέρη τῶν οὐσιῶν οὐσίαι εἰσὶν, ὥς φησιν Πορφύριος, ἡ γὰρ οὐσία προβάλλεται δύναμιν, καὶ ἡ δύναμις ἐνέργειαν,\footnote{ἡ γὰρ οὐσία] Cp. Damascius, περὶ ἀρχῶν, p. 183. éd. Kopp.} καὶ ἡ ἐνέργεια τὰ ἐνεργήματα. Αἱ τοίνυν δυνάμεις αἱ οὐσιώδεις ἑκουσίως προέρχονται, καὶ ἀχώριστοί εἰσι τῶν οὐσιῶν.\footnote{Après τῶν οὐσιῶν, AE Lb continuent, sans division et sans titre, avec le morceau suivant.}

\bigskip
\centerline{\EightStarTaper}
\centerline{\EightStarTaper\EightStarTaper}
\bigskip

\subsubsection[3. --- 33. < ΠΕΡΙ ΑΙΤΙΩΝ. >]{3. --- 33. < ΠΕΡΙ ΑΙΤΙΩΝ.\footnote{Titre ajouté dans M : περὶ ἐτίον (main du 15\textsuperscript{e} siècle ? ).} >}
\paragraph{}
\emph{Suite du texte précédent.}

\bigskip

Τέσσαρα γάρ εἰσιν αἴτια κατὰ τὸν φυσικὸν Ἀριστοτέλην παντὸς γενητοῦ · ποιητικὸν, ὑλικὸν, (f. 137 r.) ὀργανικὸν καὶ εἰδικὸν,\footnote{Sur ποιητικόν, Cp. Aristote, Génération et Corruption, 1, 7 ; sur ὑλικόν, Métaphys. 1, 3 ; sur ὀργανικόν, Morale à Eudème, 7, 10 ; sur εἰδικόν, Physique, 2, 3.} οἷον ἡ θύρα ποιητικὸν αἴτιον ἔχει τὸν τέκτονα τὸν ποιήσαντα, ὑλικὸν,\footnote{κόλλα M.} ξύλον, σίδηρον, κόλλαν, ὀργανικὸν σκέπαρνον, τέρετρον, καὶ τὰ λοιπὰ, εἰδικὸν, αὐτὸ τὸ ἔνυλον εἶδος θύρας, ἢ ἄλλο τι. Κατὰ δὲ Πλάτωνα\footnote{Πλάτωνα] Cp. Platon. Timée, p. 37 D ( ? ).} καὶ ἕτερα δύο εἰσίν, παραδειγματικὸν καὶ ἀποτελεσματικόν.

\bigskip
\centerline{\EightStarTaper}
\centerline{\EightStarTaper\EightStarTaper}
\bigskip

\subsubsection{3. --- 34. Enchainement de la Vierge.}
\paragraph{}
\emph{Suite du texte précédent (sans titre).}

\bigskip

1. Ὑδραργύρου πῦρ πυρὶ κρατοῦντες, καὶ πνεῦμα πνεύματι συνάψαντες,\footnote{E mg. : \emph{Corrige} ἡμεῖς δὲ διὰ τοῦ ἰοῦ ὐδραργύρου πῦρ ... et au-dessous : 1 A. ιος αργυρου (\emph{sic}). Réd. adoptée par Lb. --- κροτοῦντες M.} ἵνα δεσμεύσωμεν τὴν φυγαδοδαίμονα κόρην διὰ χειρῶν.\footnote{ἵνα δὲ τμήσωμεν M.} Διαφόρων ὀστέων Περσῶν κατακαυθέντων διὰ τῆς τοῦ πυρὸς βίας,\footnote{διαφέρων ὁστᾶ περσ. κατακαυθέντα M ; (διαφ. γὰρ ὀστέων περσῶν --- πν.) \emph{sic} E.} ἀπώλεσεν τὴν ἰδίαν πνευμάτωσιν.\footnote{ἀπώλεσαν M.}

2. Καὶ αὖθις ἀναγάγωμεν τὰ δύο σώματα καὶ συνερχομένων τῇ μίξει καὶ μεταμορφουμένων, εἰς παλιγγενεσίαν τρέπονται · ὁ ἄψυχος\footnote{ἀγαγὼν Lb. --- καὶ συν.] συν. γὰρ AE Lb.} ψυχοῦται, καὶ ὁ ἀσώματος σωματοῦται, καὶ ἕτερόν τι οὐ δέχονται.\footnote{καταμεταμορφ. Lb. --- τρέπεται AE Lb.}

\bigskip
\centerline{\EightStarTaper}
\centerline{\EightStarTaper\EightStarTaper}
\bigskip

\subsubsection{3. --- 35. Les Hommes Métalliques.}
\paragraph{}
\emph{Suite du texte précédent (sans titre).}

\bigskip

Οὖτος ὁ χαλκάνθρωπος ὃν ὁρᾷς ἐν τῇ πηγῇ μετεβλήθη τοῦ σώματος,\footnote{οὗτος ὁ χ.] Rapprocher ce texte du morceau 3, 1, 5 ; Τὸν γὰρ ἱερέα τὸν χαλκάνθρωπον ...} καὶ γέγονεν ἀσημάνθρωπος. Μετ ᾽ ὀλίγας οὖν ἡμέρας βλέπεις αὐτὸν καὶ χρυσάνθρωπον · πότιζε δὲ αὐτὸν μετὰ ὀξάλμης · οὕτω γὰρ γίνεται λευκὸν καὶ ἁρμόδιον.

\bigskip
\centerline{\EightStarTaper}
\centerline{\EightStarTaper\EightStarTaper}
\bigskip

\subsubsection{3. --- 36. ΚΑΔΜΙΑΣ ΠΛΥΣΙΣ.}
\paragraph{}
\emph{Transcrit sur} M, f. 137 r. --- \emph{Collationné sur} A, f. 110 v. ;--- \emph{sur} E, f. 38 v. ;--- \emph{sur} Lb, page 133. --- \emph{Chap.} 30 \emph{dans} E, 31 \emph{dans} Lb, \emph{de la compilation du Chrétien}.

\bigskip

Λαβὼν καδμίαν τὴν ἐν τῷ χαλκῷ βλισκομένην βοτρυΐτην,\footnote{βλισκομένην] (F. l. βλισσομένην) οἰκονομουμένην Lb.} κόψον · σεῖσας, λείωσον ἐπιμελῶς · εἶτα βαλὼν, τρίψον καὶ εἰς ὕδωρ\footnote{σεῖσαν M.} βάλε · καὶ ἐν τῷ ὕδατι πάλιν τρίψον τῷ δοίδυκι · εἶτα λείωσον τῇ χειρί · καὶ ὅταν εὖ ἔχῃ, ἔασον ἀπο- (f. 137 v.) καταστῆναι.\footnote{ἀποκαθεσθῆναι M.} Καὶ ἀποσειρώσας, πάλιν βάλε ὕδωρ, καὶ τὸ αὐτὸ ποίει πολλάκις, ἕως ὕδωρ\footnote{ἀποσυρώσας A E Lb, ici et plus loin. --- ἕως ἂν εἰς ὕδωρ μὲνῃ Lb.} μείνῃ καὶ ἀπομφολύγωτον · καὶ ἀποσειρώσας, ξήρανον ἐν ἡλίῳ.\footnote{ἀποπομφυλογώσας ὀλίγον Lb.}

\bigskip
\centerline{\EightStarTaper}
\centerline{\EightStarTaper\EightStarTaper}
\bigskip

\subsubsection[3. --- 37. ΠΕΡΙ ΒΑΦΗΣ.]{3. --- 37. ΠΕΡΙ ΒΑΦΗΣ.\footnote{Titre omis AE Lb. E Lb l'insèrent dans le texte après ἐργάσηται.}}
\paragraph{}
\emph{Transcrit sur} M, f. 137 v. --- \emph{Collationné sur} A, f. 111 r. ;--- \emph{sur} E, f. 39 r. ;--- \emph{sur} Lb, p. 133. --- \emph{Suite du chap.} 30 (E), 31 (Lb) \emph{dans la compilation du Chrétien}. (\emph{Cet article compte néanmoins comme chap.} 31 \emph{dans} E.)

\bigskip

Ἐὰν μὴ ἐπιεικῶς ἐργάσητε μέλαιναν βαφὴν, ἐκφέρει ἄφευκτον τὴν\footnote{ἐργάσηται mss. --- Réd. de Lb : ἐργ. περὶ βαφῆς, κ. μελ. β. ἐκφέρῃ ἄκρ. τὴν ἐργ. αὐτοῦ. --- μελαίνην M. --- ἐκφέρῃ mss.} ἐργασίαν τοῦ ἀργύρου. Οἱ Ἀγαθοδαιμονῖται καλοῦσιν < καταβαφὴν > τὴν οὕτω λειουμένην · τὴν δὲ ἕψησιν ἐκάλουν βαφὴν. Ἄλλο γὰρ θέλουσιν εἶναι βαφὴν, καὶ ἄλλο καταβαφήν. Βαφὴν οὖν λέγουσι\footnote{λέγειν M.} τὸν ἄργυρον, καταβαφὴν δὲ τὸν χρυσόν. Καὶ ἐπὶ τῆς καύσεως τοῦτο\footnote{ἄργυρον] ἄσημον A E Lb ; E mg. : signe de l'argent. --- καύσεως] οὐσίας A E Lb. E mg. aj. δὲ τῆς καύσεως.} εὑρήσεις · ἄλλην καῦσιν βαφικὴν, καὶ ἄλλην καταβαφικὴν, καὶ τὰ ἄλλα πάντα ἕως ἀραιώσεως καὶ παρατροπῆς, καὶ τῶν ἄλλων πάντων τῷ\footnote{ἀρεώσεως M ; ἀρεόσεως A.} λόγῳ διυποπτεύουσι.

\bigskip
\centerline{\EightStarTaper}
\centerline{\EightStarTaper\EightStarTaper}
\bigskip

\subsubsection{3. --- 38. ΠΕΡΙ ΞΑΝΘΩΣΕΩΣ.}
\paragraph{}
\emph{Transcrit sur} M, f. 137 v. --- \emph{Collationné sur} A, f. 111 r. ;--- \emph{sur} E, f. 39 r. ;--- \emph{sur} Lb, p. 137. --- \emph{Chap.} 32 \emph{de la compilation du Chrétien dans} E Lb.

\bigskip

« Οὐ πᾶσιν ἔδοξεν, ὦ γῦναι, ἀπὸ τῆς λευκώσεως αὐτίκα συνάπτειν τὴν ξάνθωσιν. Ἑψόμενον γὰρ τὸ λευκὸν σύνθεμα ἐπιπολὺ ἐπὶ τὸ ξανθὸν τρέπεται. » Καὶ μετ ᾽ ὀλίγον · « Ἄλλοι τι περιττόν τι\footnote{καὶ μετ ᾽ ὀλίγον om. A E Lb, qui lisent ensuite καὶ τινὲς τὶ περιττὸν τούτων ἐποίησαν.} τούτων ἐποίησαν. Ἐάσαντες γὰρ ἕως ψυγῇ, κατήνεγκαν καὶ ἐλείωσαν ἐν ἡλίῳ ὕδωρ θεῖον ξανθὸν, ἃς ἐδιδάχθησαν ἡμέρας, καὶ μετὰ τοῦτο\footnote{ἡλίῳ] signe du soleil et de l'or MAE ; χρυσῷ Lb. --- ἃς] ἐφ ᾽ ἃς AE Lb.} ἕψησαν καὶ ὤπτησαν. » Καὶ μετ ᾽ ὀλίγον · « Τὸ δὲ ἀπολελυμένον ὕδωρ\footnote{ἀπολύμενον M ; ἀπολυμένον A. Corr. conj.} θεῖον, τὸ δι ᾽ ἀσβέστου μέρη δύο, καὶ θείου μέρος ἓν, τὸ ἐν χύ- (f. 138 r.) τρᾳ\footnote{ὕ. θείου Lb. --- δι ᾽ ἀσβ. ἐποίησαν Lb. --- μερῶν M.} ἑψημένον καὶ ἀποσειρούμενον · καὶ πάλιν ἑψούμενον,\footnote{ἀποσυρούμενον Lb.} τουτέστι τὸ ὕδωρ τὸ θεῖον, τὸ εἰς ἄμφω χρώματα βαλλόμενον.\footnote{Réd. de Lb : τὸ ὕδ. τοῦ θείου, τὸ εἰς θεῖον ὕδωρ χρώματι βαλλόμενον, τὸ ἀέριον ὕδωρ φημί. --- Après βαλλόμενον, M continue avec le fragment d'Agatharchide (voir la notice du ms. M.)} »

\bigskip
\centerline{\EightStarTaper}
\centerline{\EightStarTaper\EightStarTaper}
\bigskip

\subsubsection[3. --- 39. ΤΟ ΑΕΡΙΟΝ ΥΔΩΡ.]{3. --- 39. ΤΟ ΑΕΡΙΟΝ ΥΔΩΡ.\footnote{Voir Olympiodore, 2, 4, 33, 34 et 35.}}
\paragraph{}
\emph{Transcrit sur} A. f. 111 r. --- \emph{Collationné sur} E, f. 39 v. ;--- \emph{sur} Lb, p. 137. --- E \emph{et d'après lui} Lb \emph{continuent le texte précédent sans séparation}.

\bigskip

1. Πρώτων ὑγρῶν τινος δεῖται τὸ τοιοῦτον σύνθεμα, ἵνα, φησὶν,\footnote{Réd. de Lb : Πρὸ δὲ τῶν ὐγρῶν τίνος δεῖται ... Πρῶτον ὑγροῦ τινος δεῖται Ol.} ἡ ὕλη φθαρεῖσα ἀμετάτρε- (f. 111 v.) πτον τὸ εἶδος φυλάξῃ, καὶ ἐκ\footnote{καὶ ἐκ τούτου τοῦ φθ. Lb. (Cp. Ol. § 35).} τούτου φθαρεῖσα ἐσήμανεν ἐπὶ χρόνου τινὸς, διὰ τὸ « εἰς τοῦτο σήπεται. » Σῆψις γὰρ οὐ γίνεταί ποτε, εἰ μὴ δι ᾽ ὑγροῦ τινος. Ὁ γὰρ κατάλογος τῶν ὑγρῶν, φησὶν, ἐπιστεύθη τὸ μυστήριον.

2. Περὶ δὲ τῶν ψάμμων · ὅτι περὶ αὐτῶν πάντες φροντίζουσιν λόγον,\footnote{φροντιζ., λόγον ἄξωμεν Lb.} ἄρξομαι πάλιν τῆς ἐξ αὐτῶν μαρτυρίας, χάριν τῆς σῆς δυσπιστίας.

3. Ζώσιμος τοίνυν, ἐν τῇ τελευταίᾳ ἀποχῇ πρὸς Θεοσέβειαν ποιούμενος τὸν λόγον, φησίν · « Ὅλον τῷ τῆς Αἰγύπτου βασιλεῖ, ὦ γύναι,\footnote{Ὅλον ... Début du livre intitulé περὶ τελευταίας ἀποχῆς (3, 51) dont les variantes sont désignées ici par un astérisque. F. l. ὃλον τὸ τῆς Αἰν. τὸ βασίλειον comme dans et dans Ol. § 35.} ἀπὸ τῶν δύο τεχνῶν τούτων καθέστηκεν, τῶν τε μερικῶν καὶ τῶν\footnote{μερικῶν] κερικῶν, \emph{alias} κυρικῶν $\star$. F. l. καιρικῶν.} φυσικῶν καὶ ψάμμων. Ἡ γὰρ ἀλλοιουμένη θεία τέχνη, τουτέστιν ἡ\footnote{ἀλλοιουμένη] καλουμένη.} δογματικὴ περὶ ἧς ἀσχολοῦνται ἅπαντες οἱ ζητοῦντες τὰ χειροτμήματα\footnote{περὶ ἣν ἀσχολ. $\star$. La suite se sépare de la τελευταία ἀποχή pour se rapprocher de la citation faite par Ol.} ἅπαντα καὶ τὰς τέχνας, τὰς τέσσαράς φημι < αἳ > δοκοῦσι τοῦ\footnote{τὰς τιμίας τέχνας Ol. --- τι ποιεῖν Ol. ; δοκεῖ τὸ πᾶν ποιεῖν Lb.} ποιεῖν, μόνοις ἐξεδώθη τοῖς ἱερεῦσιν. Ἡ γὰρ φυσικὴ ψαμμουργικὴ βασιλέων ἦν, ὥστε καὶ ἐὰν συμβῇ ἱερεῖ σοφῷ λεγόμενον, ἑρμηνεύσαντα\footnote{ἱερέα ἢ σοφὸν λεγ. Ol.} τοῖς ἐκ τῶν παλαιῶν, ἢ ἀπὸ προγόνων, ἐκληρονόμησαν καὶ ἔσχον. Καὶ ἰδὼν ταύτης τὴν ἀκολουθίαν τὸ συνετὸν οὐκ ἐποίει · ἐτιμωρεῖτο γὰρ, ὥσπερ οἱ τεχνῖται οἱ ἐπιστάμενοι βασιλικὸν τύπτειν νόμισμα οὐχ ἑαυτοῖς τύπτουσι, ἐτιμωροῦντο οὗτοι. »

3. Τοῦτό ἐστι τὸ παρὰ τῶν ἀρχαίων γραφῶν φημιζόμενον κοσμικὸν μήνυμα, ἡ μυστικὴ ἡ τῶν Αἰγυπτίων καὶ ἱερογραμματέων Αἰγύπτου,\footnote{μήνυμα] μίμημα Lb.} θυεία, ἀνθ ᾽ ἧς ἡ τῶν φύσεων συγγένεια τέρπει τὰς ὁμοουσίους φύσεις.\footnote{Après θυεία] addition de Lb : καὶ αὖθις ἐπὶ τὸ προκείμενον ὁ λόγος · ἡ τ. φύσεων ...} Τοῦτό ἐστι τὸ ὀρφα- (f. 112 r.) ϊκὸν ὁμοούσιον, καὶ ἡ ἑρμαϊκὴ λύρα,\footnote{ὀρφεικὸν Lb. F. l. ὀρφικὸν.} ἐν ᾗ τῶν οὐσιῶν ποθεινή τε καὶ ἐναρμόνιος ἀποτελεῖται συμπλοκή. Μιγνύμεναι γὰρ καὶ ὡς προσῆκεν ἀπὸ τῆς < γῆς > ἐπὶ τὸν οὐράνιον χορὸν, καὶ ἀμείβοντος αὐτὰς πυρὸς ἀνατρέχουσιν.\footnote{χῶρον Lb.}

4. Κἀντεῦθεν μεταξὺ μελάνσεως καὶ λευκώσεώς ἐστιν ἡ ταριχεία καὶ τῶν εἰδῶν ἡ πλύσις μεταξὺ δὲ λευκώσεως καὶ ξανθώσεώς\footnote{πλύνσις mss. ici et presque partout.} ἐστιν ἡ χοωποίησις · καὶ οὕτω ξανθώσεως καὶ ἰώσεως, μέσος δέ ἐστιν\footnote{Réd. de Lb : καὶ οὗτός ἐστιν ὁ τρόπος τῆς ξ. καὶ τῆς ἰ.} ὀ τοῦ συνθέματος διχασμός. Τῆς δὲ λευκώσεως πέρας ἡ διὰ τοῦ ὀργάνου μασθωτοῦ οἰκονομία.

5. Μελάνωσις α\textsuperscript{η} τοῦ χωρισθῆναι τὸ ὑγρὸν ἐκ τοῦ σποδίου · ταριχεία β\textsuperscript{α} τοῦ σποδίου ὑγροῦ · πλύσις εἰδῶν τρίτη, ἑπτάκις καέντων\footnote{Et suiv. ταριχεία δὲ ... τρίτη δὲ et ainsi de suite Lb.} ἐν τῇ ἀσκαλωνίτιδι γάστρᾳ, ἥτις ἐστὶν α\textsuperscript{η} λεύκωσις καὶ ἀπομελάνωσις τῶν εἰδῶν. Λεύκωσις δ\textsuperscript{η}, ἥτις μιχθεῖσα λευκοῖς ὀλίγοις ὕδασιν, ἢ ξανθοῖς, ποιεῖ κηρίον πρὸς τὸ ζητούμενον χειροποητοῖς.\footnote{κυρίον A. --- Réd. de Lb : ποιεῖ κηρίον, καὶ πρὸς τὸ ξηρίον τὸ ζητ., χειροποιεῖ.} Ε\textsuperscript{η} ἐπὶ ξάνθωσιν ἡ λεύκωσις φέρουσα, ἡ ξάνθωσις. ΣΤ\textsuperscript{η} ὡς πρόκειται ὁ διχασμὸς τοῦ συνθέματος. Ζ\textsuperscript{η} ἥτις μερισθεῖσα εἰς δύο, καὶ τὸ μὲν ἓν μέρος διχαζόμενον καὶ ἰούμενον, μαλάττει, λειοῖ καὶ πηγνύει.\footnote{Réd. de Lb : μαλάσσει, καὶ λύει · τὸ δὲ ἕτερον μέρος πηγνύει.}
\begin{center}
\emph{Addition marginale du ms.} A \emph{seul} :
\end{center}
\paragraph{}
6. Ἄλλοι δὲ, φησὶν, περὶ γρώμ < ατος > καὶ ἑψήσεως καὶ ἔργου μυστικῆς θεωρίας. Ἀρχ < ὴ > μὲν ὁ χαλκὸς ἐμβαλλόμενος μετὰ τῆς οἰκονομίας ἐν τῷ ἐργαλείῳ τῆς πρ < ά > ξεως ἐπιδείκνυται ὁμμάτων τέρψιν, ἐν δὲ τῷ χρονίζειν γιν < ο > μένης ἀπαμαυρώσε < ως > μετὰ τοῦ κόμμεως χρυσ < ὸν > σύνθετον, χρυσὸν ζώμιον καὶ τὰ ἐξῆς.\footnote{F. l. χρυσοζώμιον. (Cp. 3, 16, 6.)}

\bigskip
\centerline{\EightStarTaper}
\centerline{\EightStarTaper\EightStarTaper}
\bigskip

\subsubsection{3. --- 40. ΠΕΡΙ ΛΕΥΚΩΣΕΩΣ.}
\paragraph{}
\emph{Transcrit sur} M, f. 118 r. --- \emph{Collationné sur} B, f. 90 v. ;--- \emph{sur} A. f. 14 v. (= A) ;--- \emph{sur} A, f. 92 (= A\textsuperscript{2}) (\emph{mêmes leçons}) ;--- \emph{sur} A, f. 250 v. (= A\textsuperscript{3}) ;--- \emph{sur} K, f. 5 v. ;--- \emph{sur} Lc, p. 217.

\bigskip

1. Γιγνώσκειν ὑμᾶς θέλω ὅτι πάντων ἐστὶν κεφάλαιον ἡ λεύκωσις\footnote{Ce 1\textsuperscript{er} § forme le début de 3, 54. --- Δεῖ γιν. AK Lc. --- Réd. de A\textsuperscript{3} : περὶ λευκώσεως χρὴ γιν. ἡμᾶς. --- M mg. sur une ligne verticale : τέλιον μυ φησὶν.} · μετὰ δὲ τὴν λεύκωσιν, εὐθὺς ξανθοῦται τὸ τέλειον μυστήριον.

2. Ἡ λεύκωσις καῦσίς ἐστιν · ἡ δὲ καῦσις, ἀναζωπύρωσις · αὐτὰ\footnote{Cp. les §§ 2 et 3 avec Synésius (2, 3, 4).} γὰρ ἑαυτὰ καίουσι καὶ ἀναζωπυροῦσι, καὶ αὐτὰ ἑαυτὰ ὀχεύει,\footnote{Réd. de A\textsuperscript{3} : ὀχεύουσι καὶ ἀναζωοπυροῦσι καὶ ἐγγυοποιεῖ.} καὶ ἐγκυοποιεῖ καὶ ἀποτίκτει τὸ ζητούμενον ζῶον κατὰ τοὺς φιλοσόφους.

3. Ἐὰν λευκώσῃς, εὐκόλως βάψεις · εἰ δὲ καὶ ἰώσεις ἢ κινναβαρίσεις, μακάριος ἔσῃ, ὦ Διόσκορε · τοῦτο γάρ ἐστιν τὸ λυτρούμενον\footnote{διόσκωρε M.} πενίας, τῆς ἀνιάτου νόσου.\footnote{ἐκ πενίας A.}

\bigskip
\centerline{\EightStarTaper}
\centerline{\EightStarTaper\EightStarTaper}
\bigskip

\subsubsection[3. --- 41. ΒΙΒΛΟΣ ΑΛΗΘΗΣ ΣΟΦΕ ΑΙΓΥΠΤΙΟΥ ΚΑΙ ΘΕΙΟΥ ΕΒΡΑΙΩΝ ΚΥΡΙΟΥ ΤΩΝ ΔΥΝΑΜΕΩΝ ΣΑΒΑΩΘ. ΣΩΣΙΜΟΥ ΘΗΒΑΙΟΥ ΜΥΣΤΙΚΗ ΒΙΒΛΩΣ.]{3. --- 41. ΒΙΒΛΟΣ ΑΛΗΘΗΣ ΣΟΦΕ ΑΙΓΥΠΤΙΟΥ\footnote{αἰγύπτου A Laur.} ΚΑΙ ΘΕΙΟΥ ΕΒΡΑΙΩΝ ΚΥΡΙΟΥ ΤΩΝ ΔΥΝΑΜΕΩΝ ΣΑΒΑΩΘ.\footnote{θεῖον A.} ΣΩΣΙΜΟΥ ΘΗΒΑΙΟΥ ΜΥΣΤΙΚΗ ΒΙΒΛΩΣ.}
\paragraph{}
\emph{Transcrit sur} A, f. 251 r. --- \emph{Contenu aussi dans Laur.}, art. 32. --- \emph{Toutes les νariantes insérées dans le texte sont des corrections conjecturales}.

\bigskip

1. < Ο > ΤΗΣ ΥΔΡΑΡΓΥΡΟΥ ΣΤΑΘΜΟΣ. --- Ἀγαθοδαίμων · πέψον, ῥύου τὸν χρυσὸν, καὶ ἐπιβάλλεται ὁ χαλκός · καὶ γίνεται τὸ δίχυτον πέταλον Μαρίας,\footnote{Cp. 3, 12, 1, p. 149, l. 2.} ἵνα πυρὸς καταβαφῆς ἐλαίῳ πίπτῃ < ἢ > μέλιτι,\footnote{F. l. ἵνα πρὸς καταβαφὴν ἐλ. πέπτῃ.} καὶ θραβαθὴ καὶ ἀναληφθείη ὑδράργυρος ὡσεὶ διὰ καμ < άτ > ου. Ὁ χαλκὸς\footnote{θραβαθὴ] F. l. θραυσθῇ ( ? ) --- F. l. ἀναληφθῇ ἡ ὑδρ.} πάλιν ἰὸς ἴσος συγχωνεύεσθαι τῷ χρυςῷ εἰς ὑδράργυρον σταθμοῦ.\footnote{F. l. ἰῷ ἴσος (\emph{M. B.}). --- F. l. συγχωνευέσθω. --- F. l. εἰς ὑδραργύρου σταθμὸν.} Καὶ ἡ Μαρία · « Ὁπόταν οὖν γένηται μάλαγμα καθ ᾽ ἑαυτὸ, ἢ δι ᾽ ὀξάλμης, καὶ πεφθῇ, συλλείου τῷ θείῳ, ἤγουν αἰθάλῃ θείου, ἢ ληκυθίῳ,\footnote{αἰθάλης A.} καὶ κηροτακίδι · καὶ ἐπίβαλε ἢ συλλείου καὶ βλέπε εἰ ἐτελείωσας · εἰ δὲ\footnote{εἰ] ἢ A.} μὴ ἐτελείωσας ξανθῷ τινι ἰὸν ἡμῶν, ὃς ἦν μετὰ τοῦ προβαφίου,\footnote{ὃς] ὁ A.} καὶ ὁποιὸν χρυσόν ἐστι τέλειον, ἵνα μὴ ξανθωθέντα αὐτόν · ἐπίβαλε πάλιν\footnote{F. l. ὁποιὸς χρυσός ἐ. τέλειος.} σὺν τῷ προβαφίῳ ἢ συλλείου < μετὰ > τραπέντος ἀργύρου, τοῦ κελοῦ\footnote{κελοῦ] lire εἰκέλου, comme dans 3, 43, 1.} ἀστράπτοντος, τοῦ ἰοῦ μέρος αʹ, τοῦ ὠμοῦ μύσεως, προβαφίου,\footnote{ὁμοῦ A.} ὡς εἶπεν, χαλκοῦ τὸ μέρος λύει.

2. Πέπτεται, κἂν γὰρ μὴ ἔχῃ ὑδράργυρον δεῖ πέπτειν, ὅτι πρὸ τοῦ\footnote{πέμπεται A. --- δὴ πίπτειν A.} πυρὸς οὐ βαφή · τὸ δὲ ἀπὸ τῶν ὑλῶν καθάρσιον, ἵνα δείξῃ (f. 251 v.) ὅτι ἐστὶ καθαρόν. [Πείραζε δὲ ἀπὸ τῶν ὑλῶν καθάρσιον, ἵνα δείξῃ ὅτι ἐστὶ καθαρόν ·] πείραζε δὲ ἢ καὶ χώνευε · ἂν ἔχῃς τὰς δύο ἀγωγὰς,\footnote{ἔχεις A.} καὶ τὴν Ἰουδαίων καὶ τοῦ ... μὴ ὀκνήση οὖν πειράζειν κατὰ μέρος πάντα οἷα ὑπεθέμην σοι. Οὐ γὰρ ἀμφιβολίας < αἰτία > ἐστὶν ἡ ὑπόθεσις, ἀλλ ᾽ ἵνα συ πειράσῃς ἔσοι ἡ τύχη ἐνήλατός ἐστιν ἢ εἰς πάνυ εὐτυχής.\footnote{ἔσοι] F. l. εἴ σοι. --- ἐνίλατος A. --- F. l. ἢ εἶ πάνυ εὐτυχής.} Ἐμπεσὼν εἰς τὰ μαθήματα ταῦτα, οὐκ ἔστι ἔσοι ἀτυχής · ἀλλὰ γὰρ\footnote{F. l. οὐκέτι ἔσῃ.} νικήσεις μεθόδῳ πενίαν, τὴν ἀνίατον νόσον, μάλιστα ἐὰν εὐεὶ εἰσοὶ\footnote{νικήσεις μεθόδῳ πενίαν ... ] Cp. Synésius, 2, 3, 4. --- ἀνίαρον A. --- εὐεὶ εἰσοὶ ... ] F. l. εὗ εἴσῃ καὶ φροντίσῃς διῶξαι.} καὶ φροντίσῃς, διῶξον τοὺς κωλύτας, ὅτι διὰ τῶν μυρίων βίβλων, καλὸν\footnote{F. l. καλῶς.} λευκωθεὶς καὶ ξανθωθεὶς ὁ χαλκὸς, εἰς τὴν δίπλωσιν χύμεντος μόνον\footnote{F. l. χυμευτὸς.} ἐστὶν ἐπιτήδειος, καὶ ἰωθῇ, καὶ διὰ μυρίων μεθοδευθῇ μόνον χύμεντός\footnote{F. l. καὶ ἰώσει A. F. l. κἂν ἰωθῇ.} ἐστιν ἁρμόδιος, ὁ δὲ χαλκὸς ἡμῶν, τουτέστιν τὸ πᾶν σύνθεμα · ὅπερ\footnote{F. l. ὅδε.} μὲν ἦν ἡ λημματικὴ (καὶ αὐτῇ αὐτοῖς ἐδήλωσαν), ἡ ἀπὸ αἰῶνος\footnote{αὐτῇ] F. l. αὐτὴν.} ζητουμένη καταβαφὴ, καὶ μὴ εὑρισκομένη εἰ μὴ ὧδε · Καὶ τίς ἡ\footnote{τίς] τι A.} αἰτία αὐτοῦ ἐπιτήδειος, ἐδήλωσα σοι περὶ τοῦ χαλκάνθου στίχον · λέγει ὅτι ὥδε καὶ ὁ χαλκὸς βάπτει, καὶ ὁ μόλυβδος, καὶ πᾶν τὸ δεκτικὸν\footnote{F. l. λέγων.} τῆς βαφῆς.

\bigskip
\centerline{\EightStarTaper}
\centerline{\EightStarTaper\EightStarTaper}
\bigskip

\subsubsection[3. --- 42. ΒΙΒΛΟΣ ΑΛΗΘΗΣ ΣΟΦΕ ΑΙΓΥΠΤΙΟΥ ΚΑΙ ΘΕΙΟΥ ΕΒΡΑΙΩΝ ΚΥΡΙΟΥ ΤΩΝ ΔΥΝΑΜΕΩΝ ΣΑΒΑΩΘ.]{3. --- 42. ΒΙΒΛΟΣ ΑΛΗΘΗΣ ΣΟΦΕ ΑΙΓΥΠΤΙΟΥ ΚΑΙ ΘΕΙΟΥ\footnote{αἰγύπτου A Laur. Corr. conj. --- θείου A. Corrigé d'après Laur. cité par Bandini, Catalogue de la Laurentienne.} ΕΒΡΑΙΩΝ ΚΥΡΙΟΥ ΤΩΝ ΔΥΝΑΜΕΩΝ ΣΑΒΑΩΘ.}
\paragraph{}
\emph{Transcrit sur} A, f. 260 r. --- \emph{Contenu aussi dans Laur.}, art. 36. --- \emph{Les νariantes insérées dans le texte sont des corrections conjecturales}.

\bigskip

1. Λόγος βίβλου ἀληθὴς Σοφὲ Αἰγυπτίου, καὶ θείου Ἑβραίων κυρίου τῶν δυνάμεων σαβαώθ. Δύο γὰρ ἐπιστῆμαι καὶ σοφίαι εἰσίν · ἡ τῶν Αἰγυπτίων καὶ ἡ τῶν Ἑβραίων βεβαιοτέρα ἐστὶν δικαιοσύνης θείας\footnote{θείας] θξ (sc. θεὸς ? ) A.} · ἡ γὰρ τῶν ἀγαθωτάτων ἐπιστήμη τε καὶ σοφία κυριεύει ἀμφοτέρων ἐκ τῶν αἰώνων ἔρχεται · ἀβασίλευτος γὰρ αὐτῶν ἡ γενεὰ καὶ αὐτόνομος\footnote{ἐκ] ἐκτέων A.} · ἄϋλός τε καὶ μηδὲν ζητοῦσα τῶν ἐνύλων καὶ παναφθόρων\footnote{ζητὸν A. --- Παμαφθόρων A. F. l. παμφόρων.} σωμάτων · ἀπαθῶς γὰρ ἐργάζεται · νῦν δωρεὰς δὲ εὐχῇ, χημείας σύμβολον φέρεται < ἐκ > κοσμοποιΐας, τοῖς τε σώζουσιν καὶ καθαιροῦσιν τὴν ἐν τοῖς στοιχείοις συνδεθεῖσαν θείαν ψυχὴν, μᾶλλον δὲ θεῖον πνεῦμα φυραθὲν τῇ σαρκὶ, ὑποδείγματος χάριν, ὥσπερ ὁ ἥλιος ἄνθος\footnote{φυραθέντι σαρκὴ A.} πυρὸς καὶ ἥλιος οὐράνιος, καὶ δεξιὸς ὀφθαλμὸς τοῦ κόσμου, οὕτω καὶ ὁ χαλκὸς, ἐὰν ἄνθος γένηται διὰ τῆς καθάρσεως, ἥλιός ἐστιν ἐπίγειος, βασιλεὺς ὢν ἐπὶ γῆς, ὡς ὁ ἥλιος ἐν οὐρανῷ.

2. Οὐδαμοῦ εὑρίσκω τὰς παντελείας καταβαφὰς λαμβανούσας ἥλιον, οἷον τὴν Δημοκρίτου, καὶ τὴν μονάδα τὴν παραδιδοῦσαν τὴν σκυθικὴν κώμαριν · τῆς δὲ τελείας εὑρίσκω λαμβάνουσαν, οἷον τὴν Ἴσιδα,\footnote{σκ. καὶ καμρὶν A. --- F. l. τὰς δὲ τελ. εὑρ. λαμβανούσας. --- F. l. ἴσιδος.} ἣν προσφωνεῖ ὁ Ἡρῶν. Εὑρίσκω ἡλίου ἐξίωσιν · χρυσοζώμιον καὶ ἀργυροζώμιον\footnote{προσφωρεῖ A. --- ἡλίου en toutes lettres. F. l. χρυσοῦ. --- ἀργυροζ.] σεληνοζύμιον A avec le signe de la lune ou de l'argent au-dessus du mot.} ἐπὶ σελήνην ποιεῖ σελήνης, ἵνα σαπῇ μετὰ τοῦ σιδηροχάλκου\footnote{σελήνην puis σελήνης, surmontés du signe, A. F. l. ἐπὶ ἀργύρου π. ἄργυρον.} · ὁμοίως αὗται εἰς τὰς (f. 260 v.) σήψεις ἀργύρωσιν λαμβάνουσιν.\footnote{ἀργύρωσιν] signe de l'argent surmonté de σιν A.} Ὁμοίως δὲ καὶ εἰς οὐ μόνον καὶ διπλώσεις καὶ τριπλώσεις λαμβάνουσιν, καὶ χρυσοῦ καὶ ἀργύρου [καὶ] τὰς μίξεις · ὥστε χρὴ < διὰ > τῶν μεθοδειῶν, ἄνευ χρυσοῦ καὶ ἀργύρου ἐργάσασθαι καὶ τὰς διπλώσεις μὴ χωρίζειν χρυσὸν ἢ ἄργυρον, ὡς καὶ πορνείαν καὶ μῆνιν\footnote{πορνίαν καὶ μήνην B.} · χρυσὸν οὐ λαμβάνουσιν τὸ μείζω ὅτι ἐὰν τὸν χαλκὸν ἀσκίαστον ποιήσῃς,\footnote{F. l. τὸν μείζω.} λευκανεῖς τοῖς λευκαίνουσιν φαρμάκοις, καὶ ξανθώσεις τοῖς ξανθοῦσιν\footnote{ξανθωνοῦσιν A. --- βάψει A.} φαρμάκοις, καὶ βάψεις τὴν καδμίαν ἢ κιννάβαριν χρυσὸς ποιεῖται\footnote{F. l. τῇ καδμείᾳ ἢ κινναβάρει.} εἰς τὰ ἡφαίστεια προσεφώνησα, εἰς σκοροποιία, ἐν ᾗ τὸ πᾶν μυστήριον\footnote{F. l. σκωριοποιΐαν (mot supposé).} τῆς καταβαφῆς κέκρυπται.

3. Τοῦ δὲ χαλκοῦ λευκανθέντος καὶ μελανωθέντος καὶ ξανθωθέντος, βάπτεις τὸν ἄσημον, χρυσὸν ὁρῶν, ἢ τὸν λευκανθέντα χαλκόν\footnote{βάπτει A.} · ἀπὸ γὰρ τοῦ χαλκοῦ γίνεται ὅλα τὰ εἴδη, λέγω κιννάβαριν, καθμίαν, χρυσὸν, σαθὴν ( ? ), καὶ ὅσα ἄλλα. Ὁ γὰρ μόλυβδος εἰς πολλὰ τρέπεται\footnote{σαθὴν] σαθ suivi d'un signe figurant un C couché, surmonté de l'abréviation de ὴν ou de ὶν, A. --- F. l. ὡς γὰρ ὁ μόλ.} · οὕτως καὶ ὁ ἐξ αὐτοῦ χαλκὸς ὁ στεφανίτης. Εὑρήσεις δὲ εἰς τὰ\footnote{ἐξ αὐτὸν A. Les papyrus offrent des exx. de ἐξ avec l'accusatif.} ἐφέπεια τὰς ποιήσεις χρυσοῦ, ἔκ τε τούτων ἐπιπλοκαὶ ὅλα τὰ εἴδη\footnote{ἐφέπεια] F. l. ἡφαίστεια. --- F. l. ἐπιπλοκῶν.} γίνεται · ἀλλήλων γάρ εἰσιν αἱ οὐσίαι οἰκονομίαι · πολλαὶ δὲ μορφαὶ ἐν οἰκονομίαις · ὅλα δὲ κρίναντες βελτίοσιν χρῶ.

\bigskip
\centerline{\EightStarTaper}
\centerline{\EightStarTaper\EightStarTaper}
\bigskip

\subsubsection[3. --- 43. ΖΩΣΙΜΟΥ ΠΡΟΣ ΘΕΟΔΩΡΟΝ ΚΕΦΑΛΑΙΑ.]{3. --- 43. ΖΩΣΙΜΟΥ ΠΡΟΣ ΘΕΟΔΩΡΟΝ ΚΕΦΑΛΑΙΑ.\footnote{Titre dans A : Περὶ αἰτησίου λίθου τουτέστιν ἐκ τοῦ παντὸς γινομένου. Début du texte : ὡς αἰτησίου λίθου καὶ ταῦτα πολὐ χρησίμου.}}
\paragraph{}
\emph{Transcrit sur} M, f. 179 r. ;--- \emph{Collationné sur} A, f. 237 r. ;--- \emph{sur} K, f. 89 r. ;--- \emph{sur} Lc, p. 231 ;--- \emph{sur} E, f. 182 v. (\emph{texte écrit dans} E \emph{par le copiste de} La, Lb, Lc, \emph{probablement d'après} Lc. --- \emph{Contenu aussi dans Laur.}, art. 29 ; \emph{dans le Vind.}, art. 12. --- \emph{Sauf indication spéciale, les variantes de} Lc \emph{existent aussi dans} E.

\bigskip

1. Περὶ ἐτησίου, τουτέστιν ἐκ τοῦ παντὸς συνισταμένου, ὡς ἐτησίου λιθου, καὶ ταῦτα πολυχρησίμου. Πρὸς γὰρ τὰς οἱκονομίας ἕτερον χρῶμα δείκνυσιν · ἄλλο ἀπὸ κηροτακίδος καὶ ἄλλο ἀπὸ τῆς ἐλαιώσεως,\footnote{Réd. de Lc : ἐλαιώσεως λευκὸν ἢ μέλαν, ἢ ξ. ἢ ἡπ.} ξανθὸν ἢ μέλαν ξανθὸν, ἢ ἡπατίζον, ἢ σμυρνίζον, ἢ κηρίζον, ἢ ὅσα οἶδας · ἢ μέλαν, χρυσῷ εἰκέλιον, ἀστράπτον, ὡς καὶ ἐπὶ μέλανσιν\footnote{χρ. ἐϊκελλον M ; χρυσοείκελον Lc.} ποιεῖ, ὡς καὶ εἰς ξάνθωσιν. Ὁ ξανθὸς γίνεται καὶ αἱματώδης καὶ ἀρραγὴς, καὶ τὸ τελευταῖον ὡς κρόκος ξηρὸς. Καὶ ἐὰν δὶς ἢ τρὶς τῷ θείῳ καῇ κατὰ τὰς αὐτῶν γραφὰς, καὶ ἄλλοτε ἐπ ᾽ ὀλίγον βολβίτοις, ταῦτά εἰσιν τὰ χρώματα τὰ μετάτρεπτα βεβαίως ξανθούμενα τὴν\footnote{χρ. ὧν μετατρέπονται A.} πρώτην ἐπὶ τὸ βέλτιον καὶ οὐκ εἰς τὸ χεῖρον ἔχοντα. Αὗται αἱ\footnote{F. l. ἔρχοντα.} οἰκονομίαι κάτοχοι καλοῦνται βαφῶν ἀληθῶς ἀφεύκτων.

2. Περὶ τοῦ ὅτι ἡ βαφὴ, ἤτοι ἀλλοίωσις ἡ γινομένη ἐν τῇ ἰώσει,\footnote{ἤγουν ἡ ἀλλοίωσις ἡ γεν. Lc.} οὔτε λευκὴ, οὔτε ξανθὴ ἐπαγγέλλεται · τὰ γὰρ προλαβόντα δύο\footnote{λευκὴν ο. ξανθὴν MK.} θεῖα, τό τε λευκὸν καὶ ξανθὸν, ταῦτα τὰ ὀνόματα ἐπιστεύθησαν καὶ\footnote{λ. κ. ξ. εἰσι, καὶ ταῦτα Lc. --- καὶ τὰς βαφάς] κατὰ τὰς γραφὰς τῶν βαφῶν Lc.} τὰς βαφάς · αὕτη δὲ ἡ βαφὴ, ἤτοι ἀλλοίωσις ἡ σηπτικὴ, ἐπάνω\footnote{αὕτη δὲ ἡ ἀλλ. τῆς βαφῆς ἡ σηπτ. Lc.} πάντων ἐστίν.

3. Περὶ ἄλλων δύο θείων μὲν λεγομένων, οὐκ ὄντων δὲ θείων ὡς τὰ πρῶτα, ἀλλὰ συνθέματα νῦν παρ ᾽ αὐτοῖς καλούμενα θεῖα, οὐχ ὡς θεῖα, ἀλλὰ διὰ τὸ ἀποτελούμενον ἀπ ᾽ αὐτῶν θεῖον ἔργον.

4. (f. 179 v.) Περὶ τοῦ ὅτι πρῶτον ἐν τῷ συνθέματι γίνεται τὸ κατόχιμον, καὶ πυρίμαχον καὶ βαφικόν · ἀφ ᾽ ἑνὸς ἡμῖν καὶ δευτέρου\footnote{M mg. : M\textsuperscript{γ}, avec renvoi à άφ ᾽ ἑνὸς.} ἐν τῷ ἀσήμῳ τῷ φυσικῷ, τῷ βαπτομένῳ χρυσῷ τὸ λοιπὸν ἡμῖν φανερούμενον.\footnote{Signe du mercure au-dessus de ἀσήμω M. --- χρυσῷ en signe MΚ ; signe de la chrysocolle A ; εἰς χρυσὸν Lc.} Ἡ δὲ τοῦ ζητουμένου λύσις ἐστὶν αὕτη.\footnote{λῦσις M ; λεύκωσις A.}

5. Περὶ τοῦ ὅτι τὸ πρῶτον ἐν τῇ μήτρᾳ ἀφανῶς ἡμῖν γίνεται τὸ κατόχιμον ἐκ δύο, ἔκ τε σπέρματος καὶ αἵματος · καὶ πυριμαχεῖ τὸ πλασσόμενον ζῶον πρὸς τὸ τῆς μήτρας πῦρ, καὶ βάπτεται\footnote{πῦρ, καταβάπτεται Lc.} · τουτέστιν χρῶμα λαμβάνει καὶ σχῆμα καὶ μέγεθος, πάντα ἐν τῷ ἀφανεῖ. Ὅταν δὲ ἀποτεχθῇ, καὶ ἡμῖν πεφανέρωται · καὶ οὕτω χρὴ ἐργάζεσθαι, καὶ μὴ τῇ ὁμωνυμίᾳ τῶν γραφῶν ἢ ἄλλων τινῶν πλανᾶσθαι.

6. Περί σήψεως καὶ ἐξαιματώσεως καὶ ζυμιώσεως καὶ μεταβολῆς, καὶ παλιγγενεσίας · καὶ περὶ ἰώσεως καὶ ἐξιώσεως, καὶ τῶν τοῦ\footnote{παλιγγενησείας MK. --- Après ἰώσεως] καὶ μεταβολῆς add. A. --- M mg. : περὶ ἰοῦ (main du 13\textsuperscript{e} siècle).} ἰοῦ διαφόρων ὀνομάτων. Καὶ ὅτι καὶ ὁ ἰὸς λέγεται ὕδωρ θείου ἄθικτον, καὶ κώμαρις σκυθικὴ καὶ φονοειδὴς, καὶ χρυσόσπερμον · καὶ πᾶν σπέρμα, καὶ ἰὸς χαλκοῦ, καὶ ὕδωρ χαλκοῦ, καὶ ὕδωρ χαλκάνθου, καὶ ἄνθος χαλκοῦ, καὶ < φάρμακον > χαλκειῶδες, καὶ φάρμακον μελιτῶδες,\footnote{χαλκυῶδες MK ; χαλκοειδὲς A.} καὶ γλυκὺ, καὶ ἀρραγὲς, ἀντὶ τοῦ ἐγλυκισμένον, ἀπὸ τῆς τῶν\footnote{ἐγλυκισμένως MK ; ἐγλυκισμένος A.} δηλητηρίων καταφορᾶς. Καὶ οὐ μόνον ἀρσενικῶς καὶ θηλυκῶς καὶ\footnote{καταφ.] μεταφορᾶς Lc.} οὐδετέρως αὐτὸ κεκλήκασιν, ἀλλὰ καὶ ὑπὸ κοριστικῷ μέτρῳ χαλκύδριον · ἄλλοι δὲ ὕδωρ μαζυγίου · μάζα δὲ ὁ χαλκός · ἀφ ᾽ οὗ καὶ ἐν ταῖς ἰουδαϊκαῖς καὶ ἐν πάσῃ γραφῇ μαζὺς ἀνέκλειπτος, ἣν ἔλαβεν\footnote{ἐμ πάση M, comme dans les papyrus et dans les inscriptions.} Μοϋσῆς παρὰ κυρίου λόγου · παραφθαρὲν δὲ τῷ χρόνῳ τὸ\footnote{Après λόγου] λ M. F. l. π. κυριακοῦ λόγου.} ὄνομα ἑγένετο μαζύγιον · ἄλλοι (f. 180 r.) ἀπὸ τοῦ φανοῦ τοῦ ἀνασπῶντος, τοῦ ἔχοντος μαζούς.\footnote{Ce passage trouve son interprétation dans un article du papyrus X de Leyde sur le ferment métallique. Voir l'Introduction, p. 29 et 41 (\emph{M. B.}).}

7. Περὶ οἰσμοῦ, τουτέστιν ἐκφωνήσεως, ἐναποσβεννυμένου πυρός · καὶ σιγμοῦ, τουτέστιν συριγμοῦ, πνεύματος ἐκπεμπομένου ἐξ ὑποστροφῆς [ἢ σιγμοῦ, τουτέστιν πνεύματος ἐπομένου καὶ ἐφελκομένου],\footnote{ἑπομ. καὶ add. A.} ἤγουν ἀναρροφωμένου καὶ εἰσφερομένου.\footnote{Tout ceci s'interprète aussi par l'un des papyrus gnostiques (\emph{M. B.}). --- F. l. ἀνερροφωμένου.}

8. Περὶ τοῦ ὅτι εὑρόντες τινὲς τῶν ἱερέων γραφὴν ἄφθονον οὐκ ἐπίστευσαν ἐργάσασθαι, εἰ μὴ διὰ τούτων τῶν συγγραμμάτων διὰ τὴν ἀπόδειξιν.

9. Περὶ τοῦ ὅτι τὴν τέχνην τῆς ἰώσεως ἔχειν τινὰ μετουσίαν,\footnote{ἡ τέχνη τ. ἰ. ἔχει Lc. --- ἔχει A.} εἰς τὰ ἄλλα δύο βιβλία. Καὶ γὰρ εἰ κατ ᾽ εἶδός ἐστιν ἄλλη, ἀλλ ᾽ οὖν γε κατὰ γένος ἡ αὐτή. Καὶ γὰρ αὐτὴ πάλιν ἐστὶν βαφική.\footnote{καὶ γὰρ ἡ αὑτὴ Lc.}

10. Περὶ τοῦ ἐὰν λέγῃ ἐξίωσιν ἢ ἀσκιάστωσιν ἢ στροφὴν ἢ ἐκστροφὴν ἢ φύσει κεκρυμμένην ἢ ἀκαύστωσιν, περὶ τῆς λευκώσεως λέγει.

11. Περὶ τῶν οἰκονομιῶν τῶν χρησιμευόντων ἀπὸ τοῦ λευκοῦ ἐπὶ τὸ ξανθὸν, καὶ ἀπὸ τοῦ ξανθοῦ ἐπὶ τὸ λευκὸν, μάλιστα ἐπὶ τῶν\footnote{μάλιστα δὲ Lc.} θείων δεῖ ζητεῖν οἷον οὕτως, ἐν τῇ ὑστεραίᾳ < τάξει > τῶν ζωμῶν,\footnote{οἶον οὕτως] ὡς Lc. --- ὑστέρᾳ Lc. --- δεῖ ζητεῖν] ζήτι (pour ζήτει) A, puis : ἵνα γὰρ αὐτὸς ἐν τῆ ὑστέρα ἀπὸ τοῦ λευκοῦ εἰς τὸ ξανθὸν τῶν ζ. φησὶν ὁ φ.} φησὶν ὁ φιλόσοφος · « Πῆξαι ἀρσενίκου γ° αʹ, καὶ θείου γ° $\svgB$ ἢ\footnote{Réd. de Lc : πῆξον ἀρσ. οὐγγίαν μίαν καὶ θ. οὔγγίαν μίαν καὶ τῷ αὐτῷ συσταθμ., καὶ ἐ. τ. ξ. ἐπὶ τῆς συστ. --- M mg. : grosse étoile.} φλοιοῦ λίτραν τῷ αὐτῷ συστάθμιζε · ἐπὶ τοῦ ξανθοῦ, ἀντὶ τῆς\footnote{τὸ αὐτὸ συσταθμιάζειν A. --- F. l. τὰ αὐτὰ συστάθμιζε.} συσταθμίας τῶν φλοιῶν, βάλλε κρόκον καὶ ἐλύδριον, καὶ ἀντὶ τῶν λευκῶν γῶν, τὴν αὐτὴν συσταθμίαν ὤχρας καὶ σινώπιδος ἢ χαλκάνθου ἢ σώρεως. Καὶ τὰ μὴ ἔχοντα συσταθμίαν ὡς σοφὸς ἅρμοσον\footnote{ἅρμοσον] ἕνωσον A.} ὡς ἰατρῶν παῖδες. Τὰ γὰρ ὕγρα σχεδὸν ἐπίκοινά εἰσιν, πλὴν ὀλίγα ἅτινα οἶδας.\footnote{ὀλίγων Lc, mel. --- οἶδας] οἶσθα E, mel.} »

12. Περὶ τοῦ δεῖν κατανοεῖν ὅτι τε δεινὸν ὑπέστημεν κάματον\footnote{καμ.] κίνδυνον καμάτων A.} ἔστ ᾽ ἂν συνουσιωθῶσιν, τουτέστιν συγγαμήσωσιν αἱ φύ- (f. 180 v.) σεις τὸ τηνικαῦτα, καὶ ὅτι πᾶς χρήσιμος λόγος αὐτοῖς ἐφάνη\footnote{τὸ τηνικαῦτα] τὰ χρονικώτατα A ; Lc. om. --- A et Laur. ( ? ) continuent avec le morceau suivant (Καὶ ὅτι τοὺς χρησίμους ... 3, 44).} · καὶ ὅτι δεῖ ζητεῖν τοῦτον τὸν λόγον · ἢ ὅτι τέχνη ἢ ὁτιοῦν ποτέ ἐστιν τὸ τί ἐστιν, καὶ ὁποῖον τί ἐστιν, καὶ ἵνα τί ἐστιν.\footnote{ἵνα τί, pour διὰ τί, comme dans la Bible des Septante.}

13. Περὶ τοῦ ὅτι ὅλαι αἱ καταβαφαὶ τῶν ἀρχαίων ἀληθεύουσιν τῇ ἀγωγῇ τοῦ στερεοῦ συνθέματος, τουτέστι τῆς ἰώσεως. Ἐὰν γὰρ βάλῃς τῆς ἰώσεως μέρος αʹ, καὶ τῶν οἰκονομηθέντων εἰδῶν, ἤγουν ξηρίων ὧν καλοῦσιν ἐπιβαφίων, μέρος αʹ, καὶ ὀπτήσῃς, ἕξεις τὴν\footnote{ὧν καλ.] τῶν καλουμένων Lc, f. mel.} ἀλήθειαν.

14. Περὶ τοῦ ὅτι ἄκαυστόν ἐστι τὸ μηκέτι ἔχον ὃ καυθήσεται,\footnote{ὃ] F. l. ᾧ.} ἀλλ ᾽ ἀποκεκαυμένον, ὡς τὰ ξύλα καὶ οἱ χυλοὶ ἐπὶ τῶν πυρετῶν τῶν μὴ κεκριμένων.

15. Περὶ τοῦ ὅτι ἡ ὑπόσταθμις τῶν κεκαυμένων, τουτέστιν ἡ σποδὸς, αὕτη ἐστὶν τοῦ παντὸς ἐνέργεια.

16. Περὶ τῆς τῶν τεσσάρων στοιχείων εἰς ἑαυτὰ μεταβολῆς,\footnote{M mg. : grosse étoile.} καὶ ὅτι οὐ τὰ μόνον ἀπὸ γῆς καὶ ὕδατος μεταβαλλόμενα πῦρ γίνονται,\footnote{F. l. οὐ μόνον τὰ.} ἀλλ ᾽ ὅτι καὶ ἀναφέρονται · ἀνωφερὲς γὰρ τὸ πῦρ · ταύτην δὲ τὴν\footnote{Signe du cinabre au-dessus de ἀναφέρονται M.} εἰκόνα οὐκ εἰκῆ λαμβάνει, ἀλλὰ διὰ τὴν τέχνην καὶ τὰ ταύτης εἴδη. Ὅτι πρῶτον γῆ ὄντα καὶ ὕδωρ, ὕστερον γίνονται πῦρ,\footnote{Même signe au-dessus de πῦρ M.} καὶ ἄνω φέρονται · καὶ ὅτι τῇ ποιότητι μόνῃ τὰ στοιχεῖα ἐναντιοῦνται ἀλλήλοις, καὶ οὐχὶ τῇ οὐσίᾳ · ἡ γὰρ οὐσία τῇ οὐσίᾳ οὐκ ἔστιν ἐναντία, καθὸ οὐσία. Διὰ τοῦτο καὶ οὐσίας ἐκάλεσεν τὰ τέσσαρα\footnote{M mg. : série de points ascendants, avec renvoi à τέσσαρα.} γράμματα ὁ φιλόσοφος τῇ ἑνώσει τῆς οὐσιότητος ἑλκούσας τὸ ἔξωθεν\footnote{γράμματα] γράμματα \emph{vel} σώματα E. F. l. στοιχεῖα ?} διαχριόμενον φάρμακον. Καὶ ὅτι ὥσπερ τὰ στοιχεῖα εἰς ἑαυτὰ ἀναλυόμενα πάντα κατεργάζεται, οὕτω καὶ ἡ τέχνη · καὶ ὥσπερ αἱ τέσσαρες τροπαὶ μεταβαλλόμεναι νικῶσιν τὰς προτέρας κράσεις, οὕτω καὶ αἱ τέχναι ταῖς μεταβολαῖς νικῶσι τὰς φύσεις.\footnote{M mg. inf. : λίαν ἡ πυκτὶς καὶ πάνυ παγίως ξένη φίλοι.}

\bigskip
\centerline{\EightStarTaper}
\centerline{\EightStarTaper\EightStarTaper}
\bigskip

\subsubsection{3. --- 44. Sur les Divisions de l'Art Chimique.}
\paragraph{}
\emph{Texte fort corrompu dans} A (f. 238 v.) \emph{et dans Laur., manuscrits dans lesquels il est la continuation du texte précédent} (p. 217, l. 24). \emph{Nous aνons reconnu récemment qu'il se trouνe aussi dans le Philosophe anonyme} (\emph{ci-après} 6\textsuperscript{e} Partie). \emph{Nous aνons cependant cru deνoir conserνer une partie du texte et de la traduction, répondant au titre ci-dessus. A partir de la 4\textsuperscript{e} ligne, nous aνons suivi le texte de} M (fol. 181 et 182).

\bigskip

1. Καὶ ὅτι τοὺς χρησίμους λόγους αὐτοὺς δεῖ ζητεῖν · καὶ τί δεῖ\footnote{αὐτοὺς] F. l. αὐτοῦ.} φάναι τὴν τῶν λόγων, ἤ ὅτι τέχνη, ἢ ὁτι πρότερόν ἐστιν ἢ τὸ τί δέ ἐστιν, ἢ ὁποῖον τί δεῖ, < καὶ > ἵνα τί δεῖ · καὶ περὶ νοημάτων ἀνεπιγράφησαν ἃ ἦν καθέκαστα καὶ ἄτομοι πάντες, ὃν καὶ ἄπυρα, καθὼς ἔστιν εὑρεῖν\footnote{καὶ ἄτομοι πάντως καὶ ἄπειροι. M. --- στοίχων A.} ἀπειρίαν ἄτομον. Ὥσπερ δὲ δʹ ὄντων τῶν μουσικῶν γενικωτάτων στοχῶν, αʹ, βʹ, γʹ, δʹ, γίνονται παρ ᾽ αὐτοῖς τῷ εἴδει διάφοροι στοχοὶ κδʹ,\footnote{στοίχει A.} κέντροι καὶ ἶσοι καὶ πλάγιοι καθαροί τε καὶ ἄηχοι · καὶ ἀδύνατον ἄλλως\footnote{καθὰ εἴρηται καὶ ὰ ἤχει A.} ὑφανθῆναι τὰς κατὰ μέρος ἀπείρους μελῳδίας τῶν ὕμνων,\footnote{μέρους A.} ἢ θεραπειῶν ἢ ἀποκαλύψεων, ἢ ἄλλου σκέλους τῆς ἱερᾶς ἐπιστήμης, καὶ οἷον ῥεύσεως, ἢ φθορᾶς, ἢ ἄλλων μουσικῶν παθῶν ἐλευθέρας · τοῦτο κάνταῦθα ἔστιν εὑρεῖν τὸν δυνατὸν ἐπὶ τῆς μιᾶς καὶ ἀληθοῦς κυριωτάτης ὕλης τῆς ὀρνιθογονίας.
\begin{center}
\emph{Les} § \emph{2, 3, 4, se retrouνeront dans la 6\textsuperscript{e} partie}.
\end{center}
\paragraph{}
5. Καὶ ὥσπερ τετραμερῆ τὴν ἀρίστην φιλοσοφίαν, ἤτοι τὴν ὕλην\footnote{Cp. ce paragraphe avec 3, 29, 2.} ὑπὸ τῆς φύσεως δεδειγμένην εὑρίσκομεν τὴν γενικήν τε καὶ εἰδικὴν, καὶ τάξεων τὰς διαφορὰς, οὕτω καὶ τὴν καλὴν φιλοσοφίαν ζητοῦντες, τετραμερῆ ταύτην εὑρήκαμεν, τὸ πρῶτον ἔχουσαν μέλανσιν, δεύτερον λεύκωσιν, καὶ τὸ τρίτον ξάνθωσιν, καὶ τέταρτον ἴωσιν. Πάλιν δὲ,\footnote{Réd. de A : Πάλιν δὲ, ὥσπερ ἑκάστου τῶν εἰρημένων ἁπάντων ἀπὸ στίχου ἐξ ἑνὸς γενικοῦ ἕξει πλύσιν αὐτοῦ παντὸς ἡμισοστοίχειον.} ὡς ἕκαστος τῶν εἰρημένων στοχῶν ἐξ ὧν γενικῶν ἔχει πλησίον ἑαυτοῦ πάντως ἡμιστόχιον ἢ μεσόκεντρον, δι ᾽ οὗ κατὰ τάξιν προσβαίνει ἢ ἀποβαίνει, οὕτω κἀνταῦθα, μεταξὺ μελανώσεως καὶ λευκώσεώς ἐστιν ἡ ταριχεία, καὶ τῶν εἰδῶν ἡ πλύσις μεταξὺ δὲ λευκώσεως καὶ ξανθώσεώς ἐστιν ἡ χοοποίησις · τούτων ξανθώσεώς τε καὶ ἰώσεώς ἐστιν ὁ τοῦ συνθέματος διχασμός. Τῆς δὲ ἰώσεως πέρας ἡ διὰ τοῦ ὀργάνου\footnote{Après πέρας] ἀδύνατον add. A.} τοῦ μασθωτοῦ οἰκονομία, καὶ ἡ ἕνωσις τῶν μερῶν · καὶ ἀδύνατον ἄλλως, οἷον (f. 182 v.) τὴν καθ ᾽ εἱρμὸν ἐπιστήμης. Εἰ γὰρ καί\footnote{ἄλλως οἰκονομεῖθαι A. --- καθ ᾽ ἡρμῶν A. F. l. καθ ᾽ Ἑρμῆν.} τινες ξάνθωσιν ἄνευ λευκώσεως ἐπετήδευσαν, ὧν ἐστιν ὁ Πηβίχιος,\footnote{ἐπὶ τι δεύτυσαν A. --- ἂν ἐν ταρυχεῖ A.} ἀλλ ᾽ οὐκ ἄνευ ταριχείας, ἢ πλύσεως τῶν εἰδῶν, ἅτινά ἐστι μέρη τῆς τελείας λευκώσεως.\footnote{Après λευκώσεως] A ajoute ἔχει.}
\begin{center}
\emph{Le} § \emph{6 sera donné dans la 6\textsuperscript{e} partie. --- Reprise du} ms. A.
\end{center}
\paragraph{}
7. Ὅτι τὸ παρὸν βιβλίον ὀνομάζεται βίβλος μεταλλικὴ < καὶ > χυμευτικὴ περὶ χρυσοποιίας, ἀργυροποιίας, ὑδραργύρου πήξεως, ἔχων\footnote{πήξεως] ποιήσεως A. Corr. conj. --- ἔχων] F. l. ἔχουσα.} αἰ- (f. 240 v.) θάλας, βαφὰς φούρμουσαι ἀπὸ βροτισίων, ὡσαύτως\footnote{F. l. ἀφορμώσας ἀ. βροντησίων.} καὶ λίθων πρασίνων, καὶ λυχνιτῶν, καὶ ἑτέρων πάντων χρωμάτων, καὶ μαργάρων, καὶ δερμάτων ἐρυθροδανώσεις βασιλικῶν. Ταῦτα δὲ πάντα γίνονται ὑπὸ ὑδάτων θαλασσίων, ὠῶν, διὰ τέχνης μεταλλικῆς.

\bigskip
\centerline{\EightStarTaper}
\centerline{\EightStarTaper\EightStarTaper}
\bigskip

\subsubsection[3. --- 45. ΥΔΡΑΡΓΥΡΟΥ ΠΟΙΗΣΙΣ.]{3. --- 45. ΥΔΡΑΡΓΥΡΟΥ ΠΟΙΗΣΙΣ.\footnote{περὶ ἀργυροποιΐας AK.}}
\paragraph{}
\emph{Transcrit sur} M, f. 107 r. --- \emph{Collationné sur} A, f. 146 v. ;--- \emph{sur} K, f. 32. v. --- \emph{Presque toutes les νariantes de} M \emph{ont été reportées dans} K, \emph{sur la ligne}.

\bigskip

1. Λαβὼν ψιμύθιον καὶ σανδαράχην ἴσα λείωσον μετὰ ὄξους ἕως\footnote{K mg. : ὑδραργύρου ποίησις (en signes) et d'une main plus récente : \emph{cf.} 75. 75 est le plus ancien n° de E, qui toutefois ne contient pas ce morceau.} γένηται γλοιῶδες. Εἶτα βαλὼν εἰς (f. 107 v.) λωπάδα ἀγάνωτον, πώμασον πώματι χαλκῷ, περιπήλωσον, καὶ ὑπόκαιε ἄνθραξιν ἠρέμα,\footnote{ὑπόκαιε] ὑποκάπνισον, ἤγουν ὑποκαίων AK.} καὶ ὅτ ᾽ ἂν εἰκάσῃς ὅτι καλῶς ἔχει, ἀναπώμασον ἐλαφρῶς, καὶ πτερῷ ἄφελε τὴν ὑδράργυρον.

2. Λαβὼν ἄμμον τὴν χρυσίζουσαν, λείωσον, ψῦξον ἕως ἂν ξηρανθῇ,\footnote{ἄμμον] ἄμμυλλον AK. --- K mg. : ἄμμον, puis, comme ci-dessus : \emph{cf.} 75.} καὶ συμμίξας πάλιν ἅλατι, ὄπτησον ἐν καμίνῳ ἡμέραν καὶ νύκτα. Καὶ ἄρας πλῦνε ἕως < ἂν > τὸ ἄλας ἀπορρεύσῃ · καὶ πάλιν ξήρανον,\footnote{ἀπορεύσει M ; ἀπορεύση AK. Corr. conj.} καὶ φύρασον ὄξει, καὶ ἔασον βραχὺ ἕως συμπίῃ καὶ ξηρανθῇ · καὶ πάλιν δὸς\footnote{Réd. de AK : ἔασον βραχεῖναι (pour βραχῆναι) ἕως τοῦτο ἅλας συμπίη.} εἰς τὴν κάμινον μὴ ἀποπλύνας, καὶ τοῦτο ποίει καθάπαξ, φυρῶν τῷ ὄξει, καὶ διδοὺς εἰς τὴν κάμινον τετράκις ἢ πεντάκις, ἵνα γένηται ὡς μίλτος. Ἔπειτα λαβὼν ἕλκυσμα ἀσήμου ἰσόσταθμον, λείωσον καὶ\footnote{ἀσημίου AK (d'où le néogrec ἀσήμι). --- λείωσον puis le signe de l'argent AK.} ἀνάμιξον. Εἶτα χωνεύσας χώρισον, καὶ μόλυβδον ἐπίπασσε ἐπ ᾽ ἀμφοτέροις, μέχρις ἂν ἀναλωθῶσι, καὶ ψύξας εὑρήσεις τὸν μόλυβδον σκληρόν · τοῦτον ψωμαρίῳ χώνευσον · ἐκφύσησον ἵνα δείξῃ.\footnote{τοῦτον --- δείξῃ] καὶ τοῦτο τὸ ψωμάριον χων. A. Après δείξῃ (lire δέξῃ ? ), M continue seul.}

3. Λαβὼν γῆν ἀπὸ τῆς ὄχθης τοῦ ἐν Αἰγύπτῳ χρυσορρόου ποταμοῦ,\footnote{χρυσορόα M.} συμφύρασον ἀφαιρέματι ἐκ τοῦ σιλιγνοπωλίου προσείσας καὶ τοῦ\footnote{ἀφαίρεμά τι M. --- F. l. σιλιγνοπαλίου (de σίλιγνις, fleur de farine et de πάλη, même sens).} λεπτοῦ προσμίξας καὶ ποιήσας φύραμα, ἀναμίγνυε εἰς λεκάνην ὀστρακίνην,\footnote{F. l. τῷ λεπτῷ.} ἄχρις ἂν κολληθῇ βʹ ἐπιμελῶς καὶ γένηται ὡς φύραμα ἄρτου. Εἶτα ἀναλαβὼν καὶ πλάσας ἀρτίσκους, καὶ στοιβάσας ἐπιμελῶς ἐπὶ\footnote{στυβάσας M. Corr. conj.} σανίδος, ψῦξον εἰς ἥλιον ἄχρις οὗ ξηρανθῇ λίαν. Καὶ βαλὼν εἰς ὅλμον, καὶ ἀναλαβὼν, βάλε εἰς χύτραν καινήν · καὶ πωμάσας ἐπιμελῶς τὴν χύτραν, θὲς ἀπέχουσαν τοῦ χαμαὶ πα- (f. 108 r.) λαιστήν.\footnote{M mg. inf. du f. 107 v. : ἐκφύσισον (pour ἐκφύσησον) καὶ πλύνον. (14 ou 15\textsuperscript{e} siècle).} Καὶ ἀνακάλυψον αὐτὴν βολβίτοις, καὶ ὑπόκαυσον ὑποκάτω. Καὶ ὅτ ᾽ ἂν ἀποσχῇ ἡ φλόξ, ἀνακαλύψας, κίνει σιδήρῳ ἄχρις ἂν ἴδῃς ὅλον ὠπτημένον καὶ ὅμοιον σποδῷ μελαίνῃ. Ἐὰν δὲ μὴ ᾖ γεγονὼς, ἀνακινήσας\footnote{μὴ] μοι M. Corr. conj.} πάλιν τῇ αὐτῇ ἀγωγῇ καὶ ἀνακαλύψας, κατάμαθε καὶ κάθελε ἀπὸ τοῦ πυρὸς, καὶ ἔα ψυγῆναι ἡμέραν μίαν. Καὶ ἄρας δράκα ταῖς δύο χερσὶ, βάλε εἰς λεκάνην ὀστρακίνην, καὶ ἐπιβαλὼν ὑδράργυρον, κίνει τῇ χειρὶ γυμνάζων. Εἶτα ἄρας ἄλλην δράκα ἐκ τῆς χύτρας, ἐπίβαλλε ἄλλην\footnote{F. l. γυμναζόμενος.} δράκα ὕδατος, καὶ ἀπόπλυνε. Καὶ πάλιν ἑτέραν δράκα ἐπίβαλλε, καὶ ὁμοίως ἀπόπλυνε. Ποίει δὲ τοῦτο ἕως κενωθῇ ἡ χύτρα, καὶ τότε πλῦνον\footnote{M mg. : μάλαγμα, sur une ligne verticale, en lettres retournées. --- τούτω M.} καθαρῶς ἕως ἂν καταντήσῃ εἰς τὴν ὑδράργυρον. Καὶ βαλὼν εἰς ῥάκος,\footnote{ῥάκκος M ici et partout.} ἐκπίασον ἐπιμελῶς ἕως κενωθῇ · καὶ λύσας τὸ ῥάκος, εὑρήσεις τὸ στερρόν. Τοῦτο ποιήσας, σφαιρίον βάλε < εἰς > βατάνιον καινὸν, καὶ ποίησον εἰς τὸ μέσον ἐκ τῆς ἀπαλειφῆς ὡς βοθύνιον, καὶ κάθες τὸ σφαιρίον. Καὶ πωμάσας, θὲς ἵνα φθάσῃ ἴσως · καὶ τὸ περὶ τὸ ἥμισυ μέσον τοῦ βατανίου πάλιν\footnote{F. l. καὶ τῷ π. τ. ἥ. μέσῳ.} περιπώμασον τὴν χύτραν · καὶ ἔστω πρόσκολλος τῷ βατανίῳ. Καὶ ἐπιθεὶς ἐπὶ κυθρόποδος, ὑπόκαιε ξύλοις στερροῖς ἢ βολβίτοις λαμπρῶς καίων,\footnote{M mg. : πυροστάτης (1\textsuperscript{re} main) avec renvoi à κυθρόπ.} ἄχρι πυρρωθῇ λίαν τοῦ βατανίου ὁ πυθμήν. Μόνον ὕδωρ ἔστω σοι\footnote{πυρωθῆ M. Corr. conj.} παρακείμενον, ἐξ οὗ τὴν οὕσκην σπόγγῳ παράβρεχε, προσέχων μὴ τὸ\footnote{οὐσκην (sans accent) M. Le signe ῁ au-dessus de ce mot. --- M mg. : πώμα (lire πῶμα) ἐστὶν κακάβου (l. κακκάβου), de la 1\textsuperscript{re} main, avec renvoi à οὔσκην. Cp. Hésychius, \emph{voce} ὑρτάνα, ὑρτάνη (même sens).} ὕδωρ εἰς τὸ βατάνιον γένηται · ὅτ ᾽ ἂν δὲ γένηται ἔμπυρον, κάθελε τὸ βατάνιον ἐκ τοῦ πυρὸς, καὶ ἀνακαλύψας, εὑρήσεις ὃ ζητεῖς.\footnote{ὃ ζητεῖς] ὄξη τρεῖς M. Corr. conj.}

\bigskip
\centerline{\EightStarTaper}
\centerline{\EightStarTaper\EightStarTaper}
\bigskip

\subsubsection{3. --- 46. ΠΕΡΙ ΔΙΑΦΟΡΑΣ ΧΑΛΚΟΥ ΚΕΚΑΥΜΕΝΟΥ.}
\paragraph{}
\emph{Transcrit sur} A, f. 249 v. --- \emph{Toutes les variantes insérées dans le texte sont des corrections conjecturales}.

\bigskip

1. Χαλκὸν κεκαυμένον ποιοῦσίν τινες διὰ θείου, ὡς αἱ τάξεις\footnote{Ce 1\textsuperscript{er} § est une reproduction de 3, 13, avec quelques variantes, qui ont été reportées au passage cité.} τῶν ἄλλων λέγουσιν ἀσαφῶς, μόνος ὁ Δημόκριτος ἀφθόνως\footnote{ἀσαφῶς] σαφῶς M. Lu comme dans 3, 13.} ...

2. Αἰθάλη ἐστὶν δι ᾽ ἀμβίκων καιόμενον λεπτῷ πυρὶ κοβαθίων. Περὶ δὲ πήξεων τῶν κατασπωμένων σκωριδίων, τοῦτο ἐπεθύμησαν ἰδεῖν οἱ τῶν ἀρχαίων προφῆται, ἀλλ ᾽ ὅτι καὶ περὶ τῶν ψάμμων πάντες φροντίζουσι. Ὅτι ἡ ὕλη τῶν σωμάτων τετρασωμία λέγεται. Ὅτι καὶ μόλυβδον μέλανα ἐπεθύμησαν ἰδεῖν οἱ Αἰγύπτιοι · ἐν δὲ τῇ ἐργασίᾳ ἐστὶν ἀπομέλανσις. Γίνωσκε δὲ ὅτι καὶ τὰ σκωρίδιά εἰσι τὸ\footnote{Cp. Olympiodore, 2, 4, 37.} ὅλον μυστήριον · μέλανα γὰρ οἴδασιν οἱ ἀρχαῖοι τὸν μόλυβδον < ὅτι > ἐστὶν ὁ ὑπὸ οὐσίας. Καὶ πῶς γίνεται ; ἐὰν μὴ τὰ σώματα ἀσωματώσῃς καὶ ποιήσῃς τὰ δύο ἓν, οὐδὲν τὸ προσδοκώμενον ἔσται.\footnote{Cp. Ol. § 40. --- F. l. οὐδὲν τῶν προσδοκωμένων ἔσται, comme dans Ol.} Καὶ ἐὰν μὴ τὰ πάντα [τῷ] περιεκλεπτυνθῇ, καὶ ἡ αἰθάλη πνευματωθεῖσα καὶ πηχθῇ, οὐδὲν εἰς πέρας ἀχθήσεται · χαλκὸν δὲ μόλυβδον εἶναι\footnote{καὶ] F. l. μὴ.} αἱ οἰκονομίαι τῶν δύο σκωριῶν. Σκεύαζε δὲ ζωμὸν ἀπὸ μολύβδου\footnote{F.l. < δηλοῦσιν > αἱ οἰκ.} · λαβὼν νίτρου μέρη δʹ, στυπτηρίας στρογγύλης μέρος αʹ, μύσεως\footnote{Signe du cinabre au-dessus de στρογγύλης A.} μέρη δύο, ἅλατος καππαδοκικοῦ μέρη δʹ · βάλε ἐν ὄξει λίαν δριμυτάτῳ, καὶ ποίησον ζωμόν · ἐν τούτοις γὰρ ἀποσκιάσεις τὰ πέταλα · οὕτως γὰρ ὁ ζωμὸς ἀρχὴ καὶ τέλος ἐδοκι- (f. 250 v.) μάσθη. Ἐὰν\footnote{ἐδοκιμάσθην A. --- Ἐὰν γὰρ ἴδῃς jusqu'à εὑρήσεις τὸ ζητ. (l. 26)] Olympiodore a cité ce passage (probablement de mémoire) en l'attribuant à Zosime (2, 4, 47).} γὰρ ἴδῃς τὰ πάντα σποδὸν γινόμενα, τότε νόει ὅτι καλῶς ἐσκεύασας\footnote{γὰρ] F. l. δὲ.} ταῦτα τῷ πυρί. Τοῦτο τὸ σκωρίδιον λείωσον καλῶς καὶ ἐξυδάτωσον καὶ ἀπόπλυνον ἑξάκις καὶ ἑπτάκις ἐν γλυκοῖς ὕδασι καθ ᾽ ἑκάστην χωνείαν ποιῶν · διὰ γὰρ τῆς δυνάμεως τοῦ ψάμμου καὶ αἱ χωνείαι γίνονται · διὰ γὰρ ταύτης τῆς πλύσεως γλυκαίνεται τὸ σύνθεμα · μετὰ γὰρ τὸ τέλος τῆς ἰώσεως, ἐπιβολῆς γινομένης, γίνεται τοῦτο\footnote{γινομένων A.} καὶ βεβαία ξάνθωσις · καὶ τοῦτο ποῖων ἐκφέρεις ἔξω τὴν ἔνδον κεκρυμμένην. « Ἔκστρεψον γὰρ, φησὶν, τὴν φύσιν, καὶ εὑρήσεις τὸ\footnote{ἐκφέρει A.} ζητούμενον · ἐκστρεφομένης τῆς φύσεως, οὐκέτι λευκὸν ὁρᾷται.\footnote{φύσεις A.} »

\bigskip
\centerline{\EightStarTaper}
\centerline{\EightStarTaper\EightStarTaper}
\bigskip

\subsubsection[3. --- 47. ΖΩΣΙΜΟΥ ΠΕΡΙ ΟΡΓΑΝΩΝ ΚΑΙ ΚΑΜΙΝΩΝ.]{3. --- 47. ΖΩΣΙΜΟΥ ΠΕΡΙ ΟΡΓΑΝΩΝ ΚΑΙ ΚΑΜΙΝΩΝ.\footnote{Cp. 3, 50, 4.}}
\paragraph{}
\emph{Transcrit sur} M. f. 186 r. --- \emph{Collationné sur} K, f. 94 v. --- \emph{Contenu aussi dans le Vaticanus} 1174, f. 42.

\bigskip

1. Ἡ τῆς ὁρωμένης καμίνου διαγραφὴ κεῖται, ἧς ὁ φιλόσοφος οὐκ\footnote{Lire πρόκειται (leçon de 3, 50, 4).} ἐμνημόνευσεν, εἰ μὴ μόνον πρισμάτων καὶ τῶν ἄλλων, περὶ ὧν ἠρέμα ἐν τῷ περὶ ποσότητος πυρὸς ὑπομνήματι γεγράφηκα · ἐώρακα εἰς τὸ ἱερὸν Μέμφιδος ἀρχαῖον κατὰ μέρος κειμένην τινὰ κάμινον, ἣν οὐδὲ συνθεῖναι εὗρον οἱ μύσται τῶν ἱερῶν. Ἔρρωσο.

2. Πολλαὶ μὲν οὖν ὀργάνων κατασκευαὶ γεγραμμέναι εἰσὶν τῇ Μαρίᾳ · οὐ μόνον ὑδάτων θείων, ἀλλὰ καὶ κηροτακίδων εἴδη πολλὰ καὶ καμίνων. Τὰ οὖν τοῦ θείου ὄργανα πρὸ πάντων ἀναγκαῖον ἐκδοῦναι\footnote{ὄργανα] dernier mot de ce morceau dans le Vat.} · μάλι- (f. 186 v.) στα ἐπειδὴ καὶ αὐτῶν πρὸ πάντων χρεία, βίκος ὑέλινος, σωλὴν ὀστράκινος, πῆχος, λωπὰς, ἄγγος στενόστομον,\footnote{βῆκος MK, ici et partout.} ἐν ᾧ ἔστω ὁ σωλὴν εἰς τὸ πάχος τοῦ βικοστόμου αὐτοῦ. Καὶ ἄλλος τρόπος κομιδῆς ὕδατος θείου · ἀλλ ᾽ οὐχ ὡς τρίβικος ἔστω σωλὴν,\footnote{ἔστω] F. l. ἔσται.} ἀλλ ᾽ εἰς πυθμένα χαλκείου ἐντεθεὶς μήκους πήχεως ἢ ἑνὸς ἥμισυ · τῷ αὐτῷ τρόπῳ καὶ βίκος εἷς, καὶ ὑποκάτω λωπὰς θείου ἀπύρου, καὶ συναρμόσας, κάε. Ὁ δὲ τύπος οὗτος. Ἔχειν δὲ δεῖ ἐπὶ ὅλων\footnote{Lire καίε.} κρατῆρα ὕδατος καὶ περιψᾶν σπόγγῳ τὸ ἄγγος.\footnote{κρατῆραν MK.}

3. Καὶ ἐπὶ τῶν θείων τινὲς τῷ φανῷ < χρῶνται > καὶ τοῖς ὁμοίοις ὀργάνοις τοῖς ἔχουσι κάθισμα ὡσεὶ δρακοντῶδες. Πήσσουσιν καὶ ὑδράργυρον ξανθὴν αὐτὴν καθ ᾽ ἑαυτὴν διὰ τῆς τοῦ θείου ἀναθυμιάσεως · τῶν ἀρχαίων γραφῶν, τοῦτο παρέγνωσαν, ἀμοιροῦντος μέντοι γε τοῦ φανοῦ κρύβοντες. Καὶ ἐθαύμασα ἐπὶ ταύτῃ τῇ γραφῇ καὶ ὅτι δύο μυστήρια ἐν αὐτῇ ἐκρύβη φανερά. Καὶ οὐ ζητοῦμεν [ὅτι] πῶς τοῦ θείου ἀπύρου λευκὴ οὖσα καὶ πάντα λευκαίνουσα μόνη τῇ ὑδραργύρῳ\footnote{F. l. τὴν ὑδράργυρον ξανθὴν.} ξανθὸν ἀναδείκνυσιν μή τοι γε καῦσις αὕτη τούτῳ, ἔτι δὲ καὶ αὕτη\footnote{F. l. μέν τοί γε. --- αὐτῇ MK. Corr. conj.} λευκὴ οὖσα καὶ δυνάμει καὶ ἐνεργείᾳ, καὶ ὑπὸ λευκοῦ καιομένη, πηγνυμένη, πῶς ἐξέρχεται ξανθόν. Ἔδει οὖν πρό γε πάντων τοὺς νέους ταῦτα ζητεῖν, τὸ δὲ ἕτερον μυστήριον μὴ μόνον μετ ᾽ αὐτοῦ πήγνυσθαι, ἀλλὰ μεθ ᾽ ὅλου τοῦ συνθέματος.

4. Ἐγέλασα δὲ εἰς ἐξάκουστον γράφων ταύτην τὴν τάξιν λέγουσαν.\footnote{M mg. : groupe de quatre cercles accolés, avec point à leurs centres, et rejoints deux à deux par un angle. C'est peut-être un renvoi à 3, 50, 3.} Ἐχέτω ἡ λωπὰς, φησὶν, μνᾶν θείου ἀπύρου · καὶ ἐθαύμασα καὶ ἐν τούτῳ ὅτιπερ οὐκ ἀνεχομένη [ἢ] τοῦ φθόνου ἠξιώσας καὶ τοῦτο γραφῆναί σοι · κατέγνως μάτην τούτου φύσιν · οὐ γὰρ ἐνόησας τί\footnote{τούτου φύσιν] F. l. τοῦ φιλοσόφου. Cp. 3, 50, 3.} εἶπεν · καὶ ἐν τοῖς προτέροις ὑπομνήμασιν εἶπον ὅτι τῶν ὑδάτων ποίησιν οὐκ εἶπον, ἀλλ ᾽ ἄρσιν · ἕτερον γὰρ ποίησις καὶ ἕτερον ἄρ- (f. 187 r.) σις. Τὴν ἄρσιν < ἕκαστος > αὐτῶν εἶπεν ἀφθόνως · τὴν δὲ ποίησιν οὐδεὶς αὐτῶν ἐξέθετο · τοῦτο γὰρ ἦν τὸ ἐμφανὲς μυστήριον,\footnote{F. l. ἀφανὲς.} τουτέστιν τὸ σφόδρα κεκρυμμένον. Ἡ μὲν ἄρσις τοιάδε, ἡ διὰ τούτων τῶν ὀργάνων · ἡ δὲ ποίησις, ἤτοι σύνθεσις τούτου τοῦ ὕδατος, ἐν τῇ κατὰ πλάτος ἐκδόσει τοῦ ἔργου συγγέγραπται.\footnote{Cp. 3, 16, 10-12.}

5. Ἑξῆς καὶ τρίβικον συγγράψω. Ποίησον ἐκ χαλκοῦ ἐλατοῦ,\footnote{Cp. 3, 50, 1.} φησὶν, σωλῆνας τρεῖς · λεπτὸν τὸ ἔλασμα, ἐχέτω ἠθμοῦ πάχος ἢ\footnote{λεπτὸν] λίπανον MK. Corrigé d'après 3, 50, 1, leçon de B. --- ἰθμοῦ MK. F. l. σταθμοῦ, comme 3, 50, 1.} μικρὸν παχύτερον ὡσεὶ χαλκοῦ ἑνὸς ἥμισυ πάχος. Ποίησον οὖν\footnote{χαλκοῦ] F. l. χαλκείου \emph{vel} χαλκίου.} σωλῆνας τρεῖς τοιούτους, καὶ ποίησον χαλκεῖον μακρὸν πήχεως, ἔχον\footnote{ἔχων MK.} τὸ μῆκος παλαιστὴν, ἄνοιγμα δὲ τοῦ χαλκείου σύμμετρον · οἱ δὲ\footnote{μῆκος] F. l. βάθος (mot suppléé dans 3, 50, 1).} τρεῖς σωλῆνες ἔχοντες τὸ ἄνοιγμα, οἷον τράχηλον βίκου κούφου.\footnote{ἔχοντες F. l. ἔχουσι.} Ἱλαροῦντος δὲ ἀντίχειρας δύο εἶναι λιχανοὺς αὐταῖς ταῖς δυσὶ συναρηρότας\footnote{ἱλαροῦντος] Ce mot n'offre ici aucun sens. F. l. ἡλαρίῳ. --- Réd. proposée, d'après le texte de 3, 50, 1 : οἷον τράχηλον βίκου κούφου · ἡλαρίῳ δὲ τοὺς ἀντίχειρας δύο εἶναι λιχανοῖς αὐτοῦ τοῖς δυσὶ συναρηρότας ...} ἐκ πλευρῶν τοῦ χαλκείου περὶ τὸν πυθμένα · ἐν ᾧ πυθμένι τρεῖς τρώγλαι προσαρμόζουσαι τοῖς σωλῆσιν καὶ ἁρμοσθέντες προσκολλάσθωσαν, παραδόξως τοῦ ἄνωθεν πνεῦμα ἔχοντος · καὶ ἐπίθες τὸ\footnote{παραδόξοστοῦ M : παραδόξως τοῦ K. F. l. παραλόξως (mot supposé) τοῦ. (On connaît παραλοξαίνω).} χαλκεῖον ἐπάνω λωπάδος ὀστρακίνης, ἐχούσης τὸ θεῖον · συμπηλώσας τὰς συμβολὰς στέατι ἄρτου, ἔνθες ἐπὶ τὰ ἄκρα τῶν σωλήνων βίκους ὑελίνους μεγάλους, παχεῖς, ἵνα μὴ ῥαγῶσιν ἀπὸ τῆς θέρμης τοῦ ὕδατος. Καὶ κομίζου τὸ ἀναβαῖνον ἐν οἷς φάσκει ὁ φιλόσοφος αἴρεσθαι τὸ ὕδωρ.

6. Τὸ δὲ γίγνεσθαι ἢ συντίθεσθαι οὐκ ὀκνήσω σοι γράψαι, δέσποινα · ἔχει δὲ ἡ ποίησις τῶν ὑδάτων οὕτως. Ὕδωρ θείου, ἀρσενίκου,\footnote{Cp. 3, 25, 1.} σανδαράχης, νεφέλη, ὕδωρ φέκλης, ὕδωρ ἀσβέστου, ὕδωρ σποδοκράμβης, ὕδωρ στυπτηρίας, οὄρου, γάλακτος ὀνείου, αἰγείου · κυνὸς γάλα πολλάκις καὶ βόειον ἢ γυναικὸς ἀρσενοτόκου, κατὰ τὸν Ἀγαθοδαίμονα, καὶ ὄξος καὶ ὕδωρ θαλάσσιον καὶ μέλι, καὶ κίκινον ἢ γρὺ, καὶ οὖρον (f. 187 v.) ἄφθορον, καὶ κόμμι. Γίνεται δὲ οὕτως · ἕκαστον\footnote{ἀφθόρων MK.} ὕδωρ ὡς ἅλμη δικαία · ἐπὶ δὲ τῶν σποδῶν ὡς ἡ σαπωναρικὴ στάκτη, ἥντινα ἐν τοῖς γραφικοῖς τῶν χειροτμήτων σοι προσεφώνησα. Ἐὰν δὲ\footnote{F. l. χειροτμημάτων. (Cp. 3, 39, 3 ; 51, 1.) --- Ἐὰν δὲ ... Cp. 3, 16, 15.} μὴ δυνηθῇς συντιθέναι τῇ κοτύλῃ τοῦ ὕδατος εἴδους γ° αʹ, οἷον θείου γ° αʹ, ὕδατος καθαροῦ γ° αʹ, ἀρσενίκου γ° αʹ, ὕδατος κ° αʹ, δο ( ? ) γ° αʹ,\footnote{δο] C'est peut-être une altération du signe de la sandaraque, lequel dans BA ressemble à un Λ terminé par deux boucles. La confusion était possible dès le 11\textsuperscript{e} siècle.} ὕδατος κ°, φέκλης ὀπτῆς, ἀποσβεσθείσης εἰς ὄξος, ἀσβέστου ἀποσβεσθείσης εἰς οὐρόγαλον κ° αʹ, στυπτηρίας γ° λυθείσης εἰς ὕδωρ θαλάσσιον κ° αʹ, καὶ νίτρου πυρροῦ ὁμοίως · καὶ ἑψήσας ἰδίᾳ < καὶ > ὁμοῦ\footnote{πυροῦ MK.} τὰ ὕδατα ὀλίγον, ἵνα τὴν δύναμιν λάβῃ, ἀποσειρωσον ἢ ἀπόσταξον εἰς ἄλλην χύτραν, συνεμβάλλων τὸ μέλι καὶ τὸ ἔλαιον. Καὶ ἐὰν μὲν λευκοῦ θείου χρεία, συλλείου τῷ ὕδατι γὴν χείαν, ἀστερίτην, ἀφροσέληνον\footnote{M mg. : χρεία.} ὀπτὸν κοπτικὸν, σαμία, καρικὴ, κιμωλία ἢ στιλβάδα · καὶ βαλὼν εἰς χύτραν [καὶ] κυάνεον γενόμενον τὸ ὕδωρ · μάρμαρον ἐκ\footnote{γενάμενον MK, ici et partout.} τῆς γῆς βάλλε καὶ μύσι ὠμὸν, καὶ ἄλλο μέρος ἀσβέστου, ἵνα εἰς μέρη βʹ, κατὰ τὰς τῶν ἀρχαίων γραφὰς, ἵνα λέγηται τοῦτο τὸ δι ᾽ ἀσβέστου\footnote{ἵνα λέγεται M, leçon à retenir si l'on prend ἵνα dans le sens de \emph{où}. --- F. l. διάσβεστον (\emph{M. B.}).} · καὶ ἐπίθες τὸ ὄργανον τῇ χύτρᾳ, καὶ ἀνακόμιζε τὸ ὕδωρ, καὶ χρῶ.

7. Τὸ δὲ ξανθὸν ὕδωρ γίνεται οὕτως. Εἰς πάντα τὰ ὕδατα κατὰ τὴν συσταθμίαν τὴν πρὸς τὴν δεδηλωμένην οὐκέτι λαμβάνουσαν\footnote{πρὸς τὴν] F. l. πρόσθεν.} ἀσβέστου μέρη βʹ, ἀλὸς αʹ, καὶ ἀφεψήσασα ἓν ἕκαστον, καὶ συμμίξασα συλλείου, οὐκέτι γᾶς λευκὰς, ἀλλὰ ξανθὰς γᾶς · ξανθὸν γὰρ ὕδωρ βουλόμεθα. Αἱ δὲ γαῖ εἰσιν ὤχρα ἀττικὴ καὶ σίνωπις ποντικὴ καὶ\footnote{Cp. 3, 16, 4.} μὺσι ὀπτὸν, καὶ χάλκανθος ὀπτὴ, καὶ τὰ ὅμοια, βοτάναι πᾶσαι ἃς οἴδασι κοινῶς · καὶ λέκιθος, καὶ ὠῶν κρόκος, καὶ ἐλύδριον\footnote{λέκυθος MK.} (f. 188 r.) τὸ διπλοῦν. Τὰς μὲν πόας οὐ συνενοῖς τῷ ὕδατι, ἀλλὰ μόνον τὰς γᾶς. Καὶ μεταβάλλουσα ὡς ἔθος ἐστὶν λωπάδα, σύμβαλε τὰς βοτάνας, καὶ ἕψει τετράκις ἢ πεντάκις, ἐπιθεῖσα ἐν τῷ ὀργάνῳ, καὶ ἀνακόμιζε\footnote{ἔψει Δʹ ἡ Πρ, MK. Corr. conj. (\emph{M. B.}). --- F. l. ἕψει δʹ ἡ Μρ scil. ἡμέρας (\emph{C. E. R.})} τὸ ὕδωρ καὶ χρῶ μετὰ κόμμεως · καὶ ἀποσκεπάσασα, εὑρήσεις τὰς πόας κατακαείσας, ἀλλὰ καὶ ἀφιείσας τὸ ἴδιον βάμμα, ἤτοι τὸ ἴδιον\footnote{κατακαΐσας] MK. Corr. conj.} πνεῦμα · τούτου τοῦ ὕδατος τοῦ θείου τὸ ἄθικτον ἔχει δύναμιν καὶ φύσιν, ἐὰν ζεστῷ τῷ ὕδατι ἐπιβάψῃς ἄργυρον, ἔστω ἀνεξάλειπτον.\footnote{ἔστω] F. l. ἔσται.} Ἔρρωσο.

\bigskip
\centerline{\EightStarTaper}
\centerline{\EightStarTaper\EightStarTaper}
\bigskip

\subsubsection{3. --- 48. ΠΟΙΗΣΙΣ ΕΚ ΤΟΥΤΙΑΣ ΑΡΓΥΡΟΥ.}
\paragraph{}
\emph{Transcrit sur} M, f. 188 r. (\emph{main du} 15e-16\textsuperscript{e} \emph{siècle}.) --- \emph{Collationné sur} K, f. 96 r.

\bigskip

< Λαβὼν > τουτίας ΣΤ\textsuperscript{γ} κʹ, τρίψον ἕως ἂν γένηται χρυσός · καὶ θείου\footnote{ΣΤ\textsuperscript{γ} ] ῾ΣΤγ K, abréviation de ἑξάγιον, 6\textsuperscript{e} partie de l'once. Cp. du Cange, Glossarium infimæ græcitatis, et H. Estienne, Thesaurus, éd. Didot, \emph{voce} Ἑξάγιον.} ἀπύρου ΣΤ\textsuperscript{γ} εʹ, τρίψον ἕως ἂν γένηται μόλυβδος. Εἶτα ὠῶν ΣΤʹ λευκὰ λαβὼν, σμήξας, βάλε εἰς βικίον, καὶ ἕψει νυχθήμερα βʹ. Καὶ ἐκβαλὼν ἐὰν κόπτηται, αὖθις βαλὼν ἕψει ἡμέραν αʹ. Εἶτα λαβὼν χαλκοῦ ΣΤ\textsuperscript{γ} ιʹ, βάλε εἰς χώνην · καὶ ἐπίβαλε ἀπὸ τούτου κ° ΣΤʹ · καὶ γίνεται ἄργυρος.

\bigskip
\centerline{\EightStarTaper}
\centerline{\EightStarTaper\EightStarTaper}
\bigskip

\subsubsection{3. --- 49. ΤΟΥ ΑΥΤΟΥ ΖΩΣΙΜΟΥ ΠΕΡΙ ΟΡΓΑΝΩΝ ΚΑΙ ΚΑΜΙΝΩΝ ΓΝΗΣΙΑ ΥΠΟΜΝΗΜΑΤΑ ΠΕΡΙ ΤΟΥ Ω ΣΤΟΙΧΕΙΟΥ.}
\paragraph{}
\emph{Transcrit sur} M, f. 189 r. --- \emph{Collationné sur} K, f. 97 r. ;--- \emph{sur d'autres manuscrits à partir du} § 14 (\emph{νoir ci-après}).

\bigskip

1. Τὸ Ω στοιχεῖον στρογγύλον τὸ διμερὲς, τὸ ἀνῆκον τῇ ἑβδόμῃ\footnote{M mg. : ὁ λγ (λόγος ? ) μῦθος, d'une encre grise.} Κρόνου ζώνῃ, κατὰ τὴν ἔνσωμον φράσιν · κατὰ γὰρ τὴν ἀσώματον ἄλλο τί ἐστιν ἀνερμήνευτον. Ὁ μόνος Νικόθεος κεκρυμμένος οἶδεν\footnote{F. l. κεκρυμμένως.} · κατὰ δὲ τὴν ἔνσωμον τὸ λεγόμενον ὠκεανὸς, θεῶν, φησὶ, πάντων γένεσις καὶ σπορὰ, καθάπερ, φησὶν, αἱ μοναρχικαὶ τῆς ἐνσώμου φράσεως. Τὸ δὲ λεγόμενον μέγα καὶ θαυμαστὸν Ω στοιχεῖον περιέχει τὸν περὶ ὀργάνων ὕδατος θείου λόγον, καὶ καμίνων πασῶν μηχανικῶν [καὶ ἁπλῶν] καὶ ἁπλῶς πασῶν.

2. Ζώσιμος Θεοσεβείῃ ευηειαει. Αἱ καιρικαὶ καταβαφαὶ,\footnote{ευηειαει M ; εὐήει ἀεὶ K. F. l. χαίρειν ( ? ). Cp. 3, 51, 1. --- καιρικαὶ] κερικαὶ MK ; Cp. 3, 51, 1. Rapprocher aussi le § 11 du présent morceau.} ὦ γῦναι, εἰς χλευασμὸν ἐποίησαν τὴν περὶ καμίνων βίβλον. Πολλοὶ γὰρ εὐμένειαν ἐσχηκότες παρὰ τοῦ ἰδίου δαιμονίου, ἐπιτυγχάνειν τῶν καιρικῶν ἐχλεύασαν, καὶ τὴν περὶ καμίνων καὶ ὀργάνων βίβλον ὡς οὐκ οὖσαν ἀληθῆ. Καὶ οὐδεὶς λόγος αὐτοὺς ἀποδεικτικὸς ἔπεισεν ὅτι ἀλήθειά ἐστιν, εἰ μὴ αὐτὸς ὁ ἴδιος αὐτῶν δαίμων, κατὰ τοὺς χρόνους τῆς αὐτῶν εἱμαρμένης μεταβληθεὶς, παραλαβόντος αὐτοῦ, κακοποιοῦ δὲ εἰπεῖν · καὶ τῆς τέχνης καὶ τῆς εὐδαιμονίας αὐτῶν πάσης κωλυθείσης, καὶ ἐφ ᾽ ἑκάτερα τραπέντων τῶν αὐτῶν τύχῃ ῥημάτων, μόλις ἐκ τῶν ἐναργῶν τῆς εἱμαρμένης αὐτῶν ἀποδείξεων, ὡμολόγησαν εἶναί τι, καὶ μετ ῾ ἐκείνων ὧν πρότερον ἐφρόνουν. Ἀλλ ᾽ οἱ τοιοῦτοι οὐκ ἀποδεκτέοι οὔτε παρὰ Θεῷ οὔτε φιλοσόφοις ἀνθρώποις · πάλιν γὰρ τῶν χρόνων σχηματισθέντων κατὰ τοὺς (f. 189 v.) λεπτοὺς χρόνους, καλῶς\footnote{λεπτοὺς] F. l. ἐκλεκτοὺς.} καὶ τοῦ δαιμονίου σωματικῶς αὐτοὺς εὐεργετοῦντος, πάλιν μεταβάλλεται\footnote{F. l. μεταβάλλονται.} ἐφ ᾽ ἑτέραν ὁμολογίαν, τῶν προτέρων ἐναργῶν πραγμάτων πάντων λελησμένοι, πάντοτε τῇ εἱμαρμένῃ ἀκολουθοῦντες, καὶ εἰς τὰς λεγομένας\footnote{F. l. εἰς τὰ λεγόμενα.} καὶ εἰς τὰ ἐναντία, μηδὲν ἕτερον τῶν σωματικῶν φανταζόμενοι, ἀλλὰ τὴν εἱμαρμένην. Τοὺς τοιούτους δὲ ἀνθρώπους ὁ Ἑρμῆς ἐν τῷ περὶ φύσεων ἐκάλει ἄνοας, τῆς εἱμαρμένης μόνους ὄντας πομπὰς, μηδὲν\footnote{F. l. ἄνους. --- F. l. πομπέας.} τῶν ἀσωμάτων φανταζομένους, μήτε αὐτὴν τὴν εἱμαρμένην τοὺς αὐτοὺς ἄγουσαν δικαίως, ἀλλὰ τοὺς δυσφημοῦντας αὐτῆς τὰ σωματικὰ παιδευτήρια, καὶ τῶν εὐδαιμόνων αὐτῆς ἐκτὸς, ἄλλο φανταζομένους.

3. Ὁ δὲ Ἑρμῆς καὶ ὁ Ζωροάστρης τὸ φιλοσόφων γένος ἀνώτερον\footnote{Ζωροάστρις MK.} τῆς εἱμαρμένης εἶπον, τῷ μήτε τῇ εὐδαιμονίᾳ αὐτῆς χαίρειν, ἡδονῶν γὰρ κρατοῦσι, μήτε τοῖς κακοῖς αὐτῆς βάλλεσθαι, πάντοτε ἐναυλίαν ἄγοντες, μήτε τὰ καλὰ δῶρα παρ ᾽ αὐτῆς καταδεχόμενοι,\footnote{ἐναύλια K. F. l. ἡσυχίαν.} ἐπείπερ εἰς πέρας κακῶν βλέπουσιν. Διὰ τοῦτο καὶ ὁ Ἡσίοδος τὸν Προμηθέα εἰσάγει τῷ Ἐπιμηθεῖ παραγγέλλοντα · τίνα οἴονται οἱ ἄνθρωποι πασῶν μείζονα εὐδαιμονίαν ; γυναῖκα εὔμορφον, φησὶ, σὺν πλούτῳ πολλῷ, καί φησι · μήτε δῶρον δέξασθαι παρὰ Ζηνὸς\footnote{μήτε] Lire μήποτε ( ? ) comme dans Hésiode, Op. et D. 86.} Ὀλυμπίου, ἀλλ ᾽ ἀποπέμπειν ἐξοπίσω, διδάσκων τὸν ἴδιον ἀδελφὸν διὰ φιλοσοφίας ἀποπέμπειν τὰ τοῦ Διὸς, τουτέστι τῆς εἱμαρμένης δῶρα.

4. (f. 190 r.) Ζωροάστρης δὲ εἰδήσει τῶν ἄνω πάντων καὶ μαγείᾳ αὐχῶν, τῆς ἐνσώμου φράσεως φάσκει ἀποστρέφεσθαι πάντα τῆς εἱμαρμένης τὰ κακὰ, καὶ μερικὰ καὶ καθολικά. Ὁ μέντοι Ἑρμῆς ἐν τῷ περὶ ἀναυλίας διαβάλλει καὶ τὴν μαγείαν, λέγων ὅτι οὐ δεῖ τὸν πνευματικὸν\footnote{F. l. π. ἀναυδίας. Un des livres hermétiques est intitulé περὶ σιγῆς.} ἄνθρωπον τὸν ἐπιγνῶντα ἑαυτὸν, οὔτε διὰ μαγείας καθορθοῦν τι, ἐὰν καὶ καλὸν νομίζηται, μηδὲ βιάζεσθαι τὴν ἀνάγκην, ἀλλ ᾽ ἐᾶν ὡς ἔχει φύσεως καὶ κρίσεως · πορεύεσθαι δὲ διὰ μόνου τοῦ ζητεῖν, ἑαυτὸν καὶ θεὸν ἐπιγνῶντα, κρατεῖν τὴν ἀκατονόμαστον τριάδα · καὶ ἐᾶν τὴν εἱμαρμένην ὃ θέλει ποιεῖν, τῷ ἐᾶν τῇ σπηλῷ,\footnote{θέλειν MK Corr. conj. --- τῇ σπηλῷ] F. l. τῷ πηλῷ (\emph{M. B.}).} τουτέστιν τῷ σώματι. Καὶ οὕτως φησί · « Νοήσας καὶ πολιτευσάμενος θεάσῃ τὸν Θεοῦ υἱὸν, πάντα γινόμενον τῶν ὁσίων ψυχῶν ἕνεκεν · ἵνα αὐτὴν ἐκσπάσῃ ἐκ τοῦ χώρου τῆς εἱμαρμένης ἐπὶ τὸν ἀσώματον, ὅρα αὐτὸν γινόμενον πάντα, θεὸν, ἄγγελον, ἄνθρωπον παθητόν · πάντα γὰρ δυνάμενος πάντα ὅσα θέλει γίνεται, καὶ πατρὶ ὑπακούει διὰ παντὸς σώματος διήκων, φωτίζων τὸν ἑκάστης νοῦν, εἰς\footnote{F. l. ἑκάστου.} τὸν εὐδαίμονα χῶρον ἀνώρμησεν, ὅπουπερ ἦν καὶ πρὸ τοῦ τὸ σωματικὸν\footnote{πρὸ τοῦτο MK. Corr. conj.} γενέσθαι, αὐτῷ ἀκολουθοῦντα καὶ ὑπ ᾽ αὐτοῦ ὀρεγόμενον καὶ ὁδηγούμενον εἰς ἐκεῖνο τὸ φῶς.

5. Καὶ βλέψαι τὸν πίνακα ὃν Κέβητος γράψας, καὶ ὁ τρίσμεγας\footnote{καὶ βιτος MK. F. l. Kέβης τε ἔγραψε.} Πλάτων, καὶ ὁ μυριόμεγας Ἑρμῆς, ὅτι Θώυθος ἑρμηνεύεται τῇ ἱερατικῇ πρώτῃ φωνῇ, ὁ πρῶτος ἄνθρωπος ἑρμηνεὺς πάντων τῶν ὄντων, καὶ ὀνοματοπο- (f. 190 v.) ιὸς πάντων τῶν σωματικῶν. Οἱ δὲ Χαλδαῖοι καὶ Πάρθοι καὶ Μῆδοι καὶ Ἑβραῖοι καλοῦσιν αὐτὸν Ἀδὰμ, ᾧ ἐστιν ἑρμηνεία γῆ παρθένος, καὶ γῆ\footnote{Cp. Olympiodore (2, 4, 32).} αἱματώδης, καὶ γῆ πυρά, καὶ γῆ σαρκίνη. Ταῦτα δὲ ἐν ταῖς βιβλιοθήκαις\footnote{F. l. πυρρὰ.} τῶν Πτολεμαίων ηὕρηνται · ὃν ἀπέθεντο εἰς ἕκαστον ἱερὸν, μάλιστα τῷ Σαραπείῳ, ὅτε παρεκάλεσεν Ἀσενὰν τῶν ἀρχιεροσολύμων\footnote{ἀσεναν M. --- F. l. άρχιερέα Σολύμων.} πέμψαντα Ἑρμῆν ὃς εἱρμηνεύσε πᾶσαν τὴν Ἑβραΐδα ἑλληνιστὶ\footnote{ἑρμήνευσε M. F. l. ὁ ἑρμηνεύσας.} καὶ αἰγυπτιστί.

6. Οὕτως οὖν καλεῖται ὁ πρῶτος ἄνθρωπος ὁ παρ ᾽ ἡμῖν Θωΰθ, καὶ παρ ᾽ ἐκείνοις Ἀδὰμ, τῇ τῶν ἀγγέλων φωνῇ αὐτὸν καλέσαντες.\footnote{F. l. καλέσασι.} Οὐ μὴν δὲ ἀλλὰ καὶ συμβολικῶς διὰ τεσσάρων στοιχείων ἐκ πάσης τῆς σφαίρας αὐτὸν εἰπόντες κατὰ τὸ σῶμα. Τὸ γὰρ ἄλφα αὐτοῦ στοιχεῖον ἀνατολὴν δηλοῖ, τὸν ἀέρα · τὸ δὲ δέλτα αὐτοῦ στοιχεῖον δύσιν δηλοῖ τὴν κάτω καταδύσασαν διὰ τὸ βάρος · τὸ δὲ Μ στοιχεῖον μεσημβρίαν δηλοῖ, τὸ μέσον τούτων τῶν σωμάτων πεπαντικὸν πῦρ τὸ εἰς τὴν μέσην τετάρτην ζώνην. Οὕτως οὖν ὁ σάρκινος Ἀδὰμ κατὰ τὴν φαινομένην περίπλασιν Θωῢθ καλεῖται · ὁ δὲ ἔσω αὐτοῦ ἄνθρωπος ὁ πνευματικὸς, < ὄνι > καὶ κύρομα ἐχειον καὶ προσηγορικόν. Τὸ μὲν οὖν κύριον ἀγνοῶν διὰ τὸ τέως · μόνος γὰρ Νικόθεος ὁ ἀνεύρετος ταῦτα\footnote{F. l. ἀγνοοῦμεν εἰς τὸ τέως.} οἶδεν · τὸ δὲ προσηγορικὸν αὐτοῦ ὄνομα φῶς καλεῖται, ἀφ ᾽ οὗ καὶ φῶτας παρηκολούθησε λέγεσθαι τοὺς ἀνθρώπους.

7. Ὅτε ἦν φῶς ἐν τῷ Παραδείσῳ διαπνεόμενος ὑπὸ τῆς εἱμαρμένης, ἔπεισαν αὐτὸν ὡς ἄκακον καὶ ἀνενέργητον (f. 191 r.) ἐνδύσασθαι τὸν παρ ᾽ αὐτοῦ Ἀδὰμ, τὸν ἐκ τῆς εἱμαρμένης, τὸν ἐκ τῶν τεσσάρων στοιχείων. Ὁ δὲ διὰ τὸ ἄκακον οὐκ ἀπεστράφη. Εἰ δὲ ἐκαυχῶντο ὡς δεδουλαγωγημένου αὐτοῦ τὸν ἔξω ἄνθρωπον, δεσμὸν εἶπεν ὁ Ἡσίοδος, ὃν ἔδησεν ὁ Ζεὺς τὸν Προμηθέα. Εἶτα μετὰ\footnote{Cp. Hésiode, Théogonie, vers 521.} < τοῦτον > τὸν δεσμὸν, ἄλλον αὐτῷ δεσμὸν ἐπιπέμπει τὴν. Πανδώρην ἣν οἱ Ἑβραῖοι καλοῦσιν Εὔαν. Ὁ γὰρ Προμηθεὺς καὶ Ἐπιμηθεὺς\footnote{γὰρ] F. l. δὲ.} εἷς ἄνθρωπός ἐστι κατὰ τὸν ἀλληγορικὸν λόγον, τουτέστι ψυχὴ καὶ σῶμα. Καὶ ποτὶ μὲν ψυχῆς ἔχει εἰκόνα ὁ Προμηθεὺς, ποτὲ δὲ νοός, ποτὲ δὲ σαρκὸς, διὰ τὴν παρακοὴν τοῦ Ἐπιμηθέως ἣν παρήκουσεν τοῦ Προμηθέως τοῦ ἰδίου < ἀδελφοῦ > · φησὶ γὰρ ὁ νοῦς ἡμῶν · ὁ δὲ υἱὸς τοῦ Θεοῦ πάντα δυνάμενος, καὶ πάντα γινόμενος, ὅτε θέλει, ὠς θέλει φαίνει ἑκάστῳ · Ἀδὰμ προσῆν Ἰησοῦς Χριστὸς < ὃς > ἀνήνεγκεν,\footnote{F. l. πρώην Ἰ. Χ. ἀνήνεγκε.} ὅπου καὶ τὸ πρότερον διῆγον φῶτες καλούμενοι.

8. Ἐφάνη δὲ καὶ τοῖς πάνυ ἀδυνάτοις ἀνθρώποις, ἄνθρωπος γεγονὼς παθητὸς καὶ ῥαπιζόμενος, καὶ λάθρα τοὺς ἰδίους φῶτας συλήσας, ἅτε μηδὲν παθῶν, τὸν δὲ θάνατον δείξας καταπατεῖσθαι, καὶ ἐῶσθαι καὶ ἕως\footnote{δείξας] F. l. δόξας.} ἄρτι καὶ τοῦ τέλους τοῦ κόσμου τόποισι λάθρα, καὶ φανερὰ συλλῶν τοῖς\footnote{τόποις ἰλάθρα M. --- συλλῶν] E. l. συλλαλῶν.} ἑαυτοῦ, συμβουλεύων αὐτοῖς λάθρα καὶ διὰ τοῦ νοὸς αὐτῶν καταλλαγὴν ἔχειν τοῦ παρ ᾽ αὐτῶν Ἀδὰμ, κοπτομένου καὶ φονευομένου παρ ᾽ αὐτῶν τυφληγοροῦντος καὶ διαζηλουμένου τῷ πνευματικῷ καὶ φωτεινῷ ἀνθρώπῳ, τὸν ἑαυτῶν Ἀδὰμ ἀποκτείνουσι.

9. Ταῦτα δὲ γίνεται ἕως οὗ ἔλθῃ ὁ ἀντίμιμος δαίμων, δι ᾽ οὗ ζηλούμενος αὐτοῖς καὶ θέλων ὡς τὸ πρώην πλανῆσαι λέ- (f. 191 v.) γων ἑαυτὸν υἱὸν Θεοῦ, ἄμορφος ὢν καὶ ψυχῇ καὶ σώματι. Οἱ δὲ φρονιμώτεροι\footnote{φρονιμώτερον γενάμενοι MK.} γενόμενοι ἐκ τῆς καταλήψεως τοῦ ὄντως υἱοῦ τοῦ Θεοῦ, δίδουσιν αὐτῷ\footnote{δίδωσιν MK.} τὸν ἴδιον Ἀδὰμ εἰς φόνον τὰ ἑαυτῶν φωτεινὰ πνεύματα, σώζοντες ἴδιον χῶρον ὅπουπερ καὶ πρὸ κόσμου ἦσαν. Πρὶν ἢ δὲ ταῦτα τολμῆσαι, τὸν\footnote{πρινὴ K (forme plus moderne).} ἀντίμιμον, τὸν ζηλωτὴν, πρῶτον ἀποστέλλει αὑτοῦ πρόδρομον ἀπὸ τῆς Περσίδος, μυθοπλάνους λόγους λαλοῦντα, καὶ περὶ τὴν εἱμαρμένην ἄγοντα τοῦς ἀνθρώπους. Εἰσὶ δὲ τὰ στοιχεῖα τοῦ ὀνόματος αὐτοῦ ἐννέα,\footnote{M mg. : σηʹ, 1\textsuperscript{re} main. --- Le mot de neuf lettres ne serait-il pas φαοσφόρος (Lucifer, prince des démons « la diphthongue (αο) étant conservée ? » (Voir la note de la traduction.)} τῆς διφθόγγου σωζομένης, κατὰ τὸν τῆς εἱμαρμένης ὅρον. Εἶτα μετὰ περιόδους πλέον ἢ ἔλαττον ἑπτὰ, καὶ αὐτὸς ἑαυτῷ φύσει ἐλεύσεται.\footnote{περιόδου MK. Corr. conj. --- F. l. ἑαυτοῦ}

10. Καὶ ταῦτα μόνοι Ἑβραῖοι καὶ αἱ ἱεραὶ Ἑρμοῦ βίβλοι περὶ τοῦ φωτεινοῦ ἀνθρώπου καὶ τοῦ ὁδηγοῦ αὐτοῦ υἰοῦ Θεοῦ, καὶ τοῦ γηΐνου Ἀδὰμ, καὶ τοῦ ὁδηγοῦ αὐτοῦ ἀντιμίμου τοῦ δυσφημίᾳ λέγοντος ἑαυτὸν\footnote{λέγωντος MK.} εἶναι υἱὸν Θεοῦ πλάνῃ. Οἱ δὲ Ἕλληνες καλοῦσιν γῆϊον Ἀδὰμ Ἐπιμηθέα συμβουλευόμενον ὑπὸ τοῦ ἰδίου νοῦ, τουτέστι τοῦ ἀδελφοῦ\footnote{Cp. Hésiode, Op. et D., l. c.} αὐτοῦ μὴ λαβεῖν τὰ δῶρα τοῦ Διός. Ὅμως καὶ σφαλεὶς καὶ μετανοήσας καὶ τὸν εὐδαίμονα χῶρον ζητήσας, πάντα ἑρμηνεύει καὶ πάντα συμβουλεύει τοῖς ἔχουσιν ἀκοὰς νοεράς · οἱ δὲ τὰς σωματικὰς ἔχοντες μόνον ἀκοὰς τῆς εἱμαρμένης εἰσὶ, μηδὲν ἄλλο καταδεχόμενοι ἢ ὁμολογοῦντες.

11. Ὅσοι τὰς καιρικὰς < ποιοῦσι καταβαφὰς > εὐτυχοῦντες οὐδὲν\footnote{F. suppl. ὅσοι < δὲ. > --- Guillemets dans M jusqu'à la ligne contenant ἀνθρώποισι.} ἕτερον λέγουσι, τῆς τέχνης χλευάζοντες, ἢ τὴν μεγάλην περὶ καμίνων βίβλον · καὶ (φ. 192 ρ.) οὐδὲ τὸν ποιητὴν κατανοοῦσι λέγοντα ·
\begin{quotation}
ἀλλ ᾽ οὔπως ἅμα θεοὶ δόσαν ἀνθρώποισι\footnote{On ne retrouve ce fragment de vers ni dans Homère ni dans Hésiode.}
\end{quotation}
\paragraph{}
καὶ τὰ ἑξῆς. Καὶ οὐδὲν ἐνθυμοῦνται οὔτε βλέπουσι τὰς τῶν ἀνθρώπων διαγωγὰς, ὅτι καὶ εἰς μίαν τέχνην ἄνθρωποι διαφόρως εὐτυχοῦσι, καὶ διαφόρως τὴν μίαν τέχνην ἐργάζονται, διὰ τὰ ἤθη καὶ διάφορα σχήματα τῶν ἀστέρων μίαν τέχνην ποιεῖν. Καὶ τὸν μὲν ἄγων τεχνίτην,\footnote{ἄγων] F. l. ἀργὸν.} τὸν δὲ μόνον τεχνίτην, τὸν δὲ ὑποβεβηκότα, τὸν δὲ χείρονα, < τὸν δ ᾽ > ἀπρόκοπον, οὕτως ἐστὶν < εὑρεῖν > ἐπὶ πασῶν τῶν τεχνῶν καὶ διαφόροις ἐργαλείοις καὶ ἀγωγαῖς τὴν αὐτὴν τέχνην ἐργαζομένους καὶ διαφόρους\footnote{F. l. διαφόρως.} ἔχοντας τὸ νοερὸν καὶ ἐπιτευκτικόν.

12. Καὶ μάλιστα ὑπὲρ πάσας τὰς τέχνας, ἐν τῇ ἱερατικῇ ταῦτά ἐστι θεωρῆσαι. Φέρε εἰπεῖν κατεαγότος ὀστέου, ἐὰν εὑρεθῇ ἱερεὺς ὃς τόδε διὰ τῆς ἰδίας δεισιδαιμονίας ποιῶν, κολλᾷ τὸ ὀστοῦν, ὥστε καὶ τρισμόν ἀκοῦσαι συνερχομένων εἰς ἄλληλα τῶν ὀστέων. Ἐὰν δὲ μὴ εὑρεθῇ ἱερεὺς, οὐ μὴ φοβηθῇ ἄνθρωπος ἀποθανεῖν, ἀλλὰ φέρωνται\footnote{φέρονται MK. Corr. conj.} ἰατροὶ ἔχοντες βίβλους κατὰ ζωγράφους γραμμικὰς σκιαστὰς ἐχούσας γραμμάς · καὶ ὁσαιδηποτοῦν εἰσι γραμμαὶ, καὶ ἀπὸ βιβλίου περιδεσμεῖται ὁ ἄνθρωπος μηχανικῶς καὶ ζῇ χρόνον < τινὰ, > τὴν ὑγείαν πορισάμενος · καὶ οὐδήπου ἐφίεται ἄνθρωπος ἀποθανεῖν διὰ τὸ μὴ εὑρηκέναι ἱερέα ὀστοδέτην. Οὗτοι δὲ ἀποτυχόντες τῷ λιμῷ τελευτῶσι μὴ καταξιοῦντες τὴν ὀστοδητικὴν τῶν καμίνων διαγραφὴν νοῆσαι καὶ ποιῆσαι, ἵνα μακάριοι γενόμενοι νικήσωσι πενίαν, τὴν ἀνίατον νόσον. Καὶ ταῦτα μὲν ἐπὶ τοσοῦτον.

13. Ἐγὼ δὲ ἐπὶ (f. 192 v.) τὸ προκείμενον ἐλεύσομαι, ὡς ἔστι\footnote{F. l. ὅ ἐστι.} περὶ ὀργάνων. Λαβὼν γάρ σου τὰς ἐπιστολὰς ἃς ἔγραψας, εὗρόν σε παρακαλοῦσαν ὅπως καὶ τὴν τῶν ὀργάνων ἔκδοσίν σοι συγγράψω. Ἐθαύμασα δέ σε ὅτιπερ καὶ τὰ μὴ ὀφείλοντα συγγράφεις τυχεῖν παρ ᾽ ἐμοῦ, ἢ οὐκ ἤκουσας τοῦ φιλοσόφου λέγοντος ὅτι « ταῦτα ἐκὼν παρεσιώπησα διὰ τὸ ἀφθόνως αὐτὰ ἐγκεῖσθαι καὶ ἐν ταῖς ἄλλαις μου γραφαῖς. Σὺ δὲ παρ ᾽ ἐμοῦ ταῦτα μαθεῖν ἠβουλήθης · ἀλλὰ μὴ οἴου ἀξιοπιστότερον ἐμὶ τῶν ἀρχαίων ξυγγράψαι. Γίνωσκε ὡς οὐκ ἂν δυναίμην. Ἀλλ ᾽ ἵνα καὶ πάντα τὰ παρ ᾽ ἐκείνων λαληθέντα νοήσωμεν τοίνυν τὰ παρ ᾽ ἐκείνων σοι ὑποθήσω. Ἔχει δὲ οὕτως.
\begin{center}
\emph{Les paragraphes suiνants} (14-fin) \emph{ont été collationnés sur} B, f. 82 v. ;--- \emph{sur} C, f. 56 r. ;--- \emph{sur} A, f. 80 v. (= A ou A\textsuperscript{1}) ;--- \emph{sur} A, f. 220 r. (= A\textsuperscript{2}) ;--- \emph{sur} K, (\emph{continuation du texte précédent}).
\end{center}
\paragraph{}
14. Βίκος ὑέλεος, σωλὴν ὀστράκινος μῆκος πήχεως ἑνός. Λωπὰς\footnote{ὕελος M ; ὑάλινος BCA\textsuperscript{1. 2.} (= B etc.). Corr. conj.} ἢ ἄγγος στενόστομον ἐν ᾧ ἢ τῷ σωλῆνι τὸ πάχος βικίῳ τῷ στόματι\footnote{ἐν ᾧ --- αὐτοῦ om. B etc. --- ἢ τῷ] F. l. ἤτοι.} αὐτοῦ. Ὁ δὲ τύπος < οὗτος. > Ἔχειν δὲ δεῖ ἐπίλιθον κρατηρίαν\footnote{La figure annoncée manque. --- κρατηρίαν] F. l. κρατῆρα ou κρατήριον.} ὕδατος, καὶ παραψᾶν σπόγγῳ τὸ ἄγγος, καὶ ἐπὶ τῶν αἰθαλῶν καὶ τῆς ὑδραργύρου τὸ αὐτὸ. Ἔξεστι δὲ ἐν τῷ φανῷ καὶ τοῖς ὁμοίοις ὀργάνοις ἔχουσιν ἐγκάθισμα ὡσεὶ δρακοντῶδες πήσσειν τὴν ὑδράργυρον, καὶ ξανθὴν αὐτὴν καθιστᾶν διὰ τῆς τοῦ θείου ἀναθυμιάσεως, τῶν ἀρχαίων γραφῶν τοῦτο παρεγγυουσῶν. Ἀμοιροῦντος μὲν τοῦ\footnote{τοῦ μὲν φανοῦ ἀμ. puis le signe de Κρόνος ou du plomb B etc.} φανοῦ Κρόνον, καὶ ἐπιθαυμάσεις ἐπὶ ταύτῃ τῇ γραφῇ ὅτι δύο μυστήρια\footnote{ἐπιθαυμ.] θαυμάσεις BCA\textsuperscript{1} ; θαυμάσης A\textsuperscript{2}.} ἐν αὐτῇ ἐκρύβη φανερὰ, καὶ οὐ ζητοῦμεν [ὅτι] πῶς ἡ τοῦ θείου\footnote{ἐκρύβησαν BC ; ἐκρίθησαν A\textsuperscript{1. 2.}.} αἰθάλη λευκαίνουσα τὴν ὑδράργυρον ξανθὴν ἀναδείκνυσιν · μήτι γε\footnote{μή τοι γε B etc.} καυθείσης αὐτῆς ἐστι τοῦτο · ἔτι δὲ καὶ αὐτὴ λευκὴ οὖσα καὶ δυνάμει καὶ ἐνεργείᾳ ὑπὸ λευκοῦ καιομένη καὶ πηγνυμένη, ὅπως ξανθὴ ἔρχεται.\footnote{ὅπως] B etc. --- ἔρχεται] ἀποκαθίσταται B etc. F. l. ἐξέρχεται.}

15. Ἔδει τοίνυν τοὺς νέους πρό (f. 193 r.) γε πάντων ταῦτα ζητεῖν. Τὸ δὲ ἕτερον μυστήριον οἶμαι μὴ μόνην αὐτὴν πήγνυσθαι, ἀλλὰ καὶ μεθ ᾽ ὅλου τοῦ συνθέματος. Τὰ μέντοι ὄργανα εἰς ἃ γίνεται καὶ ὕδωρ θείου ἄθικτον, καὶ πῆξις ὑδραργύρου, καὶ μαλαγμάτων ποτίσεις, καὶ βαφὴ μαλαγμάτων, ἐστὶ ταῦτα.
\begin{center}
(\emph{Suit la formule de l'Écreνisse}. --- \emph{Voir l'Introduction de M. Berthelot}, p. 152, fig. 28).
\end{center}
\paragraph{}
16. Ὅτι ἀπὸ ἀσκιάστου χαλκοῦ ἰὸς γενόμενος ξανθωθεὶς αἰθαλοῦται · καὶ ἀποτίθεται ἐν μέλιτι λευκῷ.

17. Ὅτι καὶ τὸ μάλαγμα τὸ ἀπὸ τοῦ ἡμετέρου χαλκοῦ ξανθωθὲν ποιεῖ ἀντ ᾽ αὐτοῦ       ἧττον δέ · ὅλα δὲ αὐτὰ κεῖται παρὰ\footnote{Après αὐτοῦ] espace blanc pour 5 ou 6 lettres M seul. F. l. ὁμοίως.} Ἀγαθοδαίμονι.

18. Ὅτι καὶ τὸ μάλαγμα τὸ διὰ σκωριδίου βάλε ἐμφανῶς, καὶ\footnote{ἐμφανῶς] F. l. ἐν φανῷ.} πῆξον τῇ αἰθάλῃ τῶν θείων τῶν ἀναθυμιωμένων, ἵνα γένηται ὡς κιννάβαρις. Εἶτα βαλὼν εἰς βούκλας ἢ ληκύθια καὶ ἐκτείνας, χρῶ ὡς ἔχει ὀπίσω.\footnote{Entre nos §§ 18 et 19, les manuscrits donnent les signes du ciel, du soleil (ou de l'or), de la terre, du ciel. Les mêmes signes sont répétés dans B, au-dessus de ὅλα τὰ εἴδη.}

19. Ὡς φαίνεται οὖν, ὅλα τὰ εἴδη τὰ ἐξ αἰθαλῶν ὁ Ἀγαθοδαίμων,\footnote{γοῦν B etc.} οἷον χρυσόκολλαν, καὶ ἐτήσιον, καὶ χρυσάνθιον, καὶ ἁπλῶς\footnote{χρυσόκολλα M.} πάντα εἰς τὴν καταβαφὴν τοῦ ἀργύρου κέκραται, ὡς ἔχει αὐτοῦ ἡ\footnote{κραταιῶς ἔχει mss. Corr. conj.} ὑστέρα τάξις. Αἰθάλας δὲ βάλλει, ἵνα μὴ σκωριάσῃ ὁ ἄργυρος, ἢ ἀπουσιάσῃ τῶν παχέων σωμάτων καὶ γεωδεστέρων εἰωθότων καίεσθαι καὶ φρύγεσθαι.

\bigskip
\centerline{\EightStarTaper}
\centerline{\EightStarTaper\EightStarTaper}
\bigskip

\subsubsection{3. --- 50. ΠΕΡΙ ΤΟΥ ΤΡΙΓΒΙΚΟΥ ΚΑΙ ΤΟΥ ΣΩΛΗΝΟΣ.}
\paragraph{}
\emph{Transcrit sur} M, f. 194 r. --- \emph{Collationné sur} B, f. 83 v. ;--- \emph{sur} C, f. 57 r. ;--- \emph{sur} A, f. 81 r. (= A ou A\textsuperscript{1}) ;--- \emph{sur} A, f. 221 r. (= A\textsuperscript{2}) ;--- \emph{sur} K. f. 101 r.

\bigskip

1. Ἑξῆς δὲ τὸν τρίβικόν σοι ὑπογράψω. Καλεῖται δὲ αὕτη ἡ δι ᾽ ἀσκοῦ\footnote{δι ᾽ ἀσκοῦ] F. l. διὰ χαλκοῦ.} ἡ παρὰ Μαρίας τεχνοπαράδοτος · ἔχει δὲ οὕτως. « Ποίησον,\footnote{τεχνοπαραδότου] Cette leçon, commune aux divers mss. consultés, confirme la correction proposée ci-dessus, p. 138, l. 20. --- Ποίησον] Cp. 3, 47 (= $\star$) § 5.} φησιν, ἐκ χαλκοῦ ἐλατοῦ σωλῆνας τρεῖς, λεπτὸν τὸ ἔλασμα ἔχοντας\footnote{λεπτὸν] λεῖπον MK. --- ἔχοντας] ἔχων MK.} σταθμοῦ πάχος σμικρὸν παχύτερον ὡσεὶ χαλκοῦ τηγάνου πλακουντηρίου, μῆκος ἔχον πήχεος αʹ$\svgB$ʹ. Ποίησον οὖν σωλῆνας τρεῖς τοιούτους,\footnote{μῆκος πηχῶν αʹ $\svgB$$^{\prime\prime}$, ποίησον τ. σωλ. BC ; μῖκος πἤχος αʹ $\svgB$ʹ, ποίησον τ. σωλ. A\textsuperscript{1. 2.}.} καὶ ποίησον πάχος ἔχον τὸ μῆκος παρὰ παλαιστὴν, ἄνοιγμα δὲ τοῦ\footnote{πάχος] χαλκεῖον $\star$, f. mel. --- ἔχειν BC, f. mel. ; ἔχει A\textsuperscript{1. 2.} --- παρὰ] F. l. περὶ (environ).} χαλκείου σύμμετρον. Οἱ δὲ τρεῖς σωλῆνες ἐχέτωσαν τὸ ἄνοιγμα τραχήλου βίκου κούφου ἡλάριον, τοῦ δὲ ἀντίχειρας, ἵνα δύο λιχανοὺς\footnote{τράχηλον $\star$, f. mel. --- βίκου] λιβυκοῦ mss. Corr. d'après $\star$. --- F. l. ἡλαρίῳ δὲ τοὺς ἀντίχειρας. --- λιβάνου mss. Corr. d'après $\star$.} αὐτοῦ ταῖς δυσὶν χερσὶν συναρηρότας ἐκ πλευρῶν. Τοῦ δὲ χαλκείου\footnote{F. l. : ... ἐκ πλευρῶν τοῦδε < τοῦ > χαλκείου (leçon de $\star$).} περὶ τὸν πυθμένα, αἱ τρεῖς τρῶγλαι προσαρμόζουσαι τοῖς σωλῆσι, καὶ\footnote{τρῶγλαι] γλῶσσαι B etc.} ἁρμοσθέντες προσκολλάσθωσαν, τοῦ ἄνω παραδόξως πνεῦμα ἔχοντος.\footnote{F. l. παραλόξως (mot supposé) ; on connaît παραλοξαίνω.} Καὶ ἐπιθεὶς τὸ χαλκεῖον ἐπάνω λωπάδος ὀστρακίνης ἐχούσης τὸ θεῖον, συμπεριπηλώσας τὰς συμβολὰς στέατι ἄρτου, ἔνθες ἐπὶ τὰ ἄκρα τῶν σωλήνων βίκους ὑελοῦς μεγάλους, παχεῖς, ἵνα μὴ ῥαγῶσιν ἀπὸ\footnote{ὑέλους MK ; ὑελίνους BC ; ὑαλίνους A\textsuperscript{1. 2.}. Corr. conj.} τῆς θέρμης τοῦ ὕδατος κομιζούσης ἀνὰ μέσον. Τὸ δὲ σχῆμα τοῦτο.\footnote{ἀνὰ μέσον] τὸ ἀναβαῖνον B etc.} Λιχανὸς σωλήν.\footnote{Figure. --- Pour l'indication des figures, voir dans la traduction française les renvois à l'Introduction de M. Berthelot.}

2. Ἔστι δὲ καὶ ἄλλος τρόπος κομιδῆς ὕδατος θείου, ἀλλ ᾽ οὐχ ὡς ὁ τρίβικος. Ἔστω σωλὴν εἰς πυθμένα χαλκείου ἐντεθειμένος, μῆκος πήχεως αʹ$\svgB$ʹ. Τῷ αὐτῷ τρόπῳ καὶ βίκος εἷς · καὶ ὑποκάτω λωπὰς\footnote{Cp. 3, 47, 2.} θείου ἀπύρου, εἰς ἣν συναρμόζει τὸ χαλκεῖον καὶ περιπηλοῖ στέατι\footnote{F. l. συναρμόζεις \emph{et} περιπηλοῖς, \emph{vel} περιπήλου.} ἢ κηρῷ, ἢ πηλῷ, ἢ ὡς βούλει · καὶ καύσας, ἀνάσπα. Ὁ δὲ τύπος οὗτος.\footnote{οὕτως MK ; οἱ δὲ τύποι οὗτοι B etc. Corr. conj. --- Figure (M, f. 194 v.).}

3. (f. 195 r.). Ἐγέλασά σοι καὶ εἰς ἐξάκουστον ἐν ταῖς τάξεσι τῶν\footnote{Les mss. MK continuent seuls. Cp. 3, 47, 4.} ὀργάνων τούτων. Φησὶ γὰρ · « Εἰς ἑκάστην ἐχέτω ἡ λωπὰς μνᾶν θείου ἀπύρου. » Καὶ ἐθαύμασά σε καὶ ἐν τούτῳ ὅτιπερ οὐκ ἀνασχομένη τοῦ φθόνου ἠξίωσας καὶ ταῦτα γραφῆναί σοι. Τάχα δὲ καὶ εἰς κατάγνωσιν ἧκες τοῦ φιλοσόφου, ὅτιπερ ἐτόλμησεν εἰπεῖν ὅτι · « Ταῦτα ἑκὼν παρεσιώπησα διὰ τὸ ἀφθόνως αὐτὰ κεῖσθαι ἐν ταῖς ἄλλων γραφαῖς ... στέατι,\footnote{Espace blanc avant στέατι. F. suppl. ἄρτου \emph{vel} περιπήλου. Cp. p. précédente l. 14 et ci-dessus, l. 3.} ἢ κηρῷ, ἢ πηλῷ, ἢ ὡς βούλει, καὶ καύσας, ἀνάσπα. Ὁ δὲ τύπος οὗτος ἐν\footnote{οὕτως MK.} γραφαῖς. Καὶ ἐνκύψασα εἰς ἀκάματον φθόνον, κατέγνως τοῦ φιλοσόφου μάτην. Οὐ γὰρ ἐνόησας τί εἶπεν. Οὐκ εἶπεν γὰρ, ὡς καὶ ἐν τοῖς πρότερον\footnote{τοῖς] ταῖς M.} ὐπομνήμασιν, ὅτι « τῶν ὑδάτων ἡ ποίησις, » ἀλλὰ « ἠ ἄρσις. » Ἕτερον γάρ ἐστι ποίησις, καὶ ἕτερον ἄρσις. Οὐχ ὑδράργυρον αὐτῶν εἶπεν ἀφθόνως\footnote{οὐ puis le signe du mercure, puis μαυτον (\emph{sic}) MK ; οὐχ ὑδράργυρον αὐτὼν B etc. Corr. conj. (\emph{M. B.}). Cp. 3, 47, 4. (\emph{C. E. R.})} γεγράφθαι · τὴν δὲ ποίησιν οὐδεὶς αὐτῶν ἐξέθετο · τοῦτο γὰρ ἦν τὸ ἐμφανὲς μυστήριον, τοῦτό ἐστιν τὸ σφόδρα κεκρυμμένον. Ἡ οὖν ἄρσις\footnote{τουτέστιν $\star$, f. mel.} τοιάδε ἐστὶν, ἡ διὰ τούτων τῶν ὀργάνων καὶ τῶν ὁμοίων, τῶν ὡς ἀπὸ τοῦ νοὸς γινομένων. Καὶ μάλιστα ἐὰν [εἴ] τις προπαιδευθῇ τὰ πνευματικὰ\footnote{On ne connaît pas d'ouvrage, même perdu, d'Archimède intitulé πνευματικά.} Ἀρχιμήδους, ἢ Ἥρωνος καὶ τῶν ἄλλων καὶ τὰ μηχανικὰ αὐτῶν.

4. ΠΕΡΙ ΕΤΕΡΩΝ ΚΑΜΙΝΩΝ. --- Ἐπειδὴ ἑξῆς ὁ λόγος ἡμῖν περὶ καμίνων καὶ καταβαφῆς πρόκειται, οὐ βούλομαι πρὸς σὲ ποιεῖσθαι ἐμπεσοῦσαν ταῖς ἄλλων γραφαῖς. Καὶ γὰρ παρὰ Μαρία · « Ἡ τῆς ὁρωμένης καμίνου\footnote{Cp. 3, 47, 1.} οὐ κεῖται διαγραφὴ, ἧς ὁ φιλόσοφος οὐκ ἐμνημόνευσεν, οὐ μόνον πρισμάτων\footnote{πρησμάτων M.} καὶ τῶν ἄλλων περὶ ὧν ἠρέμα ἐν τῷ περὶ ποσότητος πυρὸς ὑπομνήματι διέλαβον. » Ἵνα οὖν μὴ δόξῃ τι λείπειν τοῖς (f. 195 v.) σοῖς γράμμασιν,\footnote{F. l. συγγράμμασι.} ἔστω παρὰ σοὶ καὶ ἡ κάμινος Μαρίας, ἧς καὶ ὁ Ἀγαθοδαίμων ἐμνημόνευσεν ἐν τῷ λόγῳ οὕτως · « Ἡ δὲ τῆς κηροτακίδος τοῦ κρεμαστοῦ θείου τάξις οὕτως γίνεται. Λαβὼν φιάλην, σμέρησον, ἢ λίθῳ παράτεμε\footnote{σμέρησον] F. l. μέρισον.} τὸ μέσον κυκλοτερῶς τὸν πυθμένα τῆς φιάλης, ἵνα ἐμβῇ κάτω ὀξύβαφον σύμμετρον. Καὶ βαλὼν ὀστράκινον ἄγγος λεπτὸν, προσηρμοσμένον τῇ φιάλῃ, ἵνα ᾖ κρεμαστὸν ἐκ τῆς φιάλης ἄνωθεν ἀπ ᾽ αὐτῆς ἀντεχόμενον · φθανέτω δὲ ἐπὶ τὴν σιδηρᾶν κηροτακίδα. Καὶ ἐπιθεὶς ὃ βούλει πέταλον, ἢ ὃ ἂν ἡ γραφὴ αἰτῇ ὑπὸ τὸ ἄγγος καὶ ὑπὸ τὴν κηροτακίδα\footnote{ὑπὸ τὸ ἄγγος] F. l. ὑπὲρ τ. ἄ. (\emph{M. B.}).} ἅμα τῇ φιάλῃ, ἵνα ἔσωθεν βλέπῃς, καὶ συμπεριπηλώσας τὰς ἁρμογὰς, ἕψε ἐφ ᾽ ἃς λέγει ὥρας ἡ ἡμετέρα ἡ τάξις. Τοῦτό ἐστι τὸ κρεμαστὸν\footnote{ἡ ἡμ.] ἢ ἡμετέρας MK.} θεῖον, καὶ κρεμαστὸν ἀρσένικον ὁμοίως. Δίδου τρυμαλίαν λεπτὴν βελόνης, μέσον τοῦ ἄγγους. »

5. Ὑαλῆ ἄλλη φιάλη ὕπωμος τε · ἢ τω δὲ τὸ ἄγγος τὸ ὀστράκινον\footnote{ὑάλη MK. Corr. conj. --- ὕπωμος] F. l. ἄπωμος. --- ἢ τῷ K. F. l. ἔστω.} ἐοικὸς τοῖς τῶν ὀρβίων κύβοις, ἀλλ ᾽ ἐοικὸς τοῖς τῶν ἀγγείων κύβοις.\footnote{ἀλλ ᾽] F. l. ἄλλως · ἐοικὸς τ. τ. ἀ. κύβοις (à considérer comme variante marginale introduite dans le texte ? ). --- Deux figures.}

(F. 196 r.) 6. Ἡ δὲ κάμινος φουρνοειδὴς, φησὶν ἡ Μαρία, ἔχουσα ἄνω τρεῖς μαζοὺς, ἢ ἀνοχὰς, ἢ σύροντας. Καῦσον δὲ καλάμοις ἑλληνικοῖς\footnote{μύζους MK. Corr. conj. --- σύροντας] F. l. σύρτας.} κατὰ πρόβασιν, νυχθήμερα δύο ἡ τρία, πρὸς ὃ ἔχει ἡ βαφή\footnote{πρόσβασιν MK. Corr. conj.} · καὶ ἄφες ἀποφρυγῆναι ἐν τῇ καμίνῳ. Κατάσπα δὲ δι ᾽ ὅλης ἡμέρας ἄσφαλτον, ἐπιβάλλων ἃ οἶδας, καὶ χαλκὸν λευκὸν ἢ ξανθόν. Δύναται δὲ ὧδε γενέσθαι, καὶ τὸ ἠθμοειδὲς ὄργανον λευκαίνει, ξανθοῖ, ἰοῖ,\footnote{F. l. ἰοῖ. Παρόπτα ... ποίει (\emph{M. B.}).} παροπτᾷ, ἀντέσματα ποιεῖ, μαλαγμάτων καταβαφὰς, καὶ ὅσα ἂν ἐπινοῇς.\footnote{ἀντέσματα] F. l. ἀνθίσματα. --- ἐπινοεῖς MK.} Ἡ δὲ ποίησις αὐτῆς αὕτη.\footnote{Figures.}

\bigskip
\centerline{\EightStarTaper}
\centerline{\EightStarTaper\EightStarTaper}
\bigskip

\subsubsection{3. --- 51. ΤΟ ΠΡΩΤΟΝ ΒΙΒΛΙΟΝ ΤΗΣ ΤΕΛΕΥΤΑΙΑΣ ΑΠΟΧΗΣ ΖΩΣΙΜΟΥ ΘΗΒΑΙΟΥ.}
\paragraph{}
\emph{Transcrit sur} A, f. 251 v. --- \emph{Contenu aussi dans Laur.}, art. 33. --- \emph{Toutes les variantes insérées dans le texte sont des corrections conjecturales}.

\bigskip

1. Ἔνθεν βεβαιοῦται ἀληθὴς βίβλος · Ζώσιμος Θεοσεβείᾳ χαίρειν.

Ὅλον τὸ τῆς Αἰγύπτου βασίλειον, ὦ γύναι, ἀπὸ τῶν δύο τούτων τῶν\footnote{Ὅλον τὸ τῆς Αἰγ. βασίλειον κ. τ. λ. jusqu'à ἄλλους Ἰουδαίους (première phrase du § 3). Morceau cité presque textuellement par Olympiodore (ci-dessus, 2, 4, 35). On a rapporté ici les principales variantes de cette citation, qui a été supprimée. --- La première phrase est citée aussi dans 3, 39 (à voir pour les variantes du présent texte).} τεχνῶν ἐστιν, τῶν τε καιρίκων, καὶ τῶν ψάμμων. Ἡ γὰρ καλουμένη\footnote{Fabricius (\emph{Biblioth. græca}, t. 12, p. 765) faisant la notice d'un ms. alchimique à lui appartenant et copié sur un « codex regius » dont la trace est perdue (peut-être la réunion de A et de K ? ), reproduit, sous le n° 20, la citation de Zosime faite par Olympiodore. Nous donnons les variantes du ms. de Fabricius, quand elle n'est pas conforme au texte de M. --- καιρικῶν] κυρικῶν A ; κερικῶν (pour καιρικῶν) καὶ τῶν φυσικῶν καὶ ψ. M dans Olympiodore ; καιρικῶν A dans Ol. ; τῶν τε κηρΰκων καὶ τῶν φυσικῶν ψ. Fabr. --- τὸν ψάμμον A.} θεία τέχνη ἢ λόγῳ δογματικῷ καὶ σοφιστικῷ ἢ τὰ πλεῖστα ὑ- (f. 252 r.) ποπίπτουσα\footnote{Après τέχνη] Réd. de M dans Ol. : περὶ ἣν ἀσχολοῦνται ἅπαντες οἱ ζητοῦντες τὰ χειροτμήματα ἅπαντα (note de Fabr. : alias χειροτεχνήματα \emph{vel} χειρόκμητα) καὶ τὰς τιμίας τέχνας, τὰς τέσσαράς φημι, δοκοῦσίν τι ποιεῖν μόνοις ἐξεδόθη τοῖς ἱερεῦσιν. Ἡ γὰρ ψαμμουργικὴ βασιλέων ἦν, ὥστε καὶ ἐὰν συμβῇ ἱερέα ἢ σοφὸν λεγόμενον ἑρμηνεύσαντα τὰ ἐκ τῶν παλαιῶν ἢ ἀπὸ προγόνων ἐκληρονόμησεν, καὶ ἔχων κ. ἰδ. τ. γν. αὐτῶν τὴν ἀκώλυτον οὐκ ἐποίει.} τοῖς ὃν φύλαξιν ἐδόθη εἰς διατροφήν · [ὃ] οὐ μόνον\footnote{τοῖς ὂν] F. l. τισιν.} δὲ αὕτη, ἀλλὰ καὶ ἅπαξ αἱ καλούμεναι τίμιαι τέσσαρες τέχναι καὶ τὰ γειροτμήματα · αἰ μέντοι καὶ ἡ δημιουργικὴ μένη βασιλέων\footnote{αἰ], f. l. καὶ. --- μένη] F. l. τέχνη} ... ὥστε καὶ ἐὰν συνευἠ, ἢ, ἐκ φωνῶν γενομένη, ἑρμηνεύηται ἐκ τῶν\footnote{συνευὴ A ; f. l. συμβῇ comme dans Ol.} στηλῶν ἔχειν προγόνων κληρονομίαν ἔχων, καὶ ἰδὼν τὴν γνῶσιν τῶν\footnote{καὶ ἰδὼν κ. τ. λ.] Réd. de L dans Ol. : καὶ εἰ καὶ εἶχε καὶ ἤδει τὴν γνώμην καὶ γνῶσιν αὐτὴν ἀκ. οὖσαν, ὅμως οὐκ ἐποίει τοῦτο, ἀλλ ᾽ ἐφοβεῖτο τιμωρίαν. (ἐφοβ. τιμ. γὰρ A).} τοιούτων ἀκωλύτων, οὐκ ἐποίει · ἐτιμωρεῖτο γὰρ, ὥσπερ οἱ τεχνῖται οἱ ἐπιστάμενοι βασιλικὸν τύπτειν νόμισμα οὐχ ἑαυτοῖς τύπτειν, ἐπεὶ\footnote{ἑαυτοῖς τύπτειν] ἑ. τύπτουσιν M dans Ol.} τιμωροῦνται, οὕτω καὶ ἐπὶ τοῖς βασιλεῦσιν τῶν Αἰγυπτίων οἱ τεχνῖται τῆς ἐψήσεως, καὶ οἱ ἔχοντες τὴν γνῶσιν τῆς ἀκολυσίας οὐχ ἑαυτοῖς\footnote{καὶ om. Fabr. --- Réd. de M dans Ol. : τ. γν. τῆς ἀμμοπλυσίας καὶ ἀκολουθίας.} ἐποίουν, ἀλλ ᾽ εἰς αὐτὸ τοῦτο ἐστρατεύοντο τοῖς Αἰγυπτίων βασιλεῦσιν,\footnote{ἐστράτευον τὸ M dans Ol. et Fabr.} εἰς τοὺς θησαυροὺς ἐργαζόμενοι · εἶχον δὲ καὶ ἰδίους ἄρχοντας ἐπικειμένους καὶ πολὺ τυραννῆς ἦν τῆς ἑψήσεως, οὐ μόνον αὐτῆς, ἀλλὰ\footnote{F. l. καὶ πολλὴ τυραννὶς ἦν. Réd. de M dans Ol. : ἐπικ. ἐπάνω τῶν θησαύρων καὶ ἀρχιστρατήγους καὶ (οἱ au lieu de καὶ L) ἐποίουν πολλὴν τυραννίην τῆς ἑψήσεως. Νόμος γὰρ ἦν Αἰγ. μηδὲ ἐγγρ. αὐτὰ τινα ἐκδιδόναι.} καὶ τῶν χρυσωρύχων. Εἴ τις γὰρ εὑρίσκεται ὀρύσσων, νόμος ἧν\footnote{χρυσορύχων A. F. l. χρυσωρυχίων.} Αἰγυπτίοις ἐγγράφως αὐτὰ ἐπιδιδόναι.\footnote{μὴ ἐγγράφως M$\star$.}

2. Τινὲς οὖν μέμφονται Δημόκριτον καὶ τοὺς ἀρχαίους < ὡς μὴ\footnote{ὡς μὴ ajouté d'après M$\star$.} > μνημονευσάντων τῶν τούτων τεχνῶν < ἀλλὰ μόνων > τῶν λεγομένων\footnote{τούτων τῶν δύο τεχνῶν M$\star$. --- ἀλλὰ μόνων ajouté d'après M$\star$. --- λεγ. κυρίων καὶ τιμ. L$\star$.} τιμίων. Τί δὲ αὐτοῖς μέμφονται ; οὐ γὰρ ἠδύναντο μέμφοντες τῶν\footnote{τί δὲ --- μεμφ.] μάτην δὲ αὐτοὺς μέμφ. M$\star$. --- μέμφοντες φίλοι ὄντες M$\star$.} βασιλέων Αἰγυπτίων, καὶ τὰ πρωτεῖα ἐν προφητείᾳ καυχῶντες, πῶς\footnote{ἐν προφητείᾳ] ἐν προφητία A ; ἐν προφητικῇ τίμῃ αὐχοῦντες ML$\star$. --- αὐχοῦντες M$\star$ ; καυχώμενοι φέρειν L$\star$.} ἠδύναντο ἄλλοις ἀναφανδὸν μαθήματα κατὰ τῶν βασιλέων δημοσίᾳ\footnote{ἄλλοις om. $\star$.} ἐκμηνύσασθαι καὶ δοῦναι ἄλλοις πλούτου τυραννίδα ; οὐδὲν ἠδύναν- (f. 252 v.) το\footnote{ἐνμυμήσασθαι A ; ἐκθέσθαι M$\star$. --- οὔτε εἰ ἠδύναντο ἐξεδίδουν M$\star$.} ἔξω δίδουν,\footnote{ἄλλοις --- ποιεῖν] Réd. de L$\star$ : ὄντα τοῖς ἄλλοις πλούτου τυραννίς τε καὶ ὄλεθρος ; οὔτε δὲ, εἴπερ ἠδύν., ἂν ἐξεδίδουν, αὐτὰ λάθρα ποιεῖν.} ἐφθόνουν γάρ · μόνοις δὲ Ἰουδαίοις ἐξέδοσαν\footnote{ἐξέδοσαν] ἐξὸν ἦν M$\star$.} λάθρα ταῦτα ποιεῖν καὶ γράφειν καὶ παραδιδόναι. Καὶ ἀμέλει γοῦν εὑρίσκομεν\footnote{παραδιδὸναι] ἐκδιδόναι M$\star$. --- κἂν μέλη A ; ἀμέλει M$\star$ ; διὸ καὶ ἀμέλει L$\star$.} Θεόφιλον τὸν Θεογένους γράψαντα τῆς χωρογραφίας χρυσωρυχεῖα,\footnote{τῆς χειρογραφίας κατορίχει A. Corrigé d'après M$\star$. (Voir ci-dessus, p. 90, l. 18). --- Fabr. a écrit τ. χ. εὐτυχείᾳ.} καὶ Μαρίας τὴν χωρογραφίαν καὶ ἄλλους Ἰουδαίους.\footnote{χωρογραφίαν] Lire καμινογραφίαν comme dans Ol.}

3. Ἀλλὰ καιρικὰς οὔτε Ἰουδαίων, οὔτε Ἑλλήνων οὐδεὶς ἐξέδωκέν\footnote{κύρικας A.} ποτε · καὶ αὐτὰς γὰρ ἐν τοῖς καθ ᾽ ἑαυτῶν χρωμάτων κατετέθεντο εἰδώλοις, παραδόντες τηρεῖν · καί γε τὴν ψαμμουργίαν πολὺ\footnote{F. l. τῆς ψαμμουργίας π. διαφερούσης τῶν καιρικῶν, οὗτοι πάνυ ...} διαφέρουσα τῶν καιρικῶν ; [οὐ] πάνυ τι ἐφθόνησαν διὰ τὸ τὴν τέχνην\footnote{κυρικῶν A.} αὐτὴν ἐξάγειν καὶ τὸν ἐπιχειροῦντα ἀποκόλαστον γίνεσθαι · εἰ γὰρ\footnote{ἐξάγην A. --- F. l. ἀκόλαστον.} ὀρύσσων κατάφορος γίνεται ἀπίων τηρούντων τὰ ἐμπόρια τῆς πόλεως\footnote{ἀπιών] F. l. ἀπὸ τῶν (\emph{M. B.}).} διὰ τὰ βασιλικὰ τέλη · ἢ τῶν καμίνων μὴ δυναμένων κρυβῆναι, ταῖς δὲ καιρικαῖς < βαφαῖς > διὰ πάντα λανθάνειν. Ὅτι ἐπεὶ καὶ οὐχ εὑρίσκεις οὐδένα τῶν ἀρχαίων, οὔτε κρυβηθὲν ἰδεῖν, οὔτε φανερῶς ἐκδίδονταί\footnote{F. l. ἐκδιδόναι.} τι περὶ αὐτῶν · μόνον δὲ Δημόκριτον εὗρον ἐν πάσῃ τῶν ἀρχαίων < τάξει > αἰνξάμενον κατ ᾽ αὐτῶν φανερῶς αὐτὰς καταλέξας.\footnote{ἐνηξάμενον A. --- F. l. καταλέξαι.} Ἀλλ ᾽ ὡσαύτως ἦν, διὰ τὸ περὶ τῶν τιμίων τεχνῶν ἤρχετο τὸ προοίμιον · καὶ\footnote{διὰ τὸ] F. l. διότι. --- ἤρχεται A.} βλέπε πανουργίαν · ἤρξατο μόνον ἀπὸ ὑδραργύρου καὶ σώματος μαγνησίας\footnote{βλέπει A. --- εἴρξατο A.} · τὰ δὲ ἄλλα πάντα τῶν καιρικῶν καὶ λέγει οὕτω · Ὤχρα ἀττικὴ, σινώπη ποντικὴ, θεῖον ἄθικτον ὅ ἐστιν [μέρη] λίτρα αʹ · καὶ λιθο- (f. 253 r.) φρύγιον, σῶριν ξανθὸν, χαλκάνθη ξηρὰ, κιννάβαριν, μίσυ ὀπτὸν, μίσυ ὠμὸν, ποιήσεις ἀνδροδάμαν, θεῖον, ἀρσένικον, καὶ σανδαράχην. Καὶ ἵνα μὴ πάντα καταλέγω < τὰ > ἐν τοῖς τέτρασιν καταλόγοις, τὰ πάντα τῶν καιρικῶν ζητούμενα εὑρήσεις · καὶ ἕνα σὲ ποιήσῃ ὅ τι\footnote{F. l. καὶ ἵνα συ ποιήσῃς ...} περὶ αὐτῶν αἰνίττεται, τὰ μὲν ὠμὰ κατέλεξεν, τὰ δὲ ὀπτὰ, ἵνα σὺν τῶν\footnote{F. l. ἵνα συνῇς. Le verbe συνίημε admet son complément au génitif.} δύο τεχνῶν · μᾶλλον δὲ ἀγαγὼν τῶν καιρικῶν μηνύσει τὰς βαφάς.\footnote{F. l. ἐπαγαγὼν.} Φησὶν γάρ · μίσυ ὠμὸν, μίσυ ὀπτὸν, σῶριν ξανθὸν, χαλκάνθη ξανθὴ, καὶ τὰ ὄμοια · ἀλλ᾽ οὶκονομηθέντα λέγει, εὶς τὰς τιμίας τέχνας καλῶς εῖπας. Καὶ διὰ τί πᾶσαι τῶν τούτων οἰκονομουμένων καὶ ξανθουμένων,\footnote{F. l. πάσας.} μὴ εἴπεις · ὑδράργυρον ξανθὴν καὶ σῶμα < μαγνησίας > ξανθόν · καὶ\footnote{F. l. μὴ εἴπας.} ἁπλῶς ὅλον τὸν κατάλογον ξανθόν ;

4. Ἀλλ ᾽ ἐκεῖνον ἴδῃ ὅπερ ἐφρόνει, καὶ ὅπερ ἔγραφεν δι ᾽ ἑνὸς συγγράμματος\footnote{F. l. ἴδε.} αἰνιγματοειδοῦς, τὰ πάντα αἰνίξασθαι ἠθέλησεν. Καὶ ἀξιοπιστοτέρας μαρτυρίας τούτων εὗρεν, ὅτι αὐτὰς αἰνίττεται. Πῶς εἰδὼς ὅτι μία βαφή ἐστι καὶ μία ἀγωγὴ, πολλὰς αὐτὰς ἐποίει λέγων · « Τούτων τῶν φύσεων οὐκ εἰσὶ μείζων ἐν βαφαῖς ; » Ἵνα δείξῃ ὅτι ἐκ\footnote{F. l. ἔστι.} τῶν αὐτῶν εἰδῶν, πολλαὶ βαφαὶ συντίθενται, καιρικῶν τοῦ σταθμοῦ\footnote{συντίθονται A.} ἐναλλασσαμένου, καὶ τὴν ποσότητα τῶν ... ... εἰδῶν ἀπὸ ἑνὸς μόνου,\footnote{Après τῶν, le signe du cuivre deux fois de suite, ici et plus loin. Nous remplaçons chaque signe par 3 points. « J'ai lu quelque part le sens νομίσματα. Peut-être faut-il lire χαλκώματα » (\emph{M. B.}). F. l. βαφαὶ \emph{vel} βαφικαὶ. Cp. p. 246, l. 2. (\emph{C. E. R.}).} ἕως ναʹ τὸν ἀριθμόν · ἅμα καὶ τῷ λέγειν, ἕως τὸν φυσικὸν, τουτέστιν ἡ τοῦ χρυσοῦ ποίησις ὕλη ἐδήλωσεν τὰς φυσικὰς βαφάς. Καὶ πάλιν οὖν λέγει · « Εἰς πολὺ ὑ- (f. 253 v.) μᾶς ἐνέβαλον κάματον, εἴ τι πολὺ ὕλη καταχώσαντες, τὰ φυσικὰ ἀπολέσαντες πάλιν\footnote{F. l. εἴ τινες πολλῇ ὕλῃ.} · δηλονότι τοῖς παρελθὸν χρόνοις τοῖς Ἑρμοῦ φυσικαὶ βαφαί ἐκαλουντο\footnote{F. l. δηλονότι τοῖς παρελθοῦσι χρ. οἷς.} αὗται μέλλουσαι γράφεσθαι κοινῇ τῇ ἐπιγραφῇ τῆς βίβλου λέγων · Βίβλος φυσικῶν βαφῶν Ἰσιδώρῳ δοθεῖσα. Ἀλλ ᾽ ὅτε ἐφθονήθησαν\footnote{F. l. λεγομένης. Un des livres hermétiques est intitulé περὶ φυσικῶν βαφῶν.} ἀπὸ τῶν τῆς σαρκὸς ... καιρικαὶ ἐγένοντο καὶ ἐλέχθησαν. Οὐ μὴν ἀλλὰ καὶ τοὺς ἀρχαίους μέμφονται < καὶ > μάλιστα Ἑρμῆν, ὅτι οὔτε δημοσίᾳ αὐτοῖς ἐκδεδώκασιν, οὔτε ἐν παραβύστῳ, οὔτε ἡνίξαντο\footnote{F. l. αὐτὰς.} ὅτι κἄν ἐστιν.\footnote{F. l. ὅτι καὶ ἐστι.}

5. Αὐτὸς δὲ μόνος ἀπέδειξεν ὁ Δημόκριτος εἰς τὸ σύγγραμμα καὶ ἠνίξατο. Αὐτοὶ δὲ ἐν ταῖς στήλαις αὐτὰ ἐνέγλυψαν ἐν τῷ σκότει\footnote{αὐτὰ] F. l. αὐτὰς.} καὶ τοῖς μυχοῖς, τοῖς συμβολικοῖς χαρακτῆρσιν, καὶ αὐτὰς καὶ τὴν\footnote{μοιχ. A.} χωρογραφίαν Αἰγύπτου, ἵνα κἄν τις τολμήσας ἐπιβῆναι τῶν μυχῶν τοὺς σκότους, τῶν πλημμελημένων ἐπιλύσεων, μὴ εὕρῃ ἐπιλύσασθαι\footnote{εὕρει A.} τὸν χαρακτῆρα μετὰ τοσαύτην τόλμην καὶ κάματον. Οἱ οὖν Ἰουδαῖοι\footnote{Sur χαρακτῆρα, une croix à l'encre rouge dans A, et à la marge, cette note rognée par le relieur : < τὸν > χαρακτῆρα < το > ῦ ναʹ . ἐὰν < ἐπ > ιβαλὸν ί ... κ τὸν πνευματικὸν · < τὶ > ς δὲ ἐκ τῶν λό < γω > ν δόξας φεύ < γ > ειν · τουτέστιν < τῶν > σαρκικῶν. (1\textsuperscript{re} main).} αὐτοὺς μιμησάμενοι, ἐν τοῖς καταθέτοις αὐτὰς τὰς καιρικὰς παραδώσαντες μετὰ τῆς αὐτῶν μυήσεως, καὶ παρακελεύονται ἐν ταῖς διαθήκαις αὐτῶν. Ἐὰν ἡμῶν εὕρῃς τοὺς θησαυροὺς, παρίδε τὸν χρυσὸν τοῖς ἐθέλουσιν ἑαυτοὺς φονεύειν, καὶ περὶ τῆς τῶν χαράττας\footnote{A mg. : ἢ χαρακτήρʹ. --- F. l. περὶ τούτων χαρακτῆρας εὑρηκὼς.} εὑρηκὼς, τὰ ὅλα χρήματα ἐν ὀλίγῳ συνάξεις · τὰ δὲ χρήματα μόνον λαβὼν, ἑαυτὸν φονεύσεις, ἐκ τοῦ (f. 254 r.) φθόνου τῶν κρατούντων βασιλέων, οὐ μόνον αὐτῶν, ἀλλὰ καὶ πάντων ἀνθρώπων.\footnote{K en rouge dans A au-dessus de μόνον et renvoi à la marge inférieure avec ces mots : πολλὰ βιβλία εὑρίσκονται < περὶ > χυμεύσεως · αον μὲν φυσικὰς βαφὰς λέγων · τὰ δὲ παραφύσεις (\emph{sic}) · τὰ δύο ψεῦδος καὶ τὴν ἀλήθειαν κατακαλυπτικήν.}

6. Δύο οὖν γένη εἰσὶν καιρικῶν ἐν < ταῖς > τῶν ὀθωνῶν ἐκδεδώκασιν,\footnote{κυρίκων A. --- F. l. ἐν < ταῖς > τῶν ὀθονῶν ἐκδόσεσιν.} ἢ κατὰ τόπον ἐφόροι τοῖς ἐαυτῶν ἱερεῦσι · τούτου ἕνεκεν καὶ καιρικαὶ\footnote{ἢ --- ἱερεῦσι] F. l. ἃς κατὰ τόπον ἐφόρουν ... --- καιρικαὶ] F. l. καιρικὰς.} ἐκάλεσαν · ἐπειδὴ καὶ καιροῖς ἐνεργοῦν τῇ θελήσει τῶν δοκώντων\footnote{δωκόντων A.} ... ... μηκέτι δὲ θελήσασιν τοὐναντίον ἐποίουν · ἐπίμικτοι οὖν ἦσαν αἱ καίρικαί ... ... τοῖς εἴδεσι · ἔκ τε τῶν γνησίων εἰδῶν τῶν καιρικῶν · τῶν ἄλλων [ἄλλων] τοῖς ἀνήκουσι ταῖς τιμίαις τέχναις. Τὸ δὲ ἄλλο γένος τῶν [τῶν] καιρικῶν γνησίων καὶ φυσικῶν τὸ Ἐρμᾶν\footnote{ἑρμὰν A. Le signe ῀ au-dessus de ce mot, et renvoi à la mg. suivi de τὸ γλυκύ (1\textsuperscript{re} main).} ἐνέγραψεν εἰς τὰς στήλας · ἀπὸγωνεὲ τὸν μόνον ξανθωμίλινον πυρὸς,\footnote{ἀπὸγωνεὲ] F. l. ἀποχώνευε (mot supposé). F. l. ξανθομήλινον (mot supposé).} ἡλιοδὸν χλωρὸν, ὠχρὸν, μέλαν, χλωρὸν καὶ τὸ ὅμοιον · καὶ αὐτὰς\footnote{F. l. καὶ τὰ ὅμοια.} δὲ τὰς γέας μυστικῶς ψάμμους ἐκάλεσαν · καὶ τὰ εἴδη τῶν χρωμάτων ἐμήνυσεν · αὗται φυσικῶς ἐνεργοῦσιν · φθονοῦνται δὲ ἀπὸ τῶν περγειῶν ... ... · ἐπὰν δέ τις μυηθεὶς ἐκδιώκει αὐτοὺς, τεύξεται τοῦ\footnote{F. l. ὑπεργείων.} ζητουμένου.

7. Οἱ οὖν ἔμφοροι ἐκδιωκόμενοι τότε παρὰ τῶν ποτε μεγάλων\footnote{F. l. ἔφοροι.} ἀνθρώπων, συνεβουλεύσαντο ἀντὶ ἡμῶν τῶν φυσικῶν πάντων ποιῆσαι,\footnote{συνευουλεύσαντο A, indice d'un ms. original du 10\textsuperscript{e} ou 11\textsuperscript{e} siècle.} ἵνα μὴ διώκωνται παρὰ τῶν ἀνθρώπων, ἀλλὰ λιτανεύωνται καὶ παρακαλῶνται, οἰκονομοῦνται διὰ θυσιῶν, ὃ καὶ πεποίηκαν · ἔκρυψαν πάντα\footnote{F. l. πεποιήκασιν.} τὰ φυσικὰ καὶ αὐτόματα, οὐ μόνον φθονοῦντες αὐτοῖς, ἀλλὰ καὶ περὶ τῆς ἑαυτῶν ζωῆς φροντίζοντες, ἵνα μὴ μαστίζωνται ἐκδιωκόμενοι καὶ λιμῷ τιμωρῶνται, θυσίας μὴ λαμβάνοντες, (f. 254 v.) ἐποίησαν\footnote{Au-dessus de ἱερεῦσι, trois points rouges dans A, et à la mg. sup. ; μετὰ τὸν χρισμὸν κἀν τοῖς ἱερεῦσιν τοῖς νομιζομένοις, ἥγουν (f. l. ἦγον) θυσίας ὅποται (l. ὅποτε) ἰληάνοντες (f. l. οἱ λεαίνοντες) ἴτε (l. εἴτε) ἱερεῖς. --- (Addition à insérer dans le texte ? )} οὕτως · ἔκρυψαν τὴν φυσικὴν καὶ εἰσηγήσαντο τὴν ἑαυτῶν ἀφύσικον, καὶ ἐξέδωκαν αὐτὰ τοῖς ἑαυτῶν ἱερεῦσι, εἴ τε δημόται ἠμέλουν τῶν θυσιῶν, ἐκώλυον καὶ αὐτοὶ τὴν ἀφύσικον φιλοτιμίαν · ὅσοι δὲ κατεκράτησαν, τὴν νομιζομένην δόξαν ... ... τοῦ αἰῶνος ὑδρογενήσαντα καὶ ἐπληθύνθησαν ἔθος καὶ νόμῳ καὶ φόβῳ αἱ θυσίαι αὐτῶν\footnote{ἔθος] F. l. ἔθει.} · οὐκέτι οὐδὲ τὰς ψευδεῖς αὐτῶν ἐπαγγελίας ἀπεπλήρουν · ἀλλ ᾽ ὅτε ἐγγενεῖ ἄρα ἀποκατάστασις τῶν κλημάτων, καὶ διεφέρετο κλήμα\footnote{ἐγγενεῖ] F. l. ἐγένετο \emph{vel} ἐγγενεῖ ἄρᾳ < ἦν > ἀποκ.} πολέμῳ, καὶ ἐλείπετο ἐκ τοῦ κλήματος ἐκείνου τὸ γένος τῶν ἀνθρώπων καὶ τὰ ἱερὰ αὐτῶν ἐρημοῦντο, καὶ αἱ θυσίαι αὐτῶν ἠμελοῦντο · τοὺς περιλειπομένους ἀνθρώπους ἐκολάκευον, ὡς δι ᾽ ὀνειράτων, διὰ τὸ ψεῦδος αὐτῶν, διὰ πολλῶν συμβούλων, τῶν [τῶν] θυσιῶν ἀντέχεσθαι · αὐτὰς\footnote{τῶν τῶν] F. l. τούτων.} δὲ πάλιν παρεχόντων τὰς ψευδεῖς καὶ ἀφυσίκας ἐπαγγελίας · καὶ ἥδοντο πάντες οἱ φιλήδονοι ἄθλιοι καὶ ἀμαθεῖς ἄνθρωποι · ὥστε καί σοι θέλουσιν ποιῆσαι, ὦ γύναι, διὰ τοῦ ψευδοπροφήτου αὐτῶν · κολακεύουσιν σε,\footnote{F. l. προσποιῆσαι.} τὰ κατὰ τόπον ... ... πεινῶντα, οὐ μόνον θυσίας, ἀλλὰ καὶ τὴν σὴν ψυχήν.\footnote{F. l. πεινῶντες.}

8. Σὺ γοῦν, μὴ περιέλκου, ὡς γυνὴ, ὡς καὶ ἐν τοὺς κατ ᾽ ἐνείαν\footnote{Cp. 3, 27, 7, où Zosime adresse à Théosébie des recommandations analogues. --- F. l. ὧ γύναι. --- F. l. ἐν τοῖς κατ ᾽ ἐνέργειαν.} ἐξεῖπόν σοι. Καὶ μὴ περιρέμβου, ζητοῦσα θεόν · ἀλλ ᾽ οἴκαδε καθέζου, καὶ θεὸς ἥξει πρὸς σὲ ὁ πανταχοῦ ὢν, καὶ οὐκ ἐν τόπῳ ἐλαχίστῳ ὡς τὰ δαιμόνια · καθεζομένη δὲ τῷ σώματι, καθέζου καὶ τοῖς πάθεσιν, ἐπιθυμίᾳ, ἡδονῇ, (f. 255 r.) θυμῷ, λύπῃ, καὶ ταῖς ιβʹ μύραις τοῦ\footnote{F. l. μοίραις. Cp. Platon, Timée, p. 41 B : οὐδὲ τεύξεσθε θανάτου μοίρας.} θανάτου · καὶ οὕτως αὐτὴν διευθύνουσα προσκαλέσῃ πρὸς ἐαυτὴν τὸ θεῖον · καὶ οὕτως ἥξει τὸ πανταχοῦ ὢν καὶ οὐδαμοῦ · καὶ μὴ καλουμένη,\footnote{τὸ] F. l. ὁ.} πρόσφερε θυσίας τοῖς ... ... μὴ τὰς προσφύρους, μὴ τὰς θρεπτικὰς αὐτῶν, καὶ προσηνεῖς, ἀλλὰ τὰς ἀποθρεπτικὰς αὐτῶν, καὶ ὰναιρετικὰς\footnote{F. l. ἀποτρεπτικὰς.} ἃς προσεφώνησεν Μεμβρῆς τῶν Ἱεροσολύμων βασιλεῖ Σολομῶντι,\footnote{Mεμβρῆς] peut-être Memphrès, roi égyptien de la 18\textsuperscript{e} dynastie (Canon d'Eusèbe, texte arménien. 1, 214).} αὐτὸς δὲ μάλιστα Σολομῶν ὅσας ἔγραψεν ἀπὸ τῆς ἑαυτοῦ σοφίας · καὶ οὕτως ἐνεργοῦσα, ἐπιτεύξῃ τῶν γνησίων καὶ φυσικῶν καιρικῶν · ταῦτα δὲ ποίει ἕως παντελειωθῇς τὴν ψυχήν. Ὅταν δὲ\footnote{κυρικῶν A. --- ἐποίει A. --- F. l. ἕως ἂν τελειωθῇς.} ἐπιγνοῦσα αὐτὴν τελειωθεῖσαν, τότε καὶ τῶν φυσικῶν τῆς ὕλης\footnote{F. l. ἐπιγνῷς αὐτὴν.} κατάπτησον, καὶ καταδραμοῦσα ἐπὶ τὸν Ποιμένανδρα καὶ βαπτισθεῖσα\footnote{F. l. ποιμάνδρα.} τῷ κρατῆρι, ἀνάδραμε ἐπὶ τὸ γένος τὸ σόν.

9. Ἐγὼ δὲ ἐπὶ τὸ προκείμενον ἐλεύσομαι τῆς σῆς ἀτελειώτητος · ἀλλ ᾽ ὀλίγῳ ἐπέκτειναι καὶ ἀνένεγκαι χρῆμα τὸ ζητούμενον · ἤνεγκεν\footnote{ἐπεκτεῖναι καὶ ἀνενέγκαι A. --- ἤνεγκεν] F. l. ἀνάγκη (\emph{M. B.}).} μὴ ἐλαττεῖ ( ? ) καὶ ἐνήλατος εὑρίσκεται.

Ἄκουσον αὐτοῦ λέγοντος καὶ μετ ᾽ ὀλίγα · ἓν πρᾶγμά ἐστὶν δύο\footnote{A mg. : Une main.} ὠὰ καταποτασόμενος, καὶ διαφόρως γενόμενον, τὸ μὲν ὑγρὸν καὶ\footnote{F. l. καταποτισόμενος (mot supposé).} ψυχρὸν, τὸ δὲ ξηρὸν καὶ ψυχρὸν, καὶ τὰ δύο ἓν ἔργον ποιοῦσιν. Ἔστιν οὖν κατανοῆσαι τοῖς δύο ὠθιακοῖς χρώμασιν καὶ ἐκπλαγῆναι\footnote{ὠθιακοῖς] F. l. ὠοθιακοῖς (mot supposé). On connaît ὠοθυτικά, les mystères de l'œuf : (\emph{M. B.}).} τὰς τῶν χρωμάτων ἀμοιβὰς τὰς ἀπὸ τῶν ὠθιακῶν, καὶ τῶν\footnote{ὠθιακῶν] F. l. ὠοθιακῶν (\emph{M. B.}).} φθασάντων · καὶ γενέσεις τῶν χρωμάτων ὅτι παρὰ τὸ ἐλάμνεσθαι\footnote{ἐλάμνεσθαι] F. l. ἐλαύνεσθαι.} ὕλη ἐστὶν, καὶ μεθ ᾽ ἕτερα καὶ αὐταὶ παρατηρήσεις καὶ οὐχ ὁμοίαι ἐξέρχονται · διὰ τί ; (f. 255 v.) οὐχ ὅτι φθονοῦνται ; φθονοῦνται μήτις\footnote{A mg. inf. du f. 255 r. : grosse étoile, puis : ὧδε ὁ νοῦς ὁ νοεῖν δυνάμενος καλῶς καὶ ὕγιος (pour ὑγιῶς ? ).} ἐξ αὐτῶν νοήσας τὴν ὁδὸν τῶν καιρικῶν εὕρῃ. Ἀλλ ᾽ ἐρεῖ τις ὅτι οὐ μόνον τὰ ὀνόματα, ἀλλὰ καὶ πᾶσα τέχνη πάντοτε οὐχ ὁμοία ἐξέρχεται, ἀλλὰ καὶ ποτὲ μὲν καλῶς, ποτὲ δὲ ἐναντίως. Νέον, φημί\footnote{νέον] ναἵ A. F. l. ναί.} · ἀλλ ᾽ ἴσασιν οἱ τεχνῖται οἱ ἰδόντες τῶν σφαλμάτων τὰ αἴτια, ὅτι τόδε παρὰ τόδε ἐποιήσαμεν, καὶ τοῦδε ἡμελήσαμεν, καὶ τοῦδε ῥαθυμότερον\footnote{τοῦδε ῥαθυμότερον] F. l. τόδε ῥαθυμότερον.} ἐποιήσαμεν.

10. Ἐγὼ δὲ ἐπὶ τὸ προκείμενον ἐλεύσομεν. Εἰσὶν οὖν δύο ἀγωγαὶ τῶν καιρικῶν βαφῶν, μία ἀπὸ ὠμῆς, καὶ μία ὀπτὴ, < αἳ > εἴδη βάλλουσιν.\footnote{βάλλουσιν] F. l. βάπτουσιν.} Ἀλλ ᾽ ἡ μὲν ὀπτὴ πολλοῦ μόχθου ἀπολέλυται, παμπόλλου δὲ ἐπιτυχίας χρῄζει, καὶ μετὰ βραχὺ, ὡς εἶπεν ἡ θεία Μαρία. Τῆς οὖν ὀπτῆς διαφοραὶ πολλαί εἰσιν ὑγρῶν καὶ φώτων · αἱ μὲν γὰρ αὐτῶν σὺν ὕδατι ὀπτοῦνται, αἱ δὲ οἴνῳ · τὰ μὲν γὰρ αὐτῶν ἄνθραξιν γίνεται ἐν ποσότητι χρόνον, τὰ δέ φυσῶνται πάλιν τῇ ποσότητι, τὰ δὲ λασοτίοις,\footnote{τῇ] τὴν A. F. l. τινὶ. --- A mg. τὴν καραλήνα] λέγ < ει, > avec renvoi à λασοτίοις.} τὰ δὲ φούρνοις · καὶ ἄλλα ἠστείαι, καὶ ἄλλα ἄλλοις καὶ μετὰ καὶ\footnote{F. l. ἱστίαις (feux de chiffons ? ) (\emph{M. B.}).} τῶν πάντων ἁπλῶς πολλὰ οἶον ἐπὶ τοῦ μέλανος τὰ τῆς διαφορᾶς ὠῶν οὕτως μέλαν κοράκων, κορυννίων, κατακοραὶς βάθει, τεφρῶδες ἐν ταῖς\footnote{F. l. κορωνῶν, κατακορὲς.} ζωγραφουμέναις ὀθώναις, ποιεῖ δένδρα, ἢ πέτρας, ἢ ὕδατα, ἢ ζῶα, πάντα ὁμοίως, καὶ τῶν ἄλλων χρωμάτων τῶν προλεχθέντων, ὧν ἔχεις\footnote{ ὧν] ὦ A. F. l. ὦ < γύναι. >} τὰς ἀποδείξεις ἐν κάππα στοιχείῳ · καὶ ἡ ποσότης τῶν χρωμάτων. Ἐὰν γὰρ ἀκούσῃς ὤχραν ξανθὴν, μὴ ἀπλῶς ὑπολά- (f. 256 r.) βῃς, καὶ μεταπαρασκευάσαντα μυστικῶς πρὸς μόνον τοὺς κωλύτας ἔχειν · τὰ γὰρ ζητούμενα πάντα ἐν τῇ τέχνῃ κατώρθωσαν.

11. Ἔχουσιν οὖν φύσιν αὕται αἱ βαφαὶ καὶ πολλὰ σήπτεσθαι, καὶ ὀλίγα, τουτέστιν γίγνεσθαι καὶ ἐν καμινίοις ὑελοψικοῖς, καὶ ἐν χωνείαις μεγάλαις καὶ μικραῖς, καὶ ἐν διαφόροις ὀργάνων < διὰ > φώτων,\footnote{F. l. ὀργάνοις.} καὶ ἐν ποσότητι αὐτῶν · καὶ ἡ πεῖρα ἀναδείξει, μετὰ καὶ τῶν ψυχικῶν πάντων κατορθωμάθων. Ἔχεις οὖν τῶν φώτων τὰς ἀποδείξεις ἐν τῷ\footnote{ἐν τῷ Ω στοιχείῳ] Cp. le morceau 3, 49.} Ω στοιχείῳ, καὶ πάντων τῶν ζητουμένων · ἔνθεν ἀπάρξομαι, πορφυρόστολε γύναι.

\bigskip
\centerline{\EightStarTaper}
\centerline{\EightStarTaper\EightStarTaper}
\bigskip

\subsubsection{3. --- 52. ἙΡΜΗΝΕΙΑ ΠΕΡΙ ΠΑΝΤΩΝ ΑΠΛΩΣ ΚΑΙ ΠΕΡΙ ΤΩΝ ΦΩΤΩΝ.}
\paragraph{}
\emph{Transcrit sur} A, f. 264 r. --- \emph{Collationné sur} B, f. 88 r. (\emph{à partir du} § 2).

\bigskip

1. Βλέπε δὲ μὴ πλανηθῇς καὶ τὸν μόλυβδον καὶ τὸν χαλκὸν < οὐ > μόνον ξανθώσῃς, ἀλλὰ καὶ τὰ μεταλλικὰ εἴδη, τὰ λεγόμενα χρυσοζώμιον, καὶ χρύσολον, ἅτινά εἰσιν τὸν ἀριθμὸν πλέον ἢ ἔλαττον οηʹ · οηʹ δὲ\footnote{F. l. χρυσόϋλον (\emph{M. B.}).} πλέον ἢ ἔλαττον εἶπον, ὅτι ἔλαβεν ὑδράργυρον. Δεῖ δὲ γινώσκειν πεῖραν καὶ τὴν δύναμιν μνημονεύει περὶ τῶν φώτων < καὶ > διοπτᾶν ἢ\footnote{F. l. < ὅτι > πεῖρα. --- A mg. : μνημονεύει περὶ τῶν φώτων. --- F. l. μνημονεύειν. --- F. l. δεῖ ὀπτᾶν.} εἰσκρίνοντα τὸν σίδηρον. Οἱ μὲν γὰρ ἡμίωριον μόνον ὄπτησαν, οἱ δὲ\footnote{εἰμίωρον A.} ὥραν αʹ, ἄλλοι δὲ βʹ, ἕτεροι γʹ, τινὲς δὲ καὶ δʹ.

2. Ἐλαφρὰ φῶτα πᾶσαν τὴν τέχνην ἀναφέρει, καὶ τὰ χρώματα\footnote{Le ms. B (titre : περὶ φώτων) donne seulement la phrase ἐλαφρὰ --- ἀναφέρει, puis notre morceau 3, 10, 1, et continue celui-ci avec notre § 3.} ὄπτα, καὶ ἔα τέως ἀποψυγῇ · ἐν ὑέλοις βλέπῃς τὸ γινόμενον · οὕτως\footnote{ἑᾶτε ἕως A.} ξανθοῦται διὰ τῆς λειώσεως καὶ ἑψήσεως.

3. Τοῦτο τὸ θεῖον ὕδωρ τὸ δίχρωμον, τὸ λευκὸν καὶ ξανθὸν, μυρίοις\footnote{θεῖον om. A. --- καὶ] τὸ B.} κεκλήκασιν ὀνόμασιν. Ἄνευ οὖν τοῦ θείου ὕδατος οὐδέν ἐστιν · τὸ γὰρ (f. 264 v.)\footnote{Cp. 3, 10, 2 et 21, 1.} ὅλον σύνθεμα δι ᾽ αὐτοῦ ἀναλαμβάνεται, καὶ δι ᾽ αὐτοῦ ὀπτᾶται, καὶ δι ᾽ αὐτοῦ καίεται, καὶ δι ᾽ αὐτοῦ πήγνυται, καὶ δι ᾽ αὐτοῦ ξανθοῦται, καὶ δι ᾽ αὐτοῦ σήπτεται, καὶ δι ᾽ αὐτοῦ βάπτεται, καὶ δι ᾽ αὐτοῦ ἰοῦται\footnote{σήπεται B, mel.} καὶ ἐξιοῦται, καὶ ἑψεῖται. Φησὶν γάρ · Ἐπιβαλὼν ὕδωρ θείου ἄθικτον, καὶ κόμμι ὀλίγον, πᾶν σῶμα βάψεις. Ὅσα γὰρ ἀπὸ ὕδατος ἔσχον γένεσιν,\footnote{βάπτεις B. --- γέννησιν A.} ταῦτα τοῖς ἀπὸ τοῦ πυρὸς ἀντιπάσχει. Ὥστε ἄνευ τοῦ καταλόγου τῶν\footnote{ὥστε] ὡς ὅτι A.} ὑγρῶν πάντων, οὐδέν ἐστιν ἀσφαλές.

4. Ἐμνημόνευσαν δέ τινες, τάχα δὲ καὶ οἱ ὅλοι, ὅτι δεῖ τοῦτο τὸ\footnote{Cp. 3, 10, 3.} ὕδωρ ζύμης χάριν καταφθεῖραι τῷ ὁμοίῳ τὸ ὅμοιον τοῦ μέλλοντος βάπτεσθαι σώματος, εἴτε ἀργύρου, εἴτε χρυσοῦ. Ἐὰν ἄργυρον ἐθέλῃς\footnote{ἐθέλῃς βάπτειν] ἡ ἀ λῆς βάπτην A. Corr. conj.} βάπτειν,\footnote{εἴτε --- ἢ χρυσόν]. Texte omis ici dans B et dans 3, 10.} ἀργύρου πέταλα συσσήπτει · ἐὰν χρυσὸν, χρυσοῦ πέταλα\footnote{συνσήπτει A.} · ὁ γὰρ Δημόκριτος · Ἐπίβαλλε, φησὶν, χρυσοῦ ὕδωρ κοινοῦ, καὶ βάψεις, καὶ χρυσὸν καὶ καταβάψεις · ὁ γὰρ εἷς ζωμὸς καὶ τὰ ἀμφότερα σήπει\footnote{F. l. σήπειν.} κατηγορεῖται. Ζυμοὶ τοίνυν χρὴ ἐκ τοῦ ὁμοίου τὸ ὕδωρ τοῦ θείου ἢ ἄργυρον,\footnote{F. l. ζυμοῦν.} ἢ χρυσόν. Ὡς γὰρ ἡ ζύμη τοῦ ἄρτου, ὀλίγη οὖσα, τοσοῦτον\footnote{Cp. 3, 21, 3.} φύραμα ζυμοῖ, οὕτω καὶ τὸ μικρὸν ἢ ἀργύρου ἢ χρυσοῦ < διὰ > τοῦ\footnote{Réd. de B : τὸ μικρὸν puis le signe de l'or surmonté de la finale ου, puis τὸ πᾶν μέλλει ξηρίον ζυμοῦν (fin du texte dans ce ms. qui reprend plus bas avec le morceau 3, 53).} ὄξους ἐστίν.\footnote{A mg. après cette ligne : λίπι (λείπει), puis les 7 dernières lignes du f. 264 et les 9 premières du f. 265, laissées en blanc.}

\bigskip
\centerline{\EightStarTaper}
\centerline{\EightStarTaper\EightStarTaper}
\bigskip

\subsubsection{3. --- 53. La Céruse.}
\paragraph{}
\emph{Transcrit sur} A (\emph{continuation du texte précédent}).

\bigskip

1. ... ... δύναμις · μετὰ δὲ τὴν ἐργασίαν τὸ ψιμμίον ὕδατι ὑετίῳ γλυκιζόμενον, καὶ ἐώμενον καταστῆναι · τὸ δὲ ὕδωρ ἀπόχεε ἀπ ᾽ αὐτοῦ, καὶ εὑρίσκεται πάνυ λευκότατον · καὶ ἡ λιθάργυρος ἡ κοινὴ μολύβδου ἐστὶν, θαυμαστὴν δύναμιν ἔχει, κοινωνίαν ποιούμενος τῷ ὄξει · ἡ γὰρ καὶ αὐτὸ ἀσώματον\footnote{τὸ puis le signe de ὄξος, puis ἡ γὰρ ... (f, l. εἰ γὰρ ... )} εὑρίσκεται, ἁλμιζόμενον δὲ καὶ γλυκιζόμενον, καὶ αὐτὴ λευκοτάτη εὑρίσκεται καὶ πάνυ παρεμφαίνουσα τὸ ψιμμίθιον. Θαυμάζω δὲ καὶ τὸ σηρικὸν πῶς ἐν τῷ πυρὶ ξανθοῦται, καὶ τὸ σανδαράχην δύναμιν\footnote{F. l. σανδαράχιν (forme néogrecque de σανδαράχιον ? ).} ἔγει θαυμαστήν.

\bigskip
\centerline{\EightStarTaper}
\centerline{\EightStarTaper\EightStarTaper}
\bigskip

\subsubsection{3. --- 54. ΠΕΡΙ ΛΕΥΚΩΣΕΩΣ.}
\paragraph{}
\emph{Transcrit sur} A (\emph{continuation, sans titre, du texte précédent}). --- \emph{Même texte, aνec le titre, dans} B, f. 90 v., \emph{et dans} K, f. 5 v., \emph{jusqu'à} μυστήριον (ligne 3).

\bigskip

1. Γινώσκειν ὑμᾶς θέλω ὅτι πάντων ἐστὶν κεφάλαιον ἡ λεύκωσις,\footnote{Γινώσκειν --- μυστήριον] même texte 3, 40, 1.} μετὰ δὲ τὴν λεύκωσιν εὐθὺς ξανθοῦται τὸ τέλειον μυστήριον,\footnote{Après μυστήριον, B et K continuent avec le texte de 3, 40, 2 et 3.} [τοῦτό ἐστιν ἴωσις, πάλιν διὰ τοῦ ὄξους, τὰς θείας δυνάμεις ἀποτελοῦσιν.\footnote{F. l. ἀποτελοῦσα.} Ἐμφανήσω ὑμῖν πρῶτον κεφάλαιον τοῦ ἐλαίου θείου. Διηγήσομαι δὲ\footnote{F. l. θειώδους (\emph{M. B.})} ὑμῖν [λευκὸς] τὰς λευκώσεις τῶν μολύβδων ἀπεργάσας, ἢ < τοῦ > πνεῦμα\footnote{Au lieu de λευκὸς, il faudrait peut-être lire πῶς δεῖ et plus loin ἀπεργάσασθαι.} βάπτειν ἡ γέννησις, ἵνα πνεῦμα βά- (f. 265 v.) ψειν · ἄνευ\footnote{F. l. τὴν γένεσιν, ἵνα βάψῃ.} γὰρ τῶν μολύβδων οὐκ ἔστιν τέλειον · ὁ γὰρ μόλυβδος πᾶσαν οὐσίαν ἐξετάζει. Καὶ θαυμαστῶς ἀνεγράψατο ὁ φιλόσοφος τῇ λοξῇ διηγήσει\footnote{θαυμαστὸς A.} · ἐὰν τὰ ἐξετάζοντα εἰς τὰς οὐσίας εἰσκριθῶσιν, ἀνεξάλειπτον ἔχει τῶν ( ? ) τὴν φύσιν.\footnote{τῶν] F. l. τοῦτο}

2. Γινώσκειν ὑμᾶς θέλω ὅτι ἡ τελεία ἐξέτασις τὸ ὄξος ἐστίν · β\textsuperscript{ον}ʹ ἐξέτασις ὅτι μόλυβδον περὶ τοῦ β\textsuperscript{ου}ʹ κεφαλαίου ἔφη ὁ φιλόσοφος, ἐὰν τὰ ἐξετάζοντα εἰς τὰς οὐσίας εἰσκριθῶσιν, ἀνεξάλειπτον ἔχει τὴν φύσιν.

\bigskip
\centerline{\EightStarTaper}
\centerline{\EightStarTaper\EightStarTaper}
\bigskip

\subsubsection{3. --- 55. ΕΡΜΗΝΕΙΑ ΠΕΡΙ ΤΩΝ ΦΩΤΩΝ.}
\paragraph{}
\emph{Transcrit sur} A (\emph{continuation du texte précédent}).

\bigskip

1. Ἑρμηνεύσω ὑμᾶς σὺν προφήταις περὶ τῶν φώτων τὴν δύναμιν\footnote{F. l. ὑμῖν.} πᾶσιν, ἵνα τελείως τὰς παραδόσεις ἐργάσασθαι, διὰ τὸ μὴ ἀποτυχίαν\footnote{F. l. ἐργάσησθε.} γίγνεσθαι ὑμῖν. Περὶ τῶν φώτων γὰρ ἐξέθετο ὁ φιλόσοφος, ὡς ὅτι ἓν εἶδος πολλὰ ἀνατρέπει φῶτα · τὰ φῶτα γάρ εἰσιν τὰ ἐναντία πάσης ἐργασίας · ἐπὶ τῶν προγυμνασθέντων ὑμῖν παραδίδωμι, τῇδε τῇ ἀκολουθίᾳ · εἰ μὲν διὰ ὑελίνων ἀγγῶν ἑψοῦνται τὰ θειώδη, ἀναγκαῖον\footnote{εἰ] ἡ A.} χρήσασθαι τοῖς φωσὶν οἷς κέχρηνται οἱ σκιογράφοι, εἴ τίς ἐστι κηροτάκις. Ἀναγκαῖον οὖν τὸ ἄγγος τὸ ὑέλινον διὰ πηλοῦ κεραμικοῦ ἐπιδερματίδα < ἔχειν > ἡμιδακτυλαίαν, ἵνα μὴ τὸ ἄγγος ῥῆξιν ὑπομένῃ\footnote{εἰ μὴ δακτυλαίαν A. --- ὑπομίνει A.} διὰ τῆς θέρμης, οὕτως διαπραξαμένους ὡς ἐᾶται τὰ μέτρα τῶν φώτων\footnote{διαπραξαμένοις ὡς ἐᾶτε A.} · Ἐὰν δὲ μέλλῃς παροπτᾶν τὰ ἐπὶ τὸ ξανθὸν ἀγόμενα, ἀναγκαῖον ὑμᾶς χρήσασθαι τοῖσδε τοῖς φωσὶ, ἡ μὲν τοῦ ζώου εἰσχικῷ\footnote{F. l. ἡμᾶς. --- F. l. εἰ μὲν.} ( ? ) καμινίῳ παροπτᾶν, ὅταν κομίσῃς αὐτοῦ τὰ ἐπὶ τὰ ξανθὰ ἀγόμενα, (f. 266 r.) ἐν τῇ καμίνῳ ἐπὶ ὥρας ΣΤʹ · ὥρας δὲ λέγω τὰς κεκραμένας ... ἀπέχεται, καὶ φῶτα τὰ ἐπὶ τὸ ξανθὸν ἄγονται.

\bigskip
\centerline{\EightStarTaper}
\centerline{\EightStarTaper\EightStarTaper}
\bigskip

\subsubsection{3. --- 56. ΠΕΡΙ ΑΙΘΑΛΩΝ.}
\paragraph{}
\emph{Transcrit sur} M, f. 116 v. --- \emph{Collationné sur} B, f. 89 r. ;--- \emph{sur} A, f. 14 r. (= A\textsuperscript{1}) ;--- \emph{sur} A, f. 91 r. (= A\textsuperscript{2}) ;--- \emph{sur} Κ, f. 4 v. --- \emph{sur} Lc, p. 205. --- \emph{Variantes de} M \emph{ajoutées en marge de} K.

\bigskip

1. Αἰθάλαι δὲ λέγονται διὰ τὸ ἀπὸ κάτωθεν < εἰς > ἄνω τὰς τέφρας,\footnote{λέγεται BA\textsuperscript{1. 2.} K. --- ἀπὸ τῶν κατ. BA\textsuperscript{1. 2.}. Lc (= B etc.). --- τὰς τέφρας] αἱ τέφραι M.} πρὸς ὕψος ἀναπέμπεσθαι τὰς οὐσίας, ἥτις δηλοῖ τὴν τῶν ὑδάτων ἀναγωγήν.\footnote{ἥτις --- ἀναγωγὴν] ἤγουν τὴν τῶν ὑδ. ἀγωγὴν Lc.} Καὶ πάλιν αἰθάλαι λέγονται διὰ τὸ ἀπὸ τῶν κάτω ἐπὶ τὸ ὕψος χωρεῖν. Ποιήσαντες αὐτοῦ τὴν διήγησιν ἐν τῇ τῶν αἰθαλῶν ἤγουν σταγόνων\footnote{Réd. de Lc : ποίησ. οὖν αὐτῶν τινες τὴν διήγ.} ἐκμυζήσει, τὰς σκωρίας τε ἀπὸ τῆς χύτρας ἄραντες ἐλείωσαν, καὶ βαλόντες αὐτὰς τὰς ἀπ ᾽ αὐτῶν ἐξελθούσας ψυχάς. Ψυχαὶ γὰρ\footnote{καὶ βαλ. --- ψυχάς] ἐκβαλόντες ἀπ ᾽ αὐτῶν τὰς ἀπ ᾽ αὐτῶν ἐξ. ψ. Lc.} αὗται τῶν σωμάτων ἀφ ᾽ ὧν ἐξῆλθον, πάλιν ἀνεκομίσαντο ταύτας διὰ\footnote{πάλιν] διὸ καὶ πάλιν Lc.} τοῦ μασθωτοῦ, φάσκοντες ταύτην εἶναι (f. 117 r.) τὴν ἴωσιν, ἀναλογήσαντες\footnote{M mg. inf. (main du 15\textsuperscript{e} siècle) : ἔψυσι (ἔψησις). ἰόσεις (ἰώσεις). ὄπτησης (ὄπτησις). ἀνάσπασης (ἀνάσπασις). ἐλλίοσης (ἐλλείωσις). --- μαστωτοῦ Lc. --- Réd. de Lc : καὶ ἀναλογίσ. τὰς πολυχρονίους σήψεις προσέπλ. καὶ συνέπλεξαν αὐτὰς ταῖς λοιπαῖς αἰθάλαις · ἡμεῖς δὲ καλοῦμεν αὐτὰ σώματα κ. θ.} < ἐκ > τῶν πολυχρονίων σήψεων. Καὶ προσέπλεξαν < μετὰ > τῶν λοιπῶν αἰθαλῶν, ἃς καλοῦσι σώματα, καὶ ἡμεῖς σῶμα, καὶ θεῖα,\footnote{σῶμα] σώματα A\textsuperscript{1}.} καὶ θειώδη, καὶ πέταλα χαλκοῦ ἢ ἀσήμου ἢ χρυσοῦ. Καὶ οὕτως εἰργάσαντο\footnote{ἀσήμου] ἀργύρου en signe A\textsuperscript{1} ; en toutes lettres BA\textsuperscript{2} K Lc. --- καὶ οὕτως] τινὲς δὲ Lc.} τὴν βαφὴν ἐπὶ τῶν ὑπηρετικῶν ὑλῶν, τῆς δευτέρας αὐτῶν ὑποστάθμης οὐδένα ἀποτίσαντες λόγον.

2. Καὶ ἀπέδειξεν τὸ διὰ τῶν τεφρῶν ἀποσταζόμενον ὕδωρ, εἰπών\footnote{καί τις ἀπέδειξε Lc.} · « Καὶ θεὶς τὸ ὄργανον, ἀνακομίζου τὰς τέφρας. » Εἰ οὖν ἡ τέφρα ἐστὶ\footnote{εἰ --- ἐστὶ] ἡ τέφρα τοίνυν ἐστὶ Lc.} τὸ διοργανισθὲν ὕδωρ, καὶ διὰ τοῦτο καὶ ὁ Ἀγαθοδαίμων · « Ὅλως\footnote{ὁ Ἀγαθ. φησί Lc.} ἡ τέφρα ἐστιν. » Ἕψησις δὲ αὕτη τυγχάνει ἢ καὶ ὄπτησις, ἥτις λείωσις\footnote{Au lieu de ἥτις, M donne un trait surmonté de 2 points : $\svgC$. --- ἡ καὶ ὄπτ. ἥτις καὶ ... Lc.} ὀνομάζεται · δηλονότι διὰ σήψεως, καὶ ἀνασπάσεως, καὶ ἰώσεως,\footnote{ὀνομάζονται M. --- Réd. de Lc, après ce mot : Οἱ ἀρχαῖοι δέ φασι διὰ σήψεως, κ. ἀ. κ. ἰ. κ. παρ. τὸ πᾶν ἀπαρτίζ.} καὶ παροπτήσεως, λέγοντες οἰ ἀρχαῖοι τὸ πᾶν ἀπαρτίζεσθαι. Καὶ ἀδύνατόν ἐστιν ἄλλως οἰκονομεῖσθαι τὴν ποίησιν τοῦ συνθέματος. Τὴν γὰρ ἕψησιν,\footnote{τὴν ποίησιν] τὴν puis le signe de πυρίτης M ; τὸν puis le même signe BA\textsuperscript{1. 2.} K ; τὸν πυρίτην (en toutes lettres) Lc. Corr. conj.} καὶ ἀνάσπασιν λείωσιν οἴδασιν οἱ ὑποφῆται τῆς ἐπιστήμης · καὶ τὴν ἴωσιν, ἕψησιν, τὴν δὲ ἕψησιν καὶ ἀνάσπασιν λείωσιν ὠνόμασαν διὰ τὴν\footnote{ὠνόμασαν --- διὰ τὸ θερμ.] ὠνόμασαν · διὰ δὲ τὸ θερμ. Lc.} ἄγαν ἐκλέπτυνσιν. Καὶ πάλιν τὸ πῦρ ὠνόμασαν διὰ τὸ θερμαίνειν καὶ καίειν καὶ φωτίζειν < καὶ > παιδίου παίγνιον καὶ γυναικὸς ἔργον ἔφασαν\footnote{καὶ καίειν om. A\textsuperscript{1. 2.} K ; hab. B Lc.} οἱ παλαιοὶ τὸ ζητούμενον τοῖς νοήμοσιν. Ἀλλ ᾽ οὐ διὰ τοῦτο ἀναγκασθησόμεθα πάντως διὰ πυρὸς τὴν ἴωσιν κατεργάζεσθαι, ὡς ἐπὶ τῶν βαπτομένων λίθων, τουτέστιν ὑδάτων ἀναγωγῆς καὶ τὴν ἐκ ψυχρἄς τελουμένην\footnote{ἐπὶ τῆς τῶν βαπτ. λ. τ. ὑ. ἀγωγῆς Lc.} πορφύραν. Λέγω δὴ < ὅτι > σαφῶς ἡμᾶς ἡ πεῖρα διδάξει εἰ τὸ ἀληθὲς,\footnote{λέγω δὴ --- διδάξει] ἡ πεῖρα δὲ σαφῶς ἡμ. διδ. Lc.} ἓν ἔργον τέλειον καὶ ἄφευκτον ἐπι- (f. 117 v.) τελοῦσα ξηρίον.\footnote{ἐπὶ τέλους ξηρίον M.}

3. Μετὰ δὲ τὴν τούτου ἰοποίησιν, ἀνεκομίσαντο αἰθάλας, καὶ προσέπλεξαν\footnote{Réd. de Lc : Ἄλλοι δὲ μετὰ τὴν τ. ἰ. τὰς αἰθάλας B etc. --- καὶ προσέπλ. αὐτὰς τοῖς λειπομένοις σκωριδίοις Lc.} < μετὰ > τῶν λειπομένων σκωριῶν, καὶ οὕτως ἔσχον τὸ πέρας, ἐντεῦθεν ξηρίον τοῖς σώμασιν ἐπιβάλλοντες διὰ τὸ λέγειν\footnote{M mg. : groupe de points ; guillemets jusqu'à la fin du §.} Ζώσιμον · « Οὕτω γὰρ τὰ μὲν πνεύματα σωματοῦνται, τὰ δὲ νεκρὰ σώματα ἐμψυχοῦνται, τῆς ἀπ ᾽ αὐτῶν ψυχῆς πάλιν αὐτοῖς εἰσκριθείσης,\footnote{τῆς] καὶ τῆς Lc.} καὶ θεῖον ἔργον ἀποτελοῦσιν, ἀμφότερα ἄλληλα κρατοῦντα καὶ\footnote{ἀποτελούσης Lc. --- κρατοῦνται B etc.} ὑπ ᾽ ἀλλήλων κρατούμενα. Τὸ γὰρ φεῦγον πνεῦμα τοῦ διώκοντος\footnote{κρατ. εὑρίσκονται Lc.} σώματος ἔτυχεν, διδαχθέντος ἤδη πυριμάχειν ἐν τῷ πυρί. Καὶ τοῦτο ἐστιν, ὡς οἶμαι, τὸ τοῦ φιλοσόφου ὕδωρ ἀσβέστου ἢ σανδαράχης, ὕδωρ νίτρου, ὕδωρ φέκλης, τὸ ἀπὸ τῆς τέφρας τῶν θειωδῶν σκευαζομένων,\footnote{σκευαζόμενον Lc.} ὕδωρ πρωτόστακτον. »

4. Δεῖ οὖν αὐτὴν ἀποστάζειν ὡς τὴν σαπωναρικὴν στάκτην, καὶ ἔχειν αὐτῆς τὰ ὕδατα · σαπωναρικὴ δὲ, φησὶ, στάκτη οὐδέποτε ἐξαιθαλοῦται, ἀλλὰ καταστάζεται. Πῶς οὖν, ὦ ἀγαθοὶ, Ζώσιμός φησιν ὅτι\footnote{Πῶς οὗν, ὦ φιλόσοφοι, φησὶν Δημόκριτος BA\textsuperscript{1. 2.} K. --- ὁ Ζώσιμος δέ φησιν Lc.} οὐδαμοῦ ἕστηκεν ὁ νοῦς τῶν γραφῶν, εἰ μὴ ἐν τῷ ὀργανισμῷ τῷ ἀνασπῶντι\footnote{διοργανισμῳ A\textsuperscript{2} K ; διοργανισμοῦ Lc.} τὸν χαλκόν · καὶ ὅτι τὸ πέρας τῆς τέχνης ὧδε οὐκ ἦν, ἀλλ ᾽ ἐν\footnote{ἦν] ἔστι Lc.} τῷ διοργανισμῷ καὶ τῇ τούτου πήξει. Ἕτεροι δὲ μόνον τοῖς ληκύθοις\footnote{λεκήθοις M ; λεκύνθοις BA\textsuperscript{1. 2.} ; λεκίνθοις K ; λεκύθοις Lc. Corr. conj.} ἕχρισαν ἐπ ᾽ ἄμφω τῷ συνθέματι, καὶ ἀνακομισάμενοι τὸ ὕδωρ, προσέπλεξαν\footnote{ἐχρήσαντο B etc. F. l. λεκίθοις ἔχρισαν ( ? ) Cp. ci-après 4, 4, 15. --- ἄμφω] ἀμφοτέρῳ Lc, mel.} τῇ οἰκείᾳ ἀσβέστῳ λειώσαντες ἐν θυείᾳ, οὐ σταθμῷ, ἀλλ ᾽ ὅσον\footnote{θυΐα mss.} ὑπερέχει τὸ ξηρὸν τοῦ ὑγροῦ, δακτύλους δύο, (f. 118 r.) ἢ τρεῖς, ἢ τέσσαρας.\footnote{ὑπερέχοι ἂν Lc.}

\bigskip
\centerline{\EightStarTaper}
\centerline{\EightStarTaper\EightStarTaper}
\bigskip
\clearpage
\setcounter{footnote}{0}
\section{Traduction.}
\subsection{Troisième Partie. --- Zosime.}
\subsubsection[3. --- 1. Le Divin Zosime sur la Vertu. --- Leçon 1.]{3. --- 1. Le Divin Zosime sur la Vertu.\footnote{AK : « Sur la vertu et la composition des Eaux. »} --- Leçon 1.}
\paragraph{}
1. La composition des eaux, le mouvement, l'accroissement, l'enlèvement et la restitution de la nature corporelle, la séparation de l'esprit d'avec le corps,\footnote{Séparation des métaux d'avec les corps volatils, tels que le soufre ou l'arsenic, auxquels ils sont associés.} et la fixation de l'esprit sur le corps ; les opérations qui ne résultent pas de l'addition de natures étrangères et tirées du dehors, mais qui sont dues à la nature propre, unique, agissant sur elle-même, dérivée d'une seule espèce, ainsi que (l'emploi) des minerais durcis et solidifiés, et des extraits liquides du tissu des plantes ; tout ce système uniforme et polychrome comprend la recherche multiple et infiniment variée de toutes choses, la recherche de la nature, subordonnée à l'influence lunaire et à la mesure du temps, lesquelles règlent le terme et l'accroissement suivant lesquels la nature se transforme.

2. En disant ces choses, je m'endormis ; et je vis un sacrificateur qui se tenait debout devant moi, en haut d'un autel en forme de coupe.\footnote{Ou de fiole (voir les appareils distillatoires des fig. 11, 14, etc., \emph{Introd.}, p. 132, 138 et suiv. ; ou plutôt les appareils à kérotakis des fig. 20, 21 et suiv., \emph{Introd.}, p. 143 et suiv.). Tout ceci est la description mystique de diverses opérations chimiques de distillation, de sublimation, de coupellation, accompagnées de grillages, d'effervescences et de changements de couleur.} Cet autel avait quinze marches à monter. Le prêtre s'y tenait debout, et j'entendis une voix d'en haut qui me disait : « J'ai accompli l'action de descendre les quinze marches, en marchant vers l'obscurité, et l'action de monter les marches, en allant vers la lumière. C'est le sacrificateur qui me renouvelle, en rejetant la nature épaisse du corps. Ainsi consacré prêtre par la nécessité, je deviens un esprit. »

Ayant entendu la voix de celui qui se tenait debout sur l'autel en forme de coupe, je lui demandai qui il était. Et lui, d'une voix grêle, me répondit en ces termes : « Je suis Ion,\footnote{L : « Je suis celui qui est » : ὢν au lieu de Ἴων.} le prêtre des sanctuaires, et je subis une violence intolérable. Quelqu'un est venu au matin précipitamment, et il m'a violenté, me pourfendant avec un glaive, et me démembrant, suivant les règles de la combinaison. Il a enlevé toute la peau de ma tête, avec l'épée qu'il tenait (en main) ; il a mêlé les os avec la chair\footnote{Voir le \emph{serpent Ouroboros}, p. 23.} et il les a fait brûler avec le feu du traitement. C'est ainsi que j'ai appris, par la transformation du corps, à devenir esprit. Telle est la violence intolérable (que j'ai subie). » Comme il m'entretenait encore, et que je le forçais de me parler, ses yeux devinrent comme du sang, et il vomit toutes ses chairs. Et je le vis (changé en) petit homme contrefait, se déchirer lui-même avec ses propres dents, et s'affaisser.

3. Rempli de crainte, je m'éveillai et je songeai : « N'est-ce-pas là la composition des eaux ? » Je fus persuadé que j'avais bien compris ; et je m'endormis de nouveau. Je vis le même autel en forme de coupe, et, à la partie supérieure, de l'eau bouillonnante et beaucoup de peuple s'y portant sans relâche.\footnote{Allégorie de la condensation des vapeurs dans le récipient supérieur.} Et il n'y avait personne que je pusse interroger en dehors de l'autel. Je monte alors vers l'autel, pour voir ce spectacle. Et j'aperçois un petit homme, un barbier blanchi par les années, qui me dit : « Que regardes-tu ? » Je lui répondis que j'étais surpris de voir l'agitation de l'eau et celle des hommes brûlés et vivants. Il me répondit en ces termes : « Ce spectacle que tu vois, c'est l'entrée, et la sortie, et la mutation. » Je lui demandai encore : « Quelle mutation ? » Et il me répondit : « C'est le lieu de l'opération appelée macération ; car les hommes qui veulent obtenir la vertu entrent ici et deviennent des esprits, après avoir fui le corps. » Alors je lui dis : « Et toi es-tu un esprit ? » Et il me répondit : « Oui un esprit et un gardien d'esprits. » Pendant notre entretien, l'ébullition allant en croissant, et le peuple poussant des cris lamentables, je vis un homme de cuivre, tenant dans sa main une tablette de plomb.\footnote{Allégorie du molybdochalque, placé sur la kérotakis, ou la constituant.} Il me dit les mots suivants, en regardant la tablette : « Je prescris à tous ceux qui sont soumis au châtiment de se calmer, de prendre chacun une tablette de plomb, d'écrire de leur propre main, et de tenir les yeux levés en l'air et les bouches ouvertes, jusqu'à ce que leur vendange\footnote{Voir plus loin la vendange d'Hermès, p. 129, note 1.} soit développée. » L'acte suivit la parole et le maître de la maison me dit : « Tu as contemplé, tu as allongé le cou vers le haut et tu as vu ce qui s'est fait. » Je lui répondis que je voyais, et il me dit : « Celui que tu vois est l'homme de cuivre ; c'est le chef des sacrificateurs et le sacrifié, celui qui vomit ses propres chairs. L'autorité lui a été donnée sur cette eau et sur les gens punis. »

4. Après avoir eu cette apparition, je m'éveillai de nouveau. Je lui dis : Quelle est la cause de cette vision? N'est-ce donc pas là l'eau blanche et jaune bouillonnante, l'eau divine? Et j'ai trouvé que j'avais bien compris. Je dis qu'il est beau de parler et beau d'écouter, beau de donner et beau de recevoir, beau d'âtre pauvre et beau d'être riche. Or, comment la nature apprend-elle à donner et à recevoir? L'homme de cuivre donne et la pierre liquéfiée reçoit ; le minéral donne et la plante reçoit ; les astres donnent et les fleurs reçoivent ; le ciel donne et la terre reçoit ; les coups de foudre donnent le feu qui s'élance. Dans l'autel en forme de coupe, toutes choses s'entrelacent, et toutes se dissocient ; toutes choses s'unissent ; toutes se combinent ; toutes choses se mêlent, et toutes se séparent ; toutes choses sont mouillées, et toutes sont asséchées ; toutes choses fleurissent et toutes se déflorent. En effet, pour chacune c'est par la méthode, par la mesure, par la pesée exacte des quatre éléments que se fait l'entrelacement et la dissociation de toutes choses ; aucune liaison ne se produit sans méthode. Il y a une méthode naturelle, pour souffler et pour aspirer, pour conserver les classes stationnaires, pour les augmenter et pour les diminuer. Lorsque toutes choses, en un mot, concordent par la division et par l'union, sans que la méthode soit négligée en rien, la nature est transformée ; car la nature, étant retournée sur elle-même, se transforme : il s'agit de la nature et du lien de la vertu dans l'univers entier.

5. Bref, mon ami, bâtis un temple monolithe, semblable à la céruse, à l'albâtre, n'ayant ni commencement ni fin dans sa construction. Qu'il y ait à l'intérieur une source d'eau très pure, étincelante comme le soleil. Observe avec soin de quel côté est l'entrée du temple et prends en main une épée ; cherche alors l'entrée, car il est étroit le lieu où se trouve l'ouverture. Un serpent est couché à l'entrée, gardant le temple. Empare-toi de lui ; tu l'immoleras d'abord ; dépouille-le, et prenant sa chair et ses os, sépare ses membres ; puis réunissant les membres avec les os, à l'entrée du temple, fais-en un marchepied, monte dessus, et entre : tu trouveras là ce que tu cherches. Le prêtre, cet homme de cuivre, que tu vois assis dans la source, rassemblant (en lui) la couleur, ne le regarde pas comme un homme de cuivre ; car il a changé la couleur de sa nature et il est devenu un homme d'argent. Si tu le veux, tu l'auras bientôt (à l'état d') homme d'or.\footnote{\emph{Origines de l'Alchimie}, p. 180. Voir le \emph{serpent Ouroboros}, 1. 4, 5, p. 23. --- Ce § répète au fond, sous une forme plus sommaire et avec une allégorie moins compliquée le § 2.}

6. Ce préambule est une entrée destinée à te manifester les fleurs des discours qui vont suivre (c'est-à-dire) la recherche des vertus, du savoir, de la raison, les doctrines de l'intelligence, les méthodes efficaces, les révélations qui éclaircissent les paroles secrètes. Ainsi la vertu poursuit le Tout, en son temps et avec méthode.

7. Que signifient ces mots : « La nature triomphant des natures ? » et ceci : « Au moment où elle est accomplie, elle est prise de vertige ? » et encore : « Resserrée dans la recherche, elle prend le visage commun de l'œuvre du Tout, et elle absorbe la matière propre de l'espèce ? » Et ceci : « tombée ensuite en dehors (de) sa première apparence, elle croit mourir ? » Et ceci : « Lorsque, parlant une langue barbare, elle imite celui qui parle la langue hébraïque ; alors, se défendant elle-même, la malheureuse se rend plus légère en mélangeant ses propres membres. ? » Et ceci : « L'ensemble liquide est mené à maturité par le feu ? »

8. Appuyé sur la clarté de ces conceptions de l'intelligence, transforme la nature, et considère la matière multiple comme étant une. N'expose clairement à personne une telle propriété ; mais suffis-toi à toi-même, de crainte qu'en parlant, tu ne te détruises toi-même. Car le silence enseigne la vertu. Il est beau de voir les mutations des quatre métaux [le plomb, le cuivre l'asèm (ou l'argent), l'étain], changés en or parfait.

Prenant du sel, mouille le soufre, de façon à amener la masse en consistance de cire mielleuse. Enchaîne la force de l'un et l'autre ; ajoutes-y de la couperose et fabriques-en un acide, premier ferment de la couleur blanche, tiré de la couperose. Avec ces (substances) tu amèneras par degré le cuivre dompté à l'apparence blanche. Fais distiller par la cinquième méthode, au moyen des trois vapeurs sublimées : tu trouveras l'or attendu. Voilà comment en domptant la matière tu obtiens l'espèce unique, tirée de plusieurs espèces.\footnote{Cet alinéa est une addition étrangère à ce qui précède. C'est une recette pour attaquer le cuivre, avant de faire agir sur lui les vapeurs destinées à le teindre.}

\bigskip
\centerline{\EightStarTaper}
\centerline{\EightStarTaper\EightStarTaper}
\bigskip

\subsubsection[3. --- 2. La Chaux.]{3. --- 2. La Chaux.\footnote{Cet article se compose d'une suite de recettes obscures pour fabriquer la pierre philosophale. Les dernières sont postérieures à Zosime, comme l'indique la citation de Stephanus tirée de A (§ 2 \emph{bis}) ; à l'exception pourtant de la phrase finale du § 3, laquelle exprime très clairement la formation des sous-sels de cuivre, ou fleurs de cuivre.}}
\paragraph{}
Zosime dit au Sujet de la Chaux :

1. Je vais vous rendre (les choses) claires. On sait que la pierre alabastron\footnote{\emph{Lexique}, p. 4.} est appelée cerveau,\footnote{Voir \emph{Lexique}, p. 7 ; \emph{Œuf philosophique}, p. 19. --- \emph{Nomenclature de l'œuf}, p. 21.} parce qu'elle est l'agent fixateur de toute teinture volatile. Prenant donc la pierre alabastron, fais-la cuire une nuit et un jour ; aie de la chaux, prends du vinaigre très fort et fais bouillir : tu seras étonné ; car tu réaliseras une fabrication divine, un produit qui blanchit au plus haut degré la surface (des métaux). Laisse déposer, puis ajoute du vinaigre très fort, en opérant dans un vase sans couvercle, afin d'enlever la vapeur sublimée, à mesure qu'elle se forme au-dessus. Prenant encore du vinaigre fort, fais élever cette vapeur pendant sept jours, et opère ainsi jusqu'à ce que la vapeur ne monte plus. Laisse durant quarante jours le produit (exposé) au soleil et à la rosée, à l'époque fixée ; puis adoucis avec de l'eau de pluie. Fais sécher au soleil, et conserve.

C'est là le mystère incommuniqué, qu'aucun des prophètes n'a osé divulguer par la parole ; mais ils l'ont révélé seulement aux initiés. Ils l'ont appelé la pierre encéphale dans leurs écrits symboliques, la pierre non-pierre, la chose inconnue qui est connue de tous, la chose méprisée qui est très précieuse, la chose donnée et non-donnée de Dieu.\footnote{Voir la note de la p. 19.} Pour moi, je la saluerai du nom de (pierre) non donnée et donnée de Dieu : c'est la seule, dans notre œuvre, qui domine la matière. Telle est la préparation qui possède la puissance, le mystère mithriaque.

2. L'esprit du feu s'unit avec la pierre et devient un esprit de genre unique. Or je vous expliquerai les œuvres de la pierre. Mélangée avec la comaris, elle produit les perles, et c'est là ce que l'on a nommé chrysolithe. L'esprit opère toutes choses par la puissance de la poudre sèche. Et moi, je vais vous expliquer le mot \emph{comaris}, chose que personne n'a osé divulguer ; mais ceux-ci (les anciens) la transmettaient aux personnes intelligentes. Elle détient la puissance féminine, celle que l'on doit préférer ; car le blanchiment est devenu un objet de vénération pour tout prophète.

Je vous expliquerai aussi la puissance de la perle. Elle accomplit ses œuvres, mise en décoction dans l'huile. Elle représente la puissance féminine. Prenant la perle, tu la mettras en décoction avec de l'huile, dans un vase non bouché, sans couvercle, pendant 3 heures, sur un feu modéré. Prenant un chiffon de laine, frotte-le contre la perle, afin d'en ôter l'huile et tiens, (la perle disponible) pour les besoins des teintures ; car l'accomplissement de la (transformation) matérielle a lieu au moyen de la perle.

2 \emph{bis}. Stephanus\footnote{Cet alinéa manque dans M ; il est tiré de A. Il a été reporté plus loin dans le \emph{Texte grec}, 4, 20, 13, Traité de Comarius. On l'a conservé ici, parce qu'il indique comment les fragments de Zosime ont été augmentés par l'addition successive de morceaux étrangers. --- Le nom de Stephanus, appliqué à l'auteur d'un morceau tiré d'un traité de Comarius, mérite aussi attention : car il prouve que la confusion signalée dans l'\emph{Introd.}, p. 182, entre les œuvres de ces deux auteurs est fort ancienne.} dit : Prenez (le métal composé) des quatre éléments, (ajoutez-y l'arsenic le plus élevé\footnote{Qui s'est sublimé, en s'oxydant, à la partie supérieure du récipient ?} et le plus bas, le rugueux et le roux, le mâle et la femelle, à poids égaux, afin de les unir entre eux. Car de même que l'oiseau couve ses œufs et les mène à terme dans la chaleur, de même vous couverez et mènerez à terme votre œuvre,\footnote{L'œuf philosophique.} après l'avoir porté au dehors, arrosé avec les eaux divines, exposé au soleil et dans des lieux chauds ; après l'avoir fait cuire sur un feu doux, en le déposant dans du lait virginal.\footnote{Expression symbolique. D'après le \emph{Lexicon Alchemiæ Rulandi} (p. 272), c'est l'eau mercurielle, le mercure des philosophes, etc.} Prenez garde à la fumée. Plongez le produit dans l'Hadès\footnote{Fond des vases où les résidus s'accumulent et sont exposés directement à l'action du feu ; comme le montrent, par exemple, les fig. 20 et 21 de l'\emph{Introd.}, p. 143.} ; [ressortez-le, arrosez-le avec du safran de Cilicie, au soleil et dans des lieux chauds ; faites cuire sur un feu doux, avec du lait virginal, en dehors de la fumée. Enfoncez-le dans l'Hadès\footnote{Ceci est une répétition ; quelque copiste ayant mis bout à bout deux versions parallèles.}]. Remuez avec soin, jusqu'à ce que la préparation ait pris de la consistance, et ne puisse s'échapper du feu. Alors, prenez-en (une partie), et lorsque l'âme et l'esprit se sont unifiés (avec le corps) et ne forment plus qu'un seul être, projetez sur le corps métallique de l'argent et vous aurez de l'or, tel que n'en renferment pas les trésors des rois.

Voilà le mystère des philosophes, celui que nos pères ont juré de ne point révéler ni publier.

3. On entend par élévation, la montée des fleurs\footnote{Fleurs métalliques, se formant à la surface des métaux par oxydation, ou se sublimant (voir page 71, note 4).} : l'eau avec laquelle le produit a été arrosé s'élève et monte sans obstacle, par suite de l'association intime du corps avec le soufre.\footnote{On propose de lire : soufre, au lieu de plomb ; le signe étant pareil (voir le \emph{Texte grec}, p. 114, note de la ligne 23).} Sinon (le corps) reste au fond (du vase à sublimation ? ) Contentons-nous du mortier et du filtre pour les deux teintures.

Quant au cuivre, Zosime dit à son sujet : « Altéré par la plupart des eaux, à cause de l'humidité de l'air et de la chaleur, il augmente de volume et se couvre de fleurs, qui sont de beaucoup les plus douces ; il fructifie par l'action productrice de la nature. »

\bigskip
\centerline{\EightStarTaper}
\centerline{\EightStarTaper\EightStarTaper}
\bigskip

\subsubsection{3. --- 3. Agathodémon.}
\paragraph{}
Après l'affinage du cuivre et son noircissement, puis son blanchiment ultérieur, alors aura lieu le jaunissement solide.

\bigskip
\centerline{\EightStarTaper}
\centerline{\EightStarTaper\EightStarTaper}
\bigskip

\subsubsection{3. --- 4. Hermès.}
\paragraph{}
Si tu ne dépouilles pas les corps de leur nature corporelle et si tu ne donnes pas une nature corporelle aux êtres incorporels, rien de ce que tu attends n'aura lieu.\footnote{Cet axiome a été attribué aussi à Marie (ce volume, p. 101), et à d'autres alchimistes. Il signifie d'une part ôter aux métaux purs ou alliés leur corps, ou forme métallique, sous laquelle ils sont fixes d'ordinaire : ce que l'on réalisait en les soumettant à la sublimation, qui rend le zinc, l'antimoine et même le plomb et le cuivre volatils (c'est à-dire esprits), dans l'état d'oxydes (par l'action de l'air), de sulfures (par l'action du soufre ou des sulfures), de chlorures (par l'action du sel marin), etc. D'autre part on leur restitue leur corps, c'est-à-dire on rétablit ces chlorures, oxydes, sulfures, dans l'état métallique avec des propriétés et une coloration nouvelles, dues soit à leur purification, soit au contraire à la formation des alliages. --- On lit de même dans le traité attribué à Avicenne (\emph{Bibl. chem.} de Manget, t. 1, p. 629) : \emph{ut corporeum fiat spirituale sublimando et cum est spirituale, fiat iterate corporeum descendendo}.}

\bigskip
\centerline{\EightStarTaper}
\centerline{\EightStarTaper\EightStarTaper}
\bigskip

\subsubsection{3. --- 5. Zosime. --- Leçon 2.}
\paragraph{}
1. Enfin je fus pris du désir de monter les sept degrés et de voir les sept châtiments ; et comme il convient, en un seul des jours (fixés), j'effectuai la route de l'ascension. En m'y reprenant à plusieurs reprises, je parcourus la route. Au retour, je ne retrouvai pas mon chemin. Plongé dans un grand découragement, ne voyant pas comment sortir, je tombai dans le sommeil.

J'aperçus pendant mon sommeil un certain petit homme, un barbier revêtu d'une robe rouge et d'un habillement royal, qui se tenait debout en dehors du lieu des châtiments, et il me dit : Que fais-tu (là), ô homme ? Et moi je lui répondis : Je m'arrête ici parce que, m'étant écarté de tout chemin, je me trouve égaré. Il me dit (alors) : Suis-moi. Et moi, je vins et je le suivis. Comme nous étions près du lieu des châtiments, je vis celui qui me guidait, ce petit barbier, s'engager dans ce lieu et tout son corps fut consumé par le feu.

2. A cette vue, je m'éloignai, je tremblai de peur ; puis je me réveillai, et je me dis en moi-même : Qu'est-ce que je vois ? et de nouveau je tirai mon raisonnement au clair et je compris que ce barbier était l'homme de cuivre, revêtu d'un habillement rouge, et je (me) dis : J'ai bien compris, c'est l'homme de cuivre. Il faut d'abord qu'il s'engage dans le lieu des châtiments.

3. De nouveau mon âme désira monter le 3\textsuperscript{e} degré. Et de nouveau, seul, je suivis le chemin ; et comme j'étais près du lieu des châtiments, je m'égarai encore, ne sachant pas ma route, et je m'arrêtai désespéré. Et de nouveau, semblablement, je vis un vieillard blanchi par les années, devenu tout à fait blanc, d'une blancheur aveuglante. Il s'appelait Agathodémon. Se retournant, ce vieillard aux cheveux blancs me considéra pendant une grande heure. Et moi je lui demandai : Montre-moi le droit chemin. Il ne se retourna pas vers moi, mais il s'empressa de suivre sa propre route. En allant et venant, de ci, de là, je gagnai en hâte l'autel. Lorsque je fus arrivé en haut sur l'autel, je vis le vieillard aux cheveux blancs s'engager dans le lieu du châtiment. O démiurges des natures célestes ! Comme il fut aussitôt embrasé tout entier ! Quel récit effroyable, mes frères ! Car, par suite de la violence du châtiment, ses yeux se remplirent de sang. Je (lui) adressai la parole et lui demandai : Pourquoi es-tu étendu ? Mais lui, ayant entr'ouvert la bouche, me dit : « Je suis l'homme de plomb et je subis une violence intolérable.\footnote{Dans le § 3, il semble s'agir de la calcination de la litharge blanche, opération qui la change en minium rouge. Peut-être aussi est-ce la coupellation.} » Là-dessus, saisi d'une grande crainte, je m'éveillai et je cherchai en moi-même la raison de ce fait. De nouveau je réfléchis et je me dis : J'ai bien compris par là qu'il faut rejeter le plomb ; la vision se rapporte réellement à la composition des liquides.

\bigskip
\centerline{\EightStarTaper}
\centerline{\EightStarTaper\EightStarTaper}
\bigskip

\subsubsection{3. --- 5 bis. Ouvrage du même Zosime. --- Leçon 3.}
\paragraph{}
1. De nouveau, je remarquai le divin et sacré autel en forme de coupe, et je vis un prêtre revêtu d'une (robe) blanche, tombant jusqu'à ses pieds, lequel célébrait ces effrayants mystères, et je dis : Quel est celui-ci ? Et il me répondit : C'est le prêtre des sanctuaires. C'est lui qui a l'habitude d'ensanglanter les corps, de rendre les yeux clairvoyants et de ressusciter les morts. Alors, tombant de nouveau (à terre), je m'endormis encore. Pendant que je montais le quatrième degré, je vis, du côté de l'orient, (quelqu'un) venir, tenant dans sa main un glaive. Un autre, derrière lui, portait un objet circulaire, d'une blancheur éclatante, et très beau à voir, appelé Méridien du Cinabre.\footnote{Le Cinabre est représenté ici dans AK, comme à l'ordinaire, par un cercle avec un point au milieu. --- Voir \emph{Introd.}, p. 108 ; Pl. 2, l. 13 ; et p. 122, note 1. --- Ce signe a été aussi le signe du soleil, et plus tard de l'or.} Comme j'approchais du lieu du châtiment, il me dit que celui qui tenait un glaive, devait lui trancher la tête, sacrifier son corps et couper ses chairs par morceaux, afin que ses chairs fussent d'abord bouillies dans l'appareil, et qu'alors elles fussent portées au lieu du châtiment. M'étant réveillé de nouveau, je (me) dis : j'ai bien compris ; il s'agit des liquides dans l'art des métaux. Celui qui portait le glaive dit encore : Vous avez accompli l'ascension des sept degrés. L'autre reprit, en même temps qu'il laissait dissoudre les plombs par tous les liquides ( ? ),\footnote{Il semble qu'il s'agisse de l'absorption de la litharge fondue par les parois de la coupelle.} : « l'Art s'accomplit. »

\bigskip
\centerline{\EightStarTaper}
\centerline{\EightStarTaper\EightStarTaper}
\bigskip

\subsubsection[3. --- 6. Le Divin Zosime sur la Vertu et l'Interprétation.]{3. --- 6. Le Divin Zosime sur la Vertu et l'Interprétation.\footnote{Cet article est formé par une suite de notices et de commentaires, d'époques diverses. Les premiers sont de Zosime ; puis viennent des §§ qui rappellent le Chrétien, Stephanus et d'autres auteurs byzantins plus modernes encore, de plus en plus subtils et alambiqués. On n'a pas cru utile d'en donner la traduction absolument complète, l'impression du texte suffisant amplement pour certains passages.}}
\paragraph{}
1. Pour obéir à son penchant et en vue d'expliquer le songe qu'il avait fait,\footnote{Ce début indique que le texte actuel est un extrait. En effet on lit dans ELc : « Commentaire du Philosophe Anonyme sur le traité du divin Zosime le Panopolitain (ou le Thébain), sur la Vertu, etc. »} il dit : Je vis un autel en forme de coupe ; un esprit igné, debout sur l'autel, présidait à l'effervescence, aux bouillonnements et à la calcination des hommes qui s'élevaient. Je m'informai, au sujet du peuple qui se tenait debout, et je dis : Je vois avec étonnement l'effervescence et le bouillonnement ; comment ces hommes en ignition sont-ils vivants ? Et me répondant, il me dit : Cette effervescence que tu vois, c'est le lieu où s'exerce la macération. Les hommes qui veulent obtenir la vertu entrent ici ; ils perdent leurs corps (et) deviennent des esprits. L'exercice (à la vertu) s'explique par-là, à cause du (mot) exercer\footnote{Il y a ici un jeu de mots intraduisible, qui rappelle le double sens français du mot macération, au sens chimique et au sens moral.} ; car, en rejetant l'épaisseur du corps, ils deviennent des esprits.

2. Démocrite dit quelque chose d'analogue : « Poursuis le traitement jusqu'à ce qu'il se forme un \emph{ios} jaune comme la couleur d'or, arrivant à l'état d'esprit au moyen de l'\emph{ios} » En effet, l'\emph{ios} provenant de la substance privée de corps, par l'action du serpent, signifie l'esprit.\footnote{Le même mot \emph{ios} signifie : rouille des métaux, vertu spécifique des corps et venin des serpents. (\emph{Introd.}, p. 254).} En raison de l'accomplissement de la coloration jaune, l'\emph{ios} est appelée couleur d'or. » C'est de cette façon qu'ils se transmettent leur pensée de vive voix et la proclament, jusqu'à ce qu'ils soient parvenus à une apparence uniforme. Et il poursuit : « Traite jusqu'à ce que tu puisses faire couler » --- faire couler vient de liquéfaction et non d'extraction, car ils changent la lettre σ en τ.\footnote{On peut interpréter ceci par un jeu de mots fondé sur la ressemblance des deux termes, ῥεῦσις, écoulement, et ῥεῦτις, extraction ?} --- Il dit ainsi : « Fais couler ; » ce qu'il entend de la liquéfaction, comme nous l'avons expliqué. Quant à ses paroles : « Fais le traitement, jusqu'à ce que tu puisses faire couler ; » ceci équivaut au mot employé plus haut d'écoulement simultané.\footnote{Voir \emph{Olympiodore}, p. 78, 101 et 113, notes.}

3. L'expression de sidérite,\footnote{Variété de Pyrite. --- Voir p. 47.} nom employé aussi par ceux qui sont signalés plus bas, désigne, conformément à ce qu'il rapporte : le molybdochalque et la pierre étésienne.

La pyrite, matière employée à cause de sa faculté colorante, après qu'elle a été brûlée ou soumise à l'action du feu, signifie le cuivre (tiré de la pyrite).

Semblablement le mot argyrite s'emploie pour la matière qui reste après l'expulsion du mercure ; car le cuivre débarrassé de l'excès du mercure devient de l'argyrite\footnote{C'est-à-dire est coloré en blanc d'argent.} ; tandis que la pierre étésienne est le mercure même, selon la vraie interprétation de l'ensemble des opérations ( ? ). En effet le départ du mercure annonce la prochaine apparition de la couleur d'or par le feu.

Il dit « sidérite » à cause de la nécessité de faire intervenir la combinaison du plomb. En effet les substances combinées produisent la sidérite.\footnote{Les §§ 3 et 4 sont formés par une suite de phrases, qui semblent presque indépendantes les unes des autres ; on dirait des lambeaux d'un vieil écrit, mis bout à bout.}

4. Semblablement, qu'est-ce que le cœur du fer ? Lorsque la masse est brisée, comme il arrive pendant cette extraction --- en employant les mots conformément aux analogies --- nous trouvons la théorie manifeste, et elle nous révèle le secret.

Dans d'autres passages, Démocrite dit : « Pratique le traitement avec la saumure additionnée de vinaigre ou d'urine, ou avec les deux réunis. » Entends d'ailleurs (comme tu le comprends d'après l'écrit, ou comme la chose y est expliquée), que la chose est possible en opérant avec d'autres liquides ; attendu que rien de tout cela ne demeure (dans la préparation), ces liquides étant déversés ensuite, lors du lavage de la composition.

5. C'est à ce sujet que le très ancien Ostanès, dans ses démonstrations, dit : Quelqu'un raconte ceci sur un certain Sophar, qui vécut antérieurement en Perse. Ce divin Sophar s'exprime ainsi : « Il existe sur un pilier un aigle d'airain,\footnote{Le sens du mot aigle dans ce passage est obscur. --- Au moyen âge, on traduisait « aigle » par sublimation naturelle (\emph{Biblioth. des Philosophes Chimiques}, t. 4, p. 571 ; 1754). Mais ce sens ne paraît pas être celui d'Ostanès.} qui descend dans la fontaine pure et s'y baigne chaque jour, se renouvelant par ce régime. » Puis il dit : « L'aigle, dont nous avons donné l'interprétation, a l'habitude de se baigner chaque jour. » Comment donc, faisant entendre la même chose d'une autre manière, rejette-t-il l'ablution et le lavage quotidien ? Il faut (s'expliquer) exactement au sujet de la présente opération. Tenu dans l'incertitude à cause de la doctrine (ambiguë) du philosophe, nous devons cependant laver et rajeunir l'aigle de cuivre pendant 365 jours entiers ; comme il convient d'après la suite de son traité, car Ostanès s'exprime ainsi : « Presse la vendange.\footnote{\emph{Uvæ Hermetis} : « Eau philosophique, désigne la distillation, la solution, la sublimation, la calcination, la fixation » (\emph{Lexicon Alch. Rulandi}, p. 468). --- Ce sens est plus étendu que ne paraît être celui d'Ostanès.} » Plus bas, il explique qu'il faut entendre par là\footnote{Lc : « Lave l'\emph{ios} plusieurs fois, au moyen de l'écoulement, et c'est là le mystère. »} le lavage par écoulement ; par ce mystère, on doit comprendre l'\emph{ios}. Il ajoute, en s'exprimant très clairement : « Va vers le courant du Nil ; tu trouveras là une pierre ayant un esprit ; prends-la, coup-là en deux ; mets ta main dans l'intérieur et tires-en le cœur : car son âme est dans son cœur. » Par l'expression : « Va vers le courant du Nil, tu trouveras là une pierre ayant un esprit ; » il désigne clairement les produits lavés par les courants (d'eau), pendant la macération de notre pierre. Voilà comment tout minerai de cuivre est employé pour la génération des métaux, ainsi que tout minerai de plomb. « Tu trouveras, dit-il, cette pierre qui a un esprit ; » ce qui se rapporte à l'expulsion du mercure.

6. C'est pour ces raisons que mon excellent (maître), Démocrite, distingue lui-même et dit : « Reçois cette pierre qui n'est pas une pierre, cette chose précieuse qui n'a pas de valeur, cet objet polymorphe qui n'a point de forme, cet inconnu qui est connu de tous, qui a plusieurs noms et qui n'a pas de nom\footnote{Voir page 19, note, 1, et \emph{Zosime}, 3, 2, p. 122.} : je veux parler de l'aphrosélinon. » Car cette pierre n'est pas une pierre, et tout en étant très précieuse elle n'a aucune valeur vénale ; sa nature est unique, son nom unique. Cependant on lui a donné plusieurs dénominations, je ne dis pas absolument parlant, mais selon sa nature ; de sorte que si on l'appelle soit : être qui fuit le feu, soit : vapeur blanche, soit : cuivre blanc, on ne ment pas.

Il dit qu'elle (se réduit entièrement) en nuage condensé, attendu qu'elle fuit le feu, à la différence de tous les autres corps métalliques ; c'est la vapeur sublimée du cinabre, et seule elle blanchit le cuivre. Fais-la donc chauffer doucement et éteins-la dans du lait d'ânesse ou de chèvre. [Rends-toi compte, après avoir opéré le rapprochement, qu'elle fuit le feu, à la différence de tous les autres corps ; que c'est la vapeur sublimée du cinabre, et que seule elle blanchit le cuivre.]

7. Comment les philosophes comprennent-ils cette pensée, à savoir que (Démocrite) appelle pierre, la pyrite débarrassée de son mercure? Cet excellent philosophe (dit) : « Qui ne sait que la vapeur sublimée du cinabre est le mercure? c'est par son moyen qu'il est fabriqué. C'est pourquoi si quelqu'un, après avoir délayé le cinabre dans l'huile de natron, après l'avoir mélangé et renfermé dans des vases doubles, l'expose ensuite à un feu continu, il recueillera toute la vapeur fixée par la chaleur sur les corps (métalliques).\footnote{Lc continue, en abrégeant tout ce passage : « On l'appelle Aphrosélinon, parce que cette pierre est produite par Aphrodite (Vénus), qui est le mercure, et par Séléné (la Lune), qui est l'argent. Car de même que la lumière, etc. » comme à la p. suivante § 8, l. 4.} »

Ainsi donc la pierre, je veux dire celle au moyen de laquelle on obtient la fixation sur le corps (métallique) de la magnésie, n'est pas une vraie pierre.\footnote{Attendu que les pierres ne sont pas volatiles. Cp. \emph{Bibl. Chem.} de Manget, t. 1, p. 935.} En effet, il est dans sa nature de s'écouler (par volatilisation).

Les uns disent que le mercure est une chose plus épaisse ; les autres, que le mercure est une chose plus spirituelle : attendu que, dans le déclin de la lune, il y a décroissement de la lumière\footnote{Le mercure est exprimé par le croissant retourné ; lequel exprime aussi la lune à son déclin. --- Voir la note 4, plus loin sur le croissant direct, et p. 133 la note 1, relative au déclin, ἀπορία, et à l'effluve, ἀπόρροια, tous deux assimilés à l'écoulement.} : Ce déclin ou écoulement résulte aussi de la nature propre de tous les autres astres. Jupiter seul est appelé d'abord électrum, pendant son ascension\footnote{Allusion au rôle des trois astres (Mercure, Vénus, la Lune) compris entre la terre et le soleil ; opposé à celui de Jupiter. Mercure ou Hermès représentait l'étain, Vénus le cuivre, la Lune l'argent ; tandis que l'électrum ou asèm, corps consacré à Jupiter (\emph{Introd.}, p. 82, et 97 ; pl. 1, l. 4, p. 104), était souvent formé par l'association de ces trois métaux ;--- voir \emph{Introd.}, p. 66.} ; tout électrum étant composé au moins de trois métaux.

8. Ainsi donc, dans son sens propre, l'argent répond à l'ascension de la lune\footnote{Le croissant direct, à concavité tournée vers la droite, exprime la lune dans ses premières phases, aussi bien que l'argent.} ; comme l'a montré l'excellent Philosophe, employant les dénominations exactes, au sujet des deux argents,\footnote{L'argent proprement dit et l'argent liquide, ou mercure.} et lorsqu'il dénomme l'aphrosélinon. De même que la lumière est vue en esprit à l'opposé de la lune, tandis qu'elle naît et meurt corporellement dans cet astre\footnote{Ceci se rapporte-t-il : d'une part, au fait que la lune brille d'une lumière empruntée, qu'elle ne produit pas elle-même ? et d'autre part, à l'opposition qui existe en général entre la Lune et le Soleil dans le ciel ?} ; de même aussi, naît et meurt le (vif) argent,\footnote{Le mercure.} tiré du corps (métallique) de la magnésie ; il est esprit quant à sa nature.

Nous trouvons encore des explications sur ces choses dans le traité de la Vertu en Action de Zosime ; car lui-même demande : « Et toi, tu es donc un esprit ? « Et celui-ci répond et dit : « Je suis esprit et gardien d'esprits.\footnote{Zosime, 3, 3, 3, p.119.} » En effet celui-ci étant esprit, en raison de la substance spirituelle qui réside dans la lune,\footnote{C'est-à-dire dans l'argent, ou dans le mercure.} il reprend un corps métallique par son union avec les solides ; et il fait à ce corps un esprit qui pénètre pour ainsi dire dans sa profondeur ... ... ... (passage inintelligible).

N'as-tu pas entendu, dit-il, proférer à haute voix cette parole souvent répétée : « Défends le cuivre, combats le mercure et rends tout à fait incorporel, jusqu'à destruction : tel est l'art. » Or il n'a rien employé pour cela, sauf le mercure et la magnésie, et ces deux substances sont réunies dans la fixation. « Prends, dit-il, le mercure (et) le corps (métallique) de la magnésie ; tu obtiens l'esprit par l'expulsion du mercure. » « On le trouve, dit-il encore, vers les courants du Nil ; » ce qui signifie l'écoulement simultané par fusion, comme il a été expliqué précédemment.\footnote{V. p. 78, 101 et p. 113, texte et note 1.} Alors, ainsi qu'il le dit : « rien ne manque, rien n'est ajourné, à l'exception de la vapeur\footnote{V. p. 57. Lc ajoute après ces mots « et de l'ascension de l'eau ; c'est-à-dire excepté ce que l'on peut voir et comprendre : car nous voyons le corps (métallique) de la magnésie ; et nous comprenons sa puissance, ainsi qu'il a été énoncé. »} ; » c'est-à-dire que l'opérateur peut, grâce à sa faculté de voir et de comprendre, voir et comprendre les choses énoncées.

9. En effet, que prescrit encore Hermès,\footnote{Lc abrège tout ce passage ainsi : « Hermès dit : ce qui tombe de l'effluve lunaire. De même que la lumière de la lune croît et décroît ; de même notre argent décroît en perdant son corps, d'une façon correspondante à la lune. L'émission ou l'absorption de l'esprit résulte de la force ou de la modération du feu, qui doit être réglé afin que l'esprit soit conservé, » etc. ; dernière ligne du § 9. --- Cp. Stephanus, édition Ideler, p. 203, au bas.} lorsqu'il parle de ce qui tombe de la lune à son déclin, et dit où (cela) se trouve, où on le traite et comment cela possède une nature qui résiste au feu ? « Tu le trouveras chez moi et chez Agathodémon. » Par l'expression déclin,\footnote{L'auteur joue sur la ressemblance des mots grecs qui signifient déclin (ἀπορία) et effluve (ἀπόρροια), mots que les manuscrits mêmes confondent et échangent. Tout ce langage allégorique semble exprimer le départ par volatilisation du mercure (lune à son déclin), mercure qui a servi à amalgamer et unir les métaux et qui laisse en partant un alliage couleur d'argent. Ces phénomènes étaient rattachés à l'influence lunaire. C'est un mélange d'alchimie et d'astrologie, fondé sur des symboles et des jeux de mots.} il parle de l'écoulement, et (cela) devient plus clair par l'addition de ces mots : « ce qui tombe au déclin lunaire ; » à ceux-ci : « la substance de la lune. » En effet, le corps demeure fixé par le déclin. La nature de la magnésie lunarisée acquiert ainsi en totalité le caractère spécifique de la lune,\footnote{C'est-à-dire de l'argent.} et se développe à l'occasion du déclin (qui répond à la volatilisation du mercure). De telle sorte que le principe actif tombe (de la lune) par ce déclin, le corps (métallique) demeurant transformé.

Revenons maintenant au déclin et à la faculté de voir et de pénétrer, qui résulte du déclin, du courant et de l'écoulement, conformément à la nature séparative de l'écoulement. Prends la magnésie traitée par l'art philosophique, en la brûlant par le feu, non pendant l'incandescence ; mais pendant le déclin du feu, afin que l'esprit soit conservé et qu'il ne s'évapore pas par la violence de l'incandescence.

10. Comprends ainsi ce que dit Ostanès : « Mets ta main à l'intérieur de la pierre, et tires-en le cœur, parce que son âme est dans son cœur.\footnote{Voir page 129.} » Ainsi donc, par un semblable déclin, cette pierre rejette tout ce qui est à l'intérieur, et le fond du cœur est rejeté ; de même que l'esprit, qui est l'\emph{ios} jaune, établi en principe comme la couleur d'or ; car ces choses sont en rapport avec ce que dit aussi Démocrite :

« Traite la pyrite jusqu'à ce qu'elle soit jaune comme la couleur d'or et vérifie si le métal devient sans ombre.\footnote{C'est-à-dire d'un jaune éclatant.} S'il ne devient pas sans ombre, ne t'en prends pas au cuivre, mais à toi-même : c'est que tu n'auras pas bien opéré. Traite donc jusqu'à ce que le cuivre, devenu jaune et sans ombre, teigne tout corps en or et devienne comme la couleur d'or. »

Il faut dès lors considérer et observer s'il devient jaune sans ombre, comme la couleur d'or : s'il ne devient pas sans ombre, il ne peut teindre en jaune comme la couleur d'or. En effet, il n'est pas d'or (ou doré) quant à sa qualité, puisque ce sont certaines qualités qui rendent jaune ; car le mot qualité\footnote{ποιότης.} a pour étymologie le mot fabriquer.\footnote{ποιεῖν. --- Jeu de mots sur qui veut dire la transmutation, la fabrication de l'or.} (Le jaune) produit une teinture, en raison de sa qualité dorée ; car il est évident que les actions exercées par les qualités sont en quelque sorte incorporelles. De là découle l'action de dorer ; attendu que si la couleur ne possède pas la qualité jaune\footnote{Le grec porte « blanche ; » ce qui semble une erreur de copiste.} dans sa propre substance, elle ne peut ni faire de l'or, ni teindre en or. Mais notre or, qui possède la qualité voulue, peut faire de l'or et teindre en or. C'est là le grand mystère, (à savoir) que la qualité devient or et alors elle fait de l'or.\footnote{En d'autres termes, la qualité « or » est indépendante de la substance métallique qui en est le support. Lorsqu'on possède une matière en laquelle cette qualité réside, à la façon du principe essentiel d'une matière colorante, c'est la pierre philosophale, et l'on peut alors teindre en or les autres métaux et faire par-là de l'or véritable. Toute la théorie des alchimistes réside dans ces notions subtiles.}

11. Voilà pourquoi la Couronne des philosophes\footnote{Zosime. Ce paragraphe est un commentaire du précédent. --- Lc dit simplement : « Stephanus ; » n'ayant pas compris la métaphore. En effet Στέφανος, couronne, est le même mot grec que le nom de ce dernier philosophe.} dit que la qualité, par la transmutation, réalise ce que l'on cherche. Il nous persuade et nous invite à l'interroger, disant : « Quelle est cette qualité ? » Il répond : « la qualité de la poudre de projection réside dans les qualités dorées. Si elle n'acquiert pas la qualité dorée et ne devient pas de l'or, possédant la couleur parfaite, elle ne peut faire de l'or. » Ainsi donc, comme il le dit, vérifie si le jaune est devenu sans ombre, c'est-à-dire (un être) incorporel, un \emph{ios} jaune comme la couleur d'or. Ce qu'il faut donc vérifier, c'est si le jaune est devenu sans ombre et paraît comme la couleur d'or.

Le commentateur poursuit en exposant des discussions subtiles et alambiquées, dont nous supprimons la traduction.

12. ... ...

Si tu commences par blanchir, le jaunissement sera parfait, parfait et solide. Dans le cas où il ne serait pas exact, il faut observer que le jaunissement dépend du degré de blanchiment : si le blanchiment passe, le jaunissement passe aussi.

13. Il sera nécessaire d'observer et de surveiller le blanchiment, et de le prolonger. Hermès exige que le lavage dure pendant six mois, à partir du mois de Méchir ; Ostanès, dans son traité, parlant de l'aigle, exige une année entière. Ajoutons que les philosophes œcuméniques, les savants modernes, les exégètes de Platon et d'Aristote, résumant le compte des dissolutions et des chauffes, disent : 2 fois 8 centaines et 3 fois 3 dizaines et 4 fois, montrant que onze cent (fois) la combinaison doit être remaniée, et décomposée, pour que le blanchiment devienne parfait et s'accomplisse en vue d'un jaunissement parfait et solide. Zosime disait encore plus expressément : « Ne craignez pas de multiplier les chauffes et les expulsions de l'eau\footnote{C'est-à-dire l'expulsion du principe de la liquidité.} des corps, attendu que la chauffe mille fois répétée du cuivre le rend plus apte à la teinture. »

On n'a pas traduit la fin de ce §, qui est un développement sans intérêt.

14. Il convient d'admirer le concours des qualités ; car les actions incorporelles effectuées par leur concours ont accompli cette merveilleuse Chrysopée, par la production d'une seule substance.

La chaleur du feu, la liquidité de l'eau, le froid de l'air, toutes qualités concourant avec la solidité de la terre, ont forcé le corps (métallique) de la magnésie de passer à la mutation et à la transformation. Où sont donc ceux qui disent qu'il est impossible de changer la nature ? Car voici que la nature des solides change et acquiert la qualité dorée ; de même le molybdochalque s'est changé, en prenant la qualité dorée, et s'est rapproché du noir ; de même l'argent commun se change par notre opération en or.

Les § 15, 16, 17 sont de pures subtilités, dont nous supprimons la traduction.

18. La présente composition part de l'unité, et se constitue en triade par l'expulsion du mercure ; l'unité de constitution résulte d'une triade à éléments sépares. C'est ainsi qu'une triade unique, partagée, constituée par des éléments séparés, constitue le monde, par la providence du premier auteur, cause et démiurge de la création. Par suite, il est appelé Trismégiste, ayant envisagé suivant la triade ce qui est fait et ce qui fait. Or ce qui est fait, c'est le molybdochalque (et la) pierre étésienne ; et ce qui fait, c'est le chaud, et le froid, et le fluide, d'abord triade première indivisible, et puis unité divisée.
\begin{center}
On juge inutile de donner la traduction du commencement du § 19.
\end{center}
\paragraph{}
19. ... ... (Zosime) dit en parlant de ces matières : « Brûlez le cuivre dans la composition blanche, » afin de vous détourner de toute autre cuisson ; (il veut) convaincre ceux qui brûlent au moyen du soufre, de l'arsenic, ou de la sandaraque, que l'on ne réussit pas avec ces matières. La pyrite chauffée avec elles ne devient pas blanche, mais noire, et ne peut plus ensuite être blanchie. Mais si on la chauffe avec la composition blanche, elle blanchit et est affinée par le lavage, ainsi qu'il a été écrit.

20. A la fin (la matière) est blanchie et jaunie, comme le dit Ostanès : « En même temps que vous blanchissez, vous jaunissez. Et Zosime dit : « Veillez à ne pas négliger le moment favorable au blanchissement : car à ce moment deux choses se produisent à la fois, le blanchissement et le jaunissement. » Rien n'est blanchi d'abord et jauni plus tard ; mais on blanchit et on jaunit dans une opération continue, suivant l'unité de cette composition trisubstantielle. Telle est la répartition triadique : Par le blanchiment, par la monade conjonctive, les trois substances sont blanchies et jaunies ; (tandis que) par la triade distinctive, elles sont désunies et s'écoulent. Le livre de Démocrite s'exprimait ainsi : « Traite avec la saumure, ou le vinaigre de saumure, ou comme tu l'imagineras. » Il déclare d'abord que le cuivre ne teint pas, mais que le cuivre brûlé par l'huile de natron, après avoir subi ce traitement à plusieurs reprises, devient plus beau que l'or. Le cuivre ne teint pas, tant qu'il conserve une essence unique ; mais il est teint par sa combinaison (avec d'autres corps). Comment donc sans cette combinaison et avant que le cuivre soit teint, pourrait-on réussir à teindre les objets soumis à l'action du feu ? Mais cela suffit pour montrer pourquoi la première opération ne réussit pas.

21. Quant à nous, nous remarquerons aussi que la cuisson par l'huile de natron a été mentionnée par le Philosophe, en opposition, comme réserve et pour se faire entendre. De même que celui qui regarde dans un miroir ne regarde pas les ombres, mais ce qu'elles font entendre, comprenant la réalité à travers les apparences fictives ; de même il s'est servi, pour se faire entendre, de l'expression « par l'huile de natron, » afin de nous faire comprendre la vérité. Voilà pourquoi au lieu des mots « vinaigre de natron » il emploie la dénomination « huile de natron. » Le métal est brûlé par la composition blanche, affiné, blanchi, lavé dans le vinaigre de natron. Dans celui-ci il est en même temps jauni, c'est-à-dire blanchi à l'extérieur et jauni à l'intérieur.

22. Il faut mettre (le métal) au feu, seulement pour l'échauffer, et prendre garde qu'il ne se produise de la fumée ; car s'il se produit de la fumée, la couleur disparaît.\footnote{Il s'agit : soit d'un métal ou d'un alliage, teint en jaune d'or avec le concours d'un composant volatil, tel que le mercure, le soufre ou l'arsenic ; soit d'un alliage jaune, analogue au laiton, renfermant un composant volatil au feu, tel que le zinc. Les termes du texte sont assez vagues pour comporter ces deux sens.} C'est dans ce sens que le libéral et excellent Démocrite ... dit au sujet du cuivre : « Ne le chauffe pas trop fortement, mon ami, de peur de lui faire perdre sa beauté ; ne l'expose jamais à la flamme du feu : ce n'est pas avantageux, car il se volatilise. Expose-le au feu, comme à l'action d'un soleil ardent ; conserve-lui toute sa matière sublimable et rends-le pareil au jaune d'œuf. » Nous interprétons (cet auteur, en admettant) que par l'expression : « ne le chauffe pas trop fortement et ne l'expose jamais à la flamme du feu ; » il rejetait de ce soufflage, toute calcination et toute action directe de la flamme. Dans cette vue il modère le feu et l'air, afin d'éviter la calcination qui sépare (les composants de l'alliage), et (il a recours) à un lut résistant au feu, bien feutré, pour enduire à l'extérieur les appareils, à deux ou trois reprises, afin d'éviter la calcination, tout en réalisant l'échauffement. Non seulement il se sert de ce lut, mais encore il prend soin d'enduire les interstices des compartiments des appareils.

De même que le Démiurge, après avoir séparé le firmament de l'élément liquide, place l'eau au-dessous du firmament ; de même l'opérateur prend soin des interstices, afin que dans les appareils la composition ne soit pas calcinée et ne se dissipe pas. De même encore que (le Démiurge) a ordonné que le soleil, en accomplissant son cours, passe au-dessus de tous les êtres délicats, (sans) brûler les corps vivants, les parties molles et les corps qui flottent à la surface ; de même l'opérateur a ordonné que l'air souffle du dehors et à travers, afin que ces corps refroidis par-là soient préservés de la combustion. Et cette intelligence démiurgique, opérant entre la composition supérieure et le feu mis au-dessous, dispose les choses de façon à tempérer l'action (du feu) sur les matières placées au-dessus. Deux fois huit centaines et trois fois trois dizaines et quatre, voilà combien de fois le feu doit être suspendu. C'est ainsi qu'il faut un grand tempérament, afin d'éviter que tout le produit ne soit brûlé et toute la partie liquide perdue. Car il dit : « Tout le liquide, par la violence de l'action du feu, serait perdu. »

23. Ainsi, toute la vapeur contenue dans la composition étant conservée et celle-ci devenue de la couleur du jaune d'œuf, passons à la seconde et grande macération. C'est celle qui transforme la nature, qui révèle la nature recélée dans la profondeur intime. A ce passage se rapporte le dire de Stephanus : « le but de la philosophie, c'est la dissolution du corps, la séparation de l'âme et du corps. » Ici voyons Démocrite disant : « Rien ne manque, il n'y a plus rien à exposer, excepté la montée de la vapeur et de l'eau.\footnote{V. page 57, § 29.} » Stephanus dit à son tour : « Il ne faut pas ... (phrase inintelligible). Mais nous enlevons les eaux qui surnagent, afin de voir sa beauté, de contempler la belle forme de la beauté ineffable, la grâce du trône d'or. Que faut-il donc faire ? Comment ferons-nous l'enlèvement de l'eau\footnote{Ou la montée de l'eau. Le Texte de Stephanus, tel que nous le possédons (\emph{Ideler}, t. 2, p. 207), est assez différent et beaucoup plus développé. Le fait de la citation de Stephanus montre qu'il s'agit d'un commentateur bien plus récent que Zosime.} ? » Mais si le feu est contraire au traitement des espèces, comment faut-il (faire) autrement ? dit-il ; si le métal ne peut être chauffé sans feu, que ferons-nous ? Opérerons-nous sans feu ? Et que sera un commencement n'ayant pas de fin, dans cette opération pratique que nous décrivons ? Que voulait donc dire notre philosophe, le maître le plus complet en toutes choses, ce professeur plein de sens ? Il n'a rien omis de ce qui tend à la pratique, sans le comprendre parmi les choses qui complètent son exposition. Voilà pourquoi il dit ici : « Prenant du plomb, je ne dis pas du plomb ordinaire, mais notre plomb, étends-le sur une largeur double. Après l'avoir disposé pour l'œuvre au moyen d'un outil, opère la montée de l'eau.\footnote{Ceci se rapporte à l'emploi de la kérotakis, où le métal est soumis à l'action des vapeurs (\emph{Introd.}, p. 143 et 144).} » Fais bien attention, dit-il : si tu es embarrassé, va en Égypte, et prenant un tissu épais, lave, presse ta vendange.\footnote{Voir la vendange d'Hermès, p. 129.} » Zosime s'explique aussi en disant : « Prenant du sel, extrais le soufre blanc, en mouillant avec un jus acide. » Stephanus dit : « Lorsque tu feras la composition avec la matière, il y aura une dépense excessive. »

24. Notre libéral et parfait Stephanus, le révélateur des mystères, (dit) : « Mets sur la nature morte\footnote{\emph{Caputa mortuum}, résidu demeuré au fond des alambics ; On lit dans la \emph{Turba} : « Illum igitur fumum suæ fæci misceto, donec coaguletur. » (\emph{Bibl. chem.} de Manget, t. 1, p. 449).} la vapeur sublimée, place (le mélange) dans un sac de lin très épais et exprime toute l'eau ; le superflu sera ainsi extrait plus vite. Mets du sel de Cappadoce en quantité égale, mouille avec une liqueur acide, jusqu'à ce que le produit ait pris une consistance pâteuse ; puis fais sécher, en broyant avec du vinaigre de natron. Celui qui opère ainsi est un homme parfait ; il suit la marche prescrite dans les ouvrages, la marche indirecte et détournée. » Pour celui qui préfère adopter une voie plus agréable et dépourvue de complications, il dit : « Prends du natron 2 parties ; de l'alun rond, 1 partie ; du misy, 2 parties ; du sel de Cappadoce, 4 parties ; mets dans du vinaigre très fort et fais une liqueur. A l'aide de ces (ingrédients) tu ôteras aux feuilles (métalliques) leur éclat. Une telle liqueur suffit pour le commencement et la fin de l'expérience. »

\bigskip
\centerline{\EightStarTaper}
\centerline{\EightStarTaper\EightStarTaper}
\bigskip

\subsubsection[3. --- 7. Sur l'Évaporation de l'Eau Divine (qui fixe le Mercure).]{3. --- 7. Sur l'Évaporation de l'Eau Divine (qui fixe le Mercure).\footnote{Addition de AB.}}
\paragraph{}
1. Me trouvant une fois dans vos demeures, ô femme,\footnote{Théosébie. --- \emph{Origines de l'Alchimie}, p. 9, 64.} afin de t'entendre, j'admirais toute l'opération de ce qui est appelé chez toi le « \emph{structeur}. » Je tombai dans une grande stupéfaction, à la vue de ces effets, et je me mis à vénérer comme divin le \emph{poxamos}\footnote{Ce sont là sans doute des noms d'instruments. --- AB disent : \emph{paxamos}, le fixateur ( ? ). --- A moins qu'il ne s'agisse de Paxamos, auteur culinaire cité par Athénée (\emph{Deipn.} l. 9, p. 376 D.)} ; je pensais, (en considérant) l'intelligence de chaque artisan (de l'œuvre) ; comment, trouvant secours dans leurs devanciers, ils perfectionnaient leurs propres recherches.

Ce qui me surprenait, c'était la cuisson de l'oiseau\footnote{On lit dans la \emph{Bibl. des philos. chimiques}, p. 583 : Oiseau d'Hermès, l'esprit du feu de nature, enclos dans l'humide du mercure hermétique ... ou la chaleur naturelle unie à l'humide radical.  } soumis à la filtration ; c'était de voir comment il la subit, par le moyen de la vapeur sublimée, de la chaleur et d'un liquide approprié, alors qu'il participe à la teinture. Surpris, mon esprit revient à notre objet d'étude ; il examine si c'est par suite de l'émission de la vapeur de l'eau divine que notre composition peut être cuite et teinte. Or je cherchais si quelqu'un des anciens fait mention de cet instrument, et (rien) ne se présentait à mon esprit. Découragé, je compulsai les livres et je trouvai dans ceux des Juifs, à côté de l'instrument traditionnel nommé \emph{tribicos}, la description de ton propre instrument. Voici comment la chose est présentée.

Le mot oiseau a donc un sens emblématique. Il s'explique par le texte qui suit et par les fig. 25, 26, 27 (\emph{Introd.}, p. 149 et 150).

Prenant de l'arsenic (sulfuré), blanchis-le de la manière suivante. Fais une pâte grasse, de la largeur d'un petit miroir très mince ; perce-la de petits trous, en manière de crible, et place par-dessus, en l'ajustant bien, un petit récipient, renfermant une partie de soufre ; mets dans le crible de l'arsenic, la quantité que tu voudras. Après avoir recouvert avec un autre récipient, et avoir luté les points de jonction, au bout de 2 jours et 2 nuits, tu trouveras de la céruse.\footnote{Ce mot semble signifier ici l'acide arsénieux.} Prends-en un quart de mine et souffle pendant tout un jour, en y ajoutant un peu de bitume, etc. Telle est la construction de l'appareil.

2. Quant à moi je reviendrai à notre objet, en montrant, d'après l'écrit lui-même, qu'il n'y a pas blanchiment, puisqu'il conseille de faire durer la cuisson 2 jours et 2 nuits ; tandis qu'une heure suffit pour évaporer une grande quantité de soufre. Mais par-là, il fournit un motif à tes réflexions. En effet Agathodémon a rappelé que l'arsenic est toute la composition ; c'est celle sur laquelle j'ai fortement discouru dans le 6\textsuperscript{e} chapitre, sur la cuisson, dans mon livre sur l'Action\footnote{Ce passage montre que le livre sur l'Action était un ouvrage étendu, dont nous ne possédons que des extraits.} ; beaucoup d'autres anciens l'ont rappelée explicitement et avec intention. Mais le début de l'écrit, qu'enseigne-t-il sur le sujet présent ? Il dit : « Le blanchiment par l'arsenic s'étend jusqu'à l'arsenic non blanchi. » C'est dans le même sens que Démocrite dit : « si la flamme est trop forte, le jaune se produit ; mais (cela) ne te servira pas maintenant, car tu veux blanchir les corps (métalliques).\footnote{D'après ces deux paragraphes, on doit changer le sulfure d'arsenic en acide arsénieux par une oxydation lente : puis on emploie cet acide arsénieux à blanchir le cuivre. Le blanchiment peut se faire aussi avec l'arsenic sulfuré lui-même ; mais alors il est plus lent et plus difficile. C'est ce blanchiment par l'arsenic qui est appelé la fixation du mercure, notre arsenic métallique étant assimilé au mercure, ainsi qu'il a été dit à plusieurs reprises (\emph{Introd.}, p. 99 et 239).} »

3. Or comment y a-t-il un homme assez simple pour ne pas entendre par là toutes les espèces de l'arsenic (sulfuré) ? Et même l'arsenic lamelleux, comme l'expose l'écrit précité ?

Si les matières\footnote{Le cuivre ?} sont blanchies de cette façon, et non pas seulement à la surface, le métal sera entièrement blanc et il ne perdra pas sa couleur au feu ; il sera blanc dans l'intérieur ainsi qu'à la surface. Or comment n'est-on pas capable d'entendre l'arsenic blanchi, là où l'écrit a prescrit de le projeter et de le soumettre à l'insufflation ; cet arsenic ne contenant aucune (partie) de soufre,\footnote{On admet ici et dans les lignes suivantes que le signe du soufre a été traduit par erreur par le mot plomb ; le signe étant le même, comme il a été dit plusieurs fois.} mais s'évaporant en nature\footnote{Acide arsénieux.} sous l'action du feu ? Mais si la composition renferme du soufre, il recommande non seulement de souffler, mais encore d'ajouter du bitume, afin que par là le Tout soit désulfuré et devienne pur et brillant.

4. Voilà toutes les choses qu'il m'est permis de dire là-dessus, et vous en êtes témoins. Mais si vous y trouvez bien des ressources, vous êtes aussi des maîtres pour le reste. Je vous conseille conformément à ce que j'ai appris jusqu'ici, ayant accepté de vous, moi aussi, les fruits de l'œuvre finale. L'écrit dit qu'on opère également sur les monnaies.\footnote{Falsification. --- \emph{Introd.}, p. 33 et 57.} Or ce procédé s'exécute dans l'Écrevisse.\footnote{Voir l'appareil appelé Écrevisse (\emph{Introd.}, p. 145 et fig. 28, p. 154). D'après la formule de la fig. 28, p. 152 à 154, on y travaillait le molybdochalque et l'argyrochalque, c'est-à-dire les alliages sur lesquels s'opérait la transmutation.}

5. Pour la composition,\footnote{Le ms. M s'arrête là, ainsi que B. La suite est donnée d'après A : c'est une addition de commentateur praticien, comme le montre la citation finale de Zosime.} le vase de terre cuite a une ouverture, destinée à découvrir la coupe placée sur la kérotakis, afin que l'on puisse voir si la matière blanchit ou jaunit. Or l'ouverture du vase de terre cuite est fermée au moyen d'une autre coupe,\footnote{V. \emph{Introd.}, p. 149, 150, 151, fig. 25, 26, 27.} afin que le produit ne s'évapore pas ; et que l'alliage de l'Écrevisse\footnote{C'est-à-dire afin que l'alliage destiné à la transmutation (molybdochalque) ne perde pas sa portion volatile (mercure ? ou arsenic ? ou zinc ? ).} ne s'échappe pas par là. L'opération a lieu en un seul jour. Si la décoction est conduite autrement, ainsi que la cuisson, il faudra deux fourneaux : le premier, pour les fioles apparentes ; le second, pour les kérotakis, les vases à fixation, ou les bocaux. Si l'on veut y faire digérer l'alliage de l'Écrevisse, ou les matières analogues, on le placera sur la kérotakis, en l'y étendant, et en évitant qu'il ne coule. Le vieux Zosime disait : « Je connais une classe unique qui renferme deux opérations : l'une pour que la fluidité soit produite par l'extraction ; la seconde pour que l'humidité du plomb soit desséchée jusqu'à épuisement. Car elle se fixera et se desséchera. »

\bigskip
\centerline{\EightStarTaper}
\centerline{\EightStarTaper\EightStarTaper}
\bigskip

\subsubsection{3. --- 8. Sur la même Eau Divine.}
\paragraph{}
1. Prenant des œufs, la quantité que tu voudras, fais-les bouillir, et après les avoir cassés, ôtes-en tout le blanc\footnote{Réd. de A : « ... tout le blanc, au moye de vases de terre cuite, et le jaune. »} ; mais n'emploie pas la coquille.\footnote{Ce langage est probablement symbolique, conformément aux pages 19 et 21.} Prenant un vase de verre mâle et femelle,\footnote{Formé de deux parties s'emboîtant, dont l'une est regardée comme mâle, l'autre comme femelle.} celui qui est appelé alambic, jettes-y les jaunes des œufs,\footnote{« Les blancs et les jaunes » d'après A.} en usant de la pesée ci-après : une once de jaune ; coquille des œufs calcinée, deux carats, ni plus ni moins, mais juste comme il a été écrit. Ensuite, délaie ; puis, prenant d'autres œufs, casse-les et jette (les) dans l'alambic avec les jaunes délayés, de façon que les œufs entiers soient recouverts par les jaunes.

Lute l'alambic et son chapiteau au récipient,\footnote{Le sens du mot \emph{rogion}, employé dans ce passage, autrement dit \emph{rogé}, (p. 59), est défini par cette description.} avec beaucoup de soin ; en te servant de suif, ou de plâtre, ou bien de cire d'abeille, ou de cendre mélangée d'huile, ou de ce que tu voudras. Fais digérer dans du crottin de cheval ou d'âne, ou sur un feu de sciure de bois, ou dans un four de pâtissier. Emploie n'importe quel genre de caléfaction convenable, au degré que peut supporter la main humaine.

Que le lieu où les appareils sont installés soit à l'abri du vent, qu'il reçoive la lumière de l'est ou du sud, mais non celle du couchant, ou du nord, ou du nord-ouest, ou du nord-est, à cause du refroidissement.\footnote{Voir p. 30, § 2.} Fais digérer pendant 14 ou 21 jours, jusqu'à ce que cesse la montée des vapeurs ; et maintiens lutés avec soin les joints de l'appareil, afin de conserver l'odeur ; car si elle s'échappe, tout le travail est perdu. En effet, cette odeur est tout à fait désagréable, et c'est dans cette odeur que réside le travail.\footnote{On voit par là qu'il s'agit de la distillation d'un produit sulfuré. V. \emph{Introd.}, p. 69.}

2. La première eau qui passe (à la distillation) est blanche.

La seconde coule goutte à goutte ; elle est d'une odeur désagréable, toute pareille (au lait de chaux).\footnote{C'est-à-dire qu'elle est blanchie à la façon du lait de chaux ( ? ), par le soufre précipité, provenant de la décomposition des polysulfures ou de l'hydrogène sulfuré qui s'est volatilisé. On dit encore aujourd'hui : lait de soufre pour une liqueur analogue.} Ensuite, quand la montée de l'eau a cessé, tu enlèves le récipient dans lequel l'eau a coulé, tu (le) fermes, et tu le gardes avec soin. Découvrant l'alambic, tu te boucheras le nez à cause de l'odeur ; et tu trouveras dans le vase femelle les scories (\emph{caput mortuum}).

Ne refuse pas au mort de parvenir à la résurrection ; mais attends la résurrection du (mort) dont on a désespéré.\footnote{Ceci signifie que le sulfure, formé au fond de l'alambic (scorie ou \emph{caput mortuum}) se désagrège et blanchit à l'air.} Ensuite mélange avec la cendre d'autres jaunes d'œufs, comme dans l'art de la savonnerie ; délaie ensemble les matières humides et les matières sèches, et jette (le tout) dans un alambic. Opère comme il a été prescrit antérieurement, en changeant le récipient de l'eau, c'est-à-dire le \emph{rogion}.

Fais cela jusqu'à trois fois et tu auras d'abord la première eau blanche, comme il a été dit précédemment, cette (eau) que les anciens ont nommée eau de pluie ; puis, la seconde eau, jaune-verdâtre, qu'ils ont nommée huile de raifort ; puis la troisième eau, d'un noir verdâtre.\footnote{Addition de A : « Qu'ils ont nommée aussi huile de ricin. »}

Tu auras aussi les scories qui sont dans le têt. Lorsque tu ouvriras l'appareil, tu trouveras la première fois la scorie tournant au noir, --- la seconde fois, blanche ;--- la troisième fois, jaune.\footnote{Comparer ce texte du Traité attribué à Avicenne, \emph{Bibl. Chem. de Manget}, t. 1, p. 633 : « Et primo distilla et quod primo exit serva seorsim, quia ista est aqua. Reitera aquam per distillationem et quod distillabitur serva et ista est simplex ; pone sub fimo et serva et quod remanebit in fundo cucurbitæ, serva seorsim, quia est terra. »}

Après la première, la seconde et la troisième extractions d'eau et ouvertures de l'appareil, tu réunis les eaux des trois extractions, c'est-à-dire les eaux divines qui s'y trouvent, avec le résidu contenu dans le vase femelle. Après cela, prenant un alambic de verre, fais-y entrer les matières, bouche l'alambic avec une poterie cuite, capable de s'ajuster aux bords de l'alambic. Lute avec tout le soin possible, à l'aide d'un lut qui résiste au feu. Abandonne sur le fumier du fourneau, pendant quarante et un jours, jusqu'à ce que la décomposition ayant eu lieu, la matière teinte devienne semblable à la matière tinctoriale, et que la nature domine la nature. En effet, de cette façon, les matières sulfureuses sont dominées par les matières sulfureuses\footnote{Voir p. 20, § 12, sur \emph{l'œuf philosophique}.} et les matières humides par les matières humides correspondantes.

3. Ne prends pas souci du poids, ni de la fraîcheur des œufs, ou de leurs jaunes ; seulement, broie ensemble les matières liquides et les matières sèches, comme il a été dit précédemment, et mets-les dans l'alambic. Après le quarante et unième jour, découvre l'alambic et tu y trouveras une composition entièrement vert clair, c'est-à-dire tournée en ios. Celui qui fait l'ios, sait quelle opération il accomplit ; mais celui qui n'en fait pas ne produit rien.

Or, après le quarante et unième jour, ôte l'alambic du lieu chaud et laisse le pendant cinq jours éloigné de toute source de chaleur. Les cinq jours (écoulés), place l'alambic sur de la braise de sciure de bois et extrais-en l'eau divine ; tu la recevras, non dans ta main, mais dans un vase de verre. Puis, prenant cette eau, mets-la dans un alambic, comme il a été écrit précédemment, et fais chauffer pendant deux ou trois jours. Après avoir enlevé, délaie, et expose au soleil sur une coquille. Lorsque le produit sera devenu compacte comme du savon, fais chauffer une once d'argent, et projettes (y) de cette eau solidifiée, c'est-à-dire deux karats de poudre sèche, et tu auras de l'or.\footnote{Il semble qu'il s'agisse simplement d'une teinture superficielle de l'argent en jaune par un polysulfure. Cependant, le résidu employé comme poudre de projection contenait peut-être d'autres métaux.}

Le nombre total des jours de l'opération est de cent dix jours, d'après ce qu'ont dit Zosime, le Chrétien et Stephanus.\footnote{Ceci indique un commentateur relativement moderne.} Quant à moi, après avoir bien butiné de tous côtés comme l'abeille, et tressé une couronne avec beaucoup de fleurs, je t'en ai fait hommage, à toi mon maître. Ensuite, je t'exposerai quels sont les appareils. Portez-vous bien en Jésus-Christ, notre Dieu, maintenant, toujours et dans tous les siècles des siècles. Amen.

\bigskip
\centerline{\EightStarTaper}
\centerline{\EightStarTaper\EightStarTaper}
\bigskip

\subsubsection[3. --- 9. Zosime de Panopolis --- Mémoires Authentiques sur l'Eau Divine.]{3. --- 9. Zosime de Panopolis\footnote{Cette ligne n'existe pas dans M ; mais dans AB. --- Cet article précède immédiatement dans A, les axiomes mystiques sur le Tout, dérivés de la Chrysopée de Cléopâtre (fig. 11, \emph{Introd.}, p. 132 et fig. 13, p. 136).} --- Mémoires Authentiques sur l'Eau Divine.}
\paragraph{}
1. Ceci est le divin et grand mystère ; l'objet que l'on cherche. Ceci est le Tout. De lui (provient) le Tout, et par lui (existe) le Tout. Deux natures, une seule essence ; car l'une attire l'une ; et l'une domine l'une. Ceci est l'eau d'argent,\footnote{Mercure des philosophes et mercure ordinaire.} l'hermaphrodite, ce qui fuit toujours,\footnote{L'esclave fugitif, \emph{Servus fugitivus} des Arabes (\emph{Introd.}, p. 217 et 258).} ce qui est attiré vers ses propres éléments. C'est l'eau divine, que tout le monde a ignorée, dont la nature est difficile à contempler ; car ce n'est ni un métal, ni de l'eau toujours en mouvement, ni un corps (métallique) ; elle n'est pas dominée.

2. C'est le Tout en toutes choses ; il a vie et esprit et il est destructeur. Celui qui comprend cela possède l'or et l'argent. La puissance a été cachée, mais elle est déposée dans Erotyle.\footnote{Auteur cité dans le Papyrus W de Leide (\emph{Introd.}, p. 17).}

\bigskip
\centerline{\EightStarTaper}
\centerline{\EightStarTaper\EightStarTaper}
\bigskip

\subsubsection[3. --- 10. Conseils et Recommandations pour ceux qui Pratiquent l'Art.]{3. --- 10. Conseils et Recommandations pour ceux qui Pratiquent l'Art.\footnote{Suite d'articles sans lien. Le premier semble tiré de Démocrite (v. p. 50).}}
\paragraph{}
1. Je vous le déclare, à vous les sages : sans l'appareil propre à traiter le cuivre, et sans le temps prescrit pour l'opération de l'iosis (lequel temps est court ou long) et pour le mélange des dix espèces susdites,\footnote{Cp. Démocrite, \emph{Questions naturelles}, p. 81.} sèches ou liquides, que l'on broie ensemble, n'espérez rien faire, ô hommes, vous qui appartenez à la troupe de l'or, à la race d'or, aux enfants de la tête d'or ; vous qui êtes les amants de la sagesse et les investigateurs de la matière du jaune d'œuf.\footnote{C'est-à-dire de la teinture en jaune ou en or. } Mais vous, gens du creuset, vous vous raillez mutuellement et vous ne suivez pas mes avis, à moi qui vous engage à vous conformer aux préceptes des maîtres et à leurs écrits ; à moi qui vous fais connaître leurs opinions, révélées par la puissance de la parole divine.

2. Cette eau a deux couleurs, blanche et jaune ; ils lui ont donné mille noms divers. Sans l'eau divine, rien n'existe. Par elle toute la composition est entreprise ; par elle, elle est chauffée ; par elle, elle est brûlée ; par elle, elle est fixée ; par elle, elle est jaunie ; par elle, elle est décomposée ; par elle, elle est teinte ; par-là, elle subit l'iosis, elle est affinée et soumise à la cuisson. En effet, il dit : « En projetant l'eau de soufre natif et un peu de gomme, tu teindras un corps quelconque. » Toutes (les substances) qui tirent leur origine de l'eau, sont en opposition avec celles qui tirent leur origine du feu ; de sorte que sans le catalogue de tous les liquides, rien n'est certain. »

3. Quelques-uns l'ont rappelé, --- et peut-être même tous : il est nécessaire que cette eau, en guise de levain, détermine la fermentation destinée à produire le semblable au moyen du semblable, dans le corps métallique qui doit être teint. En effet, de même que le levain du pain, pris en petite quantité, fait fermenter une grande masse de pâte ; de même aussi ce petit morceau d'or va faire fermenter toute la matière sèche.\footnote{V. \emph{Introd.} Papyrus de Leide, p. 57.}

4. D'autres, mêlant ensemble deux espèces de choses, les résidus dorés des (substances) sulfureuses avec les matières d'or, les ont associées : les unes aux produits bruts et non fermentés, les autres aux produits cuits ensemble dans l'eau de l'iosis.

\bigskip
\centerline{\EightStarTaper}
\centerline{\EightStarTaper\EightStarTaper}
\bigskip

En haut les choses célestes, et en bas les choses terrestres ; par le mâle et la femelle l'œuvre est accomplie.\footnote{\emph{Introd.}, p. 161, au bas de la fig. 37, et p. 163. --- Olympiodore, p. 101, --- v. aussi la note de la page 124.}

\bigskip
\centerline{\EightStarTaper}
\centerline{\EightStarTaper\EightStarTaper}
\bigskip

\subsubsection{3. --- 11. Zosime de Panopolis --- Écrit Authentique.}
\paragraph{}
\emph{Sur l'art sacré et divin de la fabrication de l'or et de l'argent}\footnote{ABK au lieu de l'argent : « du mercure. » --- Cet article est un abrégé, renfermant diverses citations techniques de Marie et de Démocrite, relatives aux opérations pour teindre en or et en argent.}

\emph{Abrégé sommaire.}

1. Prenant l'âme du cuivre qui est au-dessus de l'eau du mercure, fais (en) un corps volatil ; car l'âme du cuivre retenue dans la matière en fusion monte en haut\footnote{S'agit-il de la fleur de cuivre, \emph{Introd.}, p. 232 ? ou d'une cadmie, \emph{Introd.}, p. 239 ?} ; la partie liquide reste en bas dans l'appareil à kérotakis, et doit être fixée au moyen de la gomme\footnote{Ce mot désigne la matière qui donnait la coloration jaune, assimilée au jaune d'œuf, \emph{Lexique}, p. 10. La nature de cette matière n'est pas clairement expliquée.} : c'est la fleur d'or, la liqueur d'or, etc. D'autres entendent par-là la coloration, la cuisson, l'œuvre de la doctrine mystique. Au début le cuivre projeté, après traitement dans l'appareil de la fabrication, charme les yeux. Tandis qu'il perd son éclat, on le combine avec la gomme dorée, la liqueur d'or, etc.\footnote{Les trois phrases précédentes manquent dans BK et ont été ajoutées dans AEL.} (Voilà ce que) il a écrit au sujet de la confection de l'or, laquelle est proclamée aussi la fixation.

Marie dit : « Prends l'eau de soufre et un peu de gomme, mets-la sur le bain de cendre ; on dit que c'est de cette façon que l'eau est fixée. » Marie dit encore : « Pour la préparation de la fleur d'or, place l'eau de soufre et un peu de gomme sur la feuille de la kérotakis, afin qu'elle s'y fixe. Fais digérer à la chaleur du fumier pendant quelque temps. » Après les mots « pendant quelque temps, » Marie (ajoute) : « Prends une partie de notre cuivre, une partie d'or ; amollis la feuille formée de ces deux métaux unis par fusion, pose (la) sur le soufre, et laisse (le tout) pendant 3 fois 24 heures, jusqu'à ce que le produit soit cuit.

2. Le Philosophe\footnote{Démocrite.} expose la même chose : « après avoir fixé pendant quelque temps à la chaleur du fumier, nous faisons cuire le produit en le traitant par le soufre pendant 2 ou 3 jours, jusqu'à ce qu'il se forme une préparation extrêmement jaune, que l'on transporte dans un autre vase. » Telle est la composition. En effet, après la fixation de l'eau de soufre dans un matras,\footnote{\emph{Bouclanion} : c'est le même mot que \emph{bouclé}, plusieurs fois répété. Ce mot paraît le même que βαυκάλιον, bocal. --- La figure donnée en marge de A est celle d'un matras ou fiole allongée : v. \emph{Introd.}, p. 165, \emph{fig.} 42.} on met dans un vase, et on fait cuire fortement pendant 2 ou 3 jours.

3. Tous les écrits veulent (que) le feu (soit fait) par progression. On emploie d'abord le bain de cendre ou le fumier, jusqu'à ce que l'eau de soufre se fixe. C'est ainsi qu'ils arrivent à notre mode de cuisson : « Fixe, dit-il, transforme, et change de matras\footnote{Même figure que la précédente, en marge du ms. A.} ; fais cuire, sur un feu indirect et varié. Quant à moi, j'ai dit dans mon livre du blanc : On fait cuire d'abord pendant un jour, et l'on fixe pendant quelque temps, non seulement en exposant à la vapeur, mais aussi en trempant dans l'eau de soufre. »

4. C'est pour cette raison que le Philosophe, dans le catalogue des liquides, a parlé avec intention de la vapeur ; puis de l'eau de soufre. Après avoir opéré la fixation pendant quelque temps, au moyen de la vapeur ; puis après avoir traité par l'eau de soufre, nous faisons cuire pendant un jour ; comme pour la litharge, lorsqu'on veut l'amener à l'état de céruse. On ajoute le reste de la préparation, si l'on a besoin d'or. Sinon, on souffle avec précaution pour brûler le soufre.\footnote{Soufre, au lieu du mot plomb du texte grec, le signe étant le même.} On délaie la composition et on la traite de nouveau par l'huile de natron, jusqu'à ce qu'elle perde sa fluidité. On souffle jusqu'à ce que les matières sulfureuses s'échappent, en laissant le métal éclairci.\footnote{C'est-à-dire jusqu'à ce que le métal, désulfuré par le grillage, apparaisse dans son éclat. Le commencement des opérations faites sur la kérotakis est obscur ; mais il semble qu'à la fin une désulfuration s'obtienne, en combinant le grillage (insufflation) avec l'action d'un fondant (huile de natron). Le résultat est la teinture superficielle du métal en or ou en argent, conformément à ce qui a été dit à l'occasion du Papyrus de Leide, \emph{Introd.}, p. 56, 58 à 60.} Ainsi on fait bouillir avec l'huile (de natron) désulfurant, jusqu'à ce que le produit perde sa fluidité, et après avoir grillé par insufflation, on obtient (ce que l'on cherche).

Voici comment nous parvenons au jaunissement. Après avoir délayé et employé les matières susceptibles de jaunir, telles que l'eau de soufre et la gomme ; nous fixons légèrement avec la chaleur du fumier. Puis nous faisons cuire 2 ou 3 jours, jusqu'à ce que le produit devienne jaune au plus haut degré. On place ce produit dans le reste de la préparation pendant 3, 5 ou 7 jours, jusqu'à ce qu'il ait subi l'iosis. Puis nous le projetons sur l'argent et nous teignons en or. Nous réglons le feu de façon que la vapeur commence à se fixer.

5. Après avoir fait agir l'eau de soufre sur le molybdochalque, nous faisons chauffer pendant un jour, comme il est dit dans la première classe des liquides blancs ; nous opérons sur un feu indirect, ainsi que cela se fait pour la litharge. Si nous voulons blanchir, nous opérons l'iosis de cette manière. Mais si nous avons grillé par soufflage en vue du jaunissement, nous traitons de nouveau par l'eau de soufre natif et la gomme. Après avoir fixé en exposant à la chaleur du fumier, nous faisons cuire pendant 2 ou 3 jours, jusqu'à ce que le produit devienne jaune au plus haut degré. Après l'avoir enlevé, nous transformons en ios le reste de la préparation. J'ai défini la proportion du feu.

\bigskip
\centerline{\EightStarTaper}
\centerline{\EightStarTaper\EightStarTaper}
\bigskip

\subsubsection{3. --- 12. Sur les Substances qui Servent de Support et sur les Quatre Corps Métalliques, d'après Démocrite.}
\paragraph{}
1. Les quatre corps (métalliques) servent de support,\footnote{A la teinture.} et aucun d'eux ne se volatilise. C'est pour cela qu'il n'a pas parlé de griller (par insufflation) la composition ; car si c'était utile, il en aurait fait mention expressément. En effet, il dit : « Rien n'a été omis, rien n'a été ajourné. » Il dit aussi, en parlant de la liqueur d'or : « Elle teint un corps quelconque ; » ce qui s'applique aux quatre corps. C'est aussi pour cette raison qu'il a cité son maître disant : « Teignant toutes les substances ; » montrant par là qu'il ne s'agit pas de souffler ; mais que les quatre (corps) qui servent de support sont teints et aptes à teindre. Il introduit Pammenès opérant sur le soufre\footnote{On a remplacé le mot plomb par le mot soufre dans ces deux phrases, à cause du morceau précédent et du sens général. V. p. 123, note 7. De même au paragraphe suivant.} et disant qu'il n'est pas besoin de griller ; car le soufre s'évapore lui-même teint. Marie dit : « Enlève-la (nature) sulfureuse au plomb ; partout où le soufre entre, il teint. » Elle a voulu montrer par-là que nous n'avons pas raison de griller le soufre. Elle a employé des noms étrangers aux arts dans la description de leurs opérations. Ce n'est pas ainsi que font ceux qui opèrent, lorsqu'ils parlent de notre cuivre ou bien d'un corps métallique quelconque.

On fait une feuille au moyen de deux métaux unis par fusion. Le Philosophe prend cette feuille métallique et la coupe en morceaux ; si l'alliage est fondu, cela vaut mieux. Voici ce qu'ils disent : « Ce n'est pas au moyen d'une feuille ... »

2. De cette façon, s'ils parlent de griller, ils ne parlent pas d'une opération faite en dehors, mais pendant leur propre travail. Car ils soumettent au grillage les matières cuites, afin de prendre leur (principe) propre et tinctorial. Ils rejettent les matières cuites, et font évaporer les parties inutiles.\footnote{C'est-à-dire le soufre.} Ils donnent d'autres noms aux produits purifiés. Ainsi ils grillent par insufflation, de façon à isoler le principe propre et tinctorial. Voilà comment on brûle dans les cuissons, on expulse par insufflation toutes les matières étrangères, en gardant l'esprit utile et tinctorial.

3. \textbf{Sur les poids des (substances) crues et cuites.}

D'après ce que les écrits disent à cet égard, assurément le soufre doit être expulsé par insufflation. C'est là ce que Marie a voulu faire entendre en disant : « Tu trouveras 5 parties moins le quart, c'est-à-dire moins le soufre chassé par l'insufflation. Semblablement à la fin de son exposé, elle dit que le cuivre, dans son affinage à la fonte, diminue d'un tiers de son poids. Elle dit que ces changements s'accomplissent aussi lorsqu'on blanchit et qu'on jaunit ; car les (substances) sulfureuses teignent, mais se volatilisent. Nous nous débarrassons des substances sulfureuses par volatilisation. Il en est de même des plantes, lorsqu'elles sont entièrement dissoutes ; ainsi qu'il arrive lorsqu'on les fait cuire avec l'eau de soufre, rejetant la partie ligneuse.

4. Ce n'est pas sans motif que Agathodémon dit « et unifiées ; » mais afin que, pénétrant dans la profondeur du métal de l'argent, les matières tinctoriales puissent échapper à la destruction causée par le feu. Nous nous privons donc des teintures tirées de plantes, sachant que les métaux ne peuvent en emprunter les qualités, et recevoir ainsi à fond la teinture.

Les qualités seules agissent ; car le corps ne peut pénétrer dans l'intérieur du corps. Aristote (dit)\footnote{Cp. Aristote, \emph{Physique}, 4, ch. 6, t. 2, p. 292, éd. Didot.} : « les qualités triomphent les unes les autres. » D'après Agathodémon les métaux placés en haut prennent les substances volatiles : c'est ainsi qu'il emprunte l'esprit de la chrysocolle. Ce mot esprit signifie évidemment une substance volatile et les vapeurs sublimées sont du même ordre. Telles sont : la vapeur blanche, la vapeur du cinabre, et
\begin{quotation}
« un esprit plus noir, humide, pur.\footnote{La vapeur du soufre qui noircit les métaux ? Citation des Oracles d'Apollon, qui se trouve aussi ailleurs, 3, 19 ; 3. --- Sur ces oracles, v. Olympiodore, p. 94, note 5.} »
\end{quotation}
\paragraph{}
Car toute vapeur sublimée est un esprit, et telles sont les qualités tinctoriales. Le divin Démocrite parle ainsi du blanchiment et Hermès de la fumée. Quand ces (vapeurs) leur étaient utiles, ils les admettaient dans les traitements, mais (en les désignant) par énigmes. C'est pour cela que c'est un mystère. (Ainsi il dit) : « J'ai écrit cela dans le chapitre : \emph{Si tu es intelligent}. La vapeur du soufre natif, de l'arsenic, et la vapeur blanche de cinabre ... » Agathodémon dit aussi : « (la vapeur de) l'arsenic est l'âme de la matière dorée. Après qu'il a été débarrassé de sa partie épaisse et caustique, qu'il a abandonné son corps sulfureux, prends-en alors la partie colorante. »

5. La vapeur c'est l'esprit, l'esprit qui pénètre dans les corps. L'âme diffère de l'esprit. Il appelle âme la nature primitivement sulfureuse et caustique (de l'arsenic ? ). Sous l'influence purificatrice du feu on conserve l'esprit, si l'on travaille d'après les règles de l'art ; car il ne peut être détruit. Telle est la chose utile, l'élément tinctorial. Il faut à l'opérateur une intelligence subtile, afin qu'il reconnaisse l'esprit sorti du corps et qu'il en fasse emploi, et que surveillant son départ il atteigne le but, c'est-à-dire que le corps étant détruit, (il prenne garde que) l'esprit (ne) soit détruit en même temps. Or il n'a pas été détruit ; mais il a pénétré dans la profondeur du métal, lorsque l'opérateur a accompli son œuvre.

6. Ceux qui ne reconnaissent pas quand l'œuvre est à point, interprètent mal ; car ils ne voient pas autre chose que des matières qui n'ont pas repris leur corps (métallique), des matières brûlées ou incinérées. Tandis qu'ils ne jugent que la partie visible de ces choses, les infortunés, par une sorte de punition, laissent perdre tout et ils ne réussissent pas à éviter la réduction (du produit) en cendre.\footnote{Addition de M\textsuperscript{2} B: « La qualité reste seulement avec le cuivre ; car le cuivre seul est fixé et joue le rôle de support. »} Dans aucun passage des écrits, on ne mentionne d'autre support (à la teinture), sinon le cuivre seul. Ainsi Marie dit que le cuivre est traité et plus tard brûlé. C'est dans ce sens qu'il joue le rôle de support. Tel est (le rôle du) cuivre ou de l'argent, dans notre opération. Nous ne voulons pas en tirer la qualité, et leur corps, par sa mort, devient inutile. Les plantes aussi sont inutiles, car elles sont consumées par le feu.\footnote{Ceci paraît signifier que dans la transmutation le cuivre et l'argent ne conservent ni leur qualité, ou couleur propre, ni leur corps, qui est changé dans celui d'un autre métal. --- Quant aux plantes, si on les entend au sens propre comme les teintures végétales, celles-ci sont en effet détruites par le feu. Au sens figuré, les fleurs métalliques et certaines colorations correspondantes sont également évaporées ou détruites par le feu (v. p. 159, note 2).}

7. Agathodémon dit : « La magnésie, l'antimoine et la litharge se volatilisent, après avoir perdu leur pureté. » Marie : « souffle, dit-elle, les vapeurs, jusqu'à ce que les produits sulfureux soient volatilisés avec l'ombre (qui obscurcit le métal), et que le cuivre prenne tout son éclat. » Ainsi notre cuivre reçoit d'eux la vapeur sublimée. Or la vapeur, c'est l'esprit du corps. L'âme diffère de l'esprit ...
\begin{center}
A partir de ces mots, la fin du § 7 et le § 8, dans M, sont la répétition des § 5, 6, 7 jusqu'à ces mots : « ainsi notre cuivre (reçoit) la vapeur sublimée. » Dans le texte grec, on a donné les variantes.
\end{center}
\paragraph{}
9. Démocrite a passé sous silence les poids (dans son premier livre). Il dit : « Il ne reste rien ; il n'y a plus rien à exposer, excepté la montée de la vapeur sublimée et de l'eau. Or voici ce qu'il disait au sujet des poids et du soufre, dans le livre suivant : « la liqueur blanche d'arsenic, une once, etc. Car il y a deux compositions des soufres ... (phrase inintelligible). Le cuivre sera trouvé constitué de telle manière, qu'il puisse unir sa nature (à un autre corps), et dominer avec lui et charmer conjointement. Ainsi la nature charme la nature. Car l'argent, s'unissant à tous les corps métalliques, ne les repousse pas. Quant au cuivre, il le subit volontiers, comme la jument accepte l'accouplement de l'âne, et la chienne celui du loup : ce que font tous les êtres naturels qui se ressemblent. Le cuivre se rouille et se réduit, sans quitter sa propre nature. » Démocrite, dans la classe de la magnésie, dit : « La magnésie blanchie ne laisse pas les corps métalliques se séparer, ni apparaître\footnote{En s'oxydant séparément.} dans l'ombre du cuivre. » Nous avons achevé le discours sur les poids. Bonne santé.

\bigskip
\centerline{\EightStarTaper}
\centerline{\EightStarTaper\EightStarTaper}
\bigskip

\subsubsection{3. --- 13. Sur la Diversité du Cuivre Brûlé.}
\paragraph{}
Beaucoup préparent le cuivre brûlé au moyen du soufre.\footnote{\emph{Introd.}, p. 233. --- Dioscoride, \emph{Matière médicale}, V, 87.} Les traités des autres auteurs le disent avec obscurité. Démocrite seul s'exprime avec une clarté généreuse : « Jetez sur le cuivre un quart de fer sulfuré, c'est-à-dire préparé en fondant avec la pierre magnétique, le quart ou la moitié de soufre ; coulez le produit avec le plomb provenant de l'antimoine et de la litharge. Ensuite faites brûler la composition obtenue avec la pyrite, le cuivre et le fer, afin qu'il se forme une scorie convenable. Projetez-y la vapeur sublimée de l'arsenic (sulfuré). Le métal est blanchi par la vapeur du soufre. »

En parlant de la céruse cuite avec le soufre, il veut parler du soufre pur, comme propre à changer le molybdochalque en métal étésien. Lorsqu'il dit : « La magnésie blanchie produit le même effet ; » il veut parler du cinabre traité simultanément. Mais quelqu'un objectera : il a parlé d'abord de la magnésie et de la pyrite. Oui, afin que tu apprennes ceci qu'en même temps que le cuivre, on projette le fer et le plomb et les minerais, afin que le molybdochalque devienne du cuivre étésien (doré).

\bigskip
\centerline{\EightStarTaper}
\centerline{\EightStarTaper\EightStarTaper}
\bigskip

\subsubsection{3. --- 14. Sur ce Point qu'ils donnent le Nom d'Eau Divine a tous les Liquides et que c'est une (Substance) complexe et non pas simple.}
\paragraph{}
1. « La vapeur décrite précédemment, tu la feras cuire dans l'huile. » La vapeur décrite précédemment, c'est la formule entière ; car elle paraît comprendre l'eau divine et l'huile. Ils disent qu'il faut opérer avec tous les liquides, voulant faire entendre (par-là) la liquidité. En effet, par tous ces mots : la saumure vinaigrée, ensuite l'huile, puis le miel et le lait, il faut entendre l'eau divine. Le safran par lui-même est impuissant à teindre sans le concours de l'eau divine ; ceux qui veulent teindre s'en servent. Marie parle de « la dissolution du \emph{comaris} et de la chélidoine. » Démocrite (place) dans la dernière classe des liquides blancs « l'eau de chaux qui a coulé » à travers le filtre, ou à travers une chausse.

Toutes les espèces sont traitées par macération, au moyen des liquides simples ; puis le produit est soumis au lavage. Ainsi sont lavés les corps (métalliques) solides. On les fait macérer, soit en les délayant, soit en les arrosant. Les produits délayés sont exposés au soleil et à la rosée, à la façon du soufre blanc ou de la litharge. On les fait macérer 1, ou 3, ou 5, ou 7 jours, jusqu'à désagrégation totale.

2. Ces (espèces) ayant été macérées, tu en feras des mélanges et tu soumettras ces mélanges délayés à la rosée et au soleil. Après les avoir desséchés et délayés, en les traitant par l'huile de natron, tu trouveras le plomb noir. Délaie-le, en reprenant avec le mercure, l'eau divine et la gomme ; fais cuire sur un feu léger jusqu'à ce que l'eau se soit séparée : tu délaies au soleil jusqu'à ce que la matière soit d'un beau blanc.

3. Ce travail est répété plusieurs fois par ceux qui lavent la scorie. D'après Pébichius : « Lave 2 fois 7, et 1 fois 8 plus 8, et encore plus. » Démocrite fait la même chose dans sa dernière classe, celle des liquides blancs : il lave de la même façon les feuilles (métalliques) oxydées, et il leur restitue leur éclat. Après avoir desséché, si le métal est devenu brillant, reprends la vapeur, traite les substances qui peuvent jaunir par l'eau divine et la gomme, et fixe (la teinture) sur un feu léger.\footnote{C'est une opération de teinture en or, par vernis ou par coloration superficielle. --- V. \emph{Introduction}, p. 56, 58 à 60.} Lorsque tu auras opéré la fixation, retire la substance, et laisse égoutter sur le résidu de la préparation pendant 2, ou 3, ou 7, ou 41 jours. Si tu y projettes de l'argent commun, tu le teins (aussi). Cherchons ensuite le moment qui convient.

\bigskip
\centerline{\EightStarTaper}
\centerline{\EightStarTaper\EightStarTaper}
\bigskip

\subsubsection{3. --- 15. Sur cette Question doit-on en n'importe quel Moment entreprendre l'Œuvre ?}
\paragraph{}
1. Il est nécessaire que nous recherchions quels sont les moments opportuns. Il a dit que l'esprit, soumis à l'action du soleil, doit être tiré des fleurs, et macéré depuis le matin ; alors par toute action convenable du feu, l'or devient bon pour l'usage. « Car c'est l'œuvre du soleil, dit le grand Hermès, c'est ce qui est produit par lui. » Écoute Hermès disant que l'amollissement des substances destinées à être ramollies se fait à froid. Il s'est expliqué nettement sur ce point à la fin de son écrit sur le blanchiment du plomb. Là aussi il parle de l'or. « Voilà comment opère celui qui prépare le Tout. » C'est là aussi qu'il s'est expliqué sur ce que l'on doit filtrer le Tout par n'importe quel filtre. Cela n'a pas échappé à Agathodémon, et il parle de lavage du minerai et de sa purification, (qui a lieu) lorsque le Tout délayé et liquéfié traverse le filtre ou la chausse. Hermès dit : « Elle devient comme une lessive innocente ( ? ). » S'il se forme un dépôt, c'est la preuve que les substances et les minerais ne sont pas suffisamment pulvérisés.

2. Hermès s'est expliqué fortement sur ces choses en parlant des cribles, et disant : « Si les eaux se meuvent en tous sens, le crible lui-même semble s'écouler. » Elles doivent descendre ensemble, suivant le grand Hermès ; puis elles remontent aussitôt dans l'appareil destiné à en opérer la cuisson. Nous avons exposé ces choses dans notre discours, sauf en ce qui traite du moment opportun. Le moment opportun, c'est celui de l'été, alors que le soleil a une nature (favorable) pour l'opération.

Marie s'en occupe, en décrivant les traitements du petit objet\footnote{Cp. 3, 21, 7.} : « L'eau divine sera perdue pour ceux qui ne comprennent pas ce qui a été écrit, à savoir que le produit (utile) est renvoyé vers le haut par le matras et le tube. Mais on a coutume de désigner par cette eau la vapeur du soufre et des arsenics sulfurés. A cause de cela tu m'as raillée, parce que dans un seul et même discours je t'ai exposé un si grand mystère. »

3. Cette eau divine, blanchie par des matières blanchissantes, fait blanchir. Si elle est jaunie par des matières jaunissantes, elle fait jaunir. Si elle est noircie au moyen de la couperose et la noix de galle, elle fait noircir et réalise le noircissement de l'argent et celui de notre molybdochalque. Je t'ai parlé précédemment de ce molybdochalque, à l'occasion de notre argent traditionnel. Ainsi l'eau noircie, s'attachant à notre molybdochalque, lui donne une teinture noire fixe ; et bien que cette teinture ne soit rien, tous les initiés désirent vivement la connaître. Or l'eau capable de prendre une telle couleur, produit une teinture fixe, l'huile et le miel étant éliminés.

4. Le Philosophe dit aussi qu'une petite quantité de soufre natif suffit pour brûler beaucoup d'espèces et qu'il amollit les pierres et les métaux. Dans cette eau se dissout la composition sulfurée, comme il le dit en parlant de l'Androdamas. « Si tu mets du soufre apyre, tu produis une liqueur d'or.\footnote{Page 48, § 10.} Pour la faire agir sur la composition des substances, on délaie la composition des matières sulfureuses. » De la même façon, on la fait bouillir ou cuire. « Comprends bien, dit-il, que si tu mets du soufre apyre, tu produis une liqueur d'or. Au moyen d'un feu de sciure de bois, sur la kérotakis, distille l'eau divine, jusqu'à ce qu'elle contienne (la couleur) d'or. Tu feras cuire en agitant légèrement, et en ajoutant les \emph{motaria}\footnote{C'est-à-dire le résidu de l'expression dans un linge de la sandaraque décomposée. v. Olympiodore, p. 112 et 108.} de la sandaraque faune. [Or ils ont dit les \emph{motaria}, parce que (la composition) est épaisse comme du sang]. Fais cuire le produit fortement pendant 2 ou 3 jours, et après avoir pressé, verse le résidu de la préparation dans chaque vase : et il se forme de l'ios. Pébichius a dit aussi sur cette question : « Partagez la préparation en deux parties, et mettez-en une moitié dans un vase de terre cuite et l'autre moitié sur le cuivre ; » voulant faire entendre ceci en un seul (mot) : la cuisson, par (le vase) de terre cuite, et l'iosis, par le cuivre. Or il a parlé précédemment du blanchiment, en disant que le cuivre est brûlé dans du bois de laurier ; c'est-à-dire le soufre natif (avec le cuivre) en présence des feuilles de laurier.\footnote{Voir la note 2, page suivante.} Tu peux connaître par là le mérite des anciens, combien clairement ils ont expliqué toutes choses. En paraissant cacher toutes choses, ils ont dit clairement : « D'abord, sur des flammes légères, afin que l'eau de soufre soit absorbée en même temps. » Au sujet de ces flammes, Marie disait : « les flammes progressivement ; » puis : « le feu graduellement ; » afin de faire comprendre qu'il faut opérer suivant une progression convenable, à partir (de l'instant) de la flamme.

Le moment opportun est celui de l'été. La pourpre aussi exige une époque particulière pour les dissolutions et les refroidissements. De même, la gomme en larmes, pour s'écouler spontanément, veut la nature propre de l'été. J'ai pourtant entendu dire à quelques-uns que notre opération se fait en toute circonstance, et j'hésite à le croire.\footnote{A la fin de cet article, le ms. A. renvoi à un autre qui se trouve plus loin : 3, 29, § 21.}

\bigskip
\centerline{\EightStarTaper}
\centerline{\EightStarTaper\EightStarTaper}
\bigskip

\subsubsection[3. --- 16. Sur l'Exposé détaillé de l'Œuvre discours à Philarète.]{3. --- 16. Sur l'Exposé détaillé de l'Œuvre discours à Philarète.\footnote{Ce morceau renferme des extraits plus ou moins étendus, tirés de Démocrite, et entremêlés de commentaires.}}
\paragraph{}
1. Voici dans quels termes Démocrite expose ces choses aux prophètes égyptiens : « Je t'écris, ô Philarète, pour t'exposer tout au long la puissance de l'art. Voici le catalogue des espèces : le mercure, tiré du cinabre, l'antimoine de Coptos, de Chalcédoine, d'Italie, la litharge, la céruse, le plomb, l'étain, le fer, le cuivre, la chrysocolle, le claudianon, la cadmie, la pyrite, l'androdamas, le soufre, la sandaraque, l'arsenic, le cinabre. »

2. « Les espèces suivantes sont employées pour l'or et l'argent ; car, blanchies, elles blanchissent, et jaunies, elles jaunissent. Celles qui blanchissent sont les suivantes : la terre de Chio, l'astérite, la terre de Samos, la terre de Cimole et l'aphrosélinon. »

3. « Les (espèces) qui se délaient sont celles-ci : le soufre natif, le sel de Cappadoce, les sels de toutes sortes, la fleur de sel, le calcaire, qui a été appelé aussi le suc laiteux du mûrier, (ou) du figuier,\footnote{Noms symboliques.} l'alun en lamelles, le misy, le chalcanthon, les feuilles de pêcher, les feuilles de laurier.\footnote{Ce sont les noms symboliques de quelques substances minérales, analogues aux noms donnés plus haut au calcaire et tirés de la nomenclature prophétique (\emph{Introd.}, p. 10). De semblables substances minérales sont parfois désignées dans d'autres endroits du texte sous le nom de \emph{plantes} ; probablement parce que l'on en tirait des matières colorantes, ou \emph{fleurs}, d'apparence analogue aux couleurs végétales et aux fleurs des plantes. V. p. 71, note 4, p. 80 ; p. 108, note 6 ; p. 123, note 6 ; p. 153, note 2 ; v. aussi p. 84, note 5, etc.} »

4. « Voici les (espèces) employées pour jaunir : la terre pontique, celle qui est brûlée, la terre attique, celle qui fournit le bleu mâle et le bleu femelle, commun aux deux teintures\footnote{Théophraste parle de ces deux bleus (\emph{Introd.}, p. 245). Le bleu mâle paraît être une couleur de cobalt ; le bleu femelle, une couleur de cuivre.} ; et parmi les plantes, le ricin et la fleur de carthame, la chélidoine et l'ochumenon (basilic)\footnote{V. \emph{Lexique}, p. 8, note 1.} ; et, parmi les sucs, la gomme.\footnote{\emph{Lexique}, p. 10.} » Il disait au sujet de la gomme : « les sucs sont aussi employés pour la composition blanche. »

5. Mettez en évidence les produits qui doivent être délayés plus tard, en vue de l'opération de l'iosis, et traitez (les) conformément à l'opinion d'après laquelle les corps qui n'ont pas de substance propre agissent convenablement sans feu.\footnote{V. l'article suivant, 3, 17, p. 167.}

Quelques-uns veulent employer au 2\textsuperscript{e} et au 3\textsuperscript{e} rang dans l'opération de l'iosis, les plantes, telles que la fleur de l'anagallis et la rhubarbe, et les (espèces) semblables ; quelques-uns emploient, le safran et la racine de mandragore, celle qui porte de petits tubercules. J'ajouterai que sans elle rien n'est teint, et que toutes (les espèces) sont délayées en même temps qu'elle avec la gomme, dans l'opération de l'iosis. Mais tous ont rappelé qu'il ne faut pas détruire le ferment dans cette liqueur ; et il en est de même pour le corps qui doit être teint.

6. Si tu dois teindre en argent, (il faut) faire macérer en même temps une feuille d'argent ; pour teindre en or, c'est une feuille d'or. Car le blé engendre le blé, et le lion (engendre) le lion, et l'or (engendre) l'or.\footnote{V. la \emph{lettre d'Isis}, p. 33. --- Olympiodore, p. 96.} Projette, dit-il, de l'argent commun, et tu teindras. Car une seule liqueur est désignée pour les deux (teintures).

Voici à présent ce qui regarde la teinture de la préparation.\footnote{Φάρμακον : c'est ce que les alchimistes latins appellent \emph{medicina}. C'est la liqueur destinée à la teinture des métaux ; on lui communique d'abord à elle-même une teinture convenable.} L'eau divine préparée suivant la vraie formule, celle qui est bien fabriquée, teint les préparations ; et lorsque la préparation est teinte, alors elle-même teint à son tour. C'est pour cela que les ferments, les ferments préparatoires, les ferments acides, les ferments d'or et analogues sont tenus cachés. [Or en toutes choses tout est découvert par les gens intelligents.]

7. Parlons des quatre corps qui résistent au feu, des (corps) qui servent de support (à la teinture), c'est-à-dire de la composition ultérieure. Après l'avoir composée, nous en prenons une partie, en y ajoutant de l'eau divine, jusqu'à ce que se produise la couleur et le ton du corps correspondant,\footnote{L'or ou l'argent.} selon Marie. Quand on a obtenu la composition ultérieure, les quatre corps qui servent de support, non-seulement on projette sur eux la composition du ferment d'or, mais aussi la composition de l'eau de soufre. On doit faire la projection sur les (corps) que voici : le fer, ou l'étain, ou le plomb, ou le cuivre, etc. Tous ces corps subissent la projection. Écoute ce qu'il dit dans le chapitre \emph{des deux compositions} : « Si tu projettes sur du fer, (il s'affine) ; si tu projettes, sur du cuivre, il s'affine d'abord ; si c'est sur du plomb, il perd sa fluidité ; si tu opères d'abord sur l'étain, il devient rigide. Projette ainsi, dit-il, et pour que tu ne te trompes pas, blanchis d'abord. »

8. Discourons maintenant sur l'affinage\footnote{Les mots affinage, affiné, sont employés ici, faute de mieux, pour traduire le mot grec ἐξίωσις. En réalité il s'agit de la transformation du métal préalablement changé en \emph{ios} (oxyde, sulfure, sel basique) ; et qui est régénéré avec une couleur nouvelle, provenant de la formation d'un alliage, au moins superficiel, tel qu'un arséniure ou un amalgame.} du cuivre. Les espèces employées comprennent les feuilles de pêcher et de laurier,\footnote{Voir la note 2 de la p. 159.} ainsi que les terres blanches, (les sucs) de mûrier et de figuier,\footnote{Le calcaire, d'après le texte de la p. 159, § 3.} le suc de tithymale, le natron roux, le sel de Cappadoce et les (substances) semblables. Dans cette liqueur, dit-il, dépose les écailles du cuivre,\footnote{\emph{Introd.}, p. 233.} pendant 15 jours et tu le trouveras affiné, c'est-à-dire blanchi. Telle est la composition de la liqueur du soufre blanc.

Voici ce que le Philosophe a exposé dans la dernière classe des liqueurs : « Certes le soufre blanc blanchit le cuivre. Mais s'il s'agit du soufre jaune, le cuivre est traité par la couperose et le sori ; puis, après l'avoir jauni, on met ce cuivre, en même temps que le soufre, dans du vinaigre, etc., afin qu'il devienne \emph{ios}. » Il dit en effet, que la couperose produit la couleur d'or. Si la couperose est délayée avec le soufre, la pyrite et le son, et le soufre jaune ajouté à ce mélange jaune ; et si on le laisse déposer (sur le métal, afin qu'il le ronge), le soufre produit ainsi le jaune.\footnote{La fin de cette recette confuse semble répondre à l'affinage de l'or par un mélange complexe, analogue au cément royal (\emph{Introd.}, p. 14 et 15).}

9. Qu'est-ce donc que l'affinage, ou le jaunissement ? L'affinage et le jaunissement diffèrent entre eux seulement par la couleur : c'est-à-dire que l'affinage par le soufre (est) un blanchiment ; tandis que l'opération de l'iosis est un jaunissement. Voyons ce qu'il dit encore : « Si tu veux amollir le fer, prépare des écailles\footnote{Fer oxydé des battitures (\emph{Introd.}, p. 252).} menues de fer ; dispose une couche de terre de Samos ; puis étends une seconde couche d'alun lamelleux. Tu obtiendras un métal mou et blanc. » Or, les espèces de cette nature appartiennent au (genre du) soufre blanc. Hermès, parlant du ramollissement, disait ensuite : « Et il sera blanchi. » C'est pour cette raison que le Philosophe disait : « Mets en outre la moitié de la préparation blanche, c'est-à-dire du soufre blanc.\footnote{Le mot \emph{soufre blanc} a dans tout ce passage un sens particulier. Il paraît s'agir des compositions arsénicales et sulfurées, destinées à produire soit un laiton tournant au blanc, soit un arséniure métallique complexe, analogue au tombac ; peut-être même tout alliage métallique blanc, dur et rigide.} »

10. Cherchons maintenant ce que c'est que la rigidité. Le Philosophe (dit) : « Prends du plomb blanc qui a perdu sa fusibilité, grâce à la terre de Chio et à l'alun. Ces espèces appartiennent (au genre) du soufre blanc. Or le soufre blanc, une fois blanchi, fait blanchir. » Démocrite (dit encore) : « Lorsque tu auras affiné, amolli, donné de la rigidité et ôté la fluidité, ou bien lorsque tu auras blanchi. » Le blanchiment (s'obtient) par le soufre blanc. Vois le Philosophe, pris d'un transport divin au sujet de ce soufre blanc : « Si la préparation devient semblable au marbre, il y a là un grand mystère ; car elle blanchit le cuivre, c'est-à-dire elle l'affine ; elle amollit le fer ; elle ôte à l'étain sa flexibilité, au plomb sa fluidité ; elle rend les substances solides et les teintures fixes.

Ces teintures, (ce sont) les espèces, depuis le mercure,\footnote{D'après M. --- ABKELb, l'argent.} jusqu'à la chrysocolle, celles qu'on appelle la fleur d'or. Quelques-uns ont parlé à bon droit de ce soufre, au sujet de toutes (des choses). En effet, Stephanus,\footnote{Ce passage est dû à un commentateur de date plus récente.} lorsqu'il disait : « les substances solides, » parlait des quatre corps. D'autres disaient : « c'est l'eau divine, (c'est) le grand mystère entre tous, ce qui devient semblable au marbre, ce qui blanchit toute substance, ce qui blanchit le corps du molybdochalque,\footnote{De la magnésie, B.} c'est la fumée des cobathia.\footnote{\emph{Lexique}, p. 10. --- Olympiodore, p. 91, note 4. --- \emph{Introd.} p. 245 --- En marge de M, on ajoute : l'eau du soufre apyre.} C'est là ce qui rend les teintures fixes, ce qui maintient solides les substances. » Or, si tu veux parler (de rendre) les substances solides, ce n'est pas pour que les substances amenées à une mollesse oléagineuse se crevassent, mais afin d'éviter la déperdition des (matières) qui ont coutume de disparaître par l'action du feu, depuis la vapeur sublimée jusqu'à la chrysocolle ; attendu qu'il s'agit d'obtenir des teintures. Écoute-le parler à ce sujet : « Il faut mettre, en outre, du fer, ou du cuivre, ou de l'étain, ou du plomb. » Voilà ce qu'il nomme des teintures : les quatre corps, lesquels une fois teints, teignent (à leur tour). Or ce qui teint les teintures et les choses teintes, (c'est) l'eau divine, le grand mystère, ce qui est semblable au marbre ; ce qui rend toutes choses aptes à l'opération, ce qui brûle le cuivre et le blanchit, ce qui fixe le mercure, ce qui affine, voilà le grand mystère de l'art tout entier. En effet, l'eau jaune est un mystère manifeste.

11. Mets donc un peu de gomme et tu teindras toute sorte de corps. C'est là ce qui agit dans la calcination, le blanchiment, le jaunissement, la fixation du mercure, l'iosis. Lorsqu'il parle des substances solides, en traitant de la destruction des substances, il parle (de la perte) des espèces volatiles. Or ce soufre blanc est récapitulé dans les deux compositions ; car il dit : « Si c'est sur le fer, il amollit d'abord, etc. » C'est-à-dire blanchis d'abord toutes choses, comme il a été expliqué, lorsque tu auras affiné et ramolli, rendu rigide et non fluide ; blanchis le Tout, les quatre corps qui servent de support. Tel est le début en suivant une marche unique, celle du blanchiment. Or le blanchiment (s'obtient) au moyen du soufre blanc. Le poids des soufres blancs se trouve dans la dernière classe, celle des liqueurs blanches, savoir : arsenic doré 1 once, (autant de) natron et matières semblables, pellicules des feuilles de pêcher et de laurier 1 once, (autant de) suc de mûrier, sel, etc. Il faut mêler ensemble ces matières, suivant la proportion des pesées. Le mercure va, dans les deux compositions, s'emparer de toutes (les matières), c'est-à-dire les ramollir ; j'y reviendrai à propos du cinabre.\footnote{Lc. : Au lieu du cinabre, « de l'argent. » --- Signe de l'argent \emph{couché} ABKE. V. \emph{Introd.}, p. 120, Pl. 8, l. 22. Le sens de ce symbole particulier est incertain.} Mais pour que cette amalgamation ait lieu, il ne faut pas délayer les deux compositions avec des blancs d'œufs, de l'eau de gomme blanche. Car dans ces (compositions), le mercure\footnote{Au lieu du mercure, ABK : « l'argent. » Dans Lb l'argent est à l'accusatif, c'est-à-dire que c'est lui qui est attaqué. Le mot mercure pourrait désigner ici notre arsenic (\emph{Introd.}, p. 239).} a pour effet d'attaquer tout, de s'emparer de tout, de tout amollir. Je me suis expliqué là-dessus dans (le chapitre des) \emph{molybdochalques}.

12. Quelques-uns ont adouci l'eau divine, en la rendant plus épaisse, et ont repris les compositions avec le mercure. En effet, la composition blanche contient les œufs et la gomme. D'autres mettaient le Tout dans un grand vase de verre,\footnote{\emph{Troullos}, mot à mot, truelle. C'est quelque instrument inconnu.} luté tout autour, et ils faisaient chauffer sur un feu faible ; ils y plaçaient de l'eau divine, et cuisaient comme (on fait pour) la pourpre. Il faut procéder dans la transformation comme on le fait avec le produit tiré de la mer, lorsque ce produit est changé en pourpre véritable. Par suite, le Philosophe (dit) : « La céruse a une puissance différente en raison de l'helcysma,\footnote{\emph{Helcysma}, scorie d'argent (\emph{Introd.}, p. 266). Il y a un jeu de mots fondé sur le double sens de ce mot, qui signifie à la fois : écume tirée des métaux et produit (coquillage) tiré de la mer.} selon qu'il s'agit de celle qui sert à la teinture en or, c'est-à-dire en pourpre, ou bien de celle qui sert à la teinture en blanc, c'est-à-dire en argent. » La même composition délayée possède plusieurs sortes d'actions. « Toutes les substances (métalliques), dit-il, proviennent de la seule nature du plomb ; le cuivre ajouté, tu le sais, forme toute la composition.\footnote{Molybdochalque.} » Voilà comment il a désigné la mutation par l'helcysma, dans ses démonstrations : « Après avoir fait chauffer l'eau divine. » Par ce mot « faire chauffer, » ils ont désigné la production (de la) couleur. Ils ne se sont pas bornés à unir le mercure\footnote{Lb ajoute : « Au soufre. »} ; mais, en outre, ils ont blanchi et jauni la composition, faisant chauffer sur un feu doux et ne laissant pas la fumée se dissiper par l'instrument. Car c'est en elle que réside l'esprit tinctorial. On fait cuire jusqu'à ce que la couleur soit répandue (dans toute la masse) ; les uns pendant neuf heures, d'autres pendant deux jours.\footnote{B : Un jour et une nuit. --- Lb : 12 heures.} Cela fait, on recouvre l'instrument avec une coupe et on le place sur une kérotakis, ou dans un matras, au-dessus du fourneau ; on chauffe le fourneau, à partir de ce moment, pendant un jour,\footnote{A : Un jour et une nuit.} d'autres pendant deux. On regarde à travers la coupe ce que devient la céruse, puis on enlève le produit.

13. Quelques-uns fabriquent du jaune\footnote{Ou bien : « préparent du plomb, » suivant la variante adoptée pour le \emph{Texte grec}, p. 165, l. 8.} ; ils font un trou au milieu (du vase). A la partie inférieure on ne trouve que des scories, (la vapeur) s'étant séparée à la partie supérieure ; car dans (la composition) à deux couleurs, la scorie se rencontre avec le plomb. Après avoir détaché la scorie, on obtient le corps métallique. On pulvérise cette pierre et on l'expose au soleil, jusqu'à ce qu'elle soit blanchie. On prend la moitié du poids du produit, on y ajoute du mercure et du soufre comme complément, ainsi que de la gomme blanche. On fixe sur de la cendre chaude pendant un jour entier, jusqu'à ce que l'eau divine soit complètement desséchée. On ajoute donc de l'eau divine. Lorsque toute cette eau a été consommée, on la renouvelle, et l'on fait chauffer les matras pendant une heure, (sur un feu) indirect : on obtient ainsi la céruse. La substance encore bouillante est transportée sur du soufre apyre, et sur de l'eau de soufre, pour l'autre moitié du poids : on laisse déposer pendant (deux) jours, jusqu'à ce que l'ios soit produit.

14. Quelques-uns enfouissent le vase dans le crottin de cheval, pendant le même nombre de jours. On y met du cuivre, en ajoutant après la teinture du fer blanchi,\footnote{Voir 3, 13, p. 154.} si l'on veut fabriquer de l'argent. Si c'est de l'or, on délaie de nouveau avec le produit moitié de son poids de mercure et moitié de soufre (j'entends du soufre jaune), ainsi que de l'eau de soufre natif et de la gomme. On fixe en chauffant par en dessous et l'on commence par faire cuire, pendant deux jours et deux nuits. Après avoir enlevé bouillant, on met de l'eau divine sur le résidu du soufre, et l'on fait chauffer pendant deux jours. Quand le produit est cuit à point, on ajoute de l'argent commun.

15. La préparation du blanc est celle-ci : soufre, arsenic, sandaraque, cinabre, en quantités égales, macérés d'avance ; sel de Cappadoce, autant ; fleur de sel, alun, lie de vin cuite, calcaire cuit, aphrosélinon, misy cru et cuit, natron et sel, mêlés à parties égales avec de l'eau de mer. On expose au soleil pendant un nombre convenable de jours, jusqu'à ce que la teinture devienne capable de résister au feu. Ensuite on délaie ces matières avec de l'eau divine, de façon à rendre la couleur stable à chaud. Je veux parler de l'eau blanche, (obtenue) au moyen de la chaux délayée. Après avoir rendu la couleur stable, tu la mélanges, à raison d'une mine pour une demi-mine, et la quantité suffisante d'eau divine.

16. L'eau de soufre obtenue au moyen de la chaux se fabrique de la manière suivante : Après avoir mélangé toutes les eaux du catalogue, par portions égales, ajoute des terres blanches jusqu'à ce que (le mélange) devienne très blanc. Mets dans une marmite, installe l'appareil avec du feu dessous et reçois ce qui distille. Emploie ce produit pour le délaiement du soufre et la cuisson de la composition.

17. Le soufre jaune se prépare comme il suit : soufre, arsenic, sandaraque, cinabre, sori, couperose, chalcite, misy, alun, natron, sel, bleu d'Arménie ; tout cela macéré d'avance. Délaie avec du vinaigre, en exposant au soleil pendant un nombre convenable de jours. De ce soufre tu projettes une demi-mine, pour une mine (de matière).

18. L'eau du soufre pur se prépare comme il suit : les eaux du catalogue, par portions égales ; terre pontique, terre attique, bleu d'Arménie ; on ajoute des plantes, c'est-à-dire du safran et de la chélidoine, en quantité double. Mets dans une marmite, et, après avoir joint les diverses parties de l'appareil, prends l'eau qui en sort (l'eau de soufre), destinée aux produits qui résistent au feu. Arrose la composition avec de la gomme, du mercure et de l'eau de soufre, comme je l'ai dit précédemment, le tout par moitié. Après avoir fixé sur un bain de cendres chaudes, jusqu'à ce que toute l'eau soit partie, fais cuire pendant 2 jours, jusqu'à ce que le produit soit devenu extrêmement jaune. Enlève le produit encore bouillant, mets-y le résidu de la préparation, et laisse déposer pendant un nombre convenable de jours, jusqu'à ce que le produit soit changé en ios. Après avoir desséché et pulvérisé, on conserve. C'est ce produit que l'on mêle avec l'argent commun pour teindre. Quelques-uns après avoir opéré l'iosis, enfouissent dans le crottin de cheval.

19. Il a été établi que toutes les espèces (sont) communes aux liqueurs : si ce n'est que les matières blanchies font blanchir, et les matières jaunies font jaunir. Il faut savoir qu'après avoir accompli l'œuvre on doit mêler avec la composition. Quant à savoir ce qui teint le mieux, c'est un soufre dont tout le monde a parlé. Agathodémon, notamment, disait : « Prends du soufre, tantôt blanc, tantôt jaune, tantôt noir, tantôt enfin blanc fixe, et tantôt jaune fixe. » Il a donc montré, comme on l'a dit, que toutes les espèces (sont) communes aux liqueurs ; si ce n'est que blanchies, elles font blanchir, et que jaunies, elles font jaunir.

\bigskip
\centerline{\EightStarTaper}
\centerline{\EightStarTaper\EightStarTaper}
\bigskip

\subsubsection{3. --- 17. Sur cette Question : Qu'est-ce que la Substance suivant l'Art, et qu'est-ce que la Non-Substance ?}
\paragraph{}
1. Démocrite a nommé substances les quatre corps métalliques ; il entendait par là le cuivre, le fer, l'étain et le plomb. Tout le monde les emploie dans les deux teintures (d'or et d'argent), et toutes les substances subissent les deux teintures. Toutes les substances ont été reconnues par les Égyptiens comme produites par le plomb seul ; car c'est du plomb que proviennent les trois autres corps.\footnote{On voit que les Égyptiens regardaient le plomb comme le métal fondamental ; sans doute en vertu d'une idée analogue à celle du mercure des philosophes et par ce qu'ils y faisaient résider la qualité métallique par excellence (voir, p. 102, note 2 ; p. 103, note 4, et \emph{Introd.}, p. 58).} Il a donc nommé substances les matières résistant au feu, et les matières qui n'y résistent pas : non-substances. En effet, les non-substances agissent d'une façon convenable, indépendamment du feu. Il disait qu'elles sont engendrées par l'action des appareils et de la combustion ; tandis que le vrai résidu de la préparation, préparé sans l'action du feu, produit une teinture stable en blanc ou en jaune. L'emploi de la préparation fugace obtenue par la flamme détruit le jaunissement du molybdochalque défectueux, attendu qu'il le fait disparaître. Sur ce point il ne faut pas se tromper. Vois comme il s'exprime à cet égard : « Amène à consistance visqueuse ; enduis avec la moitié de la préparation destinée à la cuisson et teins avec le reste, de façon que la couleur soit fixée sans le concours du feu. »

2. On appelle non-substances les matières sulfureuses ne résistant pas au feu. Mais l'emploi des liquides convenables leur communique la propriété de résister au feu et d'y demeurer stables : car l'eau combat l'action du feu. C'est pour cela qu'il dit : « La nature, acquérant en propre la qualité contraire, devient solide et fixe, dominante et dominée. » Ainsi elle acquiert en propre la qualité sulfureuse, celle qui donne son nom à l'eau de soufre natif. Pourquoi parle-t-il aussi du contraire ? C'est que l'eau est le contraire du feu. Sa qualité liquide empêche que les matières soumises au feu ne s'évaporent et ne se volatilisent. Elles sont comme ensevelies dans l'humidité et retenues jusqu'à ce qu'elles se teignent. L'eau retient parce qu'elle est liquide. C'est pour cela qu'il dit : « La nature acquérant en propre la qualité contraire, » etc. On a expliqué comment au moyen des liquides on obtient des produits qui résistent au feu ; or, les liquides, c'est l'eau divine.

\bigskip
\centerline{\EightStarTaper}
\centerline{\EightStarTaper\EightStarTaper}
\bigskip

\subsubsection{3. --- 18. Sur ce que l'Art a parlé de Tous les Corps en Traitant d'Une Teinture Unique.}
\paragraph{}
1. D'après le catalogue, on sait que Hermès et Démocrite ont parlé sommairement d'une teinture unique, et les autres y ont fait allusion. C'est ainsi que Africanus dit : « Ce que l'on emploie pour la teinture, ce sont les métaux, les liquides, les terres et les plantes. » Chymès l'a déclaré avec vérité : « Un est le Tout, et c'est par lui que le Tout a pris naissance. Un est le Tout, et si le Tout ne contenait pas tout, le Tout n'aurait pas pris naissance.\footnote{Voir \emph{Introd.}, p. 132, 135, 136, les axiomes de la Chrysopée de Cléopâtre.} Il faut donc que tu projettes le Tout, afin de fabriquer le Tout. » Pébichius : « Par le moyen des quatre corps. » Marie : « Par le moyen de la feuille de la kérotakis. » Agathodémon : « Après l'affinage du cuivre, (son) atténuation et (son) noircissement, et ensuite son blanchiment, alors aura lieu un jaunissement solide. » Toutes les autres (matières) sont expliquées semblablement chez eux.

2. Lorsque Marie parle de cette question, elle dit : « Il existe un grand nombre de corps métalliques, depuis le plomb jusqu'au cuivre. » Lorsqu'elle parle des diplosis, elle dit : « Il y a, en effet, deux sortes de matières employées, tantôt l'alliage de cuivre et d'argent, tantôt l'alliage d'or et d'argent ; le molybdochalque et tous les autres y sont compris.\footnote{\emph{Introd.}, p. 56, 60, 64.} » Quant à la purification de l'argent, ou à son noircissement, j'en ai parlé précédemment. Comme quoi une seule teinture s'applique à toutes (les matières), Marie seule le dit et le proclame en ces termes : « Si je parle du cuivre, ou du plomb, ou du fer, j'entends par là (leur) ios. »

\bigskip
\centerline{\EightStarTaper}
\centerline{\EightStarTaper\EightStarTaper}
\bigskip

\subsubsection{3. --- 19. Les Quatre Corps sont l'Aliment des Teintures.}
\paragraph{}
1. Voici comment : Marie dit que le cuivre est teint d'abord, et qu'alors il teint. Leur cuivre, ce sont les quatre corps. Voici les teintures : (elles comprennent) les espèces solides et liquides du catalogue, ainsi que les plantes ; les solides, depuis la vapeur sublimée jusqu'à la chrysocolle. Quant à toutes les (espèces) liquides du catalogue, en réalité, il s'agit de l'eau divine.

2. Ainsi, de même que nous sommes nourris au moyen des matières solides et liquides (réunies), et que nous sommes colorés seulement par leur qualité propre, de même se comporte leur cuivre ; et de même que nous ne sommes pas nourris au moyen de solides seuls, ou de liquides (seuls), de même aussi le cuivre ne l'est pas davantage. En effet, lorsque nous n'avons reçu (comme aliment) que de la matière solide, nous sommes enflammés, brûlés, empoisonnés ; de même aussi leur cuivre. Par contre, si nous n'avons pris que des boissons, nous sommes enivrés, nous avons la tête lourde, nous avons les joues colorées, et nous vomissons ; (de même) aussi le cuivre. Lorsqu'il a pris la couleur de l'or, par l'action de l'eau divine, il est alourdi et rejette, et aussitôt après (sa teinte) devient fugace. Mais lorsque nous avons pris en bonne proportion une nourriture composée des deux ordres de matière, solides et liquides, nous sommes alimentés raisonnablement ; nos joues se colorent raisonnablement et la faculté nutritive répartit la nourriture dans l'estomac, en raison de sa faculté de la retenir. De même aussi le cuivre, recevant les solides d'un côté à titre d'aliment, se nourrit d'autre part de l'eau divine unie à la gomme, à titre de vin ; il se colore, en raison de la faculté de retenir qui réside en lui. C'est ainsi que dans (l'ouvrage) précité, elle a dit : « Les sulfureux sont dominés et retenus par les sulfureux. » De là cette vérité : « La nature charme, vainc et domine la nature. »

3. « De même, dit-elle, que l'homme est composé des quatre éléments ; de même aussi le cuivre ; et de même que l'homme résulte (de l'association) des liquides, des solides et de l'esprit ; de même aussi le cuivre. Or Apollon, dans ses oracles, dit que l'esprit est la vapeur :

« Et un esprit plus noir, humide, pur.\footnote{Même citation, page 152.} »

4. Marie a parlé convenablement de la vapeur (en disant) : « Le cuivre ne teint pas, mais il est teint ; et lorsqu'il a été teint, alors il teint ; lorsqu'il a été nourri, il nourrit ; lorsqu'il a été complété, il complète. » Bonne santé.

\bigskip
\centerline{\EightStarTaper}
\centerline{\EightStarTaper\EightStarTaper}
\bigskip

\subsubsection[3. --- 20. Il faut employer l'Alun Rond Discours Contradictoire.]{3. --- 20. Il faut employer l'Alun Rond Discours Contradictoire.\footnote{Le sous-titre vient probablement de ce que cet article est tiré d'une discussion contradictoire. Cet article a pour but d'expliquer le blanchiment des métaux par le mercure ; la préparation de celui-ci au moyen du cinabre mis en contact avec divers métaux, et finalement l'emploi du sulfure d'arsenic (désigné par le nom d'alun rond) pour teindre le cuivre et les alliages qui en dérivent, à la façon du mercure.}}
\paragraph{}
1. Tu sais que : Un est le Tout et que du Tout naît le Tout. Or il faut savoir, comme nous l'avons démontré dans nos commentaires précédents, que les philosophes désignent sous le nom unique d'un corps tous ses dérivés ; principalement lorsqu'ils parlent du cuivre et du corps de la magnésie. Non seulement la vapeur sublimée rend le cuivre sans ombre ; mais encore le cuivre admet toutes les espèces, de même que le corps de la magnésie se fixe avec toutes. En effet il dit : « Fixe le mercure avec le corps de la magnésie.\footnote{Démocrite, \emph{Questions naturelles et mystérieuses}, p. 46.} Chercherons-nous donc à retenir la vapeur sur le Tout, afin de le fixer de cette manière ? Tous les écrits (disent) \emph{passim} : « Après avoir retenu la vapeur. » Or nous avons appris par l'expérience que s'il n'y a pas d'or, d'argent, d'étain, de plomb, la vapeur ne s'absorbe pas : que ferions-nous donc des pierres et du fer\footnote{Ces matières n'absorbent pas le mercure.} ?

2. Parmi les écrits, les uns disent : Il faut réduire le tout en bouillie et faire absorber l'eau de gomme : d'autres mettent en avant la vapeur (sublimée). Quant à moi je trouve préférable de broyer avec le cinabre. On sait que la cuisson de cette matière produit le mercure. C'est de cette façon qu'on le prépare. En effet, les espèces traitées au soleil, au moyen de l'eau ou du vinaigre, engendrent la vapeur (sublimée). Cela, nous le savons par expérience.

Tous les écrits et (notamment) Chymès et Marie parlent d'un mortier de plomb et d'un pilon de plomb.\footnote{Pour broyer le cinabre et réduire le mercure. Dans Pline, on produit cette réduction, en broyant le cinabre avec du vinaigre dans des mortiers de cuivre, avec des pilons de cuivre : \emph{H. N.} 33, 41.} On y délaie la chaux et le cinabre, avec le vinaigre, au soleil, jusqu'à ce que le mercure se développe. On produit le même effet avec l'étain. Les (espèces) chauffées, ou calcinées, ou fixées, ou teintes, sont susceptibles de fournir le mercure, si l'opération est faite suivant les préceptes de l'art. Quelle que soit celle de ces matières que l'on travaille, si elle est du cinabre en puissance, elle fournit de la vapeur et celle-ci s'échappe, le mélange étant délayé avec toutes sortes de corps.

3. On dira peut-être qu'il est préférable de broyer (le mercure) préalablement fixé et changé en ios ; attendu que les écrits ne parlent pas d'une simple fixation. Mais, suivant tous, la vapeur blanche, projetée sur notre cuivre, en fait de l'argent sans ombre. De même Stephanus, en présence de toutes les espèces, imagine qu'il s'agit d'une simple (fixation) par toutes les espèces. Mais, si l'on n'emploie qu'une simple fixation, sachez tous que l'on ne fait rien par là. En effet, la vapeur s'évapore pendant la fixation dans le feu et, l'esprit tinctorial étant perdu, on n'obtient rien ; tandis que si le cinabre est cuit avec les espèces, l'esprit n'est pas perdu. Cet esprit, c'est-à-dire la vapeur chauffée par le feu et poussée à la volatilisation, est retenu par les corps congénères qui y sont unis, notamment par l'étain.\footnote{Lb porte, au lieu de l'étain : Hermès ; le signe étant le même à l'origine (\emph{Introd.} Pl. 1, l. 7 ; p. 104). --- Ce passage signifie que le sulfure de mercure, étant réduit par un métal, ce métal fixe en même temps le mercure, si l'on opère par digestion prolongée ; tandis qu'une action brusque met à nu le mercure, qui s'évapore.}

4. D'après certain auteur, on doit se servir de l'alun rond,\footnote{C'est-à-dire employer le sulfure d'arsenic, ou son dérivé (c'est ici l'acide arsénieux, synonyme de l'alun ; v. p. 82, note 6), au lieu du cinabre ou du mercure.} au lieu de la vapeur (du mercure). Marie s'exprime conformément à cette opinion, lorsqu'elle dit : « L'infusion des teintures a lieu dans des fioles vertes ; soumises à un feu graduellement croissant. Le fourneau en forme de four a des mamelons, à sa partie supérieure. Si tu ne peux réussir, emploie le double d'alun rond, couleur de cinabre\footnote{Réalgar (\emph{Introd.}, p. 238 et 244, article Cinabre).} ; ce qui vaut mieux pour atteindre le même résultat. Avec d'autres pâtes on réussit aussi. En effet la vapeur sublimée se fixe seulement sur les quatre corps ; quelques-uns disent qu'elle est absorbée par les autres corps, avec le concours de la chrysocolle. Pour ma part, je sais bien que la chrysocolle seule ne la retient pas ; (mais) les corps métalliques morts et délayés conservent tous la vapeur.\footnote{Sans doute à la condition de les ramener simultanément à l'état métallique par des agents réducteurs ( ? ).} »

5. Il a été dit par Agathodémon que la chrysocolle et la vapeur sont amies l'une de l'autre ; (la chrysocolle) la retient ; l'une agit comme la limaille\footnote{Des métaux qui s'unissent au mercure.} ... l'autre, même broyée, n'a pas l'adhésion du cinabre.\footnote{C'est-à-dire que l'emploi de l'arsenic sublimé ne blanchit pas les métaux aussi facilement que celui du cinabre.} L'une et l'autre, étant délayées ensemble à l'état sec, s'amalgament. Mais la vapeur en puissance agit sur le cuivre en puissance\footnote{C'est-à-dire qu'au lieu d'employer le cuivre libre et le principe colorant et volatil tiré de l'arsenic à l'état libre, il faut opérer sur des composés susceptibles de les engendrer.} et ils s'unissent ainsi.

6. Il faut chercher comment la vapeur est absorbée par toutes choses, non seulement par les corps métalliques à l'état vivant et délayé, mais encore à l'état brûlé. En fait, elle est absorbée par les métaux, surtout ceux qui tirent leur origine du cuivre.\footnote{C'est-à-dire par les alliages à base de cuivre, ou supposés tels.} Si tu ne réussis pas, mets le double de cinabre. On réussit ainsi avec tout ; c'est là ce que le Philosophe veut exprimer en disant : « Il te faut comprendre toutes choses et d'abord ne pas te relâcher de l'art ; car la méditation mène au chemin véritable. » Ces choses ont été rapportées par moi, qui voulais montrer que l'alun rond agit semblablement, ainsi que l'a dit surtout la divine Marie.

\bigskip
\centerline{\EightStarTaper}
\centerline{\EightStarTaper\EightStarTaper}
\bigskip

\subsubsection[3. --- 21. Sur les Soufres.]{3. --- 21. Sur les Soufres.\footnote{B : « Sur les eaux divines. »}}
\paragraph{}
1. Ne m'as-tu pas demandé l'explication concernant les soufres, demeurant jusqu'à ce jour fidèle à ton serment ? Cette explication te sera donnée en temps opportun. Tu sais que ce n'est pas seulement le Philosophe qui a mentionné les soufres, mais encore tous les prophètes ; car, sans les soufres il n'y aura rien, c'est-à-dire sans l'eau divine. En effet toute la composition est absorbée par elle ; c'est par elle qu'elle est cuite ; par elle, qu'elle est brûlée ; par elle, qu'elle est fixée ; par elle, qu'elle est teinte ; par elle, qu'elle subit l'iosis et par elle, qu'elle est affinée.\footnote{Cp. p. 147.} Car il dit : « Mets de l'eau de soufre natif et un peu de gomme : tu teins par là toute sorte de corps. » Ecoute encore le même auteur : « Laisse descendre et le produit se forme\footnote{Cp. Stephanus, édition Ideler, p. 247, l. 21.} : c'est là le mystère manifeste. » Mais quelqu'un dira : Qu'est-ce qui ressemble à l'eau divine, parmi les sulfureux ? --- Nous lui répondrons : d'abord qu'est-ce qui a opéré avec autre chose que les eaux divines ? Or si (personne) n'a opéré autrement, c'est avec raison que mon Philosophe n'a pas parlé d'autre chose que ce que nous comprenons (par-là).

2. On appelle donc divine l'eau de soufre. Ecoute bien. On appelle divine la vapeur sublimée, émise de bas en haut. De même aussi, la cendre formée sur les parois des conduites de fumée est appelée divine. Semblablement aussi les gouttes jaillissantes des bains ; les gouttes qui se fixent aux couvercles des chaudières, on les appelle pareillement divines. Le mercure blanc, on l'appelle encore divin, parce que lui aussi est émis de bas en haut.\footnote{Cette phrase répond à l'axiome : « En haut les choses célestes, etc. » (\emph{Introd.}, p. 162 et 163) ; le nom d'eau divine correspondant aux choses célestes et en même temps au soufre, par le double sens du mot grec. --- On voit aussi par ce paragraphe quel sens compréhensif avaient les mots : soufre ou divin, eau de soufre ou eau divine ; mots entre lesquels règne une perpétuelle confusion.}

3. Les anciens\footnote{Ce qui suit se compose d'une série d'alinéas, pour la plupart sans liaison les uns avec les autres.} ont l'habitude de faire cuire les sulfureux, en les chauffant sur un feu léger dans des fioles. Or ce que le feu effectue par artifice, le soleil l'effectue par le concours de la nature divine. Le grand Hermès dit : « Le soleil qui fait tout. » Hermès dit encore partout : « Expose au soleil et délaie la vapeur au soleil. » Çà et là il désigne le soleil. Le feu solaire accomplit toutes les opérations que nous avons dit précédemment s'effectuer dans des fioles. L'autre composition est bouillie de cette façon avec la saumure jusqu'à blanchiment. Il en est de même des choses dont il nous parle comme exécutées sous la canicule et sous l'influence solaire, ainsi que nous l'enseigne l'expérience des deux procédés.

De même que le levain du pain, employé en petite quantité, fait lever une grande quantité de pâte ; de même aussi la petite feuille d'or ou d'argent engendre toute la poudre de projection (et) fait fermenter toutes choses.

Si nous entendons dire 3, 5 et 7, on veut faire entendre le total 15.

Voilà comment ils jugent à propos d'opérer. On fait tout amollir dans des vases de verre ; car les poteries de terre doivent être écartées dans l'opération de l'iosis, de crainte qu'elles n'absorbent la teinture et la fleur de la teinture. Leur nature réceptrice se sature d'abord et se teint avec la fleur d'or, et ensuite la scorie du cuivre n'absorbe plus la fleur de l'iosis.

4. Là, nous opérons la teinture dans des vases de verre, vu qu'ils se prêtent convenablement à l'iosis. Mais il ne faut pas toucher (la teinture) avec les mains, car elle est mortelle. Lorsque l'or y a été dissous, c'est le plus délétère de tous les métaux.

Les uns délaient avec l'ios, ce que tu as appris à connaître : j'entends le soufre ; ils (en) enduisent la feuille d'argent.

En opérant de cette façon, ils font chauffer progressivement l'appareil de l'art, sur un fourneau arrondi, dans un creuset disposé sur des gradins : et l'or se produit.

5. Quelques-uns, et Marie (entre autres), ont mentionné la figure d'en bas. « C'est ainsi qu'ils ont préparé, le mercure, dit-elle, ainsi que le soufre et l'ios, en délayant l'ensemble au soleil jusqu'à ce que le tout devienne ios. Ils disent que celui-ci (ainsi préparé) est plus actif. Quelques-uns ont accompli cette iosis au soleil seulement, sans rien ajouter, et ils affirment qu'ils ont obtenu l'objet de leur recherche. D'autres ont délayé avec l'eau divine, affirmant que c'est là leur soufre ;--- c'est aussi leur mercure.\footnote{Voir la note 2 de la page suivante et celle de la page 166.} J'ai admis l'opinion de ceux-ci, plutôt que celle des autres. D'autres projetaient du mercure, tantôt cru, tantôt à l'état de concrétion jaune.\footnote{\emph{Introd.}, p. 104, Pl. 1, l. 21 ; et p. 112, Pl. 4, l. 17. Est-ce l'oxyde de mercure précipité ?} Quelques-uns, après l'opération de l'iosis, n'ont rien effectué au-delà.

6. Quant aux philosophes, ils s'exprimaient par énigmes au sujet de (l'opération qui succède à) l'iosis, disant : « Pour teindre l'or, il vaut mieux opérer après l'iosis. » D'autres, parmi les hiérogrammates qui ont écrit uniquement sur cet art, en s'occupant du délaiement,\footnote{On remarquera les sens multiples du mot λειόω, et du substantif correspondant λείωσις. Il s'agit, suivant les cas : soit de polir la surface d'un métal, ou de la rendre lisse à l'aide d'un vernis ; soit de broyer une poudre ; soit de délayer cette poudre dans un liquide (délaiement = λειώσιμον dans le \emph{Dictionnaire Français-Grec moderne} de Byzantius), ou de la léviger ; soit de saupoudrer la poudre sèche, ou d'étendre la poudre délayée dans un liquide visqueux, à la surface d'un métal, lequel se trouvera verni ou teint après avoir subi l'action du feu. Dans le § présent, ce dernier sens est surtout applicable.} disaient que l'iosis seule fait tout, et principalement l'ios. Cela leur convenait ainsi. D'autres, après avoir fait cuire, faisaient chauffer et mettaient au feu, à la suite de la fonte ; ceux-ci préféraient traiter le Tout par délaiement. Ceux qui voulaient n'avoir recours qu'au blanchiment, enduisaient une feuille d'argent, faisaient chauffer et cuire. Ils polissaient jusqu'à ce que tout eût absorbé la matière délayée, en opérant avec l'eau (de soufre ? ), le mercure et quelque substance semblable.

7. Comme dans la cuisson de l'art diverses couleurs se manifestent, Agathodémon plus que tous s'est préoccupé des délaiements. En cela ils sont d'accord pour enduire le petit objet\footnote{Voir p. 157, § 2.} avec du soufre, de la chrysocolle et de la fleur de sel (délayés). « Si tu t'aperçois, dit-il, que certaines substances sont brûlées, fais chauffer et délaie au soleil, jusqu'à ce que (la couleur) se développe. Par-là, ils ont de préférence indiqué la cuisson et le délaiement. Ils agissent ainsi pour montrer la puissance de la préparation : prenant des objets d'argent et les couvrant d'un enduit jusqu'à moitié, ils font chauffer la préparation ; et lorsqu'ils enlèvent l'objet, il est doré dans la partie enduite, tandis que l'autre (partie) reste intacte.\footnote{Ce dernier § indique clairement qu'il s'agit de donner à un objet d'orfèvrerie une coloration en or superficielle, comme dans les Papyrus de Leide : \emph{Introd.}, p. 59 et 60.}

Telle est l'explication concernant l'eau divine.

\bigskip
\centerline{\EightStarTaper}
\centerline{\EightStarTaper\EightStarTaper}
\bigskip

\subsubsection{3. --- 22. Sur les Mesures.}
\paragraph{}
1. L'explication concernant les mesures met en évidence tout le mystère de la cuisson ; car c'est là la composition, c'est là le poids, c'est là le blanchiment, c'est là le jaunissement. Or, dans le discours sur la composition, ces matières (ont été traitées en passant), et il en a été de nouveau question (dans le discours) sur le cuivre et l'iosis. Il paraît employer ce plomb, lorsqu'il dit : « saupoudre avec du plomb. » Il ne parle pas du plomb simplement, mais il ajoute : « avec notre plomb noir, provenant du minerai de Coptos et de la litharge. » Or l'opération de saupoudrer me paraît être un délaiement, comme je le montre d'après tous les écrits, dans mon Traité sur l'Action, en y parlant du poids. Ils ont l'habitude de peser ensemble secrètement les choses au moyen desquelles ils brûlent, ou saupoudrent, ou projettent. Ils pèsent le plomb destiné au saupoudrage : le blanchiment est soumis à la pesée ainsi que l'ios, lors de la projection. En effet : « rejette, dit-il, la moitié de la préparation blanche, etc. »

2. Ainsi toutes choses ont été cachées dans toutes les opérations de l'art, relativement à la pesée comparative et à l'iosis. Je dis toutes choses en même temps : attendu que si le soufre prédomine dans la coupe, on ne voit pas la composition placée au-dessous, de façon à connaître quand elle est blanchie par (l'action du) soufre lui-même. C'est lorsqu'il devient blanc, que l'on reconnaît que la (composition) située au-dessous a été blanchie. Par suite, Agathodémon disait de prendre (chaque préparation de) soufre,\footnote{Au-dessus du signe du soufre, E écrit celui du mercure ; et Lb donne à la place de ces signes le nom du mercure en toutes lettres.} qu'il fût blanc ou quelconque.\footnote{Cp. p. 166, § 19.} C'est son état qui indique la cuisson. On enlève et on fait chauffer (le produit) avec le surplus du soufre ; il le sépare (en deux portions ? ), plutôt qu'il ne l'affine ; car il s'empare de (la composition) blanchie. Si on le laisse (trop longtemps), il tourne au jaune.

C'est pourquoi le soufre produisant le blanchiment, nous chercherons le poids du Tout d'après les philosophes.\footnote{Voir p. 161, § 8 ; p. 163, § 11, etc.} On prend dans la (classe) dernière des liquides, une once d'arsenic et moitié autant de natron ; des pellicules de feuilles de pêcher encore tendres, deux onces ; du sel, la moitié ; du suc de mûrier, une once. Puis on délaie tout cela avec de l'alun lamelleux et du vinaigre, ou de l'urine, ou de la lessive de chaux, jusqu'à ce qu'il se forme une liqueur. Ensuite, on teint les feuilles (métalliques ? ) ternies ; puis on fait disparaître l'ombre du métal. Il faut mettre tous les résidus, et, avant tout, une partie d'arsenic et de sandaraque, deux parties de chaux, ainsi que les eaux divines. Après avoir obtenu une liqueur blanche semblable à du marbre, on arrose avec elle ; ou bien l'on y fait cuire dans le vase (\emph{Troullon})\footnote{Cp. p. 164.} la composition susdite.

\bigskip
\centerline{\EightStarTaper}
\centerline{\EightStarTaper\EightStarTaper}
\bigskip

\subsubsection{3. --- 23. Comment on Brûle les Corps.}
\paragraph{}
1. Cherchons maintenant, d'après les philosophes, ce que c'est que brûler les corps ; car l'explication concernant les poids y aboutit et l'ensemble (de notre étude) renferme (cette question). Introduis le Philosophe disant : « Prends la vapeur (qui provient) de l'arsenic, fixe-la suivant l'usage ; ajoute du cuivre ou du fer à (la préparation) sulfureuse, et le métal blanchit. » Quelques-uns expliquent le (mot) « sulfureuse » par « brûlée ; » car ceux-ci dans leur ignorance brûlent le cuivre avec le soufre, et le fer avec la magnésie. Or ce n'est pas là brûler, mais détruire. L'opération de brûler dans le Philosophe est nommée blanchiment. De même que l'affinage et les autres opérations ont été démontrés être un blanchiment ; de même aussi l'opération de brûler dont il parle ici est un blanchiment ; dans le second (cas), c'est un jaunissement.

2. Ainsi, le Philosophe brûle le cuivre au moyen de l'eau de soufre, pratiquant une décoction, comme il a été dit précédemment. « En effet, dit-il, mets (y) la moitié de la préparation blanche : ce sera le premier degré. Fais-la cuire. Nous conservons l'autre moitié pour l'iosis. » C'est aussi pour cette raison que Pébichius, \emph{passim}, disait : « Partagez la préparation en deux parties. Brûlez le cuivre dans du bois de laurier,\footnote{Voir p. 159. --- Ce mot paraît signifier un sulfure arsénical.} c'est-à-dire dans la composition blanche ; car les corps brûlés de cette façon avec des feuilles de laurier, après avoir été cuits dans l'eau de soufre, sont blanchis en même temps. Tel est le (précepte). Emploie du cuivre ou du fer sulfuré ; par ce (procédé), il sera aussi blanchi. » Agathodémon donne le même conseil : à savoir que les corps doivent bouillir et cuire avec la vapeur dans l'eau divine. De cette façon il y a opération de brûler et blanchiment. Car à l'occasion de l'étain le Philosophe supposait la cuisson : « Tu feras cuire la vapeur indiquée précédemment dans l'huile de ricin ou de raifort, après y avoir mélangé un peu d'alun. » Il dit ensuite : « Fais les mélanges de l'étain, etc. et toutes choses seront traitées jusqu'au bout avec deux classes (de corps) seulement. » Après avoir parlé des jours, il a mentionné toutes choses ; après avoir parlé des huiles, il a mentionné l'eau divine ; à la suite de l'alun, le soufre ; à la suite de l'étain, les deux formules ; car la vapeur (sublimée) imprègne ce métal.\footnote{Toute cette description se rapporte au blanchiment des métaux par la vapeur de l'arsenic, avec le concours de la liqueur appelée eau divine.}

3. Les projections (se font) encore ici avec les liqueurs de soufre ; tandis que la cuisson concerne l'ensemble, qui (est) une combustion, ou une décoction et un blanchiment. C'est par là que les corps sont brûlés et cuits. Cette opération (est celle) qui a été proclamée de tout temps ; celle que tous les écrits enseignent en ternies mystérieux, (en prescrivant de) brûler le cuivre avec le soufre. Mais les autres (modes de) chauffage sont des destructions, plutôt que des combustions. Le cuivre, s'il est brûlé, (devient) un cuivre propre à tout et apte à la teinture ; en disparaissant, il devient électrum. Si l'on force le feu, il devient jaune, la moitié du soufre étant brûlée. Il faut le quart de magnésie. Ainsi nous ajoutons 4 onces de cuivre, 1 once de fer, 6 scrupules de magnésie ; 2 chalques\footnote{1 chalque = 8\textsuperscript{e} d'obole = 0,091. gr.  } d'étain et de plomb, de la cadmie, du claudianon, de la chrysocolle, du cinabre, en proportion du nombre d'onces des métaux. Si tu procèdes en proportions égales, par à peu près, tu peux réussir. Mais opérer dans ces conditions, c'est laborieux et peu sensé. Il faut procéder par pesées. Démocrite ayant dit : « Rien n'a été omis, rien ne manque ; » certes, par le mérite de Démocrite ! rien n'est laissé en arrière : la composition des corps dissous, c'est-à-dire la montée de l'eau divine et de la vapeur, nous l'avons exposée sincèrement ; et nous avons donné par-là l'interprétation du Livre. Maintenant que nous avons décrit la mesure pour l'acte de brûler, examinons celle du jaunissement.

Lb dit « de mercure, » au lieu d'étain ; probablement parce que le copiste a donné par erreur au signe d'Hermès le sens moderne de mercure, au lieu du sens ancien d'étain (\emph{Introd.}, p. 84).

\bigskip
\centerline{\EightStarTaper}
\centerline{\EightStarTaper\EightStarTaper}
\bigskip

\subsubsection{3. --- 24. Sur la Mesure du Jaunissement.}
\paragraph{}
1. Pourquoi Agathodémon a-t-il écrit sur ce sujet ? Ce n'est pas en vue d'enseigner la mesure, mais pour dire qu'il faut employer en safran et en chélidoine le double des autres herbes ; car celles-ci ont de plus grandes propriétés tinctoriales. Il règle la proportion, en raison du soufre blanc. L'eau tirée des soufres, des jus et des herbes, est appelée ici eau de soufre pur. C'est avec cela qu'ils arrosent et font cuire la composition blanche : elle est jaunie par là. Fais cuire, comme tu l'as entendu dire précédemment, en enlevant dès que la matière jaunit. C'est la mesure du jaunissement. Telle est l'explication concernant la mesure, annoncée plus haut.

2. Il faut savoir que pendant qu'on accomplit l'œuvre, plusieurs causes concourent, les unes visibles à l'œil nu, les autres non. Les premières sont les espèces lavées ou mélangées, le molybdochalque et les similaires, la pyrite et les similaires. Il ne faut pas que la pyrite et l'androdamas soient traités d'avance par le vinaigre, d'après ce que disent les écrits, afin d'éviter que leur partie cuivreuse ne se change en ios ;--- plus tard elle sera mélangée avec le cinabre et ses similaires. Il est permis (de les exposer) au soleil, ainsi que les autres choses semblables.

3. Marie (place) en première ligne le molybdochalque et les (procédés de) fabrication. L'opération de brûler (est) ce que tous les anciens préconisent Marie, la première, dit : « Le cuivre brûlé avec le soufre, traité par l'huile de natron, et repris après avoir subi plusieurs fois le même traitement, devient un or excellent et sans ombre. Voici ce que dit le Dieu : Sachez tous que, d'après l'expérience, en brûlant le cuivre (d'abord), le soufre ne produit aucun effet. Mais lorsque vous brûlez (d'abord) le soufre, alors non-seulement il rend le cuivre sans tache, mais encore il le rapproche de l'or. » Marie, dans la description située au-dessous de la figure, le proclame une seconde fois, et dit : « Ceci m'a été gracieusement révélé par le Dieu, à savoir que le cuivre est d'abord brûlé avec le soufre, puis avec le corps de la magnésie ; et l'on souffle jusqu'à ce que les parties sulfureuses s'en échappent avec l'ombre : (alors) le cuivre devient sans ombre. »

4. C'est ainsi que tous brûlent. C'est ainsi que dans la chimie (μᾶζα)\footnote{Voir sur le mot μᾶζα, \emph{Introd.}, p. 209 et 257, et la \emph{Diplosis de Moïse}, p. 40.} de Moïse on brûle avec du soufre, du sel, de l'alun et du soufre (j'entends le soufre blanc). Ainsi encore Chymès brûle dans beaucoup d'endroits, surtout lorsqu'il opère avec la chélidoine. Ainsi dans Pébichius, l'opération de brûler dans du bois de laurier\footnote{Voir p. 159, § 3 et note 2 ; p. 178, note 1.} est exposée énigmatiquement et par périphrase ; les feuilles de laurier signifiant le soufre blanc. Telle est l'explication concernant les mesures.

5. Voici ce que Marie a dit, çà et là, dans mille endroits : « Brûle notre cuivre avec du soufre et, après avoir été repris, il sera sans ombre. » Non seulement elle sait le brûler avec le soufre blanc, mais encore le blanchir et le rendre sans ombre. C'est aussi avec le (soufre) que Démocrite brûle, blanchit et rend sans ombre. Et encore, « non seulement ils brûlent le soufre jaune, mais ils rendent le métal sans ombre et le jaunissent. » Voici ce que dit Démocrite : « Le safran a la même action que la vapeur ; de même que la casia par rapport à la cannelle. » Dans la chimie de Moïse, vers la fin, pareillement, il y a ce texte : « Arrose avec l'eau de soufre natif, il deviendra jaune et sans ombre ; » c'est-à-dire évidemment, brûlé.

6. Telle est l'opération de brûler ; tels sont le blanchiment, le jaunissement, et dans les deux (cas), le fait de rendre (le métal) sans ombre. Brûlant et reprenant de cette manière, vous rendrez le cuivre pareil à l'or (et) sans ombre, apte à la diplosis de l'argent et de l'or.\footnote{On voit qu'il s'agit, ici comme dans les Papyrus de Leide, de fabriquer un alliage d'or, qui conserve plus les propriétés apparentes de ce métal (\emph{Introduction}, pages 20, 53 et 56).} Mais personne, à moins de connaître toute la route, ne pratiquera bien la diplosis ; autrement il agirait comme celui qui dessécherait des raisins encore verts. Quelques-uns placent, dans tous leurs pots de terre des vases de verre carrés, pour faire cuire et digérer sur la kérotakis (bain marie) ; et ils les appellent lécythes (flacons). Agathodémon prescrit de délayer fortement, en se conformant à la marche suivie par les médecins pour les collyres.

7. Tel est donc l'acte de brûler les corps ; telle l'explication concernant les mesures. L'acte de brûler est appelée blanchiment ; pour le soufre, cet acte est appelé blanchiment et destruction de l'ombre. Le blanchiment même est appelé iosis et l'affinage est aussi un blanchiment. L'acte de brûler est encore appelé jaunissement, la destruction de l'ombre, jaunissement, et l'iosis, jaunissement. Le prophète Chymès, s'écriait avec enthousiasme : « Après les projections, il faut le rendre jaune et sans ombre. » Ensuite on t'expliquera le procédé relatif à l'eau divine et à l'iosis ou décomposition.

\bigskip
\centerline{\EightStarTaper}
\centerline{\EightStarTaper\EightStarTaper}
\bigskip

\subsubsection[3. --- 25. Sur l'Eau Divine.]{3. --- 25. Sur l'Eau Divine.\footnote{Cet article est un commentaire, plus récent que les vieux auteurs. ---- Voir 3, 14, p. 155.}}
\paragraph{}
1. Il faut montrer d'abord que l'eau divine est un composé de tous les liquides, obtenu par leur mélange, et que son nom est donné à tous les liquides. De même que l'on a nommé composition solide, le produit obtenu avec chacune des compositions solides, envisagée spécialement ; de même aussi, la composition liquide, tirée de chacune des espèces liquides, est dénommée eau divine, et l'on désigne ces deux compositions par mille noms. L'eau divine est désignée par les mots : saumure, eau de mer, urine d'impubère, vinaigre, saumure acide, huile de ricin, (huile) de raifort, baume, lait de la mère d'un enfant mâle, lait de vache noire, urine de génisse et de brebis ; quelques-uns la dénomment urine d'âne ; d'autres encore, eau de chaux et de marbre, de lie de vin ; eau de soufre, d'arsenic et de sandaraque, de natron, d'alun lamelleux ; et encore lait d'ânesse, de chèvre, de chienne ; eau de cendre de choux et autres eaux produites par la cendre ; d'autres désignent aussi par ce nom l'eau de miel et d'oxymel, de vinaigre, de natron, et l'eau aérienne (rosée), celle du Nil, de l'Arction,\footnote{Plante ? (Dioscoride, \emph{Mat. méd.}, 5, 104.)} le vin Aminéen, le vin de grenade, le vin d'olivier, le cidre, la bière, enfin un liquide quelconque, pour ne pas énumérer toutes les eaux.

2. Les Anciens ont donné souvent des noms divers au blanc et au jaune. Il me paraît convenable d'exposer quelles distinctions le philosophe Pébichius faites dans sa lettre au Philosophe, sur les liqueurs jaunes. « Etends avec du vin Aminéen ... » Ils n'ont pas énuméré le vin nouveau, parmi les liqueurs destinées au blanchiment. Pébichius dit encore : « Le cidre, le vin d'olivier et le vin de grenade. » En ne distinguant pas davantage, ils n'ont pas rendu service à (leurs) auditeurs, et ils ont agi avec peu d'intelligence. En effet, en traitant des diverses espèces, le Philosophe les emploie pour le blanchiment et pour le jaunissement ; il les emploie pour les traitements que tu as entendu signaler précédemment, destinés à brûler et à faire cuire. Il dit à propos de la pyrite : « Prenant la pyrite, traite-la et délaie-la, soit avec de la saumure acide, etc. » Voilà ce qu'il entend par eau divine blanche. Ensuite, à propos du cinabre : « Rends le cinabre blanc au moyen de l'huile, ou du vinaigre et du miel, etc. » A propos de l'Androdamas, de même encore : « avec la saumure, ou la saumure acide. » Ensuite il ajoute : « Fais chauffer l'eau de soufre natif ; » afin de te faire connaître que les eaux de mer, l'urine, le vinaigre, l'huile de cinabre, l'eau de miel, tout cela c'est l'eau divine. En effet par une seule espèce il fait entendre le tout. Plus loin, dans l'article de l'Androdamas, voulant parler clairement, il disait : « Fais chauffer l'eau de soufre natif, car les liquides sont les eaux de soufre natif. »

3. « Les (matières à) projection tirées de la chaux changent de nom et de couleur, quand il s'agit du soufre blanc. Ce sont la terre de Chio, l'astérite et la sélénite, pour la classe du blanc. Quand il s'agit du jaune, projette de l'ocre attique, du minium du Pont cuit, et les similaires. »

Au sujet de la chrysocolle, il dit : « Brûlant cette matière et l'arrosant d'huile jusqu'à sept fois. » Dans la Chrysopée, il a fait blanchir d'abord chacune de ces (substances). Il emploie semblablement la litharge dans les deux compositions. Car il n'y a pas plus de deux décoctions pour accomplir l'opération. Parmi les liqueurs, il comprend la vapeur et la litharge, (mêlées) avec le miel le plus blanc. Il ne négligeait aucun des liquides ; mais il les employait dans les deux compositions. En effet il mélangeait une solution de comaris et de lentilles ( ? ), en y ajoutant une préparation de chélidoine ; et il disait obtenir la composition de l'eau divine. Il prescrit de faire bouillir l'eau de chaux (obtenue par le marbre) avec de l'huile, et la pyrite avec du miel. Il décrit l'eau divine de diverses façons, dans ses quatre livres. Dans le livre de l'Argent ; il parle de la terre de Chio, de l'astérite, de la sélénite, et de sa propre projection. Dans le livre du Jaune, il s'agit de la terre de Sinope, de l'ocre attique et de la pierre phrygienne. « Tu trouveras dans le traité des Pierres, le sang de bouc et le suc de lotos ; et, plus loin ce qui est utile ... Les sulfureux sont dominés par les sulfureux, et les liquides par les liquides correspondants.\footnote{Axiome souvent répété, p. 20 et 145.} En effet les sulfureux sont retenus par les sulfureux. »

\bigskip
\centerline{\EightStarTaper}
\centerline{\EightStarTaper\EightStarTaper}
\bigskip

\subsubsection[3. --- 26. Sur la Préparation de l'Ocre.]{3. --- 26. Sur la Préparation de l'Ocre.\footnote{Le premier paragraphe est un fragment technique, probablement fort ancien (voir Théophraste, \emph{Sur les pierres}, t. 1, p. 701, éd. Schneider ; Leipzig, 1818). On y remarquera l'assimilation du réalgar, du minium et de la rubrique avec l'ocre (voir \emph{Introd.}, p. 261).}}
\paragraph{}
1. La préparation de l'ocre se fait dans la montagne (voisine) de la mer appelée Adriatique. Il y a là des crevasses de la montagne ; à travers les fentes on voit des couches d'ocre en plaques. L'ocre est produite aussi en Babylonie dans la montagne. On voit l'ocre dans les fentes ; on l'enlève et on la fait cuire : on obtient ainsi la rubrique, que l'on appelle encore minium de Sinope. Nous, nous n'employons ni cette rubrique, ni ce minium de Sinope. Mais l'ocre indiquée ci-dessus est la véritable teinture ; à moins que le métal que l'on se propose de teindre ne soit le corps de la magnésie, ou le plomb noir.

2. Quel rang doit lui être assigné en dehors des matières tinctoriales, tous les écrits s'expliquent sur ce point. Si par conséquent tu veux lui fixer un rang, c'est là que tu trouveras le résultat cherché ; surtout si tu suis Marie et le Philosophe. Le Philosophe mentionne les pyrites, le cinabre, le claudianon, la cadmie, l'androdamas, la chrysocolle. Il dit qu'il convient de faire agir sur le molybdochalque, le cinabre, ou le corps de la magnésie, substance qui est appelée plomb noir. Si maintenant tu en viens à la Chrysopée, tu verras quelles (substances) désagrègent l'étain, le fer ou le cuivre : ce sont le cinabre, la litharge blanche. A ton tour comprends ce que tu cherches : par la magnésie, entends le molybdochalque ; par le plomb, c'est (encore) le molybdochalque. Lorsqu'ils parlent d'Argyropée ou de Chrysopée, ils entendent le molybdochalque ; c'est là le produit qu'ils traitent, puis soumettent (à la teinture). Au moment voulu, ils le fixent, après l'avoir désagrégé ; alors ils blanchissent, ou jaunissent le métal durci par eux.

3. Ils blanchissent le cuivre et, après l'avoir broyé, ils le gardent jusqu'au résultat final. L'opération faite avec le soufre et le mercure, ils l'appellent brûler. Ils appellent cuivre brûlé, ce métal rendu couleur de sang (en vue du blanchiment), teint superficiellement et à fond.\footnote{Cp. \emph{Introd.}, p. 233, le cuivre brûlé, et plus haut, p. 154 et 178.} C'est là ce qu'ils appellent brûler ; par-là (le Philosophe) fait entendre la composition totale ; il désigne sa dilution, (opérée) en vue des deux teintures. En suivant la voie directe, il a parlé d'abord du blanchiment, puis du jaunissement.

\bigskip
\centerline{\EightStarTaper}
\centerline{\EightStarTaper\EightStarTaper}
\bigskip

\subsubsection{3. --- 27. Sur le Traitement du Corps Métallique de la Magnésie.}
\paragraph{}
1. Introduisons de nouveau les Anciens. Ils disent que le cinabre produit le blanchiment de la magnésie. Pour rendre efficaces les discours antérieurs que j'ai écrits, relativement aux quatre corps qui servent de supports et à la mesure que comporte à leur sujet la composition crue et cuite,\footnote{P. 150 et 151.} il est nécessaire de faire l'application de tout cela à l'explication de la magnésie. Il faut dire comment on forme le corps (métallique) de la magnésie ; et si le blanchiment varie suivant la macération, ainsi que je te l'ai dit précédemment. Laisse-la devant le fourneau ; que le fourneau soit allumé avec du bois et des écorces de cobathia rouges,\footnote{Composé arsenical (voir plus bas).} car la fumée de ces écorces blanchit tout. Si donc tu en recueilles la fumée, la magnésie l'absorbe et elle est blanchie.

2. N'avons-nous pas rappelé dans le 7\textsuperscript{e} livre, en parlant des cobathia rouges, que nous devions apprendre d'abord de quelle magnésie parlent les philosophes ? Si c'est de la (magnésie) simple, provenant de Chypre, ou de la magnésie composée, obtenue par notre art ? En effet, en délayant la magnésie simple, ils veulent parler de la composée\footnote{Molybdochalque.} ; mais ils entendaient en même temps la simple. C'est de cette façon que l'art a été caché par le double sens attribué aux dénominations.

3. Le philosophe Hermès, après l'eau de mer, nomme le natron, le vinaigre, le sang de moucheron,\footnote{\emph{Lexique}, p. 10. Il y a ici un symbolisme et des dénominations semblables aux noms prophétiques du Papyrus V de Leide (\emph{Introd.}, p. 10 et 11) et de Dioscoride.} le suc du styrax, l'alun lamelleux, et autres substances semblables, et il dit : « Laisse-la devant le fourneau, comme je l'ai dit précédemment, avec un feu d'écorces de cobathia rouges, car la fumée des cobathia rouges blanchit tout, étant blanche elle-même.\footnote{\emph{Olympiodore}, p. 91. Dans tout ce passage existe une confusion, qui semble voulue et amenée par la nomenclature prophétique, entre le nom des écailles ou morceaux de cobathia rouges, c'est-à-dire des sulfures d'arsenic (\emph{Introd.}, p. 245) et celui des écorces et rameaux des palmiers. Rappelons que le même mot grec φοῖνιξ signifie rouge et palmier. La dernière phrase du § 2 montre le caractère intentionnel de ces confusions.} »

4. Ainsi parle Hermès ; mais nous devons savoir que le natron, le styrax, l'alun schisteux et la cendre des rameaux de palmier, c'est le soufre blanc, qui blanchit tout. Quant au sang de moucheron et au vinaigre, c'est l'eau de soufre (obtenue) avec la chaux ; les écorces des cobathia rouges, ce sont les sulfureux, principalement l'arsenic, lequel ressemble aux cobathia : ce sont là les corps employés pour teindre en or. Il dit : « La fumée des cobathia blanchit tout. » Voulant enseigner ce que c'est que les cobathia, le Philosophe dit : « La vapeur du soufre blanchit tout. »

5. Maintenant le Philosophe voulant t'enseigner (ce que c'est que) la cendre des palmiers maritimes, qui est aussi l'eau divine, s'exprime ainsi dans la seconde classe, celle des liqueurs blanches : « Ayant dissous la cendre du bois des peupliers blancs dans l'eau de soufre [ceci n'est pas pris dans un sens simple], ou dans l'eau de soufre obtenue par la chaux, laquelle provient de la cendre blanche, du marbre, ou de la chaux vive. » De même que les sulfureux ont été dits (provenir) des cobathia rouges, de même l'eau de soufre tire sa composition du soufre ; celui-ci est aussi désigné sous le nom de palmier. De plus (on voit que) le blanchiment de la magnésie composée est produit par la composition du soufre blanc et que la composition liquide du blanc est obtenue par la chaux. Ce sont là toutes (matières) dont (j'ai expliqué) la préparation, dans mon discours sur la composition ; j'en ai dit la mesure, dans le discours sur les mesures ; le mode de cuisson et la conduite du fourneau, dans le discours sur la cuisson.

6. Voilà pour le blanchiment du corps de la magnésie. Or il vous est loisible, à vous qui avez du bon sens, d'entreprendre ce qui est le mieux et de nous seconder, au lieu de nous précipiter dans ce gouffre (de difficultés). Celui qui fait quelque autre raisonnement concernant cette doctrine, demeure dans une obscurité profonde ; il agit comme un homme qui frapperait l'air avec ses mains, et la mer avec ses pieds. Ceux qui marchent dans le vide et parlent tout à fait en l'air, travaillent inutilement par des procédés qui leur sont propres (à modifier) le type du corps (métallique).

7. Mais toi, ô bienheureuse, renonce à ces vains éléments dont on trouble tes oreilles ; car j'ai oui dire que tu converses avec Paphnutia la vierge et certains hommes sans instruction.\footnote{Cette discussion finale parait être adressée par Zosime à Théosébie ; (v. \emph{Olympiodore}, p. 90). Elle est caractéristique et met au jour la personnalité des alchimistes égyptiens et leurs controverses. --- Cp. \emph{Démocrite}, p. 50. --- Les noms de Paphnutia et de Nilus méritent d'être notés. Le premier vient s'ajouter à ceux des femmes alchimistes : Marie, Cléopâtre, Théosébie. --- Nilus était d'ailleurs un nom assez répandu en Égypte : plusieurs personnages historiques l'ont porté.} Les choses que tu leur entends dire sont vaines et tu entreprends de faire des raisonnements vides de sens. Renonce à la société des gens qui ont l'esprit aveuglé et l'imagination trop enflammée. Il faut plaindre ces gens-là, et écouter le langage de la vérité, de la bouche des hommes dignes de l'annoncer. Ces gens-là ne veulent pas de secours ; ils ne supportent pas d'être instruits par des maîtres, se flattant d'être des maîtres (eux-mêmes). Ils prétendent être honorés pour leurs raisonnements vains et vides (de sens). Lorsqu'on veut leur enseigner quels sont les degrés de la vérité, ils ne supportent pas la connaissance de l'art et ils ne (la) digèrent pas. Ils désirent l'or plutôt que la raison. Échauffés par une démence extrême, ils deviennent incapables de raisonnement et ne sauraient attendre la richesse. En effet s'ils étaient guidés par la raison, l'or les accompagnerait et serait en leur pouvoir : car la raison est maîtresse de l'or. Celui qui s'y attache, qui la désire et s'y unit, trouvera l'or placé devant nous, au milieu des détours qui le tiennent caché.

8. La raison est l'indicatrice de tous les biens, comme on l'a dit quelque part.\footnote{E Lb « Comme l'a dit le Philosophe. »} La philosophie est la connaissance de la vérité, et révèle les êtres qui existent. Celui qui accepte la raison, verra par elle l'or placé devant (ses) yeux. Mais ceux qui ne supportent pas la raison marchent constamment dans le vide, et entreprennent les actes les plus ridicules. C'est ainsi que le rire fut provoqué par Nilus, ce prêtre ton ami, qui faisait cuire le molybdochalque dans un four de campagne (comme s'il avait fait cuire des pains), opérant avec les cobathia pendant toute une journée. Aveuglé des yeux du corps, il ne pensait pas que son procédé était mauvais, mais il soufflait ; et sortant (le produit) après le refroidissement, il ne montrait que de la cendre. Quand on lui demandait où était le blanchiment, embarrassé, il disait qu'il avait pénétré dans la profondeur. Ensuite il mettait du cuivre, il teignait la scorie ; car le cuivre n'étant arrêté par aucun solide, passait outre et disparaissait lui-même dans la profondeur ; de même pour le blanchiment de la magnésie. Ayant entendu ces choses (de la bouche) de ses contradicteurs, Paphnutia fut tournée en grande dérision ; et vous le serez aussi, si vous tombez dans la même démence. Embrasse pour moi Nilus, celui qui cuit avec les cobathia, et sois pleinement édifiée sur l'économie du corps de la magnésie.

\bigskip
\centerline{\EightStarTaper}
\centerline{\EightStarTaper\EightStarTaper}
\bigskip

\subsubsection{3. --- 28. Sur le Corps de la Magnésie et sur son Traitement.}
\paragraph{}
1. Voici ce que Marie expose libéralement et clairement, au sujet de ce qu'elle nomme les pains de la magnésie. Le premier degré dans la vérité du mystère se trouve expliqué dans ces (passages). Ainsi donc Marie veut que ce soit là le corps de la magnésie ; elle le proclame non seulement dans ce passage, mais dans beaucoup d'autres. Dans un autre endroit, elle dit : « Sans le concours du plomb noir, on ne saurait produire ce corps de la magnésie,\footnote{« le molybdochalque par lequel » Lb.} dont nous avons précisé et accompli la préparation. Telles sont, dit-elle, les doctrines ; » et sans se lasser, (les) enseignant pour la 2\textsuperscript{e} et 3\textsuperscript{e} fois, elle nomme corps de la magnésie le plomb noir et le molybdochalque ; à ce sujet, elle parle du cinabre,\footnote{« du cuivre, » BAKELb.} ou du plomb, et de la pierre étésienne. C'est ce corps qui produit la fusion simultanée\footnote{V. p. 78, 101, 113, 128.} de toutes les matières cuites et dorées en puissance. Les matières crues, il les cuit ; et il en opère la diplosis. Il produit, dit-elle, en puissance toutes les matières dorées par cuisson ; car ce n'est pas encore en acte. Sur ce (point) j'écrirai un autre discours ; mais pour le moment occupons-nous de notre sujet.

2. Il a donc été exposé par Marie que le corps de la magnésie, c'est le molybdochalque noir ; car il n'a pas encore été teint. « C'est ce molybdochalque que tu dois teindre, en y projetant les \emph{motaria}\footnote{V. p. 108, 112, 157.} de la sandaraque jaune, afin que l'or cuit n'existe plus (seulement) en puissance, mais en acte. » Ainsi (s'exprime) Marie, après avoir nommé pains le corps de la magnésie.

Nous devons, avant tout, montrer que le Philosophe est du même sentiment, en ce qui (concerne) le corps de la magnésie qu'on appelait : \textbf{Le Tout}. Ce molybdochalque était le plomb noir. Lorsqu'ils disaient que le mercure est fixé avec le corps de la magnésie, ils voulaient dire par le corps complet, tel qu'il a été exposé dans mon premier mémoire, et que Marie le dit plus haut du corps de la magnésie. Elle dit (encore) : « Tu trouveras du plomb noir : emploie-le après y avoir mêlé du mercure. » Or c'est lui que dénomment les classes (du Philosophe), c'est lui dont parle le Philosophe dans ses préambules : « Mêle du mercure au corps de la magnésie. » Ainsi le Philosophe lui-même désigne le plomb noir et la pyrite. Il ne parle pas (du plomb) simplement, pour que tu ne t'égares pas, mais il dit « à notre (plomb) noir. » Pour que tu ne méconnaisses pas le molybdochalque, il dit que : « le mercure seul rend le cuivre sans ombre ; il ne fixera pas (seulement) le corps de la magnésie, mais encore le cuivre. » De cette façon aussi le Philosophe désigne sous le nom du Tout, le corps de la magnésie et le plomb noir.\footnote{Le molybdochalque, ABKELb.} Dans les livres des anciens, le molybdochalque a été rangé dans une seule et même classe (avec le plomb). Ce que l'on proclame du mercure, on le proclame de toute sorte de pierres, comme je l'ai déclaré dans les premiers (chapitres).

3. C'est donc là l'or cuit en puissance. Et s'il est blanchi ou jauni, alors aussi les matières crues réagissent sur les matières cuites : c'est-à-dire que si du cuivre blanc est jeté sur du (cuivre) brut de Chypre, il produit de l'argent. Mais s'il est jauni, en le projetant sur de l'argent ordinaire brut, on produit de l'or. Après avoir mouillé avec de la couperose, du vin Aminéen et du vinaigre ordinaire, laisse pendant 14 jours : c'est là le (temps) voulu pour la fabrication de l'argent.

4. Comme on échoue souvent dans le traitement, parce qu'on ne connaît pas la vérité sur le délaiement, rappelons ce qui a été dit touchant les vapeurs : c'est la couperose qui amène la vapeur à la coloration en or. Semblablement aussi, Agathodémon, dans son enseignement sur la teinture préalable, disait ceci : « Afin que tu puisses savoir l'effet que tu produis, en arrivant à cette couperose que tu connais, c'est sa propriété tinctoriale qui amène la vapeur à développer l'or. Cela a été montré dans l'écrit sur l'affinage, et rappelé au sujet des deux (teintures). Dans le discours sur les mesures, il est dit que les pierres les plus belles et aimées de Dieu sont les pierres blanches es les pierres couleur de sang ; c'est là ce qu'on a appelé pyrite. Elles sont multicolores et de noms multiples ; les uns parlent de l'alabastron,\footnote{\emph{Lexique}, p. 4. \emph{Introd.}, p. 238.} d'autres appliquent aux deux le nom de pyrite, ainsi que je l'ai montré. En effet, nulle autre pierre que la pyrite n'est plus belle et aimée de Dieu.

5. Maintenant le discours a pour sujet le corps de la magnésie. Ce nom unique signifie toutes les choses fabriquées avec la vraie mesure de la macération nécessaire. Le cinabre\footnote{S'agit-il ici de l'hématite ? v. p. 39.} produit le véritable corps de la magnésie. Ne m'écartant pas de cette vérité, je voulais, moi aussi, égaler la capacité de celui qui a dit\footnote{Sans doute Zosime s'adressant à Théosébie.} : « O femme, je ne parlais pas (du plomb) ordinaire, afin que tu ne t'égarasses pas. » Mais comme je ne suis pas Démocrite, je te jure par son mérite que je ne m'égare pas ; et (tu ne tomberas pas dans l'erreur) sans retour de ceux qui prétendent que la cendre sans corps (métallique) a été appelée le corps de la magnésie.\footnote{Voir plus haut ce qui est dit de Nilus, p. 187.}

On a dit que le mercure est incorporel. Je dis, moi aussi, que ceux-là ont compris quelque chose. En montrant le résultat à obtenir, ils donnent la mesure de leur intelligence. Mais ils ne tiennent pas en réalité le résultat, car la cendre n'a pas été appelée le corps de la magnésie, mais l'incorporel. Or le mercure est aussi un corps (métallique). Ne va pas m'opposer cette subtilité, que ceci comprend tous les corps métalliques et que la cendre des incorporels a été appelée le corps de la magnésie, il n'en est rien. Mais que veut-il dire, si ce n'est que (les incorporels), étant de nature sulfureuse, se volatilisent ? Ce sont donc les choses fixes et non fugaces qui sont appelées des corps. C'est pourquoi Marie dit : « le corps de la magnésie est la chose secrète qui provient du plomb, de la pierre étésienne et du cuivre. »

6. Toutes les choses de cet ordre, mélangées aux matières volatiles, sont appelées corps. C'est ainsi qu'il parle du mercure, dans son traité des liquides blancs : « mêles-y de l'alun lamelleux, ou du molybdochalque, ou de la chaux, afin que le (mercure) incorporel devienne un corps. » De même, au sujet de la chrysocolle, il dit : « celle-ci aussi est fugace. » Sur le même sujet Agathodémon : « Veille, dit-il, à ce que son esprit tinctorial ne s'en aille pas. » Bien qu'elle soit volatile, on l'appelle un corps ; le Philosophe parle de ses mélanges dans la classe de la chrysocolle. « Teins toute sorte de corps avec le cuivre, l'argent, l'or. » Marie, au sujet de la chrysocolle : « ... après avoir pesé, (opère) avec du molybdochalque, pendant un jour ... » Ou bien : « prenant de la chrysocolle et du cinabre, délaie avec de la litharge blanche et fais disparaître (la nature du métal). Si le cuivre est modifié et amené à l'état de corps (métallique), projettes-y de la couleur d'or et tu auras de l'or. » Ainsi la chrysocolle reçoit cette qualification de corps, lorsqu'elle a été bien mélangée, et quoiqu'elle soit fugace par elle-même, parce que tu en fais un corps par transmutation.

7. Ainsi, convertir et transmuter,\footnote{Dans le texte grec l'auteur oppose les mots στροφή et ἐκστροφή, et les verbes correspondants. Ces mots paraissent encore signifier : convertir la nature intérieure d'un métal en or ou en argent, en en transmutant ou extrayant la nature antérieure, qui était celle du cuivre, du plomb, de l'étain ou du fer. Une semblable extraction s'exprime par le mot κατασπάω.} dans ces auteurs, signifie donner un corps aux incorporels, c'est-à-dire aux matières fugaces. Par leur transformation on obtient le molybdochalque, le plomb noir, celui qui doit être traité avec le mercure, et devenir le corps de la magnésie. Ils ne veulent pas dire, comme certains, que la mutation s'applique au fait de convertir et de transmuter le mercure. Mais lorsque les matières fugaces ont pris un corps, la conversion a lieu pour tous les corps, par leur teinture en blanc ou en jaune. En effet cette conversion est appelée transmutation, après que les incorporels ont pris un corps, par l'effet de l'art. Dans la conversion rétrograde accomplie par le feu, c'est-à-dire dans le blanchiment ou le jaunissement, les matières délayées fortement et associées par le feu, sont de nouveau rendues fugaces et redeviennent incorporelles.\footnote{Cp. p. 21.} A ce moment elles sont réduites au dernier degré de la division. La vapeur sublimée, la première des matières incorporelles, conduit ainsi à l'art suprême.

8. Ainsi donc, les matières incorporelles sont de nouveau rendues corporelles au moyen du mercure, dans l'iosis, afin que les corps soient formés ; mais après que (les matières corporelles) ont été décomposées, elles sont rendues incorporelles et l'effet se produit par une action indépendante du concours du feu.

Ailleurs on a parlé (pour cet effet) des biles\footnote{Il semble que ce soit là une expression symbolique pour désigner les matières colorantes jaunes, et surtout celles qui produisent à froid des sulfures colorés en jaune.} et autres matières semblables qui, elles aussi, sont congénères du soufre et de l'eau de soufre. Or quelle autre substance agit bien sans le secours du feu, si ce n'est l'eau divine ? C'est d'elle que Pébichius (dit) qu'elle est plus puissante que n'importe quel feu. Dans le Chapitre des Sulfureux, il est dit qu'elle agit sans le secours du feu. Marie (l'appelle) la préparation ignée.\footnote{C'est-à-dire la préparation produisant à froid les mêmes effets que le feu.} Elle dit encore que si les corps ne sont pas rendus incorporels et les incorporels corporels,\footnote{Voir p. 101.} rien de ce que l'on attend n'aura lieu : c'est-à-dire que si les matières résistant au feu ne sont pas mélangées avec celles qui s'évaporent au feu, on n'obtiendra rien de ce que l'on attend.

9. Quels sont donc les corps et les incorporels dans notre art\footnote{Voir p. 21, 101 et 191.} ?

Les incorporels sont la pyrite et ses similaires, la magnésie et ses similaires, le mercure et ses similaires, la chrysocolle et ses similaires, toutes (matières) incorporelles. Les corps sont le cuivre, le fer, l'étain et le plomb : ces (matières) ne s'évaporent pas au feu ; ce sont là les corps. Lorsque les unes (de ces matières) sont mêlées aux autres, les corps deviennent incorporels et les incorporels deviennent corps. Mélange de cette manière le mercure, celui qui est désigné dans les classes, et tu produiras ce qui est attendu, ce dont Marie a dit : « Si deux ne deviennent un ; » c'est-à-dire si les (matières) volatiles ne se combinent pas avec les matières fixes, rien n'aura lieu de ce qui est attendu. Si l'on ne blanchit et si deux ne deviennent pas trois,\footnote{V. p. 21.} avec le soufre blanc qui blanchit (rien n'aura lieu de ce qui est attendu). Mais lorsqu'on jaunit, crois deviennent quatre ; car on jaunit avec le soufre jaune. Enfin lorsqu'on teint en violet,\footnote{Ou bien lorsqu'on opère l'iosis, le mot grec ayant ce double sens.} toutes les (matière ensemble) parviennent à l'unité.

10. Que veut dire Ostanès, lorsqu'il parle de la combinaison des matières volatiles avec celles qui ne le sont pas ? « La pierre pyrite a de l'affinité pour le cuivre. » Ostanès ne parlait pas du mercure, mais du délaiement extrême, c'est-à-dire de la condition où la pyrite ne donne lieu à aucun dépôt, se trouvant entièrement liquéfiée. Il faut dès lors que tu comprennes, au sujet de l'eau et de la liquéfaction, ce que le Philosophe a développé en parlant des lavages et des délaiements. Au sujet du délaiement, il a dit : « afin que le produit devienne comme de l'eau. » Le Philosophe a dit encore : « La magnésie et l'aimant ont de l'affinité pour le fer. » Et le Maître dit encore : « le mercure a de l'affinité pour l'étain. » Le disciple dit : « le mercure s'amalgame à l'étain. » Il dit aussi : « Ceci blanchit toute sorte de corps. Le plomb aussi a de l'affinité pour la pyrite ; la pierre étésienne, pour le plomb. » Le Philosophe, en faisant ces raisonnements, disait, au sujet de notre art, que la nature charme la nature.

11. Article sur la magnésie : Après avoir tout extrait, tu trouveras un corps noir, ou du plomb noir ; souvent aussi une grande quantité de scories, à la partie supérieure. Si on les goûte, on verra qu'elles ressemblent à la lie de vin. Après les avoir rejetées, on trouve, à l'intérieur du plomb noir, le cuivre que celui-ci renferme, la magnésie qui y est contenue. On appelle celle-ci : molybdochalque ou corps de la magnésie. C'est sur celle-ci que j'ai écrit ; c'est elle que tous les écrits proclament ; c'est elle qui égare les chercheurs ; c'est ce molybdochalque que préconisent les écrits des ancêtres. D'après l'explication d'Apollon, c'est le corps de la magnésie ; c'est le cuivre, c'est le corps dont Théophile disait qu'il reçoit une couronne de cuivre ; Hermès disait de son côté : « Le corps de la magnésie dont tu désires apprendre le traitement et la mesure ... » A son sujet nous avons dit que le cinabre, c'est le blanchiment ; ou bien encore le jaunissement, lequel exige que les (matières) soient blanchies préalablement. Voilà le traitement, tel qu'il a été décrit par nous.

\bigskip
\centerline{\EightStarTaper}
\centerline{\EightStarTaper\EightStarTaper}
\bigskip

\subsubsection[3. --- 29. Sur la Pierre Philosophale.]{3. --- 29. Sur la Pierre Philosophale.\footnote{Suite de fragments, réunis à une époque relativement récente, comme le montre d'ailleurs le titre lui-même ; la dénomination expresse de \emph{pierre philosophale} n'existant pas dans auteurs antérieurs au 7\textsuperscript{e} siècle, bien que la notion même soit plus ancienne. La plupart de ces fragments reproduisent des textes déjà donnés sous forme plus développée.}}
\paragraph{}
1. Marie dit : « Si notre plomb est noir, c'est qu'il l'est devenu ; car le plomb commun est noir dès le principe. Or comment est-il formé ? Si tu ne prives pas les corps métalliques de leur état et si tu ne ramènes pas les corps privés de leur état à l'état de corps (métalliques) ; si tu ne fais pas de deux choses une seule, rien de ce que l'on attend n'a lieu.\footnote{Voir la page précédente, la page 101, etc.} Si le Tout n'est pas atténué dans le feu, si la vapeur sublimée réduite en esprit ne monte pas, rien ne sera mené à terme. » Et encore : « Je ne dis pas avec du plomb simplement, mais avec notre plomb noir. Voici comment l'on prépare le plomb noir ; c'est par la cuisson que l'on arrive (à reproduire le) plomb commun. Car le plomb commun est noir dès le principe, tandis que notre plomb devient noir, ne l'étant pas d'abord. »

2. Les philosophes ont partagé toutes les opérations de la pierre en quatre phases : 1° noircissement ; 2° blanchiment ; 3° jaunissement, et 4° teinture en violet. Entre le noircissement, le blanchiment et le jaunissement se place la lévigation ou macération et le lavage des espèces. Or il est impossible que ces choses se fassent autrement que par le traitement opéré au moyen de l'appareil à gorge\footnote{Voir \emph{Introd.}, p. 164 ; Synésius, p. 65, et p. 144.} et de l'union des parties.

3. Pélage le Philosophe dit : « Voici à quel signe on reconnaît que le commencement de la teinture en violet a lieu. C'est la teinture se produisant à l'intérieur qui est la véritable teinture en violet, laquelle a été aussi appelée ios de l'or. Si on l'accomplit, la teinture a lieu ; sinon, elle n'a pas lieu. Veille donc à ce que la teinture pénètre dans la profondeur ; sinon la teinture n'a pas lieu. »

4. L'alabastron est la pierre la plus blanche, la pierre encéphale,\footnote{\emph{Lexique}, p. 4 et 6.} celle qui est comme une paillette brûlante. Prends-la, pulvérise et fais macérer dans du vinaigre ; mets dans un linge, et enfouis le tout dans le crottin de cheval, ou dans la fiente d'oiseau, pendant 20 jours, comme dit le divin Zosime.

5. Les soufres sont au nombre de deux, la composition est une. Donc, il y a deux mercures, savoir la composition blanche et l'eau divine, selon Démocrite. L'eau divine mêlée au soufre rend les substances sulfureuses,\footnote{L'auteur joue sur le double sens de θεῖον.} parce que ces matières ont une grande affinité entre elles.

6. Synésius expose ceci dans le traité de la Chrysopée : « Démocrite a dit : « Le mercure qui (provient) du cinabre. Et dans le Traité du blanc (Argyropée) il a dit : Le mercure tiré de la sandaraque, etc.\footnote{Synésius, p. 66.} »

7. Dioscorus a dit : « De même que la cire se transforme en assimilant la couleur surajoutée, de même aussi le mercure se transforme.\footnote{Synésius, p. 66.} »

8. Il y a deux jaunissements, deux blanchiments,\footnote{Olympiodore, p. 109 ; et \emph{passim}.} deux compositions, la sèche et la liquide : la composition sèche, dans le catalogue du jaune, ce sont les plantes et les minéraux. Il y a deux compositions liquides : une dans le jaune, et une dans le blanc. Les liquides jaunes dérivent des plantes jaunes,\footnote{Cp. p. 71, 123, 153, note 2 ; p. 159, note 2, etc.} telles que le safran, la chélidoine et les similaires. Dans la composition blanche on comprend : parmi les matières sèches, toutes les matières blanches, telles que la terre de Crète, la terre de Cimole et les analogues ; parmi les liquides blancs, toutes les eaux blanches, telles que la décoction d'orge (bière ? ) et les similaires.

9. Olympiodore dit : « La macération a lieu depuis le 25 du mois de méchir jusqu'au 25 du dernier mois de l'automne\footnote{Ou du mois Mésori (voir p. 75).} ... Toutes les choses que tu peux faire macérer et lessiver, laisse-les déposer dans des vases (convenables). La macération s'exécute sur la terre limoneuse, jusqu'à ce que la partie limoneuse s'en aille et que le minerai soit isolé. Cet art ne se pratique pas au moyen du feu. »

10. Le feu est de 40 jours pour l'opération entière.

Les anciens ont caché l'art sous la multiplicité des discours\footnote{Olympiodore, p. 75 et 76.} et ils ont donné un grand nombre de dénominations à l'eau divine.\footnote{Cp. p. 101 et 182.}

11. Marie dit\footnote{Tout ce paragraphe semble formé avec des phrases disjointes, tirées des écrits de Marie ; elles sont en partie extraites d'Olympiodore, qui les avait prises directement de ces écrits (v. p. 101).} : « Si tous les corps métalliques ne sont pas atténués par l'action du feu, et si la vapeur sublimée réduite en esprit ne monte pas, rien ne sera mené à terme. »

Le molybdochalque c'est la pierre étésienne.

Dans toute l'opération la préparation est noire dès le commencement.

Lorsque tu vois tout devenir cendre, comprends alors que tu as bien opéré.\footnote{Cp. Olympiodore, p. 107.} Pulvérise cette scorie, épuise-la de sa partie soluble et lave-la six ou sept fois, dans des eaux édulcorées, après chaque fonte. On opère par fusions et selon la richesse du minerai. En effet, en suivant cette marche et le lavage, dit Marie, « la composition est adoucie et pourvue de ses éléments. »

Après la fin de l'iosis, une projection ayant eu lieu, le jaunissement stable des liquides se produit.

En faisant cela tu fais sortir au dehors la nature cachée à l'intérieur. En effet, « transforme, dit-elle, leur nature même, et tu trouveras ce que tu cherches. »

12. Les compositions sont au nombre de deux : le blanchiment et le jaunissement ; et il y a deux blanchiments et deux jaunissements,\footnote{Cp. p. 108.} l'un par délaiement et l'autre par cuisson. Le délaiement ne se fait pas d'une manière quelconque, mais seulement dans une demeure consacrée ; là existent un lac et de gros poissons.\footnote{Cp. Olympiodore, p. 109. Dans E Lb « un lieu de repos, » au lieu de « gros poissons. »}

13. Marie dit : « Joignez le mâle et la femelle et vous trouverez ce qui est cherché.\footnote{Cp. p. 147.} » Et Marie dit ailleurs : « N'allez pas toucher avec vos mains, car c'est une préparation ignée.\footnote{Cp. p. 112.} »

14. On donne plusieurs dénominations aux deux compositions, telles que, etc. (Reproduction du texte traduit en tête de la page 182.)

15. Les appareils des compositions doivent être en verre, parce que (alors) ils permettent l'iosis, sans que (les opérateurs) aient besoin de toucher avec leurs mains ; car le mercure est mortel, lorsqu'il a dissous l'or : c'est le plus délétère de tous les métaux.

16. Ce que l'on se propose dans la calcination, c'est d'abord le blanchiment, puis le jaunissement. Projette, dit-il, la moitié de la préparation blanche, pour la première opération, et fais-en une décoction de cette manière ; l'autre moitié est conservée pour l'iosis. C'est aussi pour cette raison que Pébichius dit, \emph{passim} : « Partagez en deux portions la préparation.\footnote{P. 165 et 178.} » Il disait aussi : « Renferme l'une dans un vase de terre cuite et mets l'autre avec le cuivre.\footnote{P. 158.} » Il indique, par le vase de terre cuite, la cuisson, et par le cuivre l'iosis. Il voulait parler du blanchiment, en disant : « Brûlez le cuivre sur un feu de bois de laurier, c'est-à-dire dans la composition blanche. »

17. Agathodémon dit : « Fais une décoction de l'eau divine avec la vapeur sublimée ; de cette façon, on brûle et on opère le blanchiment. » Et encore : « Faire cuire la vapeur décrite précédemment avec l'huile de ricin ou de raifort, après y avoir mêlé un peu d'alun.\footnote{P. 178.} »

18. Zosime dit : « Pour accomplir exactement la présente opération, il faut laver l'aigle d'airain, pendant les 365 jours (de l'année) entiers, » et ainsi de suite, dans tout le cours du traité.\footnote{P. 129 et 135.}

19. Le divin Sophar dit : « Je vis un aigle d'airain descendre dans la source pure, etc. » (Reproduction de cinq lignes déjà données à la page 125.)

20. La magnésie tire son étymologie du fait de mélanger (μιγνύειν) les matières unies par la combinaison.

21. Le divin Zosime dit : Démocrite, mon excellent maître, dit avec raison : « Reçois la pierre qui n'est pas une pierre. » (Reproduction d'un passage déjà donné, p. 130, jusqu'à ces mots : « lait d'ânesse ou de chèvre. » )

22. Zosime disait : « Ne redoute point de chauffer fortement ; épuise l'élément liquide des corps. Il y a mille (modes de) chauffer le cuivre\footnote{Voir p. 154 et 177.} ; ils rendent le cuivre plus apte à la teinture. Fais sortir la nature au dehors et tu trouveras ce qui est cherché ; car la nature est cachée à l'intérieur. Or, la nature étant extraite, le blanc ne se voit plus ; mais après l'expulsion du mercure indiquée précédemment, le jaune apparaît, par le jaunissement annoncé de l'ios. Où sont donc ceux qui déclarent impossible de changer la nature ? Voici que la nature est changée ; elle devient fixe et prend la qualité de l'or, en retournant vers le noir. En effet, si l'humidité provenant de l'expulsion du mercure, circulant dans la (nature) terrestre du corps solide de la poudre sèche, ne va pas dissoudre et expulser la liquidité, conformément à la propriété essentielle de cette expulsion du mercure, alors rien n'aura lieu de ce qui est attendu. Si l'on n'opère pas la dissolution et l'épuisement de l'élément liquide par l'échauffement, rien n'aura lieu de ce qui est attendu. Si le produit n'est pas dissous et échauffé, puis refroidi, rien n'aura lieu de ce qui est attendu. Mais si toutes choses sont faites à leur rang et par ordre, tu pourras espérer arriver au résultat, avec l'aide de la divine Providence. »

23. Le temps de la gestation n'est pas moindre de neuf mois, quand il n'y a pas avortement. Le temps de la cuisson pour tous les produits, (notamment) lorsqu'on opère sur des lames, n'est pas moindre de neuf heures. Tel est le mode de gestation. Quant au temps de l'opération faite sur l'autel en forme de coupe, il faut tenir compte de la macération. En effet, considère que les modes d'opérer sont au nombre de trois. Le premier mode se rapporte au mélange. Si tu m'as bien compris, il embrasse les substances pétries et fermentées, à la façon de la farine tirée du grain. De même le liquide ne sera pas vaporisé outre mesure, mais seulement selon que le besoin s'en fera sentir ; de même aussi, pour la composition. (Reproduction du § 5, p. 142, jusqu'à la fin.)

24. C'est là la pierre étésienne. Edulcore la poudre sèche (de projection) et dessèche. Fixe et affine la poudre sèche, en prenant : couperose, trois parties ; magnésie, une partie ; cuivre affiné, une partie ; poudre sèche, une partie. Délaie ensemble, en arrosant au soleil avec du vinaigre blanc, pendant sept jours ; puis fais cuire pendant deux ou trois jours. En enlevant (le produit), tu trouveras l'or teint en rouge couleur de sang. C'est là le cinabre des philosophes et l'homme d'or. La poudre de projection s'est condensée (aux dépens) des liqueurs. Si le feu est excessif, elle devient jaune ; mais (alors) elle n'est pas utile.

\bigskip
\centerline{\EightStarTaper}
\centerline{\EightStarTaper\EightStarTaper}
\bigskip

\subsubsection[3. --- 30. Sur la Composition des Matières Premières.]{3. --- 30. Sur la Composition des Matières Premières.\footnote{Ceci paraît être une lettre-dédicace, ou un épilogue de Zosime, transformé par quelque copiste en fragment « περὶ ἀφορμῶν συνθέσεως. »}}
\paragraph{}
La composition relative aux matières premières a réuni dans un seul esprit, ô Théosébie, les compositions partielles des anciens. En outre elle montre, au moyen du fait, les noms des composés (restés) ignorés dans leurs écrits, comme (par exemple) la cendre et les (matières semblables. Or, il faut savoir quelles substances, d'après le Philosophe, produisent la résistance au feu\footnote{En marge : signe du mercure.} ; que le corps allié (au mercure) le rend capable de résister au feu, et ainsi de suite. Car le sage, prenant les matières premières, poursuivra du commencement à la fin. Mais je ne pouvais placer là les produits complets, attendu que je ne les trouvais pas chez ces (auteurs) ; je ne pouvais exposer ce que (Démocrite) n'avait pas dit ; je ne pouvais faire autre chose que réunir avec vraisemblance les choses dispersées, interpréter les choses allégoriques ; tout ce qu'il est permis de faire dans des commentaires, je l'ai fait. Bonne santé.

\bigskip
\centerline{\EightStarTaper}
\centerline{\EightStarTaper\EightStarTaper}
\bigskip

\subsubsection[3. --- 31. Sur la Poudre Sèche (de Projection).]{3. --- 31. Sur la Poudre Sèche\footnote{Ceci paraît être une lettre-dédicace, ou un épilogue de Zosime, transformé par quelque copiste en fragment « περὶ ἀφορμῶν συνθέσεως. »} (de Projection).}
\paragraph{}
1. La poudre de projection véritable a trois puissances et trois actions procédant de ces puissances. (Ce sont) la teinture, la pénétration, la fixation. Le (corps) mathématique a trois dimensions, la longueur, la largeur et la profondeur. Le corps naturel est triplement étendu et (en outre) susceptible de figure ; il a la longueur, la largeur, la profondeur et la capacité de figure. De même aussi, au sujet de (notre) espèce, nous parlerons de la teinture, de la pénétration, de la fixation, et de l'éclat (durable). Or le corps a trois dimensions, nous le désignerons comme figuré, non figuré, et susceptible de prendre toutes les figures ; sa matière subissant les puissances et les actions (de la poudre de projection).\footnote{Cp. Synésius, p. 66 et 67 ; \emph{Origines de l'Alchimie}, p. 75, 265, 267.}

\bigskip
\centerline{\EightStarTaper}
\centerline{\EightStarTaper\EightStarTaper}
\bigskip

\subsubsection{3. --- 32. Sur l'Ios.}
\paragraph{}
1. La puissance propre à l'ios est complémentaire de la substance qui en est le support ; regardée comme indivisible, elle en fait partie. Sans elle, la substance demeure incomplète. En effet, les parties de substances sont elles-mêmes des substances, comme (le) dit Porphyre ; car la substance produit la puissance ; et la puissance, l'action ; et l'action, les choses en acte. Donc les puissances substantielles proviennent des substances et sont inséparables des substances.

\bigskip
\centerline{\EightStarTaper}
\centerline{\EightStarTaper\EightStarTaper}
\bigskip

\subsubsection{3. --- 33. Sur les Causes.}
\paragraph{}
1. Il y a, selon le naturaliste Aristote,\footnote{Cp. Aristote, \emph{Gener.}, 1, 7 ;--- \emph{Métaph.}, 1, 3 ;--- \emph{Morale à Eudème}, 7, 10 ;--- \emph{Physique}, 2, 3. --- Platon, \emph{Timée}, p. 37, D.} quatre causes de tout (être) engendré, savoir : les causes efficiente, matérielle, organique et spécifique. Par exemple, la porte a pour cause efficiente, le constructeur qui l'a faite ; pour cause matérielle, le bois, le fer, la colle forte ; pour cause organique, la hache, la tarière, etc. ; pour cause spécifique, l'espèce même de la matière de la porte, ou quelque autre. Selon Platon, il y a encore deux autres (causes) : la cause exemplaire et la cause finale.

\bigskip
\centerline{\EightStarTaper}
\centerline{\EightStarTaper\EightStarTaper}
\bigskip

\subsubsection{3. --- 34. Enchainement de la Vierge.}
\paragraph{}
1. Traitant le feu du mercure par le feu et alliant l'esprit à l'esprit, afin d'enchaîner par les mains la vierge, ce démon fugace.\footnote{Le mercure ? Voir page 146, note 3.  }

Dans E on lit : Au moyen de l'ios du mercure, nous triomphons du feu par le feu, et nous allions, etc.

Divers ossements des Perses ayant été calcinés par la violence du feu,\footnote{Var. M : Dispersant les ossements des Perses calcinés, etc.} ils ont perdu leur propre volatilité.

2. Ramenons les deux corps : après les avoir réunis dans le mélange et transformés, ils sont régénérés. L'être sans me devient animé ; l'être sans corps est rendu corporel, et ils n'admettent pas d'autre changement.

\bigskip
\centerline{\EightStarTaper}
\centerline{\EightStarTaper\EightStarTaper}
\bigskip

\subsubsection{3. --- 35. Les Hommes Métalliques.}
\paragraph{}
Cet homme d'airain que tu vois dans la fontaine a changé de corps et il est devenu l'homme d'asèm ; quelques jours après, tu le vois (transformé en) homme d'or.\footnote{Cp. le \emph{Serpent}, p. 23 ; Zosime, p. 120.} Arrose-le avec de la saumure acide ; de cette façon il devient blanc et convenable.

\bigskip
\centerline{\EightStarTaper}
\centerline{\EightStarTaper\EightStarTaper}
\bigskip

\subsubsection[3. --- 36. Lavage de la Cadmie.]{3. --- 36. Lavage de la Cadmie.\footnote{Ce morceau, ainsi que celui sur l'ocre, représente un extrait de quelque auteur perdu, congénère de Dioscoride, \emph{Mat. méd.} 5, 84, vers la fin.}}
\paragraph{}
1. Après avoir pris la cadmie \emph{botruitis},\footnote{\emph{Introd.}, p. 239.} qui reste dans la préparation du cuivre, divise-la en agitant. Pulvérise avec soin : ensuite broie et projette dans l'eau. Broie de nouveau dans l'eau avec le pilon, puis délaie avec la main ; lorsque le produit est à point, laisse déposer. Après avoir bien égoutté, verse de nouveau de l'eau et répète la même chose plusieurs fois, jusqu'à ce que l'eau reste sans former de mousse. Après avoir bien égoutté fais sécher au soleil.

\bigskip
\centerline{\EightStarTaper}
\centerline{\EightStarTaper\EightStarTaper}
\bigskip

\subsubsection{3. --- 37. Sur la Teinture.}
\paragraph{}
1. Si (l'on) n'a pas pratiqué convenablement la teinture noire, le travail de l'argent ne pourra plus être tempéré. Les adeptes d'Agathodémon appellent : teinture supérieure (καταβαφή), celle que l'on exécute en délayant ainsi ; quant à la décoction, ils l'appellent teinture simple (βαφή) ; car ils distinguent la teinture simple et la teinture supérieure. Ils veulent donc que la teinture simple (βαφή) soit (la teinture en) argent et la teinture supérieure (καταβαφή), (la teinture en) or. A propos de l'acte de brûler, tu trouveras ceci : « Autre chose est de brûler en vue de la teinture simple, et autre chose de brûler en vue de la teinture supérieure. Tout le reste, jusqu'à la raréfaction, l'altération (de nature), (bref) toutes les autres (opérations), ils les dissimulent dans leurs discours. »

\bigskip
\centerline{\EightStarTaper}
\centerline{\EightStarTaper\EightStarTaper}
\bigskip

\subsubsection{3. --- 38. Sur le Jaunissement.}
\paragraph{}
1. « Tous ne pensaient pas, ô femme,\footnote{Théosébie.} que le jaunissement suivît immédiatement le blanchiment ; or le plus souvent la composition blanche, quand elle est cuite, tourne au jaune. » Et un peu plus loin : « quelques-uns ont fait une chose préférable à celles-ci. En effet, laissant refroidir, ils distillaient et rectifiaient au soleil l'eau divine jaune, pendant le nombre de jours prescrit. Puis ils opéraient la décoction et la cuisson. » Et un peu plus loin : « Eau divine rectifiée, préparée avec de la chaux, deux parties, et du soufre, une partie\footnote{C'est à peu près la même formule (celle d'un polysulfure de calcium) que la recette 89 du Papyrus X de Leide ; \emph{Introduction}, pages 46 et 68.} ; on met en décoction dans un pot et on décante ; puis on met en décoction de nouveau. C'est là l'eau de soufre, que l'on projette pour obtenir les deux couleurs.\footnote{Lb ajoute : « je dis l'eau aérienne. » Le ms. M continue par l'article tiré d'Agatharchide (\emph{Introd.}, p. 185).} »

\bigskip
\centerline{\EightStarTaper}
\centerline{\EightStarTaper\EightStarTaper}
\bigskip

\subsubsection[3. --- 39. L'Eau Aérienne.]{3. --- 39. L'Eau Aérienne.\footnote{Suite de fragments indépendants les uns des autres, et reproduisant parfois des morceaux déjà imprimés, avec certaines variantes.}}
\paragraph{}
1. « Cette composition a besoin d'abord de quelques liquides, etc. (morceau tiré d'Olympiodore, p. 97, premier alinéa tout entier). »

2. Au sujet des minerais, tout le monde s'explique sur ce point. Je commencerai par reproduire le témoignage qui le concerne, à cause de ton incrédulité. Zosime, dans son livre du \emph{Compte final}, adressé à Théosébie, s'explique en disant\footnote{Cp. Olympiodore, p. 97 et Zosime, 3, 51, 1-3.} : « Pour le roi d'Égypte, ô femme, tout consistait en ces deux arts, l'art de l'analyse,\footnote{Var. : « L'art des produits royaux ; » ou bien : « L'art des matières opportunes » (astrologie ? ) ; ou bien encore : « L'art des teintures convenables. »} et l'art des produits naturels et minerais. C'est l'art divin des transformations, c'est-à-dire l'art dogmatique pour tous ceux qui s'occupent de manipulations, j'entends les quatre arts relatifs à la fabrication (des métaux). Cet art divin a été révélé aux prêtres seuls, etc. » (La suite, p. 97 jusqu'au bas de la page, et jusqu'aux mots « ils seraient châtiés, » qui commencent la page 98.)

3. C'est là l'image du monde, célèbre dans les anciens écrits, le mortier mystique des Égyptiens et des hiérogrammates d'Egypte, par lequel l'affinité des natures charme les natures consubstantielles.\footnote{Ce mot semble répondre aux discussions sur la nature du Père et du Fils dans la Trinité, au temps du Concile de Nicée ; Cp. p. 136.} Voici le consubstantiel Orphique et la lyre Hermaïque, dans laquelle s'accomplit l'agréable et harmonieuse combinaison des substances. Mélangées suivant les rites, elles s'élancent de la (terre ? ) vers le chœur céleste ; le feu opérant leur transmutation.

4. A la suite, entre le noircissement et le blanchiment, a lieu la macération et le lavage des produits ; entre le blanchiment et le jaunissement, le traitement par fusion. De la même façon, comme intermédiaire entre le jaunissement et la teinture en violet, se place la division en deux de la composition. Le terme du blanchiment, c'est le traitement par l'appareil en forme de mamelle.\footnote{Cp. Synésius, p. 65.}

5. 1° Dans le noircissement, on sépare le produit fondu de la cendre ;

2° Dans la macération, on sépare la cendre de la liqueur ;

3° Puis vient le lavage des espèces brûlées, sept fois répété dans un vase d'Ascalon ; ce lavage est le 1\textsuperscript{er} blanchiment et la disparition de la coloration en noir des espèces ;

4° Le blanchiment, par le mélange avec une petite quantité d'eau blanche ou jaune, produit ce rayon de miel,\footnote{Synésius, p. 66. --- Lb ajoute : « Et fabrique la pierre sèche, recherchée, etc. »} recherché par les manipulateurs ;

5° Le jaunissement suit ; (car) le blanchiment mène au jaunissement ;

6° Alors s'accomplit la division en deux de la composition ;

7° Celle-ci étant partagée en deux, on prend l'une des parties, laquelle transformée en ios, amollit, délaie et\footnote{Lb intercale : « Et l'autre partie. »} accomplit la fixation.

6. D'autres, dit-il,\footnote{Addition de A seul.} (se sont expliqués) sur la couleur, sur la décoction et sur l'œuvre de la théorie secrète. On commence par projeter le cuivre. Après le traitement dans le laboratoire, il réjouit les yeux ; puis, avec le temps, la teinte devient plus claire,\footnote{Il semble qu'il s'agisse ici d'une coloration superficielle, obtenue par un procédé d'orfèvre. --- \emph{Introd.}, p. 56, 58.} lorsqu'on opère avec de l'or préparé au moyen de la gomme, de la liqueur d'or, etc.

\bigskip
\centerline{\EightStarTaper}
\centerline{\EightStarTaper\EightStarTaper}
\bigskip

\subsubsection{3. --- 40. Sur le Blanchiment.}
\paragraph{}
1. Il faut que vous sachiez que la chose capitale c'est le blanchiment ; après le blanchiment, on jaunit aussitôt le mystère accompli.

2. Le blanchiment réside dans l'acte de brûler ; or brûler c'est revivifier par le feu ; car de telles (matières) se brûlent et se revivifient d'elles-mêmes\footnote{Ce texte se trouve avec des variantes importantes dans Synésius, p. 63.} ; elles se fécondent elles-mêmes et engendrent ainsi l'animal cherché par les philosophes.

3. Si tu blanchis, tu teindras facilement, et si tu teins en violet ou en cinabre, tu seras bienheureux, ô Dioscorus ; car c'est là ce qui affranchit de la pauvreté, cette maladie incurable.\footnote{Cp. Synésius, p. 63.}

\bigskip
\centerline{\EightStarTaper}
\centerline{\EightStarTaper\EightStarTaper}
\bigskip

\subsubsection[3. --- 41. Livre Véritable de Sophé l'Égyptien et du Divin Seigneur des Hébreux (et) des Puissances Sabaoth Livre Mystique de Zosime le Thébain.]{3. --- 41. Livre Véritable de Sophé l'Égyptien et du Divin Seigneur des Hébreux (et) des Puissances Sabaoth Livre Mystique de Zosime le Thébain.\footnote{Cp. \emph{Origines de l'Alchimie}, p. 58. Sophé est une forme du nom de Chéops.}}
\paragraph{}
1. Voici la mesure du mercure.

Agathodémon dit : « Fais cuire, extrais l'or. » On projette le cuivre. On obtient la feuille de Marie, formée de deux métaux\footnote{Cp. p. 148, 151.} ; on la fait cuire au feu\footnote{Sur la kérotakis.} en vue de la teinture au moyen de l'huile et du miel et on reprend par le mercure : tel est le travail (régulier). Que le cuivre, amené de nouveau à l'état d'ios, soit fondu avec l'or, suivant la mesure du mercure.

Marie dit : « Lorsque la composition s'est formée d'elle-même, ou bien par le moyen de la saumure vinaigrée et qu'on a fait cuire, délaie avec le soufre, c'est-à-dire avec le soufre sublimé, soit dans un flacon, (soit) sur une kérotakis, puis verse, ou délaie, et regarde si lu as accompli l'œuvre. Si tu ne (l') as pas accompli avec un certain jaune, emploie notre ios avec la matière qui précède la teinture : c'est là ce qui est nécessaire pour rendre l'or parfait ; autrement l'or ne jaunit pas. Projette donc de nouveau avec la matière qui précède la teinture, ou bien délaie avec l'argent transformé : du noir scintillant, 1 partie d'ios, de misy brut, ainsi que de la matière qui précède la teinture, afin de dissoudre une portion du cuivre.

2. Il est cuit ; car même s'il ne contient pas de mercure, il faut (le) cuire, attendu qu'avant l'action du feu, il n'y a pas de teinture. Il faut lui faire subir l'action purificatrice par les matières (convenables), afin de constater qu'il est pur. Essaie, ou bien fais fondre. Si tu connais les deux marches, celles des Juifs et de ... ne crains pas d'essayer, (en exécutant) en détail toutes les choses que je t'ai exposées.

Cette exposition ne donne lieu à aucune équivoque ; mais elle a pour but de t'engager à essayer si la fortune t'est favorable et si tu as tout à fait réussi. En t'appuyant sur ces (connaissances), tu n'échoueras pas ; mais par cette méthode tu vaincras la pauvreté, surtout si tu as le talent et l'habileté de surmonter les obstacles. Dans des milliers d'ouvrages on enseigne comment le cuivre est blanchi et jauni convenablement. Il n'est propre à être allié par diplosis que s'il est changé en ios. Il peut être traité méthodiquement par mille (moyens) ; mais il n'est rendu propre à l'alliage que par une seule voie, en devenant notre vrai cuivre ; c'est là toute la formule. Telle est la teinture efficace, celle qu'ils leur ont enseignée, la teinture cherchée depuis des siècles et qui ne peut être découverte autrement que de cette façon. Quel est le principe convenable pour ces effets, je te l'ai montré dans l'écrit sur la couperose. On y dit comment le cuivre teint, et l'on y parle du plomb et de tout ce qui est susceptible de recevoir la teinture.

\bigskip
\centerline{\EightStarTaper}
\centerline{\EightStarTaper\EightStarTaper}
\bigskip

\subsubsection{3. --- 42. Livre Véritable de Sophé l'Égyptien et du Divin Maitre des Hébreux (et) des Puissances Sabaoth.}
\paragraph{}
1. Discours du livre véritable de Sophé l'Égyptien, du divin Seigneur des Hébreux (et) des puissances Sabaoth. Il y a deux sciences et deux sagesses : celle des Égyptiens et celle des Hébreux, laquelle est rendue plus solide par la justice divine. La science et la sagesse des meilleurs dominent les uns et les autres ; elles viennent des siècles anciens. Leur génération est dépourvue de roi, autonome, immatérielle ; elle ne recherche rien des corps matériels et corruptibles ; elle opère sans subir d'action (étrangère), soutenue maintenant par la prière et la grâce (divine). Le symbole de la chimie est tiré de la création, (aux yeux de ses adeptes) qui sauvent et purifient l'âme divine enchaînée dans les éléments, et surtout qui séparent l'esprit divin confondu avec la chair. De même qu'il existe un soleil, fleur du feu, un soleil céleste, œil droit du monde ; de même le cuivre, s'il devient fleur (c'est-à-dire s'il prend la couleur de l'or) par la purification, devient alors un soleil terrestre, qui est roi sur la terre, comme le soleil est roi dans le ciel.

2. Voici\footnote{Je regarde le mot οὐδαμοῦ comme ajouté ici par l'erreur d'un copiste ; à moins que ce ne soit le débris d'une phrase qui a disparu.} les teintures parfaites, communiquant la vraie couleur du soleil,\footnote{C'est-à-dire de l'or.} telles que celle de Démocrite, et, l'unité qui transmet la teinture, la comaris scythique, la (teinture) parfaite (de l'argent), celle d'Isis,\footnote{Cp. p. 31, note 2, et p. 36, note 3.} celle que proclame Héron (Horus ? ) ; voici l'affinage de l'or et la liqueur d'or.

La liqueur d'argent versée sur de l'argent produit de l'argent, lorsqu'elle est mise en réaction avec le sidérochalque. Ces (teintures) communiquent (la couleur de) l'argent dans leurs réactions. Elles produisent aussi les doublements et les triplements\footnote{\emph{Introd.}, Papyrus de Leide, p. 29.} et les alliages d'or et d'argent. Ainsi il convient de travailler par des moyens artificiels, sans or ni argent ; (il convient) d'accomplir des doublements tels, que l'on ne puisse plus séparer l'or et l'argent, comme on le ferait pour des matières adultérées et discordantes, qui n'ont pas produit de l'or véritable. Ainsi quand tu auras obtenu du cuivre sans ombre, tu (le) blanchiras avec des préparations blanchissantes et tu le jauniras avec des préparations jaunissantes ; tu le teindras (avec) la cadmie ou le cinabre : c'est ainsi que l'or est fabriqué dans les temples de Vulcain.\footnote{Il s'agit sans doute des Temples de Phtha (Vulcain). Tout ce morceau semble fort ancien et contemporain du Serment d'Isis et des traités hermétiques. Sur les livres attribués à Chéops, voir la note en tête de l'article précédent.} Je l'ai proclamé en parlant de la fabrication des cendres : c'est en elle que tout le mystère de la teinture a été caché.\footnote{Cp. Olympiodore, p. 99 et à la suite.}

3. Le cuivre ayant été blanchi, noirci et jauni, tu teins l'asèm et tu obtiens l'or, à l'aide du cuivre blanchi. En effet, c'est du cuivre que naissent toutes les espèces\footnote{Le cuivre est envisagé ici comme l'agent tinctorial par excellence, le générateur de toute couleur jaune ou rouge dans les métaux ou leurs dérivés, tandis que le plomb est la matière première commune, qui se change dans les divers métaux.} : j'entends le cinabre, la cadmie, l'or, la sandaraque et le reste. Le plomb se transforme en beaucoup (de corps) et il en est de même du cuivre (destiné aux) couronnes, qui provient de ces corps. Tu trouveras dans les temples de Vulcain ( ? ) les (procédés de) fabrication de l'or. C'est des mélanges (de ces métaux) que naissent toutes les espèces. Leurs traitements engendrent les substances les unes par les autres et il se produit des formes (très diverses) dans les traitements. En les appréciant toutes, fais usage des meilleures.

\bigskip
\centerline{\EightStarTaper}
\centerline{\EightStarTaper\EightStarTaper}
\bigskip

\subsubsection[3. --- 43. Chapitres de Zosime à Théodore.]{3. --- 43. Chapitres de Zosime à Théodore.\footnote{Ce sont les titres des divers ouvrages perdus de Zosime, parfois suivis d'un extrait ou d'un bref commentaire.}}
\paragraph{}
1. Sur la (pierre) étésienne, c'est-à-dire composée du Tout, en tant que pierre étésienne,\footnote{\emph{Salmasii Plinianæ exercitationes}, 776, \emph{b}, \emph{D.} Le \emph{Lexique} (p. 6, 7, 13, 16) l'assimile à la pyrite et à la chrysolithe, au porphyre et à l'androdamas.} et par là d'une grande utilité. En effet, dans les traitements, elle fait apparaître diverses couleurs : l'une dans le traitement de la kérotakis, une autre dans l'opération de la fusion à l'état de liquide oléagineux : à savoir une couleur jaune et une couleur noire. La couleur jaune varie depuis la nuance rougeâtre du foie, la nuance de la myrrhe, celle de la cire, ou toutes celles que tu sais. La couleur noire peut être semblable à l'or et scintillante. Or ce qui est efficace pour le noircissement, l'est aussi pour le jaunissement. Le jaune devient aussi couleur de sang, très stable, et finalement pareil à du safran desséché. Si on le brûle deux ou trois fois avec du soufre, d'après ces écrits, et si on le met en digestion quelque temps dans du fumier, on obtient alors des couleurs transformées et jaunies solidement ; leur modification initiale ayant eu lieu dans le sens du mieux et non du pire. Ce sont là les traitements appelés fixateurs, pour les teintures vraiment solides.

2. Sur ce que la teinture, c'est-à-dire l'altération qui se produit dans l'iosis, n'est désignée ni comme blanche, ni comme jaune. En effet les deux soufres qui précèdent, le blanc et le jaune, ont reçu ces noms, ainsi que les teintures. Mais la teinture même, qu'il s'agisse d'un changement ou d'une décomposition, est une opération plus avancée.

3. Sur deux autres corps appelés soufres, qui ne sont pas des soufres de l'ordre des premiers, mais des compositions qu'ils désignent aujourd'hui sous les noms de sulfureuses (ou divines), non en tant que soufre, mais à cause de l'œuvre divine accomplie par ces corps.\footnote{L'auteur joue sur le double sens du mot θεῖα.}

4. Sur ce que dans la composition on forme d'abord la matière fixatrice, celle qui résiste au feu et qui est tinctoriale. La première et la seconde nous sont manifestées dans l'asèm naturel, la dernière dans l'or obtenu par teinture. Mais la solution de la question est celle-là.

5. Sur ce que dans la matrice et d'une façon invisible pour nous, la matière fixatrice se forme avec deux (éléments), la semence et le sang ; puis l'animal une fois formé résiste au feu. C'est dans le feu de la matrice qu'il est teint, c'est-à-dire qu'il reçoit une couleur, une forme et une grandeur, tout (cela) dans un lieu invisible. Mais lorsque cet être a été enfanté, il se manifeste à nous. C'est ainsi qu'il faut travailler, sans se laisser égarer par l'homonymie\footnote{Cp. p. 196 et \emph{passim}.} des écrits ou des autres préceptes.

6. Sur la décomposition ; sur la production du sang ; sur la fermentation, la transformation et la régénération ; sur l'iosis et l'affinage et les différents noms de l'ios.

Comme quoi l'ios est dit eau de soufre natif ; comaris scythique et sanglante ; semence d'or et toute semence ; ios de cuivre ; eau de cuivre et eau de couperose ; fleur de cuivre et préparation cuivrée ; préparation de miel, corps doux et indestructible, en raison de l'adoucissement, et par suite de la résistance à l'attaque des agents délétères.

On ne l'a pas appelé seulement d'un nom masculin, féminin et neutre ; mais encore on lui a donné une forme diminutive, telle que la petite eau de cuivre ; d'autres, disent l'eau de la petite masse : or la masse, c'est le cuivre. Voilà pourquoi dans les écritures juives et dans toute écriture, on parle d'une masse inépuisable\footnote{Tout ce passage paraît se rapporter à la production d'un ferment métallique, indiqué précisément dans le Papyrus X de Leide sous le titre de « masse inépuisable » (recette 7, p. 29). --- La chimie de Moïse, traité qui sera donné plus loin, est aussi désignée sous le nom de \emph{maza} (v. p. 180). Ce mot même a été employé comme synonyme de la chimie (\emph{Introduction}, p. 209, 257).} que Moïse obtenait d'après le précepte du Seigneur.

Or ce mot, corrompu par le temps, est devenu petite masse. D'autres le tirent du phanos qui sert à puiser l'eau et qui porte des mamelons.\footnote{L'auteur joue sur le mot μαζύγιον, qu'il tire tantôt de μᾶζα, masse ; tantôt de μαζός, mamelon.}

7. Sur le bruissement du feu éteint (dans l'eau ? ) ; et sur le frémissement, c'est-à-dire le sifflement produit par le retrait du souffle ; ou bien sur le souffle produit par aspiration, ou par inspiration, et expiration.\footnote{Les bruits divers résultant des diverses formes de souffle jouaient un rôle important chez les gnostiques. (Voir Papyrus de Leide W, \emph{pagina} 1, l. 42 ; \emph{pag.} 2, l. 1 et suiv. ; \emph{pag.} 3, l. 2, et \emph{passim}).}

8. Sur ce que quelques-uns des prêtres, ayant trouvé un écrit sincère, ne croyaient pas pouvoir travailler autrement que d'après les démonstrations de cet ouvrage.

9. Sur ce que l'art de l'iosis se rapporte aussi aux deux autres livres. En effet, s'il est autre, quant à l'espèce ; du moins, quant au genre, c'est le même : c'est encore l'(art) tinctorial.

10. Sur ce qui est dit de l'affinage, de l'enlèvement de l'ombre, de la transformation et de l'extraction de la nature cachée, de la régénération par le feu : tout cela s'entend du blanchiment.

11. Sur les traitements utiles, depuis le blanc jusqu'au jaune, et depuis le jaune jusqu'au blanc. Au sujet des soufres notamment, il faut rechercher ce que dit le Philosophe dans sa dernière classe des liquides. « Fixe : arsenic, 1 once ; soufre, une demi-once ; écorce, 1 livre ; pèse-les ensemble. Pour le jaune, au lieu de peser les écorces en même temps, mets du safran et de la chélidoine. Au lieu des terres blanches, le même poids d'ocre, de terre de Sinope, ou de couperose, ou de sori. Quant aux (matières) qui ne sont pas comprises dans la pesée commune, unifie(-les) avec habileté, à la façon des enfants des médecins.\footnote{C'est le \emph{fac secundum artem} des formules pharmaceutiques d'aujourd'hui. Les enfants des médecins sont les apprentis.} Les liquides sont presque (tous) vulgaires, sauf quelques-uns que tu connais. »

12. Sur ce qu'il faut comprendre que nous nous sommes chargés d'un labeur terrible, en entreprenant de réduire à une essence commune, c'est-à-dire de marier à cette heure les natures ; comme quoi tout discours nous a été révélé à nous-mêmes ; ce qu'il faut rechercher dans ce discours ; comme quoi l'art revient à ceci : qu'est-ce ? de quelle nature est-ce ? et pourquoi est-ce ?

13. Sur ce que toutes les teintures des anciens sont réalisées en suivant la marche de la composition solide, c'est-à-dire de l'iosis. Car si vous mettez une partie d'ios, et 1 partie des espèces traitées, c'est-à-dire des poudres appelées tinctoriales, et si vous faites cuire, vous aurez un résultat exact.

14. Sur ce que la matière incombustible est celle qui ne possède plus ce qui peut éprouver la combustion, mais seulement ce qui a été brûlé : il en est ainsi des bois, et (pareillement) des sucs (animaux), dans les fièvres non critiques.

15. Sur ce que le résidu des matières brûlées, c'est-à-dire la scorie, représente l'acte accompli du Tout.

16. Sur la transmutation des quatre éléments (entre eux) ; comme quoi non seulement les (matières) venant de la terre et de l'eau se changent en feu, mais encore sont emportées vers le haut\footnote{Au-dessus M donne ici le signe du cinabre, et répète ce signe au-dessus du mot feu.} ; car le feu s'élève ; or il ne prend pas cette image au hasard, mais à cause de l'art et de ses espèces. Comme quoi ces matières étant d'abord terre et eau deviennent feu, et sont portées vers le haut. En effet c'est par leur seule qualité (propre) que les éléments sont opposés entr'eux, et non par leur substance ; car la substance n'est pas contraire à la substance, en tant que substance. C'est aussi pour cette raison que le Philosophe appelait substances les quatre éléments. Pour unifier leur substantialité, elles attirent dans leur intérieur la préparation enduite à leur extérieur. De même que les éléments dissous en eux accomplissent toutes choses, de même aussi l'art ; et de même que les quatre transformations triomphent des mélanges précédents, de même aussi nos arts, par les transmutations, triomphent des natures.

\bigskip
\centerline{\EightStarTaper}
\centerline{\EightStarTaper\EightStarTaper}
\bigskip

\subsubsection{3. --- 44. Sur les Divisions de l'Art Chimique.}
\paragraph{}
1. Comme quoi il faut chercher les discours utiles eux-mêmes, et que faut-il dire au sujet de l'art des discours : ou bien que c'est un art ? ou bien avant de poser la question : qu'est-ce ? ou de quelle nature est-ce\footnote{Voir dans l'article précédent le § 12.} ? il faut demander : pourquoi est-ce ? En ce qui touche les notions, ils les exposaient chacune en particulier, et tous étaient absurdes et embarrassés ; car on peut rencontrer une difficulté indivisible.

De même que les lignes musicales les plus générales étant au nombre de quatre, Α, Β, Γ, Δ, on forme avec elles 24 lignes d'espèces diverses ; et qu'il y a aussi des centres et des lignes obliques, selon qu'il a été dit à propos des sons, et attendu qu'il est impossible de composer autrement les mélodies innombrables des hymnes, pour le service (du culte ? ), la révélation, ou quelque autre partie de la science sacrée ... (Phrase inintelligible.)

Puis vient un long développement sur la musique et sur la comparaison entre ses divisions et celles de la chimie. On n'a pas cru utile de traduire les §§ 2, 3, 4.

5. De même que si tu divises en quatre parties la philosophie par excellence, la matière étant répartie suivant sa nature, tu trouveras la (science) générale et la (science) spéciale, ainsi que les différentes classes (de sujets) ; de même aussi, en cherchant à partager exactement la philosophie (chimique) en quatre parties, nous trouvons qu'elle contient : premièrement le noircissement, secondement le blanchiment, troisièmement le jaunissement, et quatrièmement la teinture en violet.\footnote{Cp. p. 194, le § 2 qui est un résumé du texte actuel.} De même encore que chacune des parties susdites comporte des subdivisions et un triage intermédiaires entre les lignes et les points principaux de la ligne, si l'on veut procéder par ordre ; de même aussi (en chimie) entre le noircissement et le blanchiment, il y a la macération et le lavage des espèces ; entre le blanchiment et le jaunissement, il y a la lévigation. Puis, entre le jaunissement et la teinture en violet, il y a la division par moitié de la composition ... Mais la fin de la teinture en violet est impossible sans le traitement au moyen de l'appareil à gorge, et sans l'union des parties. Il est impossible de procéder autrement dans notre science ; si quelques-uns, tels que Epibéchius, ont étudié le jaunissement sans parler du blanchiment, ils ne l'ont pas fait sans parler de la macération ou du lavage des espèces, choses qui font maintenant partie (de l'étude) du blanchiment complet.

Le § 6 est sans intérêt.

7. Le présent volume est intitulé livre métallique (et) chimique sur la Chrysopée, l'Argyropée, la fixation du mercure. Ce (livre) traite des vapeurs, des teintures qui proviennent des (êtres) vivants ( ? ), ainsi que des teintures des pierres vertes, des grenats et des pierres de toutes autres couleurs, de (la fabrication) des perles, et des colorations en garance des étoffes de peau destinées à l'Empereur. Toutes ces choses sont produites avec les eaux salées et les œufs, au moyen de l'art métallique.\footnote{Ce paragraphe est étranger à ce qui précède : c'est le titre d'un ouvrage perdu, mais dont certains extraits semblent exister dans notre 5\textsuperscript{e} partie.}

\bigskip
\centerline{\EightStarTaper}
\centerline{\EightStarTaper\EightStarTaper}
\bigskip

\subsubsection{3. --- 45. Fabrication du Mercure.}
\paragraph{}
1. Prenant de la céruse et de la sandaraque par parties égales, délaie avec du vinaigre jusqu'à ce que la masse s'épaississe ; ensuite, mettant dans un vase non étamé, recouvre avec un couvercle de cuivre ; lute tout autour et fais chauffer doucement sur des charbons. Lorsque tu présumes que l'opération est à point, découvre légèrement, et, avec une barbe de plume, enlève le mercure.\footnote{Cette préparation ne saurait fournir du mercure ordinaire, mais de l'arsenic sublimé, lequel reçoit ici le nom de mercure, parce qu'il blanchit le cuivre. (\emph{Introd.}, p. 99 et 239. --- Démocrite, p. 53).}

2. Prenant du minerai couleur d'or, pulvérise, puis évapore jusqu'à ce que le produit soit bien sec. Mélangeant alors avec du sel, fais chauffer dans le fourneau pendant un jour et une nuit. Après avoir enlevé, lave, jusqu'à ce que le sel dissous se soit écoulé ; dessèche de nouveau ; pétris avec du vinaigre et abandonne un peu (de temps), jusqu'à ce que la matière soit imbibée ; puis dessèche. Remets sur le fourneau, (cette fois) sans laver et fais cela encore une fois, en pétrissant avec du vinaigre. Remets au fourneau quatre ou cinq fois, jusqu'à ce que la matière devienne comme du vermillon. Ensuite, prenant de la scorie d'asèm à poids égal, pulvérise et mélange. Puis, après avoir fait fondre, sépare (en deux parties), saupoudre du plomb avec ces deux produits (et chauffe) jusqu'à ce que ces matières soient dissipées. Après avoir fait dessécher, tu trouveras le plomb durci ; fais-le fondre par petits fragments ; souffle afin de faire apparaître le métal.\footnote{Il semble qu'il s'agisse dans ce paragraphe d'une fabrication d'asèm, dont on opère la diplosis au moyen du plomb. Cp. \emph{Introd.}, Papyrus de Leide, p. 64.}

3. Prends de la terre provenant des bords du fleuve d'Égypte qui roule de l'or, pétris-la avec un peu de son, qui provient de la (fabrication de la) fleur de farine. Après avoir agité préalablement, mélangé et fait une pâte, mélange de nouveau dans un vase de terre cuite, jusqu'à ce que les deux (substances) soient tout à fait confondues et qu'il se soit formé comme une pâte de pain. Ensuite, reprends et forme de petits pains ; puis, ayant étendu avec soin sur une planche, fais évaporer au soleil jusqu'à ce que la matière soit bien sèche. Puis mets dans un mortier ; reprends, mets dans une marmite neuve ; ferme avec soin la marmite, place-la à une distance d'une palme du sol ; recouvre de fumier et fais du feu au-dessous. Lorsque la flamme se produit, découvre, remue avec un instrument de fer, jusqu'à ce que tu voies que le tout est cuit et semblable à une cendre noire. Si la matière n'est pas devenue telle, agite de nouveau en suivant le même procédé ; recouvre, fais chauffer ensemble ; puis retire du feu et laisse refroidir pendant un jour. Ayant pris une poignée (de cette matière) avec les deux mains, jette-la dans un vase de terre cuite ; ajoute du mercure, agite méthodiquement avec la main Ensuite, ôte de la marmite une autre poignée, ajoute une mesure d'eau, et lave. Ajoute encore une autre mesure (d'eau), et lave semblablement ; (opère ainsi) jusqu'à ce que la marmite soit vidée ; alors lave avec précaution jusqu'à ce qu'on soit parvenu au mercure. Mets dans un linge, presse avec soin jusqu'à épuisement. En déliant le linge, tu trouveras la partie solide. Après avoir fait cela, mets une boulette (du produit) sur un plat neuf ; fais au milieu, en enlevant de la matière, une sorte de fossette ; déposes-y la boulette, et recouvrant, dispose le plat de telle sorte qu'il dépasse partout également, à partir de sa partie centrale et jusqu'à la moitié de sa largeur. Recouvre de nouveau la marmite ; et que celle-ci adhère au plat. Plaçant (la marmite) sur les pieds d'un support, fais chauffer sur un feu clair, avec du bois sec ou de la bouse de vache, jusqu'à ce que le fond du plat devienne brûlant. Aie de l'eau auprès de toi pour arroser la préparation avec une éponge, en veillant à ce que l'eau ne tombe pas dans le plat. Après la chauffe, retire le plat du feu et, découvrant, tu trouveras ce que tu cherches.\footnote{Cette description semble répondre à l'extraction de l'or de son minerai par amalgamation.}

\bigskip
\centerline{\EightStarTaper}
\centerline{\EightStarTaper\EightStarTaper}
\bigskip

\subsubsection{3. --- 46. Sur la Diversité du Cuivre Brulé.}
\paragraph{}
Le premier paragraphe est identique à l'article 3, 13, p. 154.

2. La vapeur sublimée est une substance brûlée au moyen des alambics, sur un feu léger de cobathia.

Quant aux fixations (au moyen) des scories tirées de la partie inférieure, c'est ce que les prophètes des anciens voulaient obtenir. Tout le monde entend par là les minerais, parce que la matière des corps (métalliques) est dite tétrasomie, et aussi parce que les Égyptiens désiraient obtenir le plomb noir.\footnote{Olympiodore, p. 95} C'est dans cette opération que réside le noircissement. Or sachez que les scories sont tout le mystère\footnote{Olympiodore, p. 99.} ; car les anciens parlent du plomb noir, parce qu'il est le support de la substance. Comment cela arrive-t-il ? Si tu ne rends pas les corps incorporels, si de deux tu ne fais pas un,\footnote{Olympiodore, p. 101.} aucun des résultats attendus ne se produira. Si toutes choses n'ont pas été atténuées, si la vapeur sublimée n'a pas été réduite à l'état d'esprit, puis fixée, rien ne sera mené à terme. Qu'il s'agisse du molybdochalque, c'est ce que montrent les traitements des deux scories. Or, prépare une liqueur avec le plomb, en prenant : natron, quatre parties ; alun rond, une partie ; misy, deux parties ; sel de Cappadoce, 4 parties ; mets (le tout) dans du vinaigre très fort et fabrique une liqueur. Dans ces (opérations), tu ôteras l'éclat aux feuilles (métalliques). C'est de cette façon que la liqueur a été reconnue principe et fin. Lorsque tu verras que tout est devenu cendre,\footnote{Olympiodore, p. 107.} comprends alors que tu as bien exécuté la préparation par le feu. Pulvérise donc cette scorie et épuise-la de sa partie soluble ; lave-la six et sept fois dans des eaux édulcorées, après chaque fonte. Ces fontes ont lieu en raison de la richesse du minerai. En suivant cette marche et ce lavage, la composition s'adoucit. Après la fin de l'opération de l'iosis, une projection étant faite, on obtient un jaunissement stable. En faisant cela, tu fais sortir au dehors la nature cachée à l'intérieur. En effet, transforme la nature, dit-il, et tu trouveras ce que tu cherches.\footnote{La fin de ce paragraphe reproduit avec des variantes notables, le § 11 de la p.196.} La nature étant transformée perd sa couleur blanche.

\bigskip
\centerline{\EightStarTaper}
\centerline{\EightStarTaper\EightStarTaper}
\bigskip

\subsubsection{3. --- 47. Sur les Appareils et les Fourneaux.}
\paragraph{}
1. Voici la description du fourneau ci-dessous ; le Philosophe n'en a pas fait mention, mais il a parlé seulement des prismes et des autres (appareils), sur lesquels j'ai écrit dans (mon) commentaire relatif à la façon de régler le feu. Dans le sanctuaire antique de Memphis,\footnote{Temple de Phta.} j'ai vu en détail un fourneau qui s'y trouvait ; j'ai reconnu qu'il n'avait pas été mis en état par les gens initiés aux choses sacrées. Bonne santé.

2. Un grand nombre de constructions d'appareils ont été décrites par Marie ; non seulement ceux qui concernent les eaux divines (ou sulfureuses), mais encore beaucoup d'espèces de kérotakis et de fourneaux. Or les appareils pour le soufre sont ceux qu'il est nécessaire d'exposer en premier lieu. Parmi eux, il faut parler d'abord du récipient en verre, avec le tube en terre, le matras udcoé, le vase à col étroit, dans lequel pénètre le tube disposé en juste proportion avec l'ouverture du récipient.\footnote{Ce sont les appareils des figures 14, 14 \emph{bis} et 15 de l'\emph{Introduction}, p. 139 et 140.}

Il y a une autre manière de recueillir l'eau divine : le tube n'est pas alors disposé comme avec le \emph{tribicos}, mais placé à l'extrémité d'un autre tube de cuivre\footnote{Figure 16, p. 140 de l'\emph{Introduction}.} ; il est long d'une coudée ou d'une coudée et demie. On y ajuste de la même manière un récipient unique et, au-dessous (du tube de cuivre), le matras contenant le soufre apyre. Après avoir tout disposé, on fait chauffer. Voici le modèle. Il faut avoir dans tous les cas, une coupe pleine d'eau et rafraîchir le vase tout autour avec une éponge.

3. En ce qui touche le soufre, quelques-uns (se servent) du phanos et des appareils semblables, qui ont une base en forme de serpent. Ils y fixent aussi le mercure jaune isolément, en le soumettant à la vapeur du soufre. En cela ils comprennent mal les écrits antiques, qui ont caché que le phanos n'a pas de rôle ici ( ? ). J'ai été surpris (en lisant) cet écrit ; car deux mystères y ont été celés. Nous ne cherchons pas comment la combustion par le soufre, qui est blanc et blanchit tout, rend jaune le seul mercure ; (comment) ce produit, étant blanc en puissance et en acte, lorsqu'il est brûlé avec un corps blanc, produit du jaune. Il fallait que les modernes recherchassent avant tout ces choses et comprissent l'autre mystère, à savoir que le mercure n'est pas fixé par le soufre seul, mais qu'il faut pour cela la composition tout entière.

4. J'ai ri, en écoutant la lecture de ton écrit qui décrit ce genre d'opérations : « Que le matras, est-il dit, contienne une mine de soufre apyre ... » je me suis étonné de ce que, ne pouvant supporter les reproches, tu aies prétendu écrire de pareilles choses ; tu as blâmé à tort ce philosophe, car tu n'as pas compris ce qu'il a dit. Dans les précédents commentaires, j'ai dit que je parlais de la fabrication des eaux, mais non de leur distillation ; car autre chose est la fabrication, autre chose la distillation. (Chacun) de ces auteurs a parlé amplement de la distillation ; mais aucun n'a exposé la fabrication ; c'était là le mystère qu'on ne devait pas révéler, celui qui a été tout à fait caché. Or la distillation est de telle nature et (s'accomplit) au moyen de tels appareils.\footnote{Ceux qu'il va décrire.} Quant à la fabrication, c'est-à-dire la composition de cette eau, elle a été décrite dans l'exposé détaillé de l'œuvre.\footnote{Cependant il va la décrire de nouveau § 6. --- Cp. 3, 16, p. 158.}

5. Je vais décrire le tribicos\footnote{Fig. 15, p. 139 de l'\emph{Introduction}.} : Fabrique, dit-il, trois tubes de cuivre laminé ; dispose la lame ductile de façon qu'elle ait l'épaisseur du couvercle, ou un peu plus : par exemple, la moitié de l'épaisseur d'une monnaie de cuivre. Fabrique donc trois tubes dans ces conditions, et fabrique un (gros tube) de cuivre,\footnote{On traduit ainsi le mot χαλκεῖον, qui désigne en effet le gros tube vertical, dans la fig. 16 de la page 140 de l'\emph{Introd.} ; σωλῆνες doit être entendu des trois tubes abducteurs, par lesquels les produits distillés s'échappent du tribicos ; βῆκος ou βίκος est le récipient, où s'écoulent les produits. Ce mot désigne aussi (fig. 14, p. 138) le chapiteau, appelé autrement φιάλη dans la fig. 11 (p. 132). Enfin λωπάς est le matras où l'on place le soufre et qui est exposé directement à l'action du feu. Ces désignations s'appliquent aux figures du manuscrit de Venise.  } long d'une coudée, ayant une palme de diamètre. L'ouverture du gros tube sera en proportion convenable ; les trois (petits) tubes ont une ouverture adaptée à celle du col du petit récipient. Vis-à-vis du tube du pouce sont les deux tubes de l'index,\footnote{Les mots ἀντίχειρος σωλήν (tube du pouce) et λιχανός σωλήν (tube de l'index), sont appliqués à des tubes différents dans les fig. 11 (p. 132 de l'\emph{Introd.}) et 15 (p. 139 de l'\emph{Introd.}). Le premier nom désigne dans les deux figures un petit tube oblique et descendant. Quant au second nom, la fig. 15 paraît indiquer le gros tube ascendant, de direction inverse, qui est désigné dans la fig. 14 (p. 138), sous le nom de « tube de terre cuite, » et dans la fig. 16 (p. 140), sous le nom de χαλκεῖον, objet dont il a été question dans la note précédente. Ces désignations ne correspondent pas exactement au texte ci-dessus, dans lequel le tube du pouce est mis en opposition avec les deux tubes de l'index : ces derniers représentant deux des petits tubes descendants du tribicos, le tube du pouce serait alors le troisième, comme dans la fig. 15.} ajustés au moyen d'une clavette, des deux côtés, près de l'extrémité du gros tube ; vers cette extrémité existent trois orifices, ajustés aux tubes ainsi raccordés (avec le gros tube). Ces orifices sont soudés d'une façon excentrique avec le récipient supérieur, celui où se rend la partie volatile.

Dans le manuscrit A, plus moderne (fig. 37, p. 161 de l'\emph{Introd.}), λωπάς a le même sens ; mais χαλκεῖον s'applique ici au chapiteau, qui a pris une forme nouvelle et caractéristique. La description du texte a cessé de répondre à cette dernière forme. La forme du λωπάς s'est également rapprochée de notre chapiteau moderne (v. p. 161 de l'\emph{Introd.}), ou plus exactement de celle du pélican, appareil distillatoire qui était encore usité au siècle dernier.

Place le gros tube de cuivre au-dessus du matras en terre cuite, qui contient le soufre. Après avoir luté les jointures avec de la pâte de farine, adapte aux extrémités des (petits) tubes des récipients en verre grands et forts, afin qu'ils ne cassent pas, en raison de la chaleur de l'eau. Porte ce qui monte dans les appareils où le Philosophe dit que l'eau s'élève.

6. Quant à la préparation et à la composition, je ne craindrai pas de t'écrire sur ce point, ô ma princesse. La fabrication des eaux comprend ce qui suit\footnote{Cp. p. 182.} : l'Eau de soufre, d'arsenic, de sandaraque ; la vapeur, l'eau de lie, l'eau de chaux, l'eau de cendre de choux, l'eau d'alun, l'eau d'urine, de lait d'ânesse, de chèvre ; parfois le lait de chienne, le lait de vache, et le lait de la femme mère d'un enfant mâle, suivant Agathodémon ; le vinaigre, l'eau de mer, le miel et le ricin ou \emph{gry} ( ? ), l'urine d'un impubère et la gomme. Leur production a lieu comme il suit. Chaque eau se prépare à la façon d'une saumure proprement dite. Quand il s'agit de l'eau de cendre, elle se prépare comme la lessive pour savonner, que j'ai décrite dans l'exposé des manipulations. Si tu ne réussis pas, opère la composition avec une cotyle d'eau. Emploie une once des espèces suivantes,\footnote{Cp. p. 165, § 15.} savoir : une once de soufre et une once d'eau pure ; une once d'arsenic et une cotyle d'eau ... ; de la lie cuite, éteinte dans le vinaigre ; de la chaux éteinte dans une cotyle d'urine de chat ; de l'alun, une once, délayé dans une cotyle d'eau de mer ; du natron roux, même quantité. Après avoir fait cuire séparément et ensemble les eaux, pendant un peu de temps, afin qu'elles prennent de la force, fais dessécher ou distiller dans un autre vase, en y mêlant le miel et l'huile. S'il est besoin de soufre blanc,\footnote{Ce mot est une désignation générique, applicable à toutes les espèces suivantes, ainsi qu'il a été dit ailleurs, voir p. 162, § 10 et note, p. 180, § 4, p. 185, § 4 et p. 186, § 5, etc.} délaie dans l'eau la terre de Chio, l'astérite, l'aphrosélinon de Coptos cuit, la terre de Samos, celles de Carie, de Cimole, ou l'antimoine ( ? ). Mettant dans un vase l'eau devenue bleue, ajoutes (y) du marbre (tiré) de la terre, du misy brut, et une autre partie de chaux ; on en emploie deux parties, suivant les écrits des anciens, où le produit est nommé l'eau double de chaux. Ajuste l'appareil sur le matras, fais monter l'eau et mets en œuvre.

7. L'eau jaune se prépare comme il suit : Soient toutes les eaux obtenues d'après les règles précédentes ; au lieu de faire l'addition de deux parties de chaux, ajoute une partie de sel, après avoir fait cuire chacune de ces eaux séparément et les avoir mélangées, délaies-y, non plus des terres blanches, mais des terres jaunes. Car nous voulons obtenir de l'eau jaune. Or, les terres jaunes sont l'ocre attique, le minium du Pont, le misy cuit, la couperose cuite, et les matières semblables ; toutes les plantes (jaunes) que l'on connaît communément,\footnote{Cp. p. 166, § 18. Sur le sens du mot plante, voir p. 71, p. 123, p. 153, note 2, p. 159, § 4 et note 2, etc.} ainsi que le jaune d'œuf, le safran des œufs et la chélidoine double. Quant aux herbes, ne les incorpore pas avec l'eau, mais seulement les terres. Puis, changeant de vase, comme on le fait d'ordinaire, ajoute les plantes et fais cuire quatre ou cinq fois, dans l'appareil. Fais monter l'eau et emploie-la, avec addition de gomme. Après avoir découvert (l'appareil), tu trouveras les herbes brûlées, ayant perdu leur teinte propre, c'est-à-dire leur esprit propre. La portion la plus pure de cette eau divine a une vertu et une nature telle que, si vous trempez l'argent dans l'eau bouillante, la teinture sera indélébile. Bonne santé !

\bigskip
\centerline{\EightStarTaper}
\centerline{\EightStarTaper\EightStarTaper}
\bigskip

\subsubsection[3. --- 48. Fabrication de l'Argent avec la Tutie.]{3. --- 48. Fabrication de l'Argent avec la Tutie.\footnote{Recette surajoutée dans le manuscrit de St-Marc et plus moderne.}}
\paragraph{}
Prenant de la tutie, environ 20 hexages (poids), broyez jusqu'à ce qu'elle devienne or\footnote{Prenne la couleur de l'or.} ; (prenant) environ 5 hexages de soufre apyre, broyez jusqu'à ce qu'il devienne plomb.\footnote{Prenne la couleur du plomb, en agissant sur les oxydes mélangés qui forment la tutie.} Ensuite prenant 6 blancs d'œufs, après avoir décapé, mettez dans l'alambic, et faites cuire pendant deux jours et deux nuits. Enlevez pour voir si la matière est bien à point ; remettez de nouveau (la matière) et faites cuire (encore) pendant un jour. Ensuite prenant du cuivre, environ 10 hexages, mettez-le dans un creuset et projetez-y 6 cotyles (de la matière ci-dessus) : vous obtenez de l'argent.\footnote{C'est-à-dire un alliage blanc.}

\bigskip
\centerline{\EightStarTaper}
\centerline{\EightStarTaper\EightStarTaper}
\bigskip

\subsubsection[3. --- 49. Du même Zosime sur les Appareils et Fourneaux. Commentaires Authentiques sur la Lettre Ω.]{3. --- 49. Du même Zosime sur les Appareils et Fourneaux. Commentaires Authentiques sur la Lettre Ω.\footnote{Ce titre est probablement celui de l'un des livres de Zosime, désignés chacun par l'une des lettres de l'alphabet. Le premier paragraphe serait le début du livre ; il roule sur une suite de jeux de mots sur l'oméga, assimilé à l'œuf philosophique et à l'océan.}}
\paragraph{}
1. L'élément Ω est rond, formé de deux parties : il appartient à la septième zone, celle de Saturne,\footnote{Saturne occupe le 7\textsuperscript{e} des cercles concentriques ou zones de l'univers, qui ont la terre pour centre commun, dans la classification des astres errants ou planètes ; Saturne correspond aussi à la lettre Ω, dans la concordance des voyelles avec ces astres ; ainsi qu'au plomb, dans la nomenclature des métaux (corps métalliques).} dans le langage des êtres corporels ; car dans le langage des incorporels, il y a une autre chose qui ne doit pas être révélée. Nicothée seul (la) sait, lui le personnage caché. Or, dans le langage des êtres corporels, cet élément est appelé l'océan, l'origine et la semence de tous les dieux. Tels les principes fondamentaux du langage des êtres corporels.\footnote{Cette multiplicité des langages mystiques, où un même sens s'exprime par des mots divers, tandis qu'un même signe répond à plusieurs sens, se retrouve dans le Papyrus W de Leide, \emph{Introd.}, p. 18. La Cabbale repose aussi sur des conventions analogues.} Sous le nom de ce grand et admirable élément Ω, on comprend la description des appareils de l'eau divine, celle de tous les fourneaux simples et machinés, de tous, absolument parlant.

2. Zosime (s'adressant) à Théosébie, lui explique ceci avec bonne volonté. « (L'exposé des) teintures convenables, ô femme, a fait tourner en ridicule mon livre sur les fourneaux. En effet, beaucoup (d'écrivains), remplis de bienveillance pour leur propre génie, se sont moqués des teintures convenables et ils ont regardé le livre sur les fourneaux et appareils comme n'étant pas conforme à la vérité. Aucun discours ne peut leur persuader ce qui est la vérité, s'il n'est inspiré par leur propre génie. Par un destin fatal, ce qu'ils avaient reçu, ils le tournaient à mal dans leur langage, au détriment de l'art et de leur propre succès, les mêmes mots étant détournés malheureusement dans les deux sens (opposés). C'est avec peine que, contraints par la nécessité des démonstrations, ils accordaient quelque point, même au sujet des choses qu'ils avaient comprises précédemment. Mais de tels auteurs ne doivent être approuvés, ni par Dieu, ni par les philosophes. Car les temps (des opérations) étant désignés dans le dernier détail, et après que le Génie les a favorisés dans l'ordre corporel,\footnote{C'est-à-dire dans l'opération de la régénération des corps métalliques.} ils refusent d'accorder un autre point, oubliant toutes les choses évidentes qui précèdent. Ils ont dû partout obéir à la destinée, pour les choses déjà dites et pour leurs contraires, sans pouvoir rien imaginer d'autre, relativement aux êtres corporels ; (je dis) rien d'autre que l'ordre fatal de la destinée. Les hommes de cette espèce, Hermès, dans le traité sur les Natures, les appelait des insensés, propres seulement à faire cortège à la destinée, mais incapables de rien comprendre aux choses incorporelles, ni même de concevoir la destinée qui les conduit avec justice. Mais ils font outrage à ses enseignements sur les êtres corporels, et ils se livrent à des imaginations étrangères à leur propre bonheur.

3. Hermès et Zoroastre ont déclaré que la race des philosophes est supérieure à la destinée. En effet, ils ne jouissent pas du bonheur qui vient de celle-ci. Dominant ses plaisirs, ils ne sont pas atteints par les maux qu'elle cause ; vivant toujours dans leur for intérieur, ils n'acceptent pas les beaux présents qu'elle offre, parce qu'ils en voient la fin malheureuse. C'est pour cette raison qu'Hésiode\footnote{\emph{Œuvres et Jours}, vers 86.} nous présente Prométhée donnant des conseils à Epiméthée : « Quel est le bonheur que les hommes jugent le plus grand de tous ? Une belle femme, dit-on, avec beaucoup d'argent. » Il dit qu'il ne reçoit aucun présent de Jupiter Olympien ; mais il les rejette, enseignant à son frère qu'il doit repousser, au nom de la philosophie, les présents de Jupiter, c'est-à-dire les dons de la destinée.

4. Quant à Zoroastre, se glorifiant de la connaissance de toutes les choses supérieures et de celles de la magie, il dit qu'il se détourne du langage des êtres corporels ; que tout ce qui vient de la destinée est mauvais, soit en détail, soit dans l'ensemble. Hermès, toutefois, parlant des choses extérieures, condamne la magie, disant que l'homme spirituel, celui qui se connaît lui-même, ne réussit en rien par la magie, et ne regarde pas comme convenable de violenter la nécessité. Mais il laisse aller (les choses), telles qu'elles vont de nature et d'autorité. Il a pour seul objet de se chercher lui-même, de connaître Dieu, et de dominer la triade innommable. Il laisse la destinée faire ce qu'elle veut, en la laissant agir sur le limon terrestre, c'est-à-dire sur le corps. Il s'exprime ainsi : « Si tu comprends et si tu te conduis convenablement, tu contempleras le fils de Dieu, devenu tout\footnote{Ce mot vague est expliqué deux lignes plus bas.} en faveur des âmes saintes. Pour tirer ton âme du sein de la région (corporelle), régie par la destinée, (et l'amener) vers la (région) incorporelle, vois comme il est devenu tout, (c'est-à-dire à la fois) Dieu, ange, et homme sujet à la souffrance. En effet pouvant tout, il devient tout ce qu'il veut ; il obéit à son père, en pénétrant tout corps, en éclairant l'esprit de chacun ; il s'est élancé dans la région heureuse, là où il était avant d'avoir pris un corps. Tu le suivras, excité et guidé par lui vers cette lumière.

5. Regarde le tableau que Cébès a tracé, ainsi que le trois fois grand Platon et le mille fois grand Hermès ; vois comment Toth interprète la première parole hiératique, lui le premier homme, interprète de tous les êtres, et dénominateur de toutes les choses corporelles. Or les Chaldéens, les Parthes, les Mèdes et les Hébreux le nomment Adam : ce qui signifie terre vierge, terre sanglante, terre ignée et terre charnelle.\footnote{Ce texte est mutilé, comme on le voit dans Olympiodore, p. 95, note 5. En effet ce qui est relatif à la terre s'applique à Ève.} Ces choses se trouvent dans les bibliothèques des Ptolémées, déposées dans chaque sanctuaire, notamment au Sérapéum ; (elles y ont été mises) lorsque Asenan, l'un des grands prêtres de Jérusalem, envoya Hermès,\footnote{Le nom d'Hermès reprend ici le sens générique, suivant lequel il était l'auteur de tous les ouvrages égyptiens. Voir Clément d'Alexandrie, cité dans les \emph{Origines de l'Alchimie}, p. 39 et 40. On remarquera que l'origine de la traduction grecque de la Bible se trouve expliquée ici autrement que dans la traduction des Septante.} qui interpréta toute la Bible hébraïque en grec et en égyptien.

6. C'est ainsi que le premier homme est appelé Toth parmi nous, et parmi eux, Adam ; nom donné par la voix des anges. On le désigne symboliquement au moyen des quatre éléments,\footnote{Le même mot signifie lettre et élément.} qui correspondent aux points cardinaux de la sphère, et en disant qu'il se rapporte au corps.\footnote{En tant que formé par la réunion des quatre éléments.} En effet, la lettre A de son nom désigne l'Orient (Ἀνατολή) et l'Air (Ἀήρ). La lettre D désigne le couchant (Δύσις), qui s'abaisse à cause de sa pesanteur. La lettre M montre le Midi (Μεσημβρία), c'est-à-dire le feu de la cuisson qui produit la maturation des corps, la 4\textsuperscript{e} zone et la zone moyenne.

Ainsi l'Adam charnel, sous sa forme apparente, est appelé Toth ; mais l'homme spirituel contenu en lui (porte un nom) propre et appellatif. Or nous ignorons jusqu'à présent quel est ce nom propre ; car Nicothée, ce personnage que l'on ne peut trouver, savait seul ces choses. Quant au nom appellatif, c'est celui de φῶς (lumière, feu) : c'est pour cela que les hommes sont appelés φῶτες (mortels).

7. Lorsqu'il était dans le Paradis sous forme de lumière (φῶς), soumis à l'inspiration de la destinée, ils lui persuadèrent en profitant de son innocence et de son incapacité d'action, de revêtir\footnote{Voir plus haut.} le (personnage d') Adam, celui qui (était soumis à) la destinée, celui qui (répond) aux quatre éléments. Lui, à cause de son innocence, ne refusa pas ; et ils se vantaient d'avoir asservi (en lui) l'homme extérieur.

C'est dans ce sens qu'Hésiode\footnote{Cp. \emph{Théogonie}, vers 521, 618.} a parlé du lien avec lequel Jupiter attacha Prométhée. Ensuite, après ce lien, il lui en envoie un autre, (c'est-à-dire) Pandore, que les Hébreux nomment Eve. Or, Prométhée et Épiméthée, c'est un seul et même homme dans le langage allégorique ; c'est l'âme et le corps. Prométhée est tantôt l'image de l'âme ; tantôt (celle) de l'esprit. C'est aussi l'image de la chair, à cause de la désobéissance d'Épiméthée, commise à l'égard de Prométhée, son propre (frère).

Notre intelligence dit : Le fils de Dieu, qui peut tout et qui devient tout lorsqu'il (le) veut, se manifeste comme il veut à chacun. Jésus-Christ s'ajoutait à Adam et (le) ramenait au Paradis, où les mortels vivaient précédemment.

8. Il apparut aux hommes privés de toute puissance, étant devenu homme (lui-même), sujet à la souffrance et aux coups. (Cependant), ayant secrètement dépouillé son propre caractère mortel, il n'éprouvait (en réalité) aucune souffrance ; et il avait semblé fouler aux pieds la mort, et la repousser, pour le présent et jusqu'à la fin du monde : tout cela en secret. Ainsi dépouillé des apparences, il conseillait aux siens d'échanger aussi secrètement leur esprit avec celui de l'Adam qu'ils avaient en eux, de le battre et de le mettre à mort, cet homme aveugle étant amené à rivaliser avec l'homme spirituel et lumineux : c'est ainsi qu'ils tuent leur propre Adam.\footnote{Ce passage, ainsi que ceux qui précèdent doivent être rapprochés des doctrines des docètes et de celles de certains gnostiques. (Cp. Renan, \emph{Histoire des Origines du Christianisme}, t. 5, p. 421, 458, 525, etc.)}

9. Ces choses se font jusqu'à ce que vienne le démon \emph{Antimimos}\footnote{Contrefacteur. --- Son intervention rappelle le manichéisme et les doctrines persanes sur les deux principes.} ; jaloux d'eux et voulant les induire de nouveau en erreur, il se dit lui-même fils de Dieu ; bien qu'étant sans forme (originale),\footnote{Comme son nom l'indique.} ni d'âme ni de corps. Mais devenus plus sensés, par suite de la prise de possession de celui qui est réellement fils de Dieu, ils lui abandonnent leur propre Adam ; immolant leurs esprits mortels, ils demeurent sauvés, dans le lieu particulier où ils se trouvaient avant (la création du) monde. Ainsi, avant d'accomplir ces choses, il envoie d'abord l'Antimimos, le rival, son précurseur, sorti de la Perse, lequel tient des discours pleins d'erreurs et de fables, et dirige les hommes suivant la destinée. Or les éléments de son nom sont au nombre de neuf, la diphtongue étant conservée,\footnote{S'agit-il d'il d'εἱμαρμένη, qui a 9 lettres et une diphtongue ; ou bien du génitif ἀντιμίμου, qui satisfait aux mêmes conditions ; ou bien encore de φαοσφόρος, Lucifer ?} suivant le but que se propose la destinée. Ensuite, après sept périodes, plus ou moins, il viendra aussi lui-même, en vertu de sa nature propre.

10. Ces choses sont dites seulement par les Hébreux, ainsi que par les livres sacrés d'Hermès sur l'homme lumineux et sur le fils de Dieu, son guide ; sur l'Adam terrestre et sur Antimimos son guide, qui se dit, par blasphème et erreur le fils de Dieu. Or les Grecs appellent l'Adam terrestre Épiméthée : ce qui veut dire conseillé par son esprit particulier, c'est-à-dire par son frère, qui lui disait de ne pas accepter les dons de Jupiter. Toutefois, s'étant abusé et repenti, et ayant cherché la région heureuse, il explique tout, et il conseille en tout ceux qui ont un entendement spirituel. Mais ceux qui n'ont qu'un entendement corporel, appartiennent à la destinée ; ils n'admettent ou ne confessent rien d'autre.

11. Tous ceux qui (font des teintures) convenables et réussissent (par hasard) ne disent pas autre chose ; ils persiflent l'art exposé dans le grand livre sur les fourneaux, et ils ne comprennent pas non plus le Poète lorsqu'il dit :
\begin{quotation}
« Mais les Dieux n'avaient pas encore donné en même temps aux hommes ... etc. »
\end{quotation}
\paragraph{}
Ils ne réfléchissent à rien et ne voient pas les divers genres de vie des hommes : comme quoi les hommes réussissent différemment dans un seul (et même) art ; comment ils opèrent différemment dans un seul (et même) art ; comment ils pratiquent un seul (et même) art, au moyen des caractères et des figures diverses des astres ( ? ). Ils ne voient pas que tel artisan est paresseux ( ? ), tel artisan isolé ; tel autre dégénère, tel devient pire, tel ne progresse pas. Il arrive aussi que l'on rencontre dans tous les arts des gens qui travaillent un même art avec des outils et des procédés différents, et qui ont à un degré différent l'intelligence et la réussite.

12. Parmi tous les arts, c'est surtout dans l'art sacré qu'il convient de considérer ces choses. Par exemple, après une fracture, si le patient rencontre un prêtre (habile, celui-ci agissant de sa propre inspiration,\footnote{C'est la pratique du prêtre rebouteur, envisagée comme supérieure à la science écrite du médecin.} réunit les fragments, de telle sorte que l'on entend le craquement des os qui se rejoignent. Si l'on ne trouve pas un tel prêtre, que le blessé cependant ne craigne pas de mourir, mais que l'en amène des médecins avec leurs livres, pourvus de dessins et de figures ombrées. Etant pansé conformément aux lignes des figures du livre, le blessé est entouré de liens mécaniquement et il continue à vivre, après avoir repris la santé. Nulle part l'homme ne se résigne à mourir, faute de trouver un prêtre qui réunisse les fractures.

Au contraire, ceux-ci, les malheureux (ignorants), se laissent mourir de faim, plutôt que d'apprendre à connaître et à pratiquer la description des fourneaux, telle qu'elle est tracée : c'est par là que, devenus bienheureux, ils triompheraient de la pauvreté, cette maladie incurable. En voilà assez sur ce chapitre.

13. Quant à moi j'arrive à mon sujet, qui concerne les fourneaux. Ayant reçu les lettres que tu as écrites, j'ai vu que tu m'invites à rédiger pour toi la description des appareils. J'ai été surpris de voir que tu écrives pour obtenir de moi la connaissance des choses qui ne doivent pas être connues ; n'as-tu pas entendu le Philosophe ; lorsqu'il dit : « Ces choses, je les ai passées volontairement sous silence, parce qu'elles sont décrites amplement dans mes autres écrits ? » Cependant tu as voulu les apprendre de moi ; ne crois pas du reste que mon écrit soit plus digne de foi que celui des anciens, et sache que je ne pourrais pas (les surpasser). Mais, afin que nous entendions tout ce qui a été dit par eux, je vais t'exposer ce que je sais. Voici ce que c'est.

14. Récipient de verre, tube de terre cuite de la longueur d'une coudée. Matras ou vase à étroite embouchure, dont le goulot est proportionné à la grosseur du tube. Voici le modèle.\footnote{Il répond à la figure 16 de la p. 140 de l'\emph{Introd}. Ce passage reproduit la fin du second alinéa du § 2 de la p. 216, avec des variantes considérables.} Il faut avoir une coupe d'eau et mouiller le vase avec une éponge. Pour les vapeurs sublimées, ainsi que pour le mercure, c'est le même vase.

On peut fixer le mercure dans le phanos (vase) et dans des appareils semblables, ayant un récipient de forme serpentine. On jaunit (le mercure) par la vapeur du soufre ; c'est là ce que conseillent les anciens écrits, le phanos ne contenant pas le soufre.\footnote{Les Ms. indiquent le plomb, sans doute par suite d'une confusion de signes.} Tu seras surpris, au sujet de cet écrit, de ce que deux mystères manifestes y ont été cachés. D'abord ne cherchons-nous pas comment la vapeur du soufre, qui blanchit (les métaux), rend (cependant) le mercure jaune, ni comment cela arrive lorsqu'il est brûlé ? Et en outre, comment ce mercure, étant blanc en puissance et en fait, devient jaune lorsqu'il est brûlé et fixé par une substance blanche ?

15. Il fallait donc que les modernes cherchassent avant tout ces choses. Quant à l'autre mystère, je pense que (le mercure) n'est pas fixé seul, mais avec toute la composition. Maintenant les appareils dans lesquels on exécute aussi la (fabrication de l'eau) de soufre natif, la fixation du mercure, l'arrosage des mélanges et leur teinture sont ceux-ci.
\begin{center}
(Suit la formule de l'Écrevisse, \emph{Introd.}, p. 152.)
\end{center}
\paragraph{}
16. L'ios qui provient du cuivre sans ombre, étant jauni, est soumis à l'action de la sublimation ; puis on le dépose dans du miel blanc.

17. La masse molle, jaunie par notre cuivre, agit en son lieu et place, mais moins fortement que ... : tout cela se trouve chez Agathodémon.

18. La masse molle obtenue avec les petites scories, mettez-la dans le phanos (vase) et fixez avec la vapeur des soufres volatilisés, afin qu'elle devienne comme du cinabre. Ensuite mettez-la dans des bocaux ou dans des coupes, étalez et employez comme ci-dessus.

Signes de : Ciel ; soleil (ou or). Terre, ciel.\footnote{Dans B ces signes sont répétés au-dessus des mots : « toutes les espèces. »}

19. Comme on le voit, toutes les espèces (provenant) des vapeurs, ont été mélangées par Agathodémon : telles sont la chrysocolle, la (pierre) étésienne, la fleur d'or et en général toutes celles qui servent dans la teinture de l'argent, ainsi que le comporte sa dernière classe. Or il emploie les vapeurs, afin d'éviter que l'argent se réduise en scorie, ou qu'il ne cède sa substance aux corps épais et terreux, susceptibles d'être calcinés et torréfiés.

\bigskip
\centerline{\EightStarTaper}
\centerline{\EightStarTaper\EightStarTaper}
\bigskip

\subsubsection{3. --- 50. Sur le Tribicos et le Tube.}
\paragraph{}
1. Je vais te décrire le tribicos. On appelle ainsi la construction en cuivre transmise traditionnellement par Marie.\footnote{Voir l'article précédent, p. 217, § 5. --- Il y a ici des variantes considérables.} Voici en quels termes : Fabrique, dit-elle, trois tubes de cuivre laminé et aminci, d'une épaisseur dont voici la mesure : ce sera à peu près celle d'une poêle en airain, à faire cuire les gâteaux ; la longueur sera d'une coudée et demie. Fabrique donc trois tubes dans ces conditions, et fabrique aussi un (gros) tube, ayant environ une palme de diamètre et une ouverture proportionnée à celle du vase de cuivre.\footnote{Ici χαλκεῖον paraît signifier chapiteau (voir la note 1 de la p. 218).} Les trois tubes auront une embouchure adaptée au col du petit récipient, au moyen d'une clavette, par le tube du pouce\footnote{Cp., la note 2 de la p. 218.} ; afin que les deux tubes de l'index s'adaptent latéralement aux deux mains.\footnote{C'est-à-dire aux deux récipients correspondants.} Vers l'extrémité du vase de cuivre, existent trois orifices, ajustés aux tubes et bien raccordés. On les soude d'une façon excentrique au récipient supérieur, destiné à recevoir la partie volatile. On place le vase de cuivre au-dessus du matras en terre cuite qui contient le soufre. Après avoir luté les jointures avec de la pâte de farine, adapte aux extrémités des tubes des récipients en verre, grands et forts, afin qu'ils ne cassent pas en raison de la chaleur de l'eau qui entraîne la matière distillée. Voici la figure : Tube de l'index.\footnote{Ceci désigne la figure 15 de la p. 139 de l'\emph{Introd.}}
\begin{center}
Le § 2 est la reproduction du premier alinéa du § 2 de l'article 3, 47, p. 216.

Le § 3 reproduit le § 4 du même article (p. 217), mais avec des variantes très importantes que l'on va donner.
\end{center}
\paragraph{}
3. J'ai ri en écoutant ce qui est relatif aux diverses classes de ces appareils. Car tu dis : Pour chaque opération, que le matras contienne une mine de soufre apyre. Et je t'ai admirée aussi en ceci que, ne supportant pas le reproche, tu aies prétendu écrire de pareilles choses. De plus tu en es venue à critiquer le Philosophe, parce qu'il a osé dire : « Ces choses je les ai passées sous silence, attendu qu'elles sont déjà exposées avec grands détails dans les écrits des autres ... (Lute) avec du suif, ou de la cire, ou de la terre grasse, ou avec ce que tu voudras, et, après avoir calciné, enlève. » Or voici la figure qui se trouve dans les écrits.

Insistant dans un sentiment d'envie indomptable, tu critiques vainement le Philosophe ; car tu n'as pas compris ce qu'il dit. Il ne veut pas parler, comme dans les commentaires précédents, de la fabrication des eaux, mais de leur distillation ; car autre chose est la fabrication, autre chose la distillation. Il a dit qu'on n'écrivait rien en détail sur leur mercure ; nul d'entre eux n'en exposait la fabrication ; car c'était là le mystère caché. C'est une chose celée avec soin. La distillation a donc lieu au moyen de ces appareils, ou d'autres similaires, imaginés par les gens intelligents ; tels sont ceux qui ont étudié auparavant les Pneumatiques d'Archimède, ou d'Héron et d'autres auteurs, ainsi que leurs écrits relatifs à la mécanique.

4. \emph{Sur d'autres fourneaux}. --- Comme la suite de notre discours a pour sujet les fourneaux et la teinture, je ne veux pas te répéter ce qui se trouve dans les écrits des autres. En effet, chez Marie, la description du fourneau présentée ici ne figure pas. Le Philosophe n'en a pas fait mention, mais seulement des prismes et des autres (appareils) dont j'ai parlé en passant, dans le commentaire sur les règles du feu.\footnote{Cp. page 216.} Afin qu'il ne puisse rien manquer à tes écrits, parles-y du fourneau de Marie, celui dont Agathodémon a fait mention, en ces termes : « Or voici la description de la classe des kérotakis destinées au soufre mis en suspension. Prenant une coupe, fais (y) des divisions, c'est-à-dire fais avec une pierre une entaille centrale et circulaire dans le fond de la coupe, afin d'y engager à la partie inférieure une saucière de dimension correspondante.\footnote{Figure 25 de la page 149 et figure 22 de la page 146 de l'\emph{Introduction}.} Dispose un vase mince de terre cuite, ajusté et suspendu à la coupe, retenu par elle dans sa partie supérieure ; et s'avançant vers la kérotakis de fer. Dispose la feuille (métallique) que tu voudras, conformément à l'écrit, au-dessus du vase et au-dessous de la kérotakis, en même temps que la coupe, de telle façon que tu puisses voir à l'intérieur. Après avoir luté les jointures, fais cuire autant d'heures que le dit notre rédaction. Voilà pour le soufre en suspension. Pour l'arsenic en suspension, on opère semblablement. Pratique un petit trou d'aiguille au centre du vase.

5. Autre coupe de verre placée au-dessous. Le vase de terre cuite sera de dimension telle qu'il s'ajuste aux parties arrondies et conforme à ces parties.\footnote{Figures 24 et 24 \emph{bis} de la page 48 de l'\emph{Introduction}.}

6. C'est le fourneau en forme de four, dit Marie, ayant à la partie supérieure trois trous (suçoirs), destinés à arrêter (les gros morceaux) et à évacuer (les parties fondues).\footnote{Ce sont les figures 20 et 21, page 143 de l'\emph{Introduction}.} Fais chauffer progressivement, en brûlant des roseaux grecs pendant deux ou trois jours et autant de nuits, selon ce que comporte la teinture, et laisse torréfier complètement dans le fourneau. Puis fais descendre pendant tout un jour de l'asphalte, en y ajoutant ce que tu sais, plus du cuivre blanc ou jaune. Or (cela) peut se faire ainsi : l'appareil en forme de crible blanchit, jaunit, produit de l'ios. Cuis légèrement, comme pour produire du fard, la teinture des mélanges et tout ce que tu pourras imaginer. Telle est la fabrication.

\bigskip
\centerline{\EightStarTaper}
\centerline{\EightStarTaper\EightStarTaper}
\bigskip

\subsubsection{3. --- 51. Le Premier Livre du Compte Final de Zosime le Thébain.}
\paragraph{}
1. Ici, se trouve confirmé le livre de la Vérité.

Zosime à Théosébie, salut !

Tout le royaume d'Égypte,\footnote{Cp. p. 203 et Olympiodore, p. 97. --- Le texte actuel offre des variantes notables.} ô Femme, dépend de ces deux arts, celui des (teintures) convenables et celui des minerais. L'art appelé divin, soit dans ses parties dogmatiques et philosophiques, soit dans la plupart des questions de moindre portée, a été confié à ses gardiens pour leur subsistance. Il en est ainsi non seulement pour cet art, mais encore pour les quatre arts appelés libéraux et pour les arts manuels. Leur puissance créatrice appartient aux rois. S'ils le permettent, celui-là l'expose de vive voix, ou l'interprète d'après les stèles, qui en a reçu la connaissance comme héritage de ses aïeux. Mais celui qui possédait la connaissance de ces choses ne fabriquait pas (pour lui-même), car il eût été puni ; de même que les artisans qui savent frapper la monnaie royale n'ont pas le droit de la frapper pour eux-mêmes, sous peine de châtiment. De même aussi, sous les rois Égyptiens, les artisans de l'art de la cuisson et ceux qui possédaient la connaissance des procédés n'opéraient pas pour eux-mêmes ; mais ils opéraient pour les rois d'Égypte, et travaillaient en vue de leurs trésors. Ils avaient des chefs particuliers placés à leur tête, et grande était la tyrannie exercée dans l'art de la cuisson, non seulement en elle-même, mais aussi en ce qui touche les mines d'or. Car en ce qui touche la fouille, c'était une règle, chez les Égyptiens, qu'il fallait une autorisation écrite.

2. Quelques-uns reprochent à Démocrite et aux anciens ... (La suite comme à la page 98, jusqu'à la fin du paragraphe.)

3. ... Quant aux teintures convenables, personne ni parmi les Juifs, ni parmi les Grecs, ne les a jamais exposées. En effet, ils les plaçaient dans les images, formées avec leurs propres couleurs et destinées à les conserver. Les opérations faites sur les minéraux diffèrent beaucoup des teintures convenables. Ils étaient très jaloux de la divulgation de l'art lui-même ; et ne laissaient pas le manipulateur sans punition. Celui qui fait une fouille sans autorisation, peut être précipité (et mis à mort) par les surveillants des marchés de la ville, chargés du recouvrement des impôts royaux. De même il n'était pas permis de mettre en œuvre secrètement les fourneaux, ou de fabriquer en secret les teintures convenables. Aussi tu ne trouveras personne parmi les anciens qui révèle ce qui est caché, et qui expose quelque chose de clair à cet égard. Je n'ai rencontré que Démocrite seul, parmi les anciens, qui ait fait entendre clairement quelque chose à cet égard, dans les énumérations de ses catalogues.

En effet, voici comment il débute, dans le préambule de sa composition sur les arts libéraux : ici observe sa malice. Il parlait seulement, au début, du mercure et du corps de la magnésie. Or les autres (substances) sont toutes de la classe des teintures convenables. Il s'exprime ainsi : « Ocre attique, minium du Pont, soufre natif : on en prend une livre ; pierre phrygienne, sori jaune, couperose sèche, cinabre, misy cuit, misy cru. Tu fabriqueras l'androdamas, le soufre, l'arsenic, la sandaraque. Pour ne pas énumérer tout ce qui est dans les quatre catalogues, tu trouveras toutes les substances propres aux (teintures) et pour que tu exécutes ce qu'il fait entendre là-dessus, il a énuméré les (substances) crues et les (substances) cuites, qui répondent aux deux arts. Il parle de préférence des teintures parmi les choses convenables. Lorsqu'il dit : misy cru, misy cuit, son jaune, couperose sèche et autres similaires ; il parle des (substances) qui ont subi un certain traitement, en s'attachant aux arts libéraux. Mais pourquoi ne parle-t-il pas de toutes ces substances, après qu'elles ont été traitées et jaunies, telles que le mercure jaune et le corps (de la magnésie) jaune, et généralement tout le catalogue jaune ?

4. Vois comment ce qu'il pensait et ce qu'il écrivait était présenté sous forme énigmatique ; il voulait tout faire entendre par énigmes. Les témoignages les plus dignes de foi qu'il ait trouvés sur ces choses, il les a fait entendre par énigmes. Comment se fait-il que sachant qu'il n'y a qu'une teinture et qu'une marche, il représentait celles-ci comme multiples, disant : « Parmi ces natures, il n'en est pas de meilleure pour les teintures ... ; » afin de montrer que les mêmes espèces peuvent servir à composer convenablement plusieurs teintures, la proportion variant suivant la quantité des espèces (destinées aux teintures [ ? ]) depuis une seule jusqu'au nombre de cinquante et une. En même temps, il parle de l'opération naturelle, c'est-à-dire de la matière de la fabrication de l'or, et il met en évidence les teintures naturelles. Il dit encore : « Je vous ai engagés dans un grand travail, si quelques-uns ayant opéré avec une quantité considérable de matière, venaient à échouer dans la fabrication des produits naturels. »

Au temps d'Hermès, on appelait teintures naturelles celles qui devaient être inscrites (plus tard) sous un titre commun, dans son ouvrage intitulé : \emph{Livre des Teintures naturelles}, dédié à Isidore.\footnote{Synonyme : Pétésis.} Lorsqu'elles avaient réussi avec les objets de cuivre, elles devenaient et étaient dites convenables. Au surplus, on reproche aux anciens et surtout à Hermès, de ne les avoir exposées, ni publiquement, ni en secret, et de ne pas avoir fait entendre ce que c'est.

5. Seul, Démocrite l'a exposé dans son ouvrage et l'a fait entendre. Mais eux, ils ont gravé ces procédés sur les stèles dans l'ombre des sanctuaires, en caractères symboliques ; ils y ont gravé ces procédés et la chorographie de l'Égypte\footnote{Voir le texte de Clément d'Alexandrie, \emph{Origines de l'Alchimie}, p. 41.} ; de telle sorte que, si quelqu'un osait affronter les ténèbres du sanctuaire pour obtenir la connaissance d'une façon illicite, il ne réussît pas à comprendre les caractères, malgré son audace et sa peine.\footnote{Note de A. « Il faut pénétrer le sens spirituel du caractère, et éviter les opinions tirées des paroles charnelles. » --- On voit à quelles imaginations donnaient lieu les vieux textes hiéroglyphiques que l'on ne comprenait plus (Cp. \emph{Introduction}, p. 135).} Mais les Juifs, ayant été initiés, ont transmis ces procédés convenables, qui leur avaient été confiés. Voici ce qu'ils conseillent dans leurs traités : « Si tu découvres nos trésors, abandonne l'or à ceux qui veulent se détruire eux-mêmes. Après avoir trouvé les caractères qui décrivent ces choses, tu réuniras toutes ces richesses en peu (de temps) ; mais si tu te bornes à prendre ces richesses, tu te détruiras toi-même, par suite de l'envie des rois qui gouvernent et de celle de tous les hommes.\footnote{Note de A. « Il y a beaucoup de livres relatifs à la chimie. Les uns parlent des teintures naturelles ; les autres, des surnaturelles : les deux ordres de livres sont mensonge et vérité dissimulée. »} »

6. Il y avait deux genres de (teintures) convenables, dans les toiles teintes,\footnote{La teinture des étoffes est ici assimilée à celle des métaux (voir \emph{Origines de l'Alchimie}, p. 242 et suiv.).} qu'ils présentaient à leurs prêtres ; voici pourquoi elles étaient appelées convenables,\footnote{Ou opportunes.} c'est parce qu'ils opéraient au moment voulu les teintures, à la volonté de ceux qui (les) attendaient ; mais pour ceux qui ne le demandaient pas, ils opéraient autrement. Les (teintures) convenables étaient obtenues par le mélange des espèces tinctoriales, en opérant avec les espèces pures. Les unes appartiennent à ces arts précieux ; quant à l'autre genre de teintures pures et naturelles, voici l'interprétation que Hermès grava sur les stèles : « Fais fondre seulement la matière jaune verdâtre, la matière jaune, la noire, la verte et les similaires. » Ils appelaient ces terres, en langage mystique, des minerais. Hermès indique aussi les espèces de couleurs : « Celles-ci agissent naturellement ; mais elles sont surpassées par les produits supraterrestres. Or si quelque initié s'en débarrasse, il obtiendra ce qu'il cherche. »

7. Ceux qui apportaient (les couleurs fabriquées) par voie surnaturelle ( ? ), étant ainsi mis de côté, conseillaient aux gens considérables d'agir contre nous tous, savants, opérant par des actions naturelles. Ils ne voulaient pas être mis de côté par les hommes,\footnote{Comme imposteurs.} mais être suppliés et adjurés de céder ce qu'ils avaient fabriqué, en retour des offrandes et des sacrifices. Ils tinrent donc cachés tous les procédés naturels, ceux qui donnent les résultats sans artifice. Ce n'était pas seulement par jalousie contre nous, mais parce qu'ils étaient soucieux de leur existence et ne voulaient pas s'exposer à être battus de verges, chassés, et à mourir de faim, en cessant de recevoir les offrandes des sacrifices. Ils opérèrent ainsi : Ils cachèrent les procédés naturels et mirent en avant les leurs, qui étaient d'ordre surnaturel\footnote{Ce curieux passage accuse la rivalité des opérateurs procédant par la magie et avec charlatanisme, contre ceux qui opéraient par la science seule et qui leur enlevaient leur clientèle.} ; ils exposèrent à leurs prêtres que les gens du peuple négligeraient les sacrifices, s'ils n'avaient plus recours aux procédés surnaturels, pour s'adresser à ceux qui possédaient cette prétendue connaissance des alliages vulgaires, cet art de fabriquer les eaux et de faire les lavages. C'est ainsi que, par l'effet de la coutume, de la loi et de la crainte, leurs sacrifices étaient très suivis. Ils n'accomplissaient même plus leurs annonces mensongères. Lorsque leurs sanctuaires venaient à être désertés et leurs sacrifices négligés, ils obtenaient encore des hommes restés (auprès d'eux), qu'ils s'adonnassent aux sacrifices, en les flattant par des songes\footnote{Les Papyrus de Leide renferment diverses formules pour procurer des songes et artifices magiques, à côté des procédés chimiques (voir \emph{Introd}, p. 13).} et d'autres tromperies, ainsi que par certains conseils. Ils revenaient sans cesse à ces promesses mensongères et surnaturelles, pour complaire aux hommes amis du plaisir, misérables et ignorants. Toi aussi, ô femme, ils veulent te gagner à leur cause, par l'intermédiaire de leur faux prophète ; ils te flattent ; étant affamés, (ils convoitent) non seulement les sacrifices, mais encore ton âme.\footnote{Ce paragraphe montre le caractère des polémiques entre Zosime et ses rivaux, polémiques dont nous avons la trace en plus d'un point de ses écrits. Cp. p. 186 et 187.}

8. Toi donc, ne te laisses pas séduire, ô femme, ainsi que je te l'ai expliqué dans le livre concernant l'Action. Ne te mets pas à divaguer en cherchant Dieu ; mais reste assise à ton foyer, et Dieu viendra à toi, lui qui est partout ; il n'est pas confiné dans le lieu le plus bas, comme les démons.\footnote{Cp. Olympiodore, p. 90.} Repose ton corps, calme tes passions, résiste au désir, au plaisir, à la colère, au chagrin et aux douze fatalités de la mort. En te dirigeant ainsi, tu appelleras à toi l'être divin, et l'être divin viendra à toi, lui qui est partout et nulle part. Sans être appelée, offre des sacrifices : non pas les (sacrifices) avantageux pour ces hommes, et destinés à les nourrir et à leur complaire ; mais des (sacrifices) qui les éloignent et les détruisent, tels que ceux qu'a préconisés Membrès, s'adressant à Salomon, roi de Jérusalem, et principalement tels que ceux qu'a décrits Salomon lui-même, d'après sa propre sagesse. En opérant ainsi, tu obtiendras les teintures convenables, authentiques et naturelles. Fais ces choses jusqu'à ce que tu sois devenue parfaite dans ton âme. Mais, lorsque tu reconnaîtras que tu es arrivée à la perfection, alors redoute (l'intervention) des éléments naturels de la matière : descendant vers le Pasteur, et te plongeant dans la méditation, remonte ainsi à ton origine.

9. Quant à moi, je viendrai au secours de ton insuffisance ; mais réfléchis et rappelle-toi la chose cherchée : il faut qu'elle n'éprouve pas d'amoindrissement, mais qu'elle suive ses degrés réguliers.

Écoute-le, quand il dit un peu plus loin : un seul produit existe, en lequel doivent se réunir deux œufs\footnote{Dans ce passage, le mot œuf est pris dans un sens mystique, comme désignant le produit d'une opération chimique. Cp. p. 18 et 19.} ; les composants sont divers ; l'un est humide et froid, l'autre sec et froid, et les deux produisent une œuvre unique. Il faut entendre ici les deux couleurs de l'œuf et admirer les changements de couleurs qui proviennent de l'œuf, ainsi que ceux qui précèdent, et toutes les générations de couleurs ; comme quoi elles indiquent l'expulsion de la matière (étrangère) ; après d'autres phénomènes, on peut les observer ; mais elles ne reparaissent pas (dans un état) semblable. Pourquoi (faut-il expliquer tout cela)\footnote{Note de A. « (Ainsi parle) les capable de comprendre, d'une manière droite et saine. »} ? N'est-ce pas parce qu'ils le cachent par jalousie ? Ils ne veulent pas que personne puisse comprendre et trouver par leur secours la voie des teintures favorables. Quelqu'un dira qu'il ne s'agit pas seulement du changement des noms, mais encore de tout l'art, qui n'est pas exposé (par tout le monde) d'une façon semblable ; il l'est tantôt d'une façon, tantôt de la façon contraire. Tout cela est nouveau, dis-je ; les artisans le savent, eux qui voient les causes des fautes commises ; ils savent que nous avons produit telle chose, plutôt que telle autre ; que nous avons négligé telle chose, et que nous avons fait telle autre chose avec plus de paresse.

10. Quant à moi, je reviendrai à mon propos. Il y a deux marches de teintures convenables, selon qu'on opère sur les espèces crues ou cuites. Le procédé de la cuisson est affranchi d'une grande fatigue ; il a besoin d'une grande adresse et il est plus court, comme l'a dit la divine Marie. Pour ce procédé de cuisson, il y a de nombreuses variétés de liquides et de feux. Tantôt on cuit avec de l'eau, tantôt avec du vin. (Parmi les feux), les uns sont obtenus avec des charbons et soutenus pendant tout le temps ; dans les autres on procède par insufflation, suivant une certaine mesure. Dans d'autres on emploie des broussailles ; dans d'autres, des fourneaux, et dans d'autres des chiffons ; ou bien l'on opère par d'autres voies : par tous ces moyens on obtient beaucoup de choses diverses. Ainsi, par exemple, pour le noir : suivant la diversité des œufs,\footnote{Voir la note 1 de la page précédente.} on peut avoir le noir des corbeaux, le noir des corneilles, le noir très foncé, la couleur gris cendré sur les toiles peintes. On y dessine aussi\footnote{Sur les étoffes peintes ?} des arbres, ou des pierres, ou de l'eau, ou des animaux, tous semblablement. Quant aux autres couleurs susdites, tu en as les démonstrations comprises sous la lettre K.\footnote{Un peu plus loin Zosime vise la section Ω (voir aussi p. 221).  } Il faut tenir compte de la proportion des couleurs ; si tu entends parler de l'ocre jaune, ne suppose pas simplement que j'aie changé la préparation et que je tienne un langage mystérieux, dans le seul but de créer des difficultés ; car dans l'art, toutes les préparations (indiquées pour notre) recherche réussissent.

Suidas rapporte que Zosime avait écrit un livre sur la chimie adressé à sa sœur Théosébie, divisé en sections désignées par les lettres de l'alphabet grec.

11. Ces teintures ont une nature propre. Elles résultent de la décomposition de produits tantôt nombreux, tantôt en petit nombre ; elles sont fabriquées dans de petits fourneaux, avec des vases de verre, ou bien dans des creusets grands et petits : on opère ainsi dans différents appareils, au moyen de feux diversement réglés. L'épreuve manifeste la bonté des produits obtenus en suivant ces divers perfectionnements. Voici que tu as les démonstrations des feux dans la lettre Ω, ainsi que celles de toutes les choses cherchées. Tel sera mon commencement, ô femme à la robe de pourpre.

\bigskip
\centerline{\EightStarTaper}
\centerline{\EightStarTaper\EightStarTaper}
\bigskip

\subsubsection{3. --- 52. Interprétation sur Toutes Choses en Général et (Notamment) sur les Feux.}
\paragraph{}
1. Veille à ne pas t'égarer et à jaunir non seulement le plomb et le cuivre, mais encore les espèces métalliques appelées liqueur d'or, or massif [etc.],\footnote{Ou la matière dorée.} lesquelles sont au nombre de 78, plus ou moins. J'ai dit 78, plus ou moins, suivant que l'on emploie (ou non) le mercure. Or il faut connaître l'épreuve et la vertu des préparations, ainsi qu'il le rappelle en parlant des feux ; il faut faire cuire, en introduisant du fer. En effet, les uns faisaient cuire une demi-heure seulement ; d'autres une heure, d'autres deux, d'autres trois, et quelques-uns même quatre.

2. Tout l'art consiste dans les feux légers\footnote{Cette phrase est restée, comme la seule trace du morceau tout entier, dans M (voir \emph{Introd.}, p. 185).} ; fais cuire les couleurs et laisse (sur le feu) jusqu'à refroidissement ; regarde dans les (vases) de verre ce qui se passe. De cette façon, (la matière) jaunit par le délaiement et par la décoction. C'est là l'eau divine, l'eau aux deux couleurs, blanche et jaune ; on lui a donné mille dénominations.

3. Sans l'eau divine, il n'y a rien : toute la composition s'accomplit par elle ; c'est par elle qu'elle est cuite ; c'est par elle qu'elle est calcinée ; c'est par elle qu'elle est fixée ; c'est par elle qu'elle est jaunie ; c'est par elle qu'elle est décomposée ; c'est par elle qu'elle est teinte ; c'est par elle qu'elle subit l'iosis et l'affinage ; c'est par elle qu'elle est mise en décoction. En effet, il dit : « En employant l'eau du soufre natif et un peu de gomme, tu teindras toute sorte de corps. Toutes les choses qui tirent leur origine de l'eau sont incompatibles avec celles qui proviennent du feu ; de telle sorte que, sans le catalogue de tous les liquides, il n'y a rien de sûr. »

4. Quelques-uns, tous peut-être, ont rappelé qu'il faut que cette eau, destinée à agir comme ferment, détruise le semblable par le semblable, en opérant sur le corps que l'on veut teindre, soit en argent, soit en or. Si tu veux teindre l'argent, fais réagir des feuilles d'argent ; si c'est l'or, des feuilles d'or. Car Démocrite (dit) : « Projette l'eau (divine) sur l'or commun, et tu donneras une teinte parfaite d'or. Une seule liqueur est reconnue comme agissant sur les deux (métaux). » Il faut donc que l'eau divine joue le rôle d'un levain produisant le semblable, soit avec l'argent, soit avec l'or. En effet, de même que le levain du pain, bien qu'en petite quantité, fait lever une grande quantité de pâte ; de même aussi, agit une petite quantité d'or ou d'argent, avec le concours de ce vinaigre.\footnote{B « ... de même un peu d'or : la poudre sèche doit faire tout fermenter. »}

\bigskip
\centerline{\EightStarTaper}
\centerline{\EightStarTaper\EightStarTaper}
\bigskip

\subsubsection{3. --- 53. La Céruse.}
\paragraph{}
1. ... puissance ; après l'opération, la céruse est adoucie au moyen de l'eau de pluie et abandonnée à elle-même. Décante l'eau et tu trouves une matière tout à fait blanche. La litharge commune, tirée du plomb, aune puissance merveilleuse quand elle est associée au vinaigre. Le plomb perd ses propriétés métalliques, étant salifié et adouci : cette litharge devient ainsi très blanche et présente tout à fait l'aspect de la céruse.\footnote{C'est une fabrication de céruse, au moyen du vinaigre, agissant soit sur le plomb, soit sur la litharge.}

J'admire aussi la rubrique (minium) ; (je vois) comment elle jaunit au feu. La sandaraque a aussi une puissance merveilleuse.\footnote{On remarquera l'analogie établie entre la formation de la céruse, matière blanche, produite au moyen de la litharge jaune et du minium rouge, et la métamorphose de la sandaraque (réalgar) rouge, en acide arsénieux blanc. Dans d'autres passages, l'acide arsénieux est même désigné par le nom de céruse.}

\bigskip
\centerline{\EightStarTaper}
\centerline{\EightStarTaper\EightStarTaper}
\bigskip

\subsubsection{3. --- 54. Sur le Blanchiment.}
\paragraph{}
1. Je veux que vous sachiez que le point capital en toutes choses, c'est le blanchiment ; aussitôt après le blanchiment, on jaunit : c'est le mystère parfait, c'est-à-dire l'iosis, laquelle s'effectue à son tour au moyen du vinaigre, agent des puissances divines. Je vous révélerai d'abord le chapitre de l'huile sulfureuse ; et je vous exposerai comment on opère les blanchiments des plombs, et quelle est l'origine de l'esprit tinctorial. Car sans les plombs on ne peut pas accomplir l'œuvre : le plomb sert à éprouver toute substance.\footnote{Par la coupellation ?} C'est ce que le Philosophe a décrit merveilleusement par un exposé indirect, en disant : « Si les substances ont subi l'action des agents qui servent à l'épreuve, (la) nature du produit est indélébile.\footnote{En d'autres termes : quand le plomb est intervenu dans la transmutation, le métal transformé résiste ensuite aux essais d'analyse faits au moyen de ce métal.} »

2. Je veux que vous sachiez d'abord que l'épreuve définitive se fait avec le vinaigre. En second lieu, c'est l'épreuve par le plomb dont le Philosophe a parlé dans son second chapitre, (en disant) : « Si les substances ont subi l'action des agents qui servent à l'épreuve, la nature du produit est indélébile. »

\bigskip
\centerline{\EightStarTaper}
\centerline{\EightStarTaper\EightStarTaper}
\bigskip

\subsubsection{3. --- 55. Explication sur les Feux.}
\paragraph{}
1. Je vous expliquerai, avec tous les prophètes, la puissance des feux, afin que votre travail soit parfait et conforme aux traditions, de façon à ne pas échouer. En effet, le Philosophe exposait, en parlant des feux, comment l'unité de l'espèce est transformée par un feu excessif ; car l'excès des feux est contraire à toute l'opération. Pour les choses auxquelles vous êtes exercés préalablement, je vous transmets les préceptes suivants : Si les matières sulfureuses sont cuites dans des vases de verre, il est nécessaire d'employer les feux dont se servent les peintres avec la kérotakis. Il est nécessaire que le vase de verre soit garni d'un lut céramique, de l'épaisseur d'un demi-doigt ; afin que le vase ne casse pas sous l'influence de la chaleur. Voici la proportion convenable pour les feux : Si tu dois faire cuire légèrement les (matières), en les poussant vers le jaune, il est nécessaire d'employer les feux modérés, tels que ceux usités dans le fourneau à fusion des figures en couleur. Lorsque tu veux opérer de façon à amener le produit au jaune, laisse dans le fourneau pendant six heures ; je parle de la durée moyenne ; cela suffit : les feux amènent ainsi le produit au jaune.

\bigskip
\centerline{\EightStarTaper}
\centerline{\EightStarTaper\EightStarTaper}
\bigskip

\subsubsection{3. --- 56. Sur les Vapeurs.}
\paragraph{}
1. On les appelle vapeurs sublimées, à cause de ce fait que les substances sont élevées de bas en haut, au-dessus des cendres, vers la partie supérieure, comme il est exposé dans le traitement des eaux. Ainsi on les appelle vapeurs sublimées, à cause de ce fait qu'elles montent du bas vers le sommet de l'appareil, et nous avons exposé comment on opère l'aspiration de ces vapeurs ou de ces gouttes condensées.

On enlève les scories de la marmite, on les délaie et on projette sur elles les âmes que l'on en a tirées.\footnote{C'est-à-dire des vapeurs sublimées : mercure, arsenic, soufre, etc.} Ces âmes tirées des corps (métalliques), ils les sublimèrent de nouveau au moyen de l'appareil en forme de mamelle, disant que c'était là l'iosis, accomplie par les réactions de longue durée. Ils combinèrent avec les autres vapeurs sublimées ce qu'ils nommaient des corps --- ce que nous appelons un corps métallique --- en opérant avec les soufres, les sulfures (agissant sur) les feuilles de cuivre, ou d'asèm,\footnote{Ms. A. « d'argent. »} ou d'or. C'est de cette façon qu'ils pratiquèrent la teinture avec les matières auxiliaires, sans tenir aucun compte de leur second traitement.

2. Il a désigné l'eau filtrée (agissant) sur les cendres, en disant : « Dispose l'appareil et apporte les cendres ; la cendre éprouve l'action de l'eau, agissant dans l'appareil. » C'est aussi pour cela qu'Agathodémon (dit) : « La cendre est tout ; c'est sur elle qu'opère la décoction ou la cuisson, ou bien ce que l'on appelle le délaiement. » Ainsi, au moyen de la décomposition, de l'extraction, de l'iosis, de la cuisson modérée, les anciens disaient que le Tout se parfait.

Il est impossible de traiter autrement la fabrication\footnote{Au lieu de la fabrication, les manuscrits portent : « la pyrite, » (ou son signe), lequel est presque le même.} de la composition. Car ce fait que la décoction et l'extraction sont un délaiement est connu des interprètes de la science. L'iosis, ils la nommèrent décoction ; la décoction et l'extraction, délaiement, destiné à produire une atténuation extrême. En outre ils parlèrent du feu, parce qu'il produit la chaleur, la combustion et la flamme. Les anciens traitaient d'enfantillage et de travail de femme la recherche des simples connaisseurs. Mais nous ne sommes pas obligés pour cette raison d'effectuer l'iosis au moyen du feu, ainsi qu'on opère pour les pierres teintes, ou dans le traitement des liquides pour la pourpre fabriquée à froid. Je dis que l'expérience nous enseignera la vérité ; elle nous conduira à accomplir l'œuvre une et parfaite et à (obtenir) la poudre de projection stable.

3. Après la fabrication de cet ios, ils transportèrent les vapeurs et (les) réunirent aux scories restantes, et de cette façon ils arrivèrent au terme. C'est alors qu'ils projetaient la poudre tinctoriale sur les corps, attendu que Zosime dit : « Ainsi les esprits prennent un corps et les corps morts sont ranimés par l'âme qui provient d'eux, et qui est de nouveau reçue en eux. Ils réalisent l'œuvre divine, deux éléments se dominant mutuellement et étant dominés l'un par l'autre. » En effet, il obtient par-là l'esprit fugace du corps poursuivant,\footnote{V. p. 105.} et il nous instruit à rendre la teinture résistante au feu, par le moyen du feu. Telle est, je pense, d'après le Philosophe, l'eau de chaux, ou de sandaraque, l'eau de natron, l'eau de lie, l'eau fabriquée avec la cendre des sulfureux, l'eau de la première distillation.

4. Il faut la rectifier comme la lessive des savonniers,\footnote{Ce mot ne doit pas être pris dans le sens de la chimie moderne.} et recueillir les eaux qui en proviennent : or la lessive des savonniers ne se réduit jamais en vapeur, mais elle est rectifiée.\footnote{Par filtration.} Comment donc, ô philosophes, Zosime\footnote{Dans BA, c'est Démocrite.} peut-il dire que le sens des écritures n'est pas compris, à moins que l'on n'emploie l'appareil qui opère l'extraction sur le cuivre ? et que le terme de l'art n'est pas celui-là, mais consiste dans l'appareil et dans la fixation qui y est accomplie ? D'autres se servaient seulement de flacons pour les deux genres de compositions.\footnote{Sans procéder par distillation, ou sublimation.} Après avoir fait monter l'eau, ils la réunissaient à la chaux ordinaire, en délayant dans un mortier ; non pas suivant une mesure précise, mais de façon que la partie sèche dépassât le liquide de deux, trois ou quatre doigts.

\bigskip
\centerline{\EightStarTaper}
\centerline{\EightStarTaper\EightStarTaper}
\bigskip

\end{document}
